\section{Problem Analysis}
\label{sec:problem_analysis}

% PROBLEMS OF NEGATIVE SAMPLING
% GENERAL
First of all, it can be said that while many Negative Sampling methods currently demonstrate high performance, the sampled negatives are often too simple and represent a trivial solution. 
The embedding models do not learn from the provided negative triples and therefore, they do not improve the embedding and suffer from the vanishing gradient or biased estimation problem \cite{zhang2021efficient}.
The vanishing gradient problem is present when the gradients of the loss functions approach zero and consequently, the model is unable to learn during the training process.
This results from the fact that most \ac{KGE} models, due to simplicity and efficiency, use \textsc{Uniform Negative Random Sampling}.

% UNIFORM RANDOM SAMPLING
In this standard technique of Negative Sampling either the head or the tail entity in a given positive triple \triple{h}{r}{t} is replaced by any other entity of the \ac{KG} which remains in the new negative triple \triple{h’}{r}{t} or \triple{h}{r}{t’}. 
Therefore, it is very likely to pick an entity which results in a zero gradient because the negative triple can be easily discriminated from the positive one \cite{cai2017kbgan}.
For example, by replacing the head entity of the positive triple \triple{h}{r}{t} = \triple{Joe Biden}{bornIn}{USA} with head entity $h' = Paderborn$ would result in the negative triple \triple{Paderborn}{bornIn}{USA} which is not very informative for the embedding.
By simply replacing the randomly selected head or tail entity of an again randomly selected entity of the \ac{KG} does not use any further information.
For example, it would have been useful either if Negative Sampling had recognized that the head entity \textit{Joe Biden} is a person and to replace it with another person.
Or by recognizing that the tail entity as well as the sampled entity \textit{Paderborn} is a place and its replacement would have led to the much more meaningful negative triple \triple{Joe Biden}{bornIn}{Paderborn}.  
Thus, while this approach is a fast and effective way to generate negative triples, it leads to a low learning factor in the embedding model.

% BERNOULLI SAMPLING
More information about the \ac{KG} and individual relations and therefore, more useful negative examples are created by \textsc{Bernoulli Sampling}.
In comparison to Uniform Negative Random Sampling, it considers types of relations between entities (one-to-many, many-to-one and many-to-many) \cite{zhang2021efficient}.
These relation types is an indicator for the sampling approach if it is better to replace the head or the tail entity.
From the above example, it would have been recognized that the relation \textit{bornIn} is a many-to-one relation.
Therefore, the head entity cannot have this relation to multiple entities, making each replaced tail entity a more useful negative triple.
Even though this is still a very fast and effective way to create negative triples, they are still easy to distinguish from the positive ones.

% OTHER INFORMATION USED
In addition to these most commonly used methods, there are others which leverage external ontological constraints such as entity types.
However, this resource does not always exist or is accessible \cite{cai2017kbgan}.
Instead of sampling entities from the \ac{KG}, other Negative Sampling methods take the approach of sampling from only a handful of selected entities.
For example, by sampling entities in the same domain, they hope to increase efficiency \cite{qiannegative}.
Due to the rapid growth and frequent updating of \acp{KG}, the constant renewal of custom clusters is essential and skilled \cite{qiannegative}. 
However, the creation of subsets of entities leads to a degradation of sampling performance.
In addition, the information needed for this is not always available or is very difficult to derive from a \ac{KG}.

%However, this insight about constantly updating the graph and making negative sampling dependent on the underlying KG provides the basis for further approaches.
%These attempt to estimate the distribution of negative triples and follow a dynamic sampling approach. 

%Many of these approaches aim to find hard negative examples that are close to positive facts from a given \ac{KG} and thus have a positive effect on the embedding learning process.


%In addition, a distinction must be made for which \ac{KGE} models the Negative Sampling is used.
%For example, the problem of too easy negative examples is less severe to models using log-softmax loss function since they sample tens or hundreds of negative triples for one positive triple.
%By sampling so many negatives for just one positive it is very likely to have a few good ones which help the model to learn the embedding \cite{cai2017kbgan}.
%On the other hand, the problem is more severe for KGE models with marginal loss function, because they use a one-to-one ratio for positive and sampled negatives and therefore, bad negatives seriously damage their performance \cite{cai2017kbgan}.





%Most of graph representation learning methods can be unified within a \ac{SampledNCE} framework comprising an encoder that generates node embeddings by learning to distinguish pairs of a positive and a negative triple \cite{MCNS}.
%Therefore, we need negative triples for embedding learning which are not available in most \acp{KG}. 

%For this reason, we are left with two options to collect them \cite{safavi2021negater}: 
%On one hand they can be obtained by human annotation which is very cost-prohibitive at scale, but leads to very informative and useful examples. 
%On the other, they can be generated ad-hoc, which in turn does not represent a great effort, but also leads to uninformative and useless negative examples.
%For this reason, we need to find a way between these two extreme approaches.





