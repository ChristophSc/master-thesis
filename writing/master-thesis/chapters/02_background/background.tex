\chapter{Related Work}
\label{ch:relatedwork}

\section{Knowledge Graphs} 
\label{sec:knowledge_graphs}

There are several definitions of a \acp{KG} in the literature. 
In the scope of this work, we work with the following definition given by \cite{ConEx, RotatE}.
Let the set of entities be represented by \entities and relations by \relations.
Then, a \ac{KG} $\kg= \{\triple{h}{r}{t} \}  \subseteq \entities \times \relations \times \entities$ can be formalised as a set of triples where each triple contains a head entity \texttt{h} and a tail entity \texttt{t} with \texttt{h}, \texttt{t} $\in$ \entities and a relation \texttt{r} $\in$ \relations which can be \cite{ConEx}
\begin{itemize}
    \item 
    \emph{symmetric} if $\triple{h}{r}{t} \iff \triple{t}{r}{h}$ for all pairs of entities $\texttt{h},\texttt{t}\in \entities$, 
   
   \item 
   \emph{anti-symmetric} if $\triple{h}{r}{t} \in \kg \Rightarrow \triple{t}{r}{h} \not \in \kg$ for all $\texttt{h} \not= \texttt{t}$, and
    
    \item 
    \emph{transitive}/\emph{composite} if $\triple{h}{r}{t}\in\kg \wedge \triple{t}{r}{y} \in \kg  \Rightarrow \triple{h}{r}{y} \in \kg$ for all $\texttt{h},\texttt{t},\texttt{y}\in \entities$.
\end{itemize}
In addition, $\texttt{r}^{-1}$ denotes the inverse of a relation \texttt{r} where for any two entities \texttt{h} and \texttt{t}, \triple{h}{r}{t} $\in$ \kg $\iff (\texttt{t},\texttt{r}^{-1},\texttt{h}) \in \kg$ \cite{ConEx}.


\section{Knowledge Graph Embeddings} 
\label{sec:knowledge_graph_embeddings}


\acfp{KGE} are low-dimensional representations of entities and relations in a \ac{KG}. 
Numerous methods have been developed in the last few years to tackle various problems such as defining a different score function to measure the distance between entities relative to their relation.
Depending on the dimensionality of the embedding space and which types of relations should be embedded, a different embedding can be chosen.
While some of them support only symmetric relations, others support antisymmetric ones as well. 
Besides this, the types of relations in a \ac{KG} differ from each other.
We can differentiate between 1-to-1, 1-to-N, N-to-1, and N-to-N relations.
Overall, all the embedding methods can be separated by three different aspects \cite{electronics9050750}:
(i) How entities and relations are represented, (ii) how the scoring function is defined and (iii) how the ranking criterion is optimized.
On the basis of this, the two different categories 
\textit{Triplet Fact-Based Representation Learning Models} (\Autoref{subsec:triplet_fact_based_representation_learning_models}) and \textit{Description-Based Representation Learning Models} (\Autoref{subsec:description_based_representation_learning_models}) can be derived \cite{electronics9050750}.

\subsection{Triplet Fact-Based Representation Learning Models} 
\label{subsec:triplet_fact_based_representation_learning_models}

Triplet Fact-Based Representation Learning Models are separated into three groups:
\begin{enumerate}
    \item 
    \textbf{Translation-Based Models} are based on word embedding algorithms and inspired by \textit{word2vec} \cite{electronics9050750}.
    Examples for \ac{KGE} models of this group are \textsc{TransE} \cite{TransE}, \textsc{TransH} \cite{TransH}, \textsc{TransR} \cite{TransR}, \textsc{TransD} \cite{TransD}
    and 
    \textsc{RotatE} \cite{RotatE}.
    The basic idea of \transe is that a functional relation \texttt{r} corresponds to a translation of embeddings, i.e. $h + r \thickapprox t$ when \triple{h}{r}{t} holds.
    If it does not hold, \texttt{t} should be far away from $h + r$ which lead to a energy-based framework $d(h+r, t)$ for some dissimilarity measure $d$ which is the $L_1$ or $L_2$ norm \cite{TransE}.
    The idea of \transe is depicted in \Autoref{fig:translationbasedmodels} (a).
    \begin{figure}[H]
      \centering
        \includegraphics[width=0.95\textwidth]{figures/Transe+TransD.pdf}
      \caption{Illustrations of (a) \transe and (b) \transd (based on: \cite{electronics9050750}).}
      \label{fig:translationbasedmodels}
    \end{figure}
    Therefore, head entity \texttt{h}, tail entity \texttt{t} and relation \texttt{r} are in the same vector space and the model tries to create the embeddings such that $h+r$ map to \texttt{t} as closely as possible.
    This leads to the scoring function
    \begin{equation}
        f_r(h,r,t) = || h + r - t ||_{l_1, l_2}
        \label{eq:transescoringfunction}
    \end{equation}
    \transd on the other hand works with two vectors for each entity and relation.
    The first one captures the meaning and the second one is used to construct the mapping matrices \cite{TransD}.
    It is illustrated in \Autoref{fig:translationbasedmodels} (b).
    
    \item 
    \textbf{Tensor Factorization-Based Models} transform triple facts into a 3D binary tensor $\mathcal{X} \in \mathbb{R}^{n \times n \times m}$ where $n$ and $m$ are the number of entities and relations \cite{electronics9050750}.
    Examples of this group are \textsc{RESCAL} \cite{RESCAL}, \textsc{DistMult} \cite{DistMult}, \textsc{ComplEx} \cite{ComplEx}, and \textsc{HolE} \cite{HolE}.
    \distmult simplifies the computational complexity of \rescal and restricts matrix $M_r \in \mathbb{R}^{d \times d}$, which is a matrix associated with the relation, to be a diagonal matrix, i.e. $M_r = diag(r), r \in \mathbb{R}^d$ \cite{electronics9050750}. 
    Therefore, its scoring function is transformed to
    \begin{equation}
        f_r(h,r) = h^{\top}diag(r)t\label{eq:distmultscoringfunction}
    \end{equation}
    Since \distmult is not able to model asymmetric relations, \complex \cite{ComplEx} extends \distmult by complex-valued embeddings \cite{electronics9050750} which leads to the scoring function
    \begin{equation}
        f_r(h,r) = Re(h^{\top}diag(r)\bar{t})
        \label{eq:complexscoringfunction}
    \end{equation}
    where $Re(\cdot)$ denotes the real part of a complex value and $\bar{t}$ is the complex conjugate of tail t \cite{electronics9050750}.
    
    \item 
    \textbf{Neural Network-Based Models} use deep learning and its representation and generalization capabilities to embed a knowledge graph by a neural network \cite{electronics9050750}.
    \textsc{ConvE} \cite{ConvE}, \textsc{HypER} \cite{HypER}, \textsc{ConEx} \cite{ConEx}, \textsc{ConvQ} and  \textsc{ConvO} \cite{demir2021convolutional} are examples for this group.
    
\end{enumerate}

\subsection{Description-Based Representation Learning Models}
\label{subsec:description_based_representation_learning_models}

Description-Based Representation Learning Models can also be broken down into three further categories:
\begin{enumerate}
    \item 
    \textbf{Textual Description-Based Models} is an extension of the traditional triplet-based model and integrates additional textual information for entities to evolve its performance.
    Examples are \ac{TKRL} \cite{TKRL} and \ac{TEKE} \cite{TEKE}.
    
    \item 
    \textbf{Relation Path-Based Models} refines the performance of an embedding model by multi-step relational paths, which reveal semantic relations between the entities.
    One example is \ac{PTransE} \cite{PTransE}.
    
     \item 
    \textbf{Other Models} that cannot be assigned to the other two categories and pay attention to further triple information such as temporal aspects.
\end{enumerate}





\section{Training Objective Types} 
\label{sec:training_objective_types}

As described in \cite{cai2017kbgan} several Training Objective Types for \ac{KGE} models exist.
Every model defines a \textit{score function} $f(h,r,t)$ which assigns a score to every possible triple in the knowledge graph.
The estimated likelihood of a triple to be true depends only on its score given by the score function.
Since different \ac{KGE} models formulate different designs for this scoring function, the interpretation of this score is also different which in turn leads to different training objectives.
The two most common forms are \textbf{Marginal Loss function} and \textbf{Log-softmax function}:
\begin{enumerate}
    \item 
    \textbf{Marginal loss function} is mostly used by translation-based models like \textsc{TransE} and \textsc{TransD}.
    Since these models work with distances, a smaller distance indicates a higher likelihood of a triple to be true.
    It is defined as 
    \begin{equation}
        L_{m}=\sum_{(h,r,t)\in\mathcal{T}}[f(h,r,t)-f(h',r,t')+\gamma]_+\label{eq:marginalloss}
    \end{equation}
    where $\gamma$ is the margin, $[\cdot]_+=\max(0,\cdot)$ is the hinge function, and $(h',r,t')\in\{(h',r,t)|h'\in\mathcal{E}\}\cup\{(h,r,t')|t'\in\mathcal{E}\}$ is a negative triple.
    
    \item \textbf{Log-softmax loss function} is commonly used for models which have a probabilistic interpretation for score function like \distmult or \complex.
    Therefore, for corrupted, negative triples the score should be lower than for positive ones.
    The loss function is defined as
    \begin{equation} \label{eq:nllloss}
        L_{l}=\sum_{(h,r,t)\in\mathcal{T}}-\log \frac{\exp f(h,r,t)}{\sum\exp f(h',r,t')}
    \end{equation}
    where $(h',r,t')\in\{(h,r,t)\}\cup Neg(h,r,t)$ and $Neg(h,r,t)\subset\{(h',r,t)|h'\in\mathcal{E}\}\cup\{(h,r,t')|t'\in\mathcal{E}\}$ is a set of sampled corrupted triples.
    Therefore, it provides a probability distribution over a set of triples where each triple $(h, r, t)$ has probability $p(h,r,t)=\frac{\exp f(h,r,t)}{\sum_{(h',r,t')}\exp f(h',r,t')}$.
    In the original \kbgan approach this distribution is provided by the generator from which one negative triple is sampled and given to the discriminator.
    
    \item 
    \textbf{Other forms} like a triple-wise logistic function \cite{ConvE} exist, but the two other ones are the most common used.
\end{enumerate}




\begin{figure*}[t]
  \centering
    \includegraphics[width=0.95\textwidth]{figures/loss_functions.PNG}
  \caption{Loss functions of \acp{KGE} (based on \cite{9207513})}
  \label{fig:overview}
\end{figure*}



\section{Training Techniques}
\label{sec:training_techniques}

Among all of there \ac{KGE} models, three commonly used approaches are used for training and differ mainly in the way how negative examples are generated \cite{Ruffinelli2020You}:

\subsection{Negative Sampling}
In Negative Sampling a set of (pseudo-) negative triples is generated by perturbing the subject, relation or object for each positive triple $(h, r, t)$ from the training data a set.
Optionally each obtained negative triple is verified, if it exists in the training \ac{KG} (false negative triples).


\begin{algorithm*}[t]
\small
 \KwIn{Observed facts $\mathbb{D}^+ = \{((h,r,t)\}$; Loss function $\mathcal{L}$}
 \KwOut{Embeddings of Entities and relations}
 \Repeat{convergence}{
      $\mathbb{P} \longleftarrow$ a small set of positive facts sampled from $\mathbb{D}^+$\;
      $\mathbb{B}^+ \longleftarrow \emptyset$;
      $\mathbb{B}^- \longleftarrow \emptyset$\;
     
      \For{$\tau^+ = (h,r,t)\in \mathcal{P}$}{
          Generate negative triple fact $\tau^- = (h',r',t')$ \;
          $\mathbb{B}^+ \longleftarrow \mathbb{B}^+ \cup \{\tau^+\} $\;
          $\mathbb{B}^- \longleftarrow \mathbb{B}^- \cup \{\tau^-\} $\;
      }
      Update entity and relation embeddings w.r.t the gradients of loss function $\mathcal{L}$\;
      Handle additional constraints and regularization terms
}
\caption{Training Under Open World Assumption}
\label{alg:ucgan}
\end{algorithm*}



\subsection{1vsAll}:
The 1vsAll training approach omits sampling by perturbing subject and object posiitions for all positive triples from the \ac{KG}, even if these negative triples exist as positive triples.
In comparison to the Negative Sampling approach it is generally more expensive but feasible if the number of entities is not excessively large.
    
\subsection{KvsAll}:
The KvsAll training type is divided into two different steps:
At first, triples are created from non-empty rows $(h,r,t)$.
Therefore, a pair $(h,r)$ is taken and scored against all entities $t \in \entities$.
Secondly, all of these generated triples are labeled as either positive or negative.

\section{Negative Sampling Methods} 
\label{sec:negativesamplingmethods}

In the literature, several negative sampling methods are proposed to create synthetic negative examples which are used for subsequent embedding learning.
They can be separated into three different groups \cite{qianunderstanding}:


\subsection{Static Distribution-Based Sampling}
\label{subsec:static_distribution_based_sampling}

Static Distribution-Based Sampling include methods like the uniform, Bernoulli and probabilistic sampling technique from a fixed and static distribution of negative triples.	While in the uniform sampling either the head or tail entity is replaced by an entity randomly sampled from entity set \entities,
the Bernoulli sampling technique uses different probabilities for replacing head or tail entity depending on the underlying relation type.
In contrast, probabilistic sampling speeds up the training process by including a train bias.
% TODO: Descibe Random Sampling in more detail
% TODO: Descibe Bernoulli Sampling in more detail
	
\textbf{Evaluation}:\\
% STATIC-DISTRIBUTION-BASED SAMPLING
Static Distribution-Based Sampling approaches are commonly used because of their simplicity and efficiency, but ignore the dynamics in the Negative Sampling distribution which lead to the vanishing gradient problem \cite{qianunderstanding}.
This problem occurs when the gradient will be vanishingly small and accordingly, small gradients prevent changing the weight value.
This can impede the training process or, in the worst case, completely stop the model from further training.
% RANDOM UNIFORM NEGATIVE SAMPLING
While negative triples, such as those generated by randomly replacing a head or tail entity, are very likely to be negative examples, they are generally uninformative and useless.
For example, by replacing the tail entity in given positive triple (Paderborn, locatedIn, Germany) by randomly selected entity 'Apple' leads to the new negative triplet (Paderborn, locatedIn, Apple).
Even though it is a true negative triple, it is uninformative and useless for embedding learning.
The problem with 'too easy' negative triples is less severe to models using log-softmax loss function, because they usually sample a high amount of negatives for one positive triple \cite{cai2017kbgan}.
However, the performance of marginal loss functions can be seriously damaged by the low quality of uniformly sampled negatives since negative-to-positive ratio is always 1:1 \cite{cai2017kbgan}.

\subsection{Custom Cluster-Based Sampling}
\label{subsec:custom_cluster_based_sampling}


Custom Cluster-Based Sampling samples negative triples from small clusters which are based on closeness between entities.
Instead of sampling from the whole set of entities, they a are divided into a number of groups and randomly sampled to create negative triples. Examples are TransE-\ac{SNS} \cite{TransE-SNS} or \ac{NSCaching} \cite{zhang2019nscaching} which are based on K-Means clustering algorithm or caching techniques. 


\textbf{Evaluation}:\\
% CUSTOM CLUSTER-BASED SAMPLING
To get more efficiency in the training process and to search for suitable entities in a more targeted way is aimed with  \textit{Custom Cluster-Based Sampling} methods (\autoref{sec:negativesamplingmethods}).
By selecting negative samples only from a handful of candidates and not from the entire entity set, a better correlation between the positive and corresponding negative triple is expected.
This negative triple should be closer to the original positive triple and thus provide more valuable information for the \ac{KGE} model.
For example, Domain Sampling \cite{domainSampling} is to sample from the same domain.
For example, if it is recognized that 'Germany' is a country and that 'France' is a nearby country, the much more valuable negative triple (Paderborn, locatedIn, France) could be sampled from the  positive triple (Paderborn, locatedIn, Germany).
However, as \acp{KG} grow rapidly and are updated frequently, continuous renewing custom clusters is essential and difficult \cite{qianunderstanding}.

With these approaches of Negative Sampling from a fixed distribution two more different problems arise:
Since they ignore changes in the distribution of negative triples, they suffer from the vanishing gradient and biased estimation problem \cite{zhang2021efficient}.
The scoring functions tend to give observed (positive) triplets large values and most of the non-observed (probably negative) triplets will have smaller values during the training.
Therefore, if negative triplets are uniformly sampled, it is very likely to pick up one with zero gradients.
This leads to the following main challenges for Negative Sampling \cite{zhang2021efficient}: 
(i) The negative triple's dynamic distribution has to be captured and 
(ii) triples have to be effectively sampled from this distribution.

\subsection{Dynamic Distribution-Based Sampling}
\label{subsec:dynamic_distribution_based_sampling}

Dynamic Distribution-Based Sampling tries to model the changes the distribution of negative triples by using a \ac{GAN}-based framework which includes two components: A generator and a discriminator.
While the generator dynamically approximates the constantly updated Negative Sampling distribution to provide high-qualitative negative triples, 
the discriminator learns to distinguish positive and negative triples with its own \ac{KGE} model.
Known approaches are \ac{KBGAN} \cite{cai2017kbgan}, \ac{IGAN}  \cite{IGAN} or \ac{MCNS} \cite{MCNS}.


\textbf{Evaluation}:\\
% DYNAMIC DISTRIBUTION-BASED SAMPLING
To capture the dynamic distribution of the negative triples, the
group of \textit{Dynamic Distribution-Based Sampling} (\autoref{sec:negativesamplingmethods}) was developed.
The two approaches \ac{KBGAN} \cite{cai2017kbgan} and \ac{IGAN} \cite{IGAN} are considered as pioneering works of these Sampling methods and attempt to address the challenge of capturing the negative distribution by a \ac{GAN} \cite{zhang2021efficient}.
\acp{GAN} were originally proposed for generating examples in a continuous space such as images and adapted for the generation of hard negative examples \cite{zhang2021efficient}.
Using an adversarial training process like in \ac{KBGAN} described in \autoref{sec:kbgan}, the distribution of negative triples is thus determined.
However, this is not an efficient training process and it is not effectively sampled from this distribution.
Additionally, it increases the number of training parameters because of the generator and the model suffers from instability and degeneracy.
This results from the fact, that the \textsc{REINFORCE} gradient has high variance for training the generator and only a few negative triples lead to large gradient \cite{zhang2021efficient}.
Consequently, the \ac{GAN}-based models have to put a lot of effort to model the negative triple distribution which leads to instable performances.
Since the generator and discriminator in Negative Sampling only draw information from the two underlying embedding models, pre-training is necessary, which is a time-consuming endeavor.
However, even if only a few triple pairs (negative + positive triple) are useful for learning embedding, they are not maintained to be able to learn from them at a later time.
Moreover, no other information of the graph is included except the softmax-probability outputs of the generator and the \ac{KGE} model of the discriminator.
Randomly selecting one of the negative triples does not ensure that a useful one is selected.
For embedding, the distinction between negative and positive triples from areas of a \ac{KG} about which no or incomplete information are available could be particularly important.

% DYNAMIC DISTRIBUTION-BASED NEGATIVE SAMPLING
KBGAN
\cite{zhang2021efficient}:
- avoid vanishing gradient problem 
-> but more complex and harder to train
- waste time on additional parameters to fit the full distribution of negative triplets
Problems: - increases the number of parameters
- learning suffers from instability and degeneracy
-> REINFORCE gradient has high variance
- only a few negatives with high gradient -> a lot of effort find them (model distribution of negatives)
-> instable performance
- pretraining necessary
- other approaches like Self adversarual sampling (Self-Adv) tried to solve this problem by using a self embedding model for the generator
-> disadvantage: can not guarantee to sample enough negative triplets with large gradients 



In summary, there are already several approaches for Negative Sampling, but they still have problems to capture the dynamic distribution of the negative triples or effectively sample them \cite{zhang2021efficient}.
Therefore, we present a new Negative Sampling technique which aims to \textbf{improve the efficiency of Sampling by selecting more informative and useful negative examples for the embedding model by uncertainty}.

\section{KBGAN} 
\label{sec:kbgan}

In the group of \textit{Dynamic Distribution-Based Sampling}, \ac{KBGAN} is one of the pioneering works among \ac{GAN}-based approaches and contains two components which are trained in an adversarial training process.
This process of \ac{KBGAN} is depicted in \autoref{fig:overview} and described as follows:
\begin{figure*}[t]
  \centering
    \includegraphics[width=0.95\textwidth]{figures/kbgan_original.png}
  \caption{An overview of the \textsc{kbgan} framework (based on: \cite{cai2017kbgan})}
  \label{fig:overview}
\end{figure*}

\begin{enumerate}
    \item 
    At first, a set of sampled corrupted triples $Neg(h,r,t)\subset\{(h',r,t)|h'\in\mathcal{E}\}\cup\{(h,r,t')|t'\in\mathcal{E}\}$ is created by uniformly Sampling of $N_s$ entities $\in$ \entities to replace $h$ or $t$.
    For reasons of efficiency and to reduce the likelihood of creating false negatives, $N_s$ is a relatively small number compared to the number of all possible negative examples \cite{cai2017kbgan}.
    
    \item 
    Negative triples from $Neg$ are given to the generator G as input.
    
    \item 
    $G$ is a \ac{KGE} model with a softmax function to provide  probabilities of being sampled for each given negative triple, which gives us a probability distribution $p_G(h',r,t'|h,r,t)$.
    Other models are also possible at this point, but the approach would then have to be adapted, so that it outputs a probability distribution over all negative triples of the set $Neg$ as well.

    \item 
    Afterwards, one of these triples is sampled as output of G according to $p_G$, while high quality triples should have a high probability to be sampled.
    
    \item 
    Discriminator D receives as input both the sampled negative triple \triple{h'}{r}{t'} and the ground truth triple \triple{h}{r}{t}.
    The training objective of D is a marginal loss function, because these benefit most from high quality negative examples \cite{cai2017kbgan}.

    \item 
    Subsequently, D calculates the scores of both triples \triple{h}{r}{t} and \triple{h’}{r}{t'} with score function $f_D$ of D, which models distance between points or vector.
    Therefore, a smaller distance indicates a higher likelihood of truth \cite{cai2017kbgan}.
    
    \item 
    Accordingly, the objective of D is to minimize the marginal loss function $L_D$ which is defined as
    \begin{equation}
        L_D=\sum_{(h,r,t)\in\mathcal{T}}[f_D(h,r,t)-f_D(h',r,t')+\gamma]_+ \in \mathbb{R}^+
    \end{equation}
    
    \item 
    To give G feedback for the sampled negative triple \triple{h’}{r}{t'} its calculated score $f_D(h',r,t')$ is send back to G as a reward, which is defined as
    \begin{equation}
        r = -f_D(h',r,t') \in \mathbb{R}^-
    \end{equation}
    since $f_D(h,r,t) \in \mathbb{R}^+$.
    Therefore, the objective of G can be formulated as maximizing the expectation of negative distances:
    \begin{equation}
        R_G=\sum_{(h,r,t)\in\mathcal{T}}\mathbb{E}[-f_D(h',r,t')]
    \end{equation}
    
    \item
    This process continues until convergence
\end{enumerate}
The dynamic distribution of the negative triple is determined through this procedure of the adversarial training, 





%\cite{cai2017kbgan}
%- probability-based, log-loss embedding models as the generator
%- distance-based, margin-loss embedding as discriminator


\section{Evaluating Knowledge Graph Embeddings} 
\label{sec:evaluating_knowledge_graph_embeddings}

The most common evaluation protocol for \acp{KGE} is a form of question answering and called \textit{entity ranking} \cite{Ruffinelli2020You}.
After separating available data into training, validation and test sets, for a given test triple $(i,k,j)$ the task is to determine missing entities j and i.
In the sense of question answering we find the missing tail entity $(i,k,?)$ as well as the missing head entity $(?,k,j)$.
For this purpose, potential triples are ranked 
by their model score $s(i, k, j)$ in descending order, which is a collection of all pseudo-negative object scores and defined as
$$\{ s(i, k, j') : j' \in \entities \text{ and } (i, k, j') \text{ does not occur in training, validation, or test data}\}$$
Since in many \ac{KGE} models either the head or the tail entity is randomly replaced by other entities of the \ac{KG}, these corrupted triples may be true facts.
For this reason, many approaches distinguish between \textit{raw} and \textit{filtered}, where for corrupted triples it is looked whether these occur in the train, test or validation set \cite{TransE}.
Therefore, true corrupted triples are removed from the list before they are scored (except the test triplet of interest) and the collection of all pseudo-negative object scores in a filtered settings \cite{Ruffinelli2020You}.

Therefore, we obtain the filtered rank $rank(j|i, k)$ of object j and given head entity i and relation k, and $rank(i|k, j)$ for object i and given relation k and tail entity j.
Subsequently, metrics like the \ac{MRR} or the average HITS@k are computed.
There are defined as follows \cite{Ruffinelli2020You}:
\begin{equation}
    MRR = \frac{1}{2 |\mathcal{K}^{test}|} \sum_{(i,k,j) \in  \mathcal{K}^{test}} \left( \frac{1}{rank(i |k,j)} + \frac{1}{rank(j|i,k)} \right)
\end{equation}
    
\begin{equation}
    Hits@K = \frac{1}{2|\mathcal{K}^{test}|} \sum_{(i,k,j) \in  \mathcal{K}^{test}} \left( \mathds{1} (rank(i |k,j) \leq K + \mathds{1}  (rank(j|i,k) \leq K \right)
\end{equation}
where $\mathcal{K}^{test}$ is the set of test triples and indicator $\mathds{1}(E)$ is 1 if condition E is true, else 0.





\section{Uncertainty Sampling} \label{sec:uncertaintysampling}
Uncertainty Sampling originates from Active Learning, 
where labeled data for training a supervised model is obtained from a dataset of unlabeled instances.
Based on the informativeness of the unlabeled instances for the learning algorithm, a prioritization results in which they are labeled.
It is used in machine learning approaches, where unlabeled data is abundant, but it is difficult, time-consuming, or expensive to obtain labeled data \cite{Settles2009ActiveLL}.
They aim for greater accuracy with fewer labeled training instances \cite{Settles2009ActiveLL}.
These selected unlabeled instances can either be generated de novo or sampled from a given distribution.
In the literature several query strategies have been proposed with different approaches how to receive informative instances, one of them is Uncertainty Sampling \cite{Settles2009ActiveLL}.
Other query strategies for Active Learning are Query-By-Committee, Expected Model Change, Variance Reduction, Fisher Information Ration, Estimated Error Reduction and Density-Weighted Methods \cite{Settles2009ActiveLL}.
They provide different strategies to obtain informative instances from the unlabeled dataset like voting of a committee consisting of several trained models, querying the instance that would impart the greatest change to the current model if we knew its label or queries instances which minimize the learner’s future error by minimizing its variance.

In Uncertainty Sampling, given a model $\theta$ which has been trained on labeled dataset $D$, each instance $x_j$ of the unlabeled data pool $U$ will be assigned a utility score $s(\theta, x_j)$.
Subsequently, the instance with the highest score will be sampled.
Popular examples of measures for utility score include
\begin{itemize}
    \item the entropy:
     $$s(\theta, x) = - \sum_{y \in \mathcal{Y}}{p_{\theta}(y | x) \cdot log p_{\theta}(y|x)}$$

    \item the least confidence:
    $$s(\theta, x) = 1 - \max_{y \in \mathcal{Y}}{p_{\theta}(y | x)}$$
    
    \item the smallest margin:
    $$s(\theta, x) = p_{\theta}(y_m | x) - p_{\theta}(y_n|x)$$
    where
    $y_m = \argmax_{y \in \mathcal{Y}} p_{\theta}(y | x)$ 
    and 
    $y_n = \argmax_{y \in \mathcal{Y} \setminus y_m}{p_{\theta}(y | x)}$
\end{itemize}
Following three different frameworks for measuring the  uncertainty of a learner can be separated \cite{nguyen2021howtomeasure}.
While the first one has been specifically developed for the purpose of active learning, the others are more general approaches for machine learning \cite{nguyen2021howtomeasure}.
\begin{enumerate}
    \item 
    \textbf{Evidence-based uncertainty (EBU)} differentiates between uncertainty due to conflicting evidence and insufficient evidence.
    \ac{EBU} looks at the influence of individual features,
    partitions them into those that provide evidence for the positive and for the negative class, 
    and either queries the instance with the highest conflicting evidence (\ac{CEU}) or where both evidences are low (\ac{IEU}).

    \item 
    \textbf{\ac{CU}} seeks to differentiate between the reducible and irreducible part of the uncertainty in a prediction by defining a
    credal set of models with probability distributions.
    A class y is dominated by another y' if y is more likely than y' for any distribution in the credal set
    An instance x is sampled, which has the least evidence for the dominance of one of the classes \cite{nguyen2021howtomeasure}.

    \item 
    \textbf{Epistemic and aleatoric uncertainty (EAU)} are based on the use of relative likelihoods. 
    \ac{EU} samples instance for which both the positive and the negative class appear to be plausible, while \ac{AU} samples instances where none of the classes is supported.
    Therefore, the uncertainty due to either influence of the classes or lack of knowledge. 
\end{enumerate}



Active Learning and Uncertainty Sampling: \cite{5272205}
- Active Learning to minimize the amount of human labeling efforts required for a supervised classifier
- active learner has two major schemes: uncertainty sampling and committee-based sampling
- most uncertain per cluster: using density to weigh the selected examples
- or: select examples based on informativeness, diversity and density criteria
- outlier problem: uncertainty sampling often fails by selecting outliers
- motivation behind uncertainty sampling: find some unlabeled examples near decision boundaries and use them to clarify the position of decision boundaries -> most informative instances
- use density to determine whether an unlabeled example is highly representative
- 











