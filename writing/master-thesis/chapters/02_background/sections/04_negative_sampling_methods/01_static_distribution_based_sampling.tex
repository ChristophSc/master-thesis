\subsection{Static Distribution-Based Sampling}
\label{subsec:static_distribution_based_sampling}

Static Distribution-Based Sampling include methods like the uniform, Bernoulli and probabilistic sampling technique from a fixed and static distribution of negative triples.	While in the uniform sampling either the head or tail entity is replaced by an entity randomly sampled from entity set \entities,
the Bernoulli sampling technique uses different probabilities for replacing head or tail entity depending on the underlying relation type.
In contrast, probabilistic sampling speeds up the training process by including a train bias.
% TODO: Descibe Random Sampling in more detail
% TODO: Descibe Bernoulli Sampling in more detail
	
\textbf{Evaluation}:\\
% STATIC-DISTRIBUTION-BASED SAMPLING
Static Distribution-Based Sampling approaches are commonly used because of their simplicity and efficiency, but ignore the dynamics in the Negative Sampling distribution which lead to the vanishing gradient problem \cite{qianunderstanding}.
This problem occurs when the gradient will be vanishingly small and accordingly, small gradients prevent changing the weight value.
This can impede the training process or, in the worst case, completely stop the model from further training.
% RANDOM UNIFORM NEGATIVE SAMPLING
While negative triples, such as those generated by randomly replacing a head or tail entity, are very likely to be negative examples, they are generally uninformative and useless.
For example, by replacing the tail entity in given positive triple (Paderborn, locatedIn, Germany) by randomly selected entity 'Apple' leads to the new negative triplet (Paderborn, locatedIn, Apple).
Even though it is a true negative triple, it is uninformative and useless for embedding learning.
The problem with 'too easy' negative triples is less severe to models using log-softmax loss function, because they usually sample a high amount of negatives for one positive triple \cite{cai2017kbgan}.
However, the performance of marginal loss functions can be seriously damaged by the low quality of uniformly sampled negatives since negative-to-positive ratio is always 1:1 \cite{cai2017kbgan}.