\chapter{Implementation}
\label{ch:implementation}

\section{Data analysis}



\section{Our Model}

ADD AND DESCRIBE CLASS DIAGRAMM


\section{Used Knowledge Graph Embeddungs}

Additionally, we will test the impact of our Negative Sampling approach on different embeddings with the given datasets and compare in which areas our method outperforms the current state-of-the-art approaches.
For this reason, we will use the most common embeddings from the three groups of embedding types:
\textsc{TransE}, \textsc{TransH} for \textit{Translation-Based Models}, \textsc{DistMult} and \textsc{ComplEx} for \textit{Tensor Factorization-Based Models}, and \textsc{ConvE} and \textsc{ConEx} for \textit{Neural Network-Based Models}.


\section{Algorithm}

\begin{algorithm}[H]
\small
 \KwData{training set of positive fact triples $\mathcal{T}=\{(h,r,t)\}$}
 \KwIn{(Pre-trained) generator G with parameters $\theta_G$ and score function $f_G(h,r,t)$, and (pre-trained) discriminator D with parameters $\theta_D$ and score function $f_D(h,r,t)$}
 \KwOut{Adversarially trained discriminator}
 $b \longleftarrow 0$\tcp*[l]{baseline for policy gradient}
 \Repeat{convergence}{
  Sample a mini-batch of data $\mathcal{T}_{batch}$ from $\mathcal{T}$ \;
  
  $G_G \longleftarrow 0$, $G_D \longleftarrow 0$\tcp*[l]{gradients of parameters of G and D}
  $r_{sum} \longleftarrow 0$\tcp*[l]{for calculating the baseline}
  Uniformly randomly sample $N_s$ negative triples for each positive triple $(h,r,t) \in \kg$: $Neg(h,r,t)=\{(h_i',r,t_i')\}_{i=1\dots N_s}$\;
  
  $score_{min} \longleftarrow \argmin_{(h',r,t')\in Neg}{score(h',r,t')}$\;
  
  $score_{max} \longleftarrow \argmax_{(h,r,t)\in \kg}{score(h,r,t)}$ \;
   
  \For{$(h,r,t)\in \mathcal{T}_{batch}$}{
    Classify all negative triples by calculating their probability of being a positive triple:  $\mathds{P}(y = 1|f_G(h, r, t)) = \frac{f_G(h,r,t) - score_{min}}{score_{max} - score_{min}}$\;
    
    Calculate uncertainty scores $u_i \in Neg$ with $i \in [1, N_s]$ of all negative triples\;
    
    Obtain their probability of being sampled: $p_i=\frac{exp^{u_i}}{\sum_{j=1}^{N_s} exp^{u_j}}$\;
    
    Sample one negative triple $(h_s',r,t_s')$ from $Neg(h,r,t)$  based on uncertainty scores  according to $\{p_i\}_{i=1\dots N_s}$. Assume its probability to be $p_s$\;

    $G_D \longleftarrow G_D + \nabla_{\theta_D}[f_D(h,r,t)-f_D(h_s',r,t_s')+\gamma]_+$\tcp*[l]{accumulate gradients for D}
    
    $r \longleftarrow -f_D(h_s',r,t_s'), r_{sum} \longleftarrow r_{sum} + r$\tcp*[l]{$r$ is the reward}
    
    $G_G \longleftarrow G_G + (r-b)\nabla_{\theta_G} \log p_s$\tcp*[l]{accumulate gradients for G}
  }
  $\theta_G \longleftarrow \theta_G+\eta_G G_G, \theta_D \longleftarrow \theta_D-\eta_D G_D$\tcp*[l]{update parameters}
  $b \leftarrow r_{sum} / |\mathcal{T}_{batch}|$\tcp*[l]{update baseline}
 }
 \caption{\usgan algorithm with sampling method \ussoftmax.}
 \label{alg:ucgan}
\end{algorithm}

\subsection{Hyperparameter Optimization}

\begin{table}[H]
    \centering
    \begin{tabular}{lll}
        \toprule
        
        Model & 
        Hyperparameters & 
        Constraints or Regularizations \\
    
        \midrule
        
        \textsc{DistMult} & $k=50, \lambda=0.1$ & L2 regularization: $L_{reg}=L+\lambda||\Theta||_2^2$ \\ 
        
        \textsc{ComplEx}  & $k=25, \lambda=0.1$ & L2 regularization: $L_{reg}=L+\lambda||\Theta||_2^2$ \\
        
        \textsc{TransE} & $L_1$ distance, $k=50, \gamma=3$ & $||\mathbf{e}||_2\leq 1,||\mathbf{r}||_2\leq 1$ \\ 
        
        \textsc{TransD} & $L_1$ distance, $k=50, \gamma=3$ & $||\mathbf{e}||_2\leq 1,||\mathbf{r}||_2\leq 1,||\mathbf{e_p}||_2\leq 1,||\mathbf{r_p}||_2\leq 1$  \\ 
        
        \bottomrule
    \end{tabular}
    \caption{Hyperparameter settings for generator and discriminator models.
    For generator models a learning rate of $\lambda=0.1$ is used. 
    Constraints and regularizations are taken from \cite{cai2017kbgan}.
    All hyperparameters are shared among all datasets. 
    $\Theta$ represents all parameters in the model.}
    \label{tab:hyperparams}
\end{table}


\subsection{Processing on $PC^2$}
Since these datasets are large and running algorithms on them are computationally expensive, we use the servers of \ac{PC2} \footnote{https://pc2.uni-paderborn.de/} to ensure faster processing of the learning process of our model.


\subsection{Datasets}

Initially, the approach will be implemented on the local computer and tested with small data sets such as \textsc{KINSHIP} or \textsc{UMLS}.
Subsequently, we want to compare our achieved accuracy with state-of-the-art approaches.
Therefore, we want to use \ac{MRR} and Hit@10 metrics on datasets \textsc{WN18}, \textsc{WN18RR}, \textsc{FB15K}, \textsc{FB15K237}, \textsc{YAGO3-10}.



\begin{table}[h]
    \centering
    \begin{tabular}{llllllll}
        \toprule
        
        \textbf{Dataset} & \textbf{\#rel.} & \textbf{\#ent.} & 
        \textbf{\#heads} & \textbf{\#tails} &
        \textbf{\#train} & \textbf{\#val} & \textbf{\#test} \\
    
        \midrule
        
        \umls 
        & 46 & 135 & 789 & 834 & 5,216 
        & 652 & 661 \\
        \kinship & 25 & 104 & 1,496 & 1,739 
        & 8,544 & 1,068 & 1,074 \\
        \wnrr 
        & 11 & 40,943 & 42,853 & 66,166 & 86,835 & 3,034 & 3134  \\
        \wn 
        & 18 & 40,943 & 90,445 & 90,334 & 141,442 & 5,000 & 5,000  \\
        \textsc{FBK-237} 
        & 237 & 14,541 & 59,734 & 102,188 & 272,115 & 17,535 & 20,466 \\
        \textsc{FB15k} 
        & 1,345 & 14,951 & 133,909 & 174,097 & 483,142 & 50,000 & 59,071 \\
        \textsc{YAGO3-10}
        & 37 & 123,143 & 86,009 & 307,939 & 1,079,040 & 4,978 & 4,982 \\

        \bottomrule
    \end{tabular}
    \caption{Statistics of datasets we used in the experiments. ``rel.'': relations, ``ent.'': entities}
\label{tab:datasets}
\end{table}

