\section{Probabilities of Positive and Negative Triples} 
\label{sec:probabilities}


6) Probabilities 
- probabilities are needed for Uncertanty Sampling 
-> model in uncertain about how to label a triple instance (positive or negative)
- binary classification
- y=0: triple is a positive
- y=1: triple is a negative
- min.score: P(y=1)=0, P(y=0)=1 
- max.score: P(y=1)=1, P(y=0)=0

- we need different classifiers, which indicate with which probability a given triple is positive or negative.
- Either a classifier with its uncertainty, i.e. a probability around 0.5 of being either positive or negative.
- or several classifiers, where the uncertainty can be recognized by a different classification of triples. 
- Consequently, we need probabilities instead of simple scores to be able to calculate uncertainty.
- 

Therefore, we need probabilities $\mathds{P}(y_i | x; \theta)$ for the classes $y_i \in \{0, 1\}$.

Calculate $Generator\_Score$ of each triple $(h',r,t') \in Neg$.
Remember triple $(h',r,t')_{min}$ which achieves the minimum negative score which is defined as
\begin{equation}
    score_{min} := \argmin_{(h', r, t') \in Neg}{Generator\_Score(h', r, t')}
\end{equation}
In the next step, $score_{min}$ is the lower bound for the probability of a triple to be positive (y=1), so it is considered to be a probability of 0.
\begin{equation}
    \mathds{P}(y = 1|(h', r, t')_{min}) := 0
\end{equation}

\begin{figure*}[t]
  \centering
    \includegraphics[width=0.75\textwidth]{figures/positives_negatives1.PNG}
  \caption{Example of Uncertainty Sampling for positive and negative instances. Simplified to two-dimensional embedding space d1 and d2 and borders b1 and b2 between positive and negative instances.}
  \label{fig:informativeinstances}
\end{figure*}


\begin{figure*}[t]
  \centering
    \includegraphics[width=0.75\textwidth]{figures/positives_negatives2.PNG}
  \caption{Example of Uncertainty Sampling for positive and negative instances. Simplified to two-dimensional embedding space d1 and d2 and borders b1 and b2 between positive and negative instances.}
  \label{fig:informativeinstances}
\end{figure*}