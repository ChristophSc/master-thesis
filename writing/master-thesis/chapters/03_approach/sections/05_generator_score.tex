\section{Generator Score} 
\label{sec:generatorscore}


5) Generator Score

- combines all the information from KGE model of generator + feature functions
Option1: minimum and maximum of all triples
- min.score: minimum score of all triples (-> should be a positive one)
- max.score: maximum score of all triples (-> should be a negative one)
Option2: left and right boundary of the uncertain range for positives and negatives
- min.score: left boundary of the negative triples (if they are higher than the positive ones)
- max.score: right boundary of the positive triples (if they are lower than the negative ones)

Calculate $Generator\_Score$ (equation (2)) for all positive triples $(h, r, t) \in \mathcal{T}_{batch}$.
\begin{equation}
    score_{max} := \argmax_{(h,r,t) \in \mathcal{T}_{batch}}{Generator\_Score(h,r,t)}
\end{equation}
Assuming $(h,r,t)_{max}$ achieved the highest score, 
it is considered to have a probability of being positive (y=1) of 1:
\begin{equation}
    \mathds{P}(y = 1|(h, r, t)_{max}) := 1
\end{equation}

To incorporate more information in the score function than just the embedding of a triple, we want to extend the original score function $f_G$ by several information about entities and relations in the \ac{KG}.
Therefore, our new score function $Generator\_Score$ is defined as follows:
\begin{equation}
    Generator\_Score(h, r, t)=
    \begin{cases}
         \lambda_1 \text{PEER(\textit{h, r})} + \lambda_2 \text{POP(\textit{t})} + \lambda_3 \text{PIVO(\textit{h, t})} + \lambda_4 f_G(\textit{h, r, t})
         \\ \ \ 
         if\ \ \ \neg(\textit{h, r, t})
         \\ \\
         \lambda_1 \text{PEER(\textit{h, r})} + \lambda_2 \text{FRQ(\textit{r})} + \lambda_3 \text{PIVO(\textit{h, t})} + \lambda_4 f_G(h, r, t)
         \\ \ \ 
         if\ \ \ \neg \exists (\textit{h, r, \_})
         \\
    \end{cases}
\end{equation}
where $\lambda_i \in [0, 1]$ for $i \in [1,4]$ are hyperparameters and are set to randomly generated values at the beginning but can be optimized at a later stage.
Furthermore, $\neg (h, r, t)$ is a \textit{grounded negative statement} and is satisfied if $(h, r, t) \notin$ \ac{KG} and $\neg\exists(h, r, \_)$ is a \textit{universally negative statement} which is satisfied if there exists no $t$ such that $(h, r, t) \in KG$ \cite{arnaout2020enriching}.
\textit{PEER}, \textit{POP}, \textit{PIVO} and \textit{FRQ} are functions that provide additional information about the \ac{KG}, its entities and relations such that the $Generator\_Score$ can make more reliable statements about whether the given triple $(h, r, t)$ is positive or not.
However, these components of the function can also be replaced or extended by further components known from for example Cluster-Based Sampling methods that give more information about a \ac{KG} and its structure.
For example, the K-Means algorithm \cite{qianunderstanding} or other techniques known from Relational Sampling, Nearest Neighbor Sampling or Near miss Sampling can be considered \cite{kotnis2017analysis}.