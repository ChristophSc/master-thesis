\chapter{Approach}
\label{ch:approach}

\section{Idea} \label{sec:idea}



% description and general introduction to solution
With our approach, we aim to improve the training process of the embedding model by incorporating more information into the Negative Sampling process and thus selecting negative examples that are more informative and more valuable for the embedding model.
This is achieved by including Uncertainty Sampling in state-of-the-art approaches.
Uncertainty Sampling is the most commonly used query framework among all the other strategies frameworks (\autoref{sec:uncertaintysampling}) from Active Learning.
In addition, it is easier to implement than some other frameworks, because, for example, it does not require to train multiple models as is the case with the Query-by-Committee approach.
Nevertheless, it provides promising results for many Active Learning scenarios and is therefore also used for our approach to sample negative triples.

Since \ac{KBGAN} captures the dynamic distribution of negative examples, but suffers particularly from non-efficient Sampling, our first considerations are based on this approach.
This addresses the problem of instability and degeneracy of the training process, so that it is more likely to sample negative examples with a higher gradient.
However, it is also conceivable for other existing approaches with an inefficient Negative Sampling process by replacing it with Uncertainty Sampling.

% ways to improve KBGAN
In general, in \ac{KBGAN} there are two different ways to improve the approach and to allow Sampling of negative triples with a higher gradient:
First, this can be achieved by improving the general quality of the negative examples in subset $Neg$.
In the original approach, $Neg$ is generated only by Uniform Sampling, so that with high probability there are many useless negative triples.
Second, the adversarial learning process can be optimized by learning faster with less Sampling which is addressed by our approach.
To achieve this, we want to sample more informative negative examples.
Informative at this point means particularly interesting triples for the embedding model because those that are difficult to classify as positive or negative triple.

For Uncertainty Sampling, we have several measures and frameworks at our disposal which are stated in \autoref{sec:uncertaintysampling}.
Entropy-based Uncertainty Sampling represents the standard approach and is used in for binary classification problems.
Since we are dealing with negative and positive triples and their distinction, our first approach will start with a binary classification problem for which we define label 0 for negative triples and 1 for positive triples.
For this we use standard entropy-based Uncertainty Sampling, which works with the probability that a given triple is positive. 
Further possibilities arise for example by using other Uncertainty Sampling variants and methods like \ac{EBU}, \ac{CU} or \ac{EAU} which set a different focus in Sampling and represent more complex alternatives, as additional factors are included in the calculation.
By choosing entropy-based Uncertainty Sampling we sample the instance whose posterior probability of being positive is nearest to 0.5 \cite{Settles2009ActiveLL}.
In our case, since we have a binary classification between negative triples ($y=0$) and positive triples ($y=1$), $<$ 0.5 represents a negative and $\geq$ 0.5 a positive triple.
The result is that we query examples that are close to 0.5, or in other words: 
we query triples where the embedding model is most uncertain how to label it.

The structure of our approach is illustrated in Figure \ref{fig:architecture}.
It can be seen that Uncertainty Sampling replaces the original Sampling process from a probability distribution.
\begin{figure}
    \centering
    \includegraphics[width=\textwidth]{figures/architecture.png}
    \caption{Structure of our Negative Sampling approach including two components:
    A generator G which samples negative triples by Uncertainty Sampling and a discriminator D which scores the difference of positive triple $(h,r,t)$ and negative triple $(h',r,t')$ by given embedding score function $f_D$}
    \label{fig:architecture}
\end{figure}
In the original \ac{KBGAN} approach, a softmax function is applied to obtain probabilities of being sampled for each generated negative triple.
However, for Uncertainty Sampling we need probabilities of each triple to be positive. 
Therefore, we need probabilities $\mathds{P}(y_i | x; \theta)$ for the classes $y_i \in \{0, 1\}$.
To incorporate more information in the score function than just the embedding of a triple, we want to extend the original score function $f_G$ by several information about entities and relations in the \ac{KG}.
Therefore, our new score function $Generator\_Score$ is defined as follows:
\begin{equation}
    Generator\_Score(h, r, t)=
    \begin{cases}
         \lambda_1 \text{PEER(\textit{h, r})} + \lambda_2 \text{POP(\textit{t})} + \lambda_3 \text{PIVO(\textit{h, t})} + \lambda_4 f_G(\textit{h, r, t})
         \\ \ \ 
         if\ \ \ \neg(\textit{h, r, t})
         \\ \\
         \lambda_1 \text{PEER(\textit{h, r})} + \lambda_2 \text{FRQ(\textit{r})} + \lambda_3 \text{PIVO(\textit{h, t})} + \lambda_4 f_G(h, r, t)
         \\ \ \ 
         if\ \ \ \neg \exists (\textit{h, r, \_})
         \\
    \end{cases}
\end{equation}
where $\lambda_i \in [0, 1]$ for $i \in [1,4]$ are hyperparameters and are set to randomly generated values at the beginning but can be optimized at a later stage.
Furthermore, $\neg (h, r, t)$ is a \textit{grounded negative statement} and is satisfied if $(h, r, t) \notin$ \ac{KG} and $\neg\exists(h, r, \_)$ is a \textit{universally negative statement} which is satisfied if there exists no $t$ such that $(h, r, t) \in KG$ \cite{arnaout2020enriching}.
\textit{PEER}, \textit{POP}, \textit{PIVO} and \textit{FRQ} are functions that provide additional information about the \ac{KG}, its entities and relations such that the $Generator\_Score$ can make more reliable statements about whether the given triple $(h, r, t)$ is positive or not.
However, these components of the function can also be replaced or extended by further components known from for example Cluster-Based Sampling methods that give more information about a \ac{KG} and its structure.
For example, the K-Means algorithm \cite{qianunderstanding} or other techniques known from Relational Sampling, Nearest Neighbor Sampling or Near miss Sampling can be considered \cite{kotnis2017analysis}.

Assuming $\mathcal{T} \subset $ \ac{KG} is a set of positive triples and $\entities_{\mathcal{T}}$, $\relations_{\mathcal{T}}$ are its sets of entities and relations.
Additionally we have the function $peergroup(e) \subseteq \mathcal{T}$ which returns a set of peers for given entity $e \in \entities_{\mathcal{T}}$ by its embedding. 
Then the $Generator\_Score$ function contains the following features from \cite{arnaout2020enriching}:
\begin{itemize}
    \item 
    \emph{\ac{PEER}:} 
    Relative frequency of peers within a peer group that is related to different objects, e.g. 0.9 of persons are married. 
    \begin{equation}
        PEER(h,r) = \frac{|\{p | p \in peergroup(h), (p, r, \_) \in \mathcal{T}\}|}{|peergroup(h)|}
    \end{equation}

    \item
    \emph{\ac{POP}:} 
    The popularity of the tail entity $t$ in $\mathcal{T}$. 
    \begin{equation}
        POP(t) = \frac{|\{t | (\_, t, \_) \in \mathcal{T}\}|}{|\mathcal{T}|}
    \end{equation}

    \item 
    \emph{\ac{FRQ}:} 
    Frequency of a relation/predicate $r$ in $\mathcal{T}$. 
    \begin{equation}
        FRQ(r) = \frac{|\{r | (\_, r, \_) \in \mathcal{T}\}|}{|\mathcal{T}|}
    \end{equation}
    
    \item 
    \emph{\ac{PIVO}:} 
    Textual background information about an entities $h$ and $t$ and is a pivoting classifier like in \cite{arnaout2020enriching}.
    
    \item 
    \emph{$f_G$:} 
    $Generator\_Score$ function of the \ac{KGE} model from generator like in the original \ac{KBGAN} approach (e.g. \textsc{DistMult}  or \textsc{ComplEx}).
    
\end{itemize}
Input of the algorithm of our approach is like in original \ac{KBGAN} a pre-trained generator G with parameters $\theta_G$ and score function $f_G(h,r,t)$, 
and a pre-trained discriminator D with parameters $\theta_D$ and score function $f_D(h,r,t)$.
The training process can be described in the following steps:
\begin{enumerate}
    \item 
    If $\mathcal{T} \subset $ \ac{KG} is the training set of positive fact triples, sample a mini-batch of data $\mathcal{T}_{batch} \subset \mathcal{T}$.
 
    \item  Calculate $Generator\_Score$ (equation (2)) for all positive triples $(h, r, t) \in \mathcal{T}_{batch}$.
    \begin{equation}
        score_{max} := \argmax_{(h,r,t) \in \mathcal{T}_{batch}}{Generator\_Score(h,r,t)}
    \end{equation}
    Assuming $(h,r,t)_{max}$ achieved the highest score, 
    it is considered to have a probability of being positive (y=1) of 1:
    \begin{equation}
        \mathds{P}(y = 1|(h, r, t)_{max}) := 1
    \end{equation}
    
    \item 
    Create a set of negative triples $Neg(h,r,t)=\{(h_i',r,t_i')\}_{i=1\dots N_s}$ by uniformly randomly sample $N_s$ negative triples by replacing head $h$ or tail entity $t$ for given positive triple $(h, r, t)$.
    
    \item 
    Calculate $Generator\_Score$ of each triple $(h',r,t') \in Neg$.
    Remember triple $(h',r,t')_{min}$ which achieves the minimum negative score which is defined as
    \begin{equation}
        score_{min} := \argmin_{(h', r, t') \in Neg}{Generator\_Score(h', r, t')}
    \end{equation}
    In the next step, $score_{min}$ is the lower bound for the probability of a triple to be positive (y=1), so it is considered to be a probability of 0.
    \begin{equation}
        \mathds{P}(y = 1|(h', r, t')_{min}) := 0
    \end{equation}
    
    \item 
    To sample the negative triple $(h',r,t')^{*}_{ENT}$, we need to calculate the entropy-based uncertainty, which is defined as: 
    $$(h',r,t')^{*}_{ENT} = \argmax_{(h',r,t') \in Neg)} -  \sum_{i}{\mathds{P}(y_i | (h',r,t')) log \mathds{P}(y_i|(h',r,t'))}$$
    Since we have a binary classification with $y_i \in \{0,1\}$:\\
    $$= \argmax_{(h',r,t') \in Neg)} - \mathds{P}(y = 1| (h',r,t')) log \mathds{P}(y = 1|(h',r,t'))$$
    $$- \mathds{P}(y = 0| (h',r,t')) log \mathds{P}(y = 0|(h',r,t'))$$
    $$= \argmax_{(h',r,t') \in Neg)} - \mathds{P}(y = 1| (h',r,t')) log \mathds{P}(y = 1|(h',r,t'))$$
    $$- (1 - \mathds{P}(y = 1|(h',r,t') log(1 - \mathds{P}(y = 1|(h',r,t'))$$
    To obtain the probabilities for a triple of being positive ($y=1$) we define
    \begin{equation}
        \mathds{P}(y = 1|(h, r, t)) := \frac{Generator\_Score(h, r, t) - score_{min}}{score_{max} - score_{min}} \in [0, 1]
    \end{equation}
    and accordingly the probability of a triple to be negative ($y=0$) as
    \begin{equation}
        \mathds{P}(y = 0|(h, r, t)) := 1 - \mathds{P}(y = 1|(h, r, t)) \in [0,1]
    \end{equation}
    
    \item 
    From this step on, everything is the same as the original \ac{KBGAN} approach:
    The generated negative triple $(h',r,t')$ as well as the positive triple $(h, r, t)$ are sent to the discriminator.
    
    \item 
    The discriminator distinguishes both triples by given scoring function $f_D$, which is usually a translation-based \ac{KGE} model like \textsc{TransE} or \textsc{TransD}.
    
    \item 
    The reward defined by $r = - f_D(h',r,t')$ of the current triple pair is calculated and added to the reward sum $r_{sum}$.
    
\end{enumerate}
These adversarial training steps are repeated until convergence, such that the generator improves the quality of sampled negative triples and discriminator improves embedding over time.
Output of our algorithm is the adversarially trained discriminator and its embedding.
By incorporating additional information from the graph and Sampling triples by uncertainty that have the most added value to the embedding model, we expect promising results.

Initially, the approach will be implemented on the local computer and tested with small data sets such as \textsc{KINSHIP} or \textsc{UMLS}.
Subsequently, we want to compare our achieved accuracy with state-of-the-art approaches.
Therefore, we want to use \ac{MRR} and Hit@10 metrics on datasets \textsc{WN18}, \textsc{WN18RR}, \textsc{FB15K}, \textsc{FB15K237}, \textsc{YAGO3-10}.
Since these datasets are large and running algorithms on them are computationally expensive, we use the servers of \ac{PC2} \footnote{https://pc2.uni-paderborn.de/} to ensure faster processing of the learning process of our model.
Additionally, we will test the impact of our Negative Sampling approach on different embeddings with the given datasets and compare in which areas our method outperforms the current state-of-the-art approaches.
For this reason, we will use the most common embeddings from the three groups of embedding types:
\textsc{TransE}, \textsc{TransH} for \textit{Translation-Based Models}, \textsc{DistMult} and \textsc{ComplEx} for \textit{Tensor Factorization-Based Models}, and \textsc{ConvE} and \textsc{ConEx} for \textit{Neural Network-Based Models}.



\begin{table}[H]
    \centering
    \begin{tabular}{lll}
        \toprule
        
        Model & 
        Hyperparameters & 
        Constraints or Regularizations \\
    
        \midrule
        
        \textsc{DistMult} & $k=50, \lambda=0.1$ & L2 regularization: $L_{reg}=L+\lambda||\Theta||_2^2$ \\ 
        
        \textsc{ComplEx}  & $k=25, \lambda=0.1$ & L2 regularization: $L_{reg}=L+\lambda||\Theta||_2^2$ \\
        
        \textsc{TransE} & $L_1$ distance, $k=50, \gamma=3$ & $||\mathbf{e}||_2\leq 1,||\mathbf{r}||_2\leq 1$ \\ 
        
        \textsc{TransD} & $L_1$ distance, $k=50, \gamma=3$ & $||\mathbf{e}||_2\leq 1,||\mathbf{r}||_2\leq 1,||\mathbf{e_p}||_2\leq 1,||\mathbf{r_p}||_2\leq 1$  \\ 
        
        \bottomrule
    \end{tabular}
    \caption{Hyperparameter settings for generator and discriminator models.
    For generator models a learning rate of $\lambda=0.1$ is used. 
    Constraints and regularizations are taken from \cite{cai2017kbgan}.
    All hyperparameters are shared among all datasets. 
    $\Theta$ represents all parameters in the model.}
    \label{tab:hyperparams}
\end{table}

\begin{table}[h]
    \centering
    \begin{tabular}{llllllll}
        \toprule
        
        \textbf{Dataset} & \textbf{\#rel.} & \textbf{\#ent.} & 
        \textbf{\#heads} & \textbf{\#tails} &
        \textbf{\#train} & \textbf{\#val} & \textbf{\#test} \\
    
        \midrule
        
        \umls 
        & 46 & 135 & 789 & 834 & 5,216 
        & 652 & 661 \\
        \kinship & 25 & 104 & 1,496 & 1,739 
        & 8,544 & 1,068 & 1,074 \\
        \wnrr 
        & 11 & 40,943 & 42,853 & 66,166 & 86,835 & 3,034 & 3134  \\
        \wn 
        & 18 & 40,943 & 90,445 & 90,334 & 141,442 & 5,000 & 5,000  \\
        \textsc{FBK-237} 
        & 237 & 14,541 & 59,734 & 102,188 & 272,115 & 17,535 & 20,466 \\
        \textsc{FB15k} 
        & 1,345 & 14,951 & 133,909 & 174,097 & 483,142 & 50,000 & 59,071 \\
        \textsc{YAGO3-10}
        & 37 & 123,143 & 86,009 & 307,939 & 1,079,040 & 4,978 & 4,982 \\

        \bottomrule
    \end{tabular}
    \caption{Statistics of datasets we used in the experiments. ``rel.'': relations, ``ent.'': entities}
\label{tab:datasets}
\end{table}








