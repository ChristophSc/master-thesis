% !TEX root = ../my-thesis.tex
%
\pdfbookmark[0]{Abstract}{Abstract}
\addchap*{Abstract}
\label{sec:abstract}
%
Knowledge graphs present an incomplete knowledge repository that store facts about the real world.
These facts are represented in a network structure where entities are stored as nodes and relations as edges.
These knowledge graphs are used for a large number of downstream tasks such as structured search and link prediction.
As the volume of knowledge graphs increased significantly in recent years, the management of large knowledge graph systems have become more important.
For this reason, knowledge graph embedding models have been presented to embed entities and relations in a continuous low-dimensional vector space.
Learning these embeddings requires a distinction between positive and negative triples.
Since only positive information is stored in knowledge graphs, there are several training types available to find negative examples.
One of these is negative sampling. 
However, commonly used negative sampling approaches provide low-quality negative examples.
This can impede the training process, lead to a worse embedding and thus directly impact downstream tasks.
In active learning and many areas of machine learning, sampling through uncertainty offers the possibility of sampling particularly informative instances for a model and thus making the learning process more efficient.
In this thesis, we are interested in leveraging uncertainty means in the process of negative sampling.
For this reason, several methods of uncertainty sampling are analyzed first.
Subsequently, these sampling methods replace the original sampling method of the generative adversarial network-based approach KBGAN. 
Finally, the new approach is evaluated in detail on a total of seven different data sets and the experimental results are compared.
To ensure reproducibility, we provide scripts, datasets, and results of this work on our project page \url{https://github.com/ChristophSc/master-thesis}.

% \vspace*{20mm}
\clearpage
{\usekomafont{chapter}Abstract (Deutsch)}
\label{sec:abstract-german}\\

Wissensgraphen stellen einen unvollständigen Wissensspeicher dar, der Fakten über die reale Welt speichert.
Diese Fakten werden in einer Netzstruktur dargestellt, in der Entitäten als Knoten und Beziehungen als Kanten gespeichert sind.
Diese Wissensgraphen werden für eine Vielzahl nachgelagerter Aufgaben wie strukturierte Suche und Linkvorhersage verwendet.
Da das Volumen der Wissensgraphen in den letzten Jahren erheblich zugenommen hat, ist die Verwaltung großer Wissensgraphen-Systeme immer wichtiger geworden.
Aus diesem Grund wurden Modelle zur Einbettung von Wissensgraphen vorgestellt, die Entitäten und Beziehungen in einen kontinuierlichen niedrigdimensionalen Vektorraum einbetten.
Das Lernen dieser Einbettungen erfordert eine Unterscheidung zwischen positiven und negativen Tripeln.
Da in Wissensgraphen nur positive Informationen gespeichert sind, gibt es mehrere Trainingsarten, um negative Beispiele zu finden.
Eine davon ist das negative Sampling. 
Die üblicherweise verwendeten Negativ-Sampling-Ansätze liefern jedoch Negativbeispiele von geringer Qualität.
Dies kann den Trainingsprozess behindern, zu einer schlechteren Einbettung führen und sich somit direkt auf nachgelagerte Aufgaben auswirken.
Beim aktiven Lernen und in vielen Bereichen des maschinellen Lernens bietet das Sampling durch Unsicherheit die Möglichkeit, besonders informative Instanzen für ein Modell zu sampeln und damit den Lernprozess effizienter zu gestalten.
In dieser Arbeit sind wir daran interessiert, Unsicherheitsmittel im Prozess des negativen Samplings zu nutzen.
Aus diesem Grund werden zunächst verschiedene Methoden des Unsicherheitssamplings analysiert.
Anschließend ersetzen diese Sampling-Methoden die ursprüngliche Sampling-Methode des generativen adversen netzbasierten Ansatzes KBGAN. 
Schließlich wird der neue Ansatz an insgesamt sieben verschiedenen Datensätzen detailliert evaluiert und die experimentellen Ergebnisse werden verglichen.
Um die Reproduzierbarkeit zu gewährleisten, stellen wir Skripte, Datensätze und Ergebnisse dieser Arbeit auf unserer Projektseite \url{https://github.com/ChristophSc/master-thesis} zur Verfügung.