\subsection{WN18 Dataset}
\label{subsec:methods_wn18}


The courses of evaluation metrics on \textsc{WN18} dataset are very similar to that on \textsc{WN18RR} dataset.
Interestingly, the model pairs with a \transd model in the discriminator also show a significantly worse performance as can be seen in \Autoref{fig:advtrain_wn18_usmax_ussoftmax}.
\begin{figure}
    \centering
    \begin{minipage}{.5\textwidth}
      \centering
      \includegraphics[width=0.9\linewidth]{figures/results/gan_train/not_pretrained/uncertainty/max/entropy/wn18/uncertainty_wn18_mrrs.png}
    \end{minipage}%
    \begin{minipage}{.5\textwidth}
      \centering
      \includegraphics[width=0.9\linewidth]{figures/results/gan_train/not_pretrained/uncertainty/max_distribution/entropy/wn18/uncertainty_wn18_mrrs.png}
    \end{minipage}
    \begin{minipage}{.5\textwidth}
      \centering
      \includegraphics[width=0.9\linewidth]{figures/results/gan_train/not_pretrained/uncertainty/max/entropy/wn18/uncertainty_wn18_hit10.png}
    \end{minipage}%
    \begin{minipage}{.5\textwidth}
      \centering
      \includegraphics[width=0.9\linewidth]{figures/results/gan_train/not_pretrained/uncertainty/max_distribution/entropy/wn18/uncertainty_wn18_hit10.png}
    \end{minipage}%
    \caption{Adversarial training on \textsc{WN18} dataset. 
    Left Figures shows Sampling Negative triples with Uncertainty Max, Right Figures shows Sampling by Uncertainty Distribution.
    Shown are validation H@10 and MRRs for 1000 epochs.}
    \label{fig:advtrain_wn18_usmax_ussoftmax}
\end{figure}
The maximum of the metrics is already reached after about 100 epochs.
But unlike learning on the \textsc{WN18RR} dataset, the discriminator does not achieve better Hit@10 values over time on \textsc{WN18} dataset.
Furthermore, \ussoftmax shows an even clearer difference in the models where \transd is the discriminator for the course of validation MRR.
This also shows that adversarial training with a \transd discriminator is worse for \usmax but better for \ussoftmax.
At Hit@10, no differences between the model pairs can be seen in the \ussoftmax.
\textcolor{red}{ADD REASON}

The progression of losses and rewards is also similar to the \textsc{WN18RR} dataset (\Autoref{fig:advtrain_wn18_usmax_ussoftmax_losses_rewards}).
\begin{figure}
    \centering
    \begin{minipage}{.5\textwidth}
      \centering
      \includegraphics[width=0.9\linewidth]{figures/results/gan_train/not_pretrained/uncertainty/max/entropy/wn18/uncertainty_wn18_losses.png}
    \end{minipage}%
    \begin{minipage}{.5\textwidth}
      \centering
      \includegraphics[width=0.9\linewidth]{figures/results/gan_train/not_pretrained/uncertainty/max_distribution/entropy/wn18/uncertainty_wn18_losses.png}
    \end{minipage}
    \begin{minipage}{.5\textwidth}
      \centering
      \includegraphics[width=0.9\linewidth]{figures/results/gan_train/not_pretrained/uncertainty/max/entropy/wn18/uncertainty_wn18_rew.png}
    \end{minipage}%
    \begin{minipage}{.5\textwidth}
      \centering
      \includegraphics[width=0.9\linewidth]{figures/results/gan_train/not_pretrained/uncertainty/max_distribution/entropy/wn18/uncertainty_wn18_rew.png}
    \end{minipage}%
    \caption{Losses and rewards during adversarial training on \textsc{WN18} dataset. 
    Left figures show Sampling Negative triples with \usmax, right figures show \ussoftmax.
    Shown are validation Hit@10 and MRRs for 1000 epochs.}
    \label{fig:advtrain_wn18_usmax_ussoftmax_losses_rewards}
\end{figure}
However, it is striking that the model pair \dismult + \transd have a considerable drop in rewards after 700 epochs.
\textcolor{red}{ADD REASON}


