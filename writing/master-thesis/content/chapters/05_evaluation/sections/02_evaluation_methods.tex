\section{Evaluation of Uncertainty Sampling Methods}
\label{ch:evaluation:sec:evaluation_methods}
%
To determine the best approach of sampling by uncertainty, we first compare the two methods \usmax and \ussoftmax.
For this purpose, we compare the course of the validation metrics of the two methods in different data sets in \Autoref{subsec:methods_results}.
This is followed by a conclusion of uncertainty sampling methods in  \Autoref{subsec:methods_conclusion}.
%
\subsection{Results on Datasets} \label{subsec:methods_results}

\subsection{UMLS Dataset}

\subsubsection{Uncertainty Sampling Max vs.Uncertainty Sampling Distribution}

\begin{figure}
    \centering
    \begin{minipage}{.5\textwidth}
      \centering
      \includegraphics[width=0.9\linewidth]{figures/results/gan_train/not_pretrained/uncertainty/max/entropy/umls/gan_train_uncertainty_umls_mrrs.png}
    \end{minipage}%
    \begin{minipage}{.5\textwidth}
      \centering
      \includegraphics[width=0.9\linewidth]{figures/results/gan_train/not_pretrained/uncertainty/max_distribution/entropy/umls/gan_train_uncertainty_umls_mrrs.png}
    \end{minipage}
    \begin{minipage}{.5\textwidth}
      \centering
      \includegraphics[width=0.9\linewidth]{figures/results/gan_train/not_pretrained/uncertainty/max/entropy/umls/gan_train_uncertainty_umls_hit10s.png}
    \end{minipage}%
    \begin{minipage}{.5\textwidth}
      \centering
      \includegraphics[width=0.9\linewidth]{figures/results/gan_train/not_pretrained/uncertainty/max_distribution/entropy/umls/gan_train_uncertainty_umls_hit10s.png}
    \end{minipage}%
    \caption{Adversarial training on \umls dataset. 
    Left Figures shows Sampling Negative triples with Uncertainty Max, Right Figures shows Sampling by Uncertainty Distribution.
    Shown are validation H@10 and MRRs for 5000 epochs.}
    \label{fig:advtrain_umls_not_pretrained_uncertainty_max}
\end{figure}






\textbf{KINSHIP Dataset}
\label{subsubsec:methods_kinship}\\
%
A sharp increase in evaluation metrics can also be seen on the \kinship dataset, especially at the beginning of the training (\Autoref{fig:advtrain_kinship_usmax_ussoftmax}).
\begin{figure}[H]
    \centering
    \begin{minipage}{.5\textwidth}
      \centering
      \includegraphics[width=0.9\linewidth]{figures/results/gan_train/not_pretrained/uncertainty/max/entropy/kinship/1k_epochs/uncertainty_kinship_mrrs.png}
    \end{minipage}%
    \begin{minipage}{.5\textwidth}
      \centering
      \includegraphics[width=0.9\linewidth]{figures/results/gan_train/not_pretrained/uncertainty/max_distribution/entropy/kinship/1k_epochs/uncertainty_kinship_mrrs.png}
    \end{minipage}
    \begin{minipage}{.5\textwidth}
      \centering
      \includegraphics[width=0.9\linewidth]{figures/results/gan_train/not_pretrained/uncertainty/max/entropy/kinship/1k_epochs/uncertainty_kinship_hit10.png}
    \end{minipage}%
    \begin{minipage}{.5\textwidth}
      \centering
      \includegraphics[width=0.9\linewidth]{figures/results/gan_train/not_pretrained/uncertainty/max_distribution/entropy/kinship/1k_epochs/uncertainty_kinship_hit10.png}
    \end{minipage}%
    \caption{Adversarial training on \kinship dataset. 
   Left figures show sampling negative triples with \usmax, right figures show \ussoftmax.
    Shown are validation MRR and Hit@10 values for 1000 epochs.}
    \label{fig:advtrain_kinship_usmax_ussoftmax}
\end{figure}
With both \usmax and \ussoftmax, the values increase steadily up to about 100 epochs.
In contrast to the \umls data set, however, the models with a \transd discriminator perform significantly better than those with a \transe discriminator.
So they do not seem to depend on the generator model as with the \umls dataset.
Furthermore, this difference between the different models in the discriminator can also be seen in the \ussoftmax sampling approach.
Equal to the \umls dataset, however, is that with \usmax there is a drop in the evaluation metrics after reaching the maximum at 100 epochs, but the values continue to increase slightly with \ussoftmax.
Furthermore, there is a significant increase in the Hit@10 metric with the model pairs with a \transd discriminator.
These reach a value of over 65\% with \ussoftmax, whereas they remained below this with \usmax and the other model pairs.

The learning behavior of the two sampling methods with the different model pairs can also be seen in the progression of the losses and rewards depicted in \Autoref{fig:advtrain_kinship_usmax_ussoftmax_losses_rewards}.
\begin{figure}[H]
    \centering
    \begin{minipage}{.5\textwidth}
      \centering
      \includegraphics[width=0.9\linewidth]{figures/results/gan_train/not_pretrained/uncertainty/max/entropy/kinship/1k_epochs/uncertainty_kinship_losses.png}
    \end{minipage}%
    \begin{minipage}{.5\textwidth}
      \centering
      \includegraphics[width=0.9\linewidth]{figures/results/gan_train/not_pretrained/uncertainty/max_distribution/entropy/kinship/1k_epochs/uncertainty_kinship_losses.png}
    \end{minipage}
    \begin{minipage}{.5\textwidth}
      \centering
      \includegraphics[width=0.9\linewidth]{figures/results/gan_train/not_pretrained/uncertainty/max/entropy/kinship/1k_epochs/uncertainty_kinship_rew.png}
    \end{minipage}%
    \begin{minipage}{.5\textwidth}
      \centering
      \includegraphics[width=0.9\linewidth]{figures/results/gan_train/not_pretrained/uncertainty/max_distribution/entropy/kinship/1k_epochs/uncertainty_kinship_rew.png}
    \end{minipage}%
    \caption{Losses and rewards during 1000 epochs of adversarial training on \kinship dataset. 
    Left figures show sampling negative triples with \usmax, right figures show \ussoftmax.}
    \label{fig:advtrain_kinship_usmax_ussoftmax_losses_rewards}
\end{figure}
Especially for the losses and rewards, there is a clear difference between the different model pairs.
While the losses initially decrease for all model pairs, they increase again with \usmax after 100 epochs for almost all model pairs.
Only with the model pair \complex and \distmult they continue to decrease.
The same applies to the rewards with \usmax.
With \ussoftmax, on the other hand, the clear distinction of the model pairs with \transd discriminator can be seen again. 
However, although the rewards decrease with \ussoftmax with \transd discriminator, the evaluation metrics for exactly these model pairs take a high value.

\input{content/chapters/05_evaluation/sections/02_methods/03_methods_wn18rr}

\input{content/chapters/05_evaluation/sections/02_methods/04_methods_wn18}

\input{content/chapters/05_evaluation/sections/02_methods/05_methods_fb15k237}

\textbf{FB15k Dataset}
\label{subsubsec:methods_fb15k}\\
%
When looking at the validation curves on the \textsc{FB15k} dataset, a difference can be seen in the different model pairs (\Autoref{fig:advtrain_fb15k_usmax_ussoftmax}).
In this case, however, with \usmax the model pairs perform significantly better with a \distmult generator and with \ussoftmax with a \transd discriminator.
\begin{figure}[H]
    \centering
    \begin{minipage}{.5\textwidth}
      \centering
      \includegraphics[width=0.9\linewidth]{figures/results/gan_train/not_pretrained/uncertainty/max/entropy/fb15k/1k_epochs/uncertainty_fb15k_mrrs.png}
    \end{minipage}%
    \begin{minipage}{.5\textwidth}
      \centering
      \includegraphics[width=0.9\linewidth]{figures/results/gan_train/not_pretrained/uncertainty/max_distribution/entropy/fb15k/1k_epochs/uncertainty_fb15k_mrrs.png}
    \end{minipage}
    \begin{minipage}{.5\textwidth}
      \centering
      \includegraphics[width=0.9\linewidth]{figures/results/gan_train/not_pretrained/uncertainty/max/entropy/fb15k/1k_epochs/uncertainty_fb15k_hit10.png}
    \end{minipage}%
    \begin{minipage}{.5\textwidth}
      \centering
      \includegraphics[width=0.9\linewidth]{figures/results/gan_train/not_pretrained/uncertainty/max_distribution/entropy/fb15k/1k_epochs/uncertainty_fb15k_hit10.png}
    \end{minipage}%
    \caption{Adversarial training on \textsc{FB15k} dataset. 
    Left figures show sampling negative triples with \usmax, right figures show \ussoftmax.
    Shown are validation MRR and Hit@10 values for 1000 epochs.}
    \label{fig:advtrain_fb15k_usmax_ussoftmax}
\end{figure}
Both evaluation metrics show a clear increase over the entire validation process.
However, it is noticeable that with \ussoftmax both MRR and Hit@10 achieve higher values than \usmax.

This significantly stronger increase in the evaluation metrics with a \distmult generator for \usmax and a \transd discriminator for \ussoftmax can also be seen in the development of the rewards (\Autoref{fig:advtrain_fb15k_usmax_ussoftmax_losses_rewards}).
\begin{figure}[H]
    \centering
    \begin{minipage}{.5\textwidth}
      \centering
      \includegraphics[width=0.9\linewidth]{figures/results/gan_train/not_pretrained/uncertainty/max/entropy/fb15k/1k_epochs/uncertainty_fb15k_losses.png}
    \end{minipage}%
    \begin{minipage}{.5\textwidth}
      \centering
      \includegraphics[width=0.9\linewidth]{figures/results/gan_train/not_pretrained/uncertainty/max_distribution/entropy/fb15k/1k_epochs/uncertainty_fb15k_losses.png}
    \end{minipage}
    \begin{minipage}{.5\textwidth}
      \centering
      \includegraphics[width=0.9\linewidth]{figures/results/gan_train/not_pretrained/uncertainty/max/entropy/fb15k/1k_epochs/uncertainty_fb15k_rew.png}
    \end{minipage}%
    \begin{minipage}{.5\textwidth}
      \centering
      \includegraphics[width=0.9\linewidth]{figures/results/gan_train/not_pretrained/uncertainty/max_distribution/entropy/fb15k/1k_epochs/uncertainty_fb15k_rew.png}
    \end{minipage}%
    \caption{Losses and rewards during 1000 epochs of adversarial training on \textsc{FB15k} dataset. 
    Left figures show sampling negative triples with \usmax, right figures show \ussoftmax.}
    \label{fig:advtrain_fb15k_usmax_ussoftmax_losses_rewards}
\end{figure}
However, even if the rewards for the other model pairs decrease, the evaluation metrics still improve for \ussoftmax.

\textbf{YAGO3-10 Dataset}
\label{subsubsec:methods_yago3_10}\\
%
On \textsc{YAGO3-10} dataset all model pairs seem to perform differently with \usmax, while they are almost equally with \ussoftmax (\Autoref{fig:advtrain_yago3_10_usmax_ussoftmax}).
The model pairs with \complex generator, and especially \complex and \transd achieve better results with \usmax.

Looking at the development of losses and rewards on the \textsc{YAGO3-10} dataset, losses decrease over time for both sampling methods, but rewards do not seem to increase over time (\Autoref{fig:advtrain_yago3_10_usmax_ussoftmax_losses_rewards}).
On the contrary, these seem more likely to stagnate or even reduce over time after having a high peak at the beginning.
In addition, the rewards of \distmult and \transd with \usmax are different than other model pairs with a high variance during training.
\clearpage
\begin{figure}[H]
    \centering
    \begin{minipage}{.45\textwidth}
      \centering
      \includegraphics[width=0.9\linewidth]{figures/results/gan_train/not_pretrained/uncertainty/max/entropy/yago3_10/1k_epochs/uncertainty_yago3_10_mrrs.png}
    \end{minipage}%
    \begin{minipage}{.45\textwidth}
      \centering
      \includegraphics[width=0.9\linewidth]{figures/results/gan_train/not_pretrained/uncertainty/max_distribution/entropy/yago3_10/1k_epochs/uncertainty_yago3_10_mrrs.png}
    \end{minipage}
    \begin{minipage}{.45\textwidth}
      \centering
      \includegraphics[width=0.9\linewidth]{figures/results/gan_train/not_pretrained/uncertainty/max/entropy/yago3_10/1k_epochs/uncertainty_yago3_10_hit10.png}
    \end{minipage}%
    \begin{minipage}{.45\textwidth}
      \centering
      \includegraphics[width=0.9\linewidth]{figures/results/gan_train/not_pretrained/uncertainty/max_distribution/entropy/yago3_10/1k_epochs/uncertainty_yago3_10_hit10.png}
    \end{minipage}%
    \caption{Adversarial training on \textsc{YAGO3-10} dataset. 
    Left figures show sampling negative triples with \usmax, right figures show \ussoftmax.
    Shown are validation MRR and Hit@10 values for 1000 epochs.}
    \label{fig:advtrain_yago3_10_usmax_ussoftmax}
\end{figure}

\begin{figure}[H]
    \centering
    \begin{minipage}{.45\textwidth}
      \centering
      \includegraphics[width=0.9\linewidth]{figures/results/gan_train/not_pretrained/uncertainty/max/entropy/yago3_10/1k_epochs/uncertainty_yago3_10_losses.png}
    \end{minipage}%
    \begin{minipage}{.45\textwidth}
      \centering
      \includegraphics[width=0.9\linewidth]{figures/results/gan_train/not_pretrained/uncertainty/max_distribution/entropy/yago3_10/1k_epochs/uncertainty_yago3_10_losses.png}
    \end{minipage}
    \begin{minipage}{.45\textwidth}
      \centering
      \includegraphics[width=0.9\linewidth]{figures/results/gan_train/not_pretrained/uncertainty/max/entropy/yago3_10/1k_epochs/uncertainty_yago3_10_rew.png}
    \end{minipage}%
    \begin{minipage}{.45\textwidth}
      \centering
      \includegraphics[width=0.9\linewidth]{figures/results/gan_train/not_pretrained/uncertainty/max_distribution/entropy/yago3_10/1k_epochs/uncertainty_yago3_10_rew.png}
    \end{minipage}%
    \caption{Losses and rewards during 1000 epochs of adversarial training on \textsc{YAGO3-10} dataset. 
    Left figures show sampling negative triples with \usmax, right figures show \ussoftmax.}
    \label{fig:advtrain_yago3_10_usmax_ussoftmax_losses_rewards}
\end{figure}
%
\subsection{Summary and Conclusion of Uncertainty Sampling Methods} 
\label{subsec:methods_conclusion}
% summary
In summary, \usmax and \ussoftmax achieve similar results for many embedding model pairs and datasets.
Additionally, it can be seen that both sampling methods are very dependent on the chosen embedding models for the generator and discriminator.
However, \ussoftmax's training appears to be more stable and consistent than \usmax's, since \usmax has high fluctuations of the results depending on the chosen models and performs badly especially on some model pairs.
Furthermore, a strong dependence of the training on the dataset can be seen.
In many cases, loss is reduced over time and reward is increased, just as learning behavior is expected.
And from the learning curves can be seen, that even if the reward is not increased over time, an increase of evaluation metrics can occur.

% conclusion
In conclusion, adversarial training seem to work with both sampling methods.
Since training is very dependent on datasets and model pairs, hyperparameters need to be tuned for each dataset and model pair independently.
Since the development of the approach was mainly done on the \umls dataset, \ussoftmax performs particularly well on it.
No additional hyperparameter tuning was done for the other and large datasets.
This could be the reason why the training works in general, but does not outperform any results.
Since the training with \ussoftmax is more stable and not so dependent on the models and their hyperparameter optimization, the further evaluation will be performed with this sampling method.
\clearpage