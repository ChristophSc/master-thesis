\section{Evaluation of Uncertainty Sampling Methods}
\label{ch:evaluation:sec:evaluation_methods}

To determine the best approach of sampling by uncertainty, we first compare the two methods \textsc{UncertaintySamplingMax} and \textsc{UncertaintySamplingSoftmax}.
For this purpose, we compare the course of the validation metrics of the two methods in the different data sets in \Autoref{subsec:methods_umls} to \ref{subsec:methods_fb15k237}.

\subsection{UMLS Dataset}

\subsubsection{Uncertainty Sampling Max vs.Uncertainty Sampling Distribution}

\begin{figure}
    \centering
    \begin{minipage}{.5\textwidth}
      \centering
      \includegraphics[width=0.9\linewidth]{figures/results/gan_train/not_pretrained/uncertainty/max/entropy/umls/gan_train_uncertainty_umls_mrrs.png}
    \end{minipage}%
    \begin{minipage}{.5\textwidth}
      \centering
      \includegraphics[width=0.9\linewidth]{figures/results/gan_train/not_pretrained/uncertainty/max_distribution/entropy/umls/gan_train_uncertainty_umls_mrrs.png}
    \end{minipage}
    \begin{minipage}{.5\textwidth}
      \centering
      \includegraphics[width=0.9\linewidth]{figures/results/gan_train/not_pretrained/uncertainty/max/entropy/umls/gan_train_uncertainty_umls_hit10s.png}
    \end{minipage}%
    \begin{minipage}{.5\textwidth}
      \centering
      \includegraphics[width=0.9\linewidth]{figures/results/gan_train/not_pretrained/uncertainty/max_distribution/entropy/umls/gan_train_uncertainty_umls_hit10s.png}
    \end{minipage}%
    \caption{Adversarial training on \umls dataset. 
    Left Figures shows Sampling Negative triples with Uncertainty Max, Right Figures shows Sampling by Uncertainty Distribution.
    Shown are validation H@10 and MRRs for 5000 epochs.}
    \label{fig:advtrain_umls_not_pretrained_uncertainty_max}
\end{figure}






\subsection{WN18RR Dataset}

\begin{figure}
    \centering
    \begin{minipage}{.5\textwidth}
      \centering
      \includegraphics[width=0.9\linewidth]{figures/results/gan_train/not_pretrained/uncertainty/max/entropy/wn18rr/uncertainty_wn18rr_mrrs.png}
    \end{minipage}%
    \begin{minipage}{.5\textwidth}
      \centering
      \includegraphics[width=0.9\linewidth]{figures/results/gan_train/not_pretrained/uncertainty/max_distribution/entropy/wn18rr/uncertainty_wn18rr_mrrs.png}
    \end{minipage}
    \begin{minipage}{.5\textwidth}
      \centering
      \includegraphics[width=0.9\linewidth]{figures/results/gan_train/not_pretrained/uncertainty/max/entropy/wn18rr/uncertainty_wn18rr_hit10.png}
    \end{minipage}%
    \begin{minipage}{.5\textwidth}
      \centering
      \includegraphics[width=0.9\linewidth]{figures/results/gan_train/not_pretrained/uncertainty/max_distribution/entropy/wn18rr/uncertainty_wn18rr_hit10.png}
    \end{minipage}%
    \caption{Adversarial training on \textsc{WN18RR} dataset. 
    Left Figures shows Sampling Negative triples with Uncertainty Max, Right Figures shows Sampling by Uncertainty Distribution.
    Shown are validation H@10 and MRRs for 1000 epochs.}
    \label{fig:advtrain_not_pretrained_wn18rr_pretrained}
\end{figure}








\subsection{WN18 Dataset}

\subsubsection{Uncertainty Sampling Max}

\begin{figure}
    \centering
    \begin{minipage}{.5\textwidth}
      \centering
      \includegraphics[width=0.9\linewidth]{figures/results/gan_train/not_pretrained/uncertainty/max/entropy/wn18/gan_train_uncertainty_wn18_losses.png}
    \end{minipage}%
    \begin{minipage}{.5\textwidth}
      \centering
      \includegraphics[width=0.9\linewidth]{figures/results/gan_train/not_pretrained/uncertainty/max/entropy/wn18/gan_train_uncertainty_wn18_rewards.png}
    \end{minipage}
    \begin{minipage}{.5\textwidth}
      \centering
      \includegraphics[width=0.9\linewidth]{figures/results/gan_train/not_pretrained/uncertainty/max/entropy/wn18/gan_train_uncertainty_wn18_mrrs.png}
    \end{minipage}%
    \begin{minipage}{.5\textwidth}
      \centering
      \includegraphics[width=0.9\linewidth]{figures/results/gan_train/not_pretrained/uncertainty/max/entropy/wn18/gan_train_uncertainty_wn18_hit10s.png}
    \end{minipage}%
    \caption{Adversarial training without Pretraining on \textsc{WN18} dataset. 
    Negative Triples are sampled by Uncertainty Max.
    Shown are the training losses and rewards and validation H@10 and MRRs for 5000 epochs.}
    \label{fig:advtrain_wn18_not_pretrained_uncertainty_max}
\end{figure}

\subsubsection{Uncertainty Sampling Distribution}

\begin{figure}
    \centering
    \begin{minipage}{.5\textwidth}
      \centering
      \includegraphics[width=0.9\linewidth]{figures/results/gan_train/not_pretrained/uncertainty/max_distribution/entropy/wn18/gan_train_uncertainty_wn18_losses.png}
    \end{minipage}%
    \begin{minipage}{.5\textwidth}
      \centering
      \includegraphics[width=0.9\linewidth]{figures/results/gan_train/not_pretrained/uncertainty/max_distribution/entropy/wn18/gan_train_uncertainty_wn18_rewards.png}
    \end{minipage}
    \begin{minipage}{.5\textwidth}
      \centering
      \includegraphics[width=0.9\linewidth]{figures/results/gan_train/not_pretrained/uncertainty/max_distribution/entropy/wn18/gan_train_uncertainty_wn18_mrrs.png}
    \end{minipage}%
    \begin{minipage}{.5\textwidth}
      \centering
      \includegraphics[width=0.9\linewidth]{figures/results/gan_train/not_pretrained/uncertainty/max_distribution/entropy/wn18/gan_train_uncertainty_wn18_hit10s.png}
    \end{minipage}%
    \caption{Adversarial Training without Pretraining on \textsc{WN18} dataset. 
    Negative Triples are sampled by Uncertainty Distribution.
    Shown are the training losses and rewards and validation H@10 and MRRs for 5000 epochs.}
    \label{fig:advtrain_not_pretrained_wn18_pretrained}
\end{figure}







\subsection{FB15k237 Dataset}

\subsubsection{Uncertainty Sampling Max}

\begin{figure}
    \centering
    \begin{minipage}{.5\textwidth}
      \centering
      \includegraphics[width=0.9\linewidth]{figures/results/gan_train/not_pretrained/uncertainty/max/entropy/fb15k237/gan_train_uncertainty_fb15k237_losses.png}
    \end{minipage}%
    \begin{minipage}{.5\textwidth}
      \centering
      \includegraphics[width=0.9\linewidth]{figures/results/gan_train/not_pretrained/uncertainty/max/entropy/fb15k237/gan_train_uncertainty_fb15k237_rewards.png}
    \end{minipage}
    \begin{minipage}{.5\textwidth}
      \centering
      \includegraphics[width=0.9\linewidth]{figures/results/gan_train/not_pretrained/uncertainty/max/entropy/fb15k237/gan_train_uncertainty_fb15k237_mrrs.png}
    \end{minipage}%
    \begin{minipage}{.5\textwidth}
      \centering
      \includegraphics[width=0.9\linewidth]{figures/results/gan_train/not_pretrained/uncertainty/max/entropy/fb15k237/gan_train_uncertainty_fb15k237_hit10s.png}
    \end{minipage}%
    \caption{Adversarial training without Pretraining on \textsc{FB15k237} dataset. 
    Negative Triples are sampled by Uncertainty Max.
    Shown are the training losses and rewards and validation H@10 and MRRs for 5000 epochs.}
    \label{fig:advtrain_fb15k237_not_pretrained_uncertainty_max}
\end{figure}

\subsubsection{Uncertainty Sampling Distribution}

\begin{figure}
    \centering
    \begin{minipage}{.5\textwidth}
      \centering
      \includegraphics[width=0.9\linewidth]{figures/results/gan_train/not_pretrained/uncertainty/max_distribution/entropy/fb15k237/gan_train_uncertainty_fb15k237_losses.png}
    \end{minipage}%
    \begin{minipage}{.5\textwidth}
      \centering
      \includegraphics[width=0.9\linewidth]{figures/results/gan_train/not_pretrained/uncertainty/max_distribution/entropy/fb15k237/gan_train_uncertainty_fb15k237_rewards.png}
    \end{minipage}
    \begin{minipage}{.5\textwidth}
      \centering
      \includegraphics[width=0.9\linewidth]{figures/results/gan_train/not_pretrained/uncertainty/max_distribution/entropy/fb15k237/gan_train_uncertainty_fb15k237_mrrs.png}
    \end{minipage}%
    \begin{minipage}{.5\textwidth}
      \centering
      \includegraphics[width=0.9\linewidth]{figures/results/gan_train/not_pretrained/uncertainty/max_distribution/entropy/fb15k237/gan_train_uncertainty_fb15k237_hit10s.png}
    \end{minipage}%
    \caption{Adversarial Training without Pretraining on \textsc{WN18} dataset. 
    Negative Triples are sampled by Uncertainty Distribution.
    Shown are the training losses and rewards and validation H@10 and MRRs for 5000 epochs.}
    \label{fig:advtrain_not_pretrained_fb15k237_pretrained}
\end{figure}







\subsubsection{Conclusion of Uncertainty Sampling Methods}

In conclusion, \usmax and \ussoftmax perform equally for many embedding model pairs and datasets.
Just in some cases \ussoftmax performs better and increase MRR and Hit@10 metrics slightly over time and in addition, performance does never not decrease, it always stays on same level even if it does not increase either.
Additionally, especially for some model pairs \usmax performs bad.
Furthermore, it can be seen that adversarial training performs better with larger data sets, because the course of losses and rewards for smaller datasets is very different than normal courses of losses and rewards.

