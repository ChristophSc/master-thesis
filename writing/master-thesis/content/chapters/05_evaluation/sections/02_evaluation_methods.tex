\section{Evaluation of Uncertainty Sampling Methods}
\label{ch:evaluation:sec:evaluation_methods}
%
To determine the best approach of sampling by uncertainty, we first compare the two methods \usmax and \ussoftmax.
For this purpose, we compare the course of the validation metrics of the two methods in different data sets in \Autoref{subsec:methods_results}.
This is followed by a conclusion of uncertainty sampling methods in  \Autoref{subsec:methods_conclusion}.
%
\subsection{Results on Datasets} \label{subsec:methods_results}
\subsection{UMLS Dataset}

\subsubsection{Uncertainty Sampling Max vs.Uncertainty Sampling Distribution}

\begin{figure}
    \centering
    \begin{minipage}{.5\textwidth}
      \centering
      \includegraphics[width=0.9\linewidth]{figures/results/gan_train/not_pretrained/uncertainty/max/entropy/umls/gan_train_uncertainty_umls_mrrs.png}
    \end{minipage}%
    \begin{minipage}{.5\textwidth}
      \centering
      \includegraphics[width=0.9\linewidth]{figures/results/gan_train/not_pretrained/uncertainty/max_distribution/entropy/umls/gan_train_uncertainty_umls_mrrs.png}
    \end{minipage}
    \begin{minipage}{.5\textwidth}
      \centering
      \includegraphics[width=0.9\linewidth]{figures/results/gan_train/not_pretrained/uncertainty/max/entropy/umls/gan_train_uncertainty_umls_hit10s.png}
    \end{minipage}%
    \begin{minipage}{.5\textwidth}
      \centering
      \includegraphics[width=0.9\linewidth]{figures/results/gan_train/not_pretrained/uncertainty/max_distribution/entropy/umls/gan_train_uncertainty_umls_hit10s.png}
    \end{minipage}%
    \caption{Adversarial training on \umls dataset. 
    Left Figures shows Sampling Negative triples with Uncertainty Max, Right Figures shows Sampling by Uncertainty Distribution.
    Shown are validation H@10 and MRRs for 5000 epochs.}
    \label{fig:advtrain_umls_not_pretrained_uncertainty_max}
\end{figure}





\textbf{KINSHIP Dataset}
\label{subsubsec:methods_kinship}\\
%
A sharp increase in evaluation metrics can also be seen on the \kinship dataset, especially at the beginning of the training (\Autoref{fig:advtrain_kinship_usmax_ussoftmax}).
\begin{figure}[H]
    \centering
    \begin{minipage}{.5\textwidth}
      \centering
      \includegraphics[width=0.9\linewidth]{figures/results/gan_train/not_pretrained/uncertainty/max/entropy/kinship/1k_epochs/uncertainty_kinship_mrrs.png}
    \end{minipage}%
    \begin{minipage}{.5\textwidth}
      \centering
      \includegraphics[width=0.9\linewidth]{figures/results/gan_train/not_pretrained/uncertainty/max_distribution/entropy/kinship/1k_epochs/uncertainty_kinship_mrrs.png}
    \end{minipage}
    \begin{minipage}{.5\textwidth}
      \centering
      \includegraphics[width=0.9\linewidth]{figures/results/gan_train/not_pretrained/uncertainty/max/entropy/kinship/1k_epochs/uncertainty_kinship_hit10.png}
    \end{minipage}%
    \begin{minipage}{.5\textwidth}
      \centering
      \includegraphics[width=0.9\linewidth]{figures/results/gan_train/not_pretrained/uncertainty/max_distribution/entropy/kinship/1k_epochs/uncertainty_kinship_hit10.png}
    \end{minipage}%
    \caption{Adversarial training on \kinship dataset. 
   Left figures show sampling negative triples with \usmax, right figures show \ussoftmax.
    Shown are validation MRR and Hit@10 values for 1000 epochs.}
    \label{fig:advtrain_kinship_usmax_ussoftmax}
\end{figure}
With both \usmax and \ussoftmax, the values increase steadily up to about 100 epochs.
In contrast to the \umls data set, however, the models with a \transd discriminator perform significantly better than those with a \transe discriminator.
So they do not seem to depend on the generator model as with the \umls dataset.
Furthermore, this difference between the different models in the discriminator can also be seen in the \ussoftmax sampling approach.
Equal to the \umls dataset, however, is that with \usmax there is a drop in the evaluation metrics after reaching the maximum at 100 epochs, but the values continue to increase slightly with \ussoftmax.
Furthermore, there is a significant increase in the Hit@10 metric with the model pairs with a \transd discriminator.
These reach a value of over 65\% with \ussoftmax, whereas they remained below this with \usmax and the other model pairs.

The learning behavior of the two sampling methods with the different model pairs can also be seen in the progression of the losses and rewards depicted in \Autoref{fig:advtrain_kinship_usmax_ussoftmax_losses_rewards}.
\begin{figure}[H]
    \centering
    \begin{minipage}{.5\textwidth}
      \centering
      \includegraphics[width=0.9\linewidth]{figures/results/gan_train/not_pretrained/uncertainty/max/entropy/kinship/1k_epochs/uncertainty_kinship_losses.png}
    \end{minipage}%
    \begin{minipage}{.5\textwidth}
      \centering
      \includegraphics[width=0.9\linewidth]{figures/results/gan_train/not_pretrained/uncertainty/max_distribution/entropy/kinship/1k_epochs/uncertainty_kinship_losses.png}
    \end{minipage}
    \begin{minipage}{.5\textwidth}
      \centering
      \includegraphics[width=0.9\linewidth]{figures/results/gan_train/not_pretrained/uncertainty/max/entropy/kinship/1k_epochs/uncertainty_kinship_rew.png}
    \end{minipage}%
    \begin{minipage}{.5\textwidth}
      \centering
      \includegraphics[width=0.9\linewidth]{figures/results/gan_train/not_pretrained/uncertainty/max_distribution/entropy/kinship/1k_epochs/uncertainty_kinship_rew.png}
    \end{minipage}%
    \caption{Losses and rewards during 1000 epochs of adversarial training on \kinship dataset. 
    Left figures show sampling negative triples with \usmax, right figures show \ussoftmax.}
    \label{fig:advtrain_kinship_usmax_ussoftmax_losses_rewards}
\end{figure}
Especially for the losses and rewards, there is a clear difference between the different model pairs.
While the losses initially decrease for all model pairs, they increase again with \usmax after 100 epochs for almost all model pairs.
Only with the model pair \complex and \distmult they continue to decrease.
The same applies to the rewards with \usmax.
With \ussoftmax, on the other hand, the clear distinction of the model pairs with \transd discriminator can be seen again. 
However, although the rewards decrease with \ussoftmax with \transd discriminator, the evaluation metrics for exactly these model pairs take a high value.
\input{content/chapters/05_evaluation/sections/02_methods/03_methods_wn18rr}
\input{content/chapters/05_evaluation/sections/02_methods/04_methods_wn18}
\input{content/chapters/05_evaluation/sections/02_methods/05_methods_fb15k237}
%
\subsection{Conclusion} 
\label{subsec:methods_conclusion}
%
In conclusion, \usmax and \ussoftmax perform equally for many embedding model pairs and datasets.
Just in some cases \ussoftmax performs better and increase MRR and Hit@10 metrics slightly over time and in addition, performance does never not decrease, it always stays on same level even if it does not increase either.
Additionally, especially for some model pairs \usmax performs bad.
Furthermore, it can be seen that adversarial training performs better with larger data sets, because the course of losses and rewards for smaller datasets is very different than normal courses of losses and rewards.

