\section{Evaluation of Uncertainty Measures}
\label{ch:evaluation:sec:evaluation_metrics}
%
In the previous section, the two uncertainty sampling methods were only compared with the uncertainty measure entropy.
Since \ussoftmax was found to be a slightly better sampling method, this section now compares and evaluates the different uncertainty metrics entropy, least confidence, margin of confidence, and ratio of confidence with \ussoftmax. 
These are also analysed for different data sets in \Autoref{subsec:measures_results} and concluded in \Autoref{subsec:measure_conclusion}.
%
\subsection{Results on Datasets} \label{subsec:measures_results}
\textbf{UMLS Dataset}
\label{subsubsec:metrics_umls}\\
%
First, we look at the different uncertainty metrics on the smallest \umls dataset.
Therefore, we take a closer look at the course of MRR values for 1000 epochs with entropy, least confidence, confidence margin and confidence ratio metrics depicted in \Autoref{fig:advtrain_metrics_umls}.
\begin{figure}[H]
    \centering
    \begin{minipage}{.5\textwidth}
      \centering
      \includegraphics[width=0.9\linewidth]{figures/results/gan_train/not_pretrained/uncertainty/max_distribution/entropy/umls/1k_epochs/uncertainty_umls_mrrs.png}
    \end{minipage}%
    \begin{minipage}{.5\textwidth}
      \centering
      \includegraphics[width=0.9\linewidth]{figures/results/gan_train/not_pretrained/uncertainty/max_distribution/least_confidence/umls/uncertainty_umls_mrrs.png}
    \end{minipage}
    \begin{minipage}{.5\textwidth}
      \centering
      \includegraphics[width=0.9\linewidth]{figures/results/gan_train/not_pretrained/uncertainty/max_distribution/confidence_margin/umls/uncertainty_umls_mrrs.png}
    \end{minipage}%
    \begin{minipage}{.5\textwidth}
      \centering
      \includegraphics[width=0.9\linewidth]{figures/results/gan_train/not_pretrained/uncertainty/max_distribution/confidence_ratio/umls/uncertainty_umls_mrrs.png}
    \end{minipage}%
    \caption{Validation MRR values during adversarial training on \umls dataset for 1000 epochs. 
    Figures show different uncertainty metrics with sampling method  \ussoftmax.
    Top left shows entropy, top right least confidence, 
    bottom left confidence margin and bottom right confidence ratio.}
    \label{fig:advtrain_metrics_umls}
\end{figure}
With all uncertainty metrics, a significant increase in the MRR value is achieved at the beginning of the training.
In addition, a constant increase is achieved with all metrics, so that this also suggests a further increase in the MRR value and thus an improvement in embedding after 1000 epochs.
For all uncertainty sampling metrics, a maximum MRR of slightly below 80\% is achieved on the \umls dataset.


If you look at the rewards over the course (\Autoref{fig:advtrain_metrics_umls_rew}), you can also see a sharp increase right at the beginning of the training.
\begin{figure}[H]
    \centering
    \begin{minipage}{.5\textwidth}
      \centering
      \includegraphics[width=0.9\linewidth]{figures/results/gan_train/not_pretrained/uncertainty/max_distribution/entropy/umls/1k_epochs/uncertainty_umls_rew.png}
    \end{minipage}%
    \begin{minipage}{.5\textwidth}
      \centering
      \includegraphics[width=0.9\linewidth]{figures/results/gan_train/not_pretrained/uncertainty/max_distribution/least_confidence/umls/uncertainty_umls_rew.png}
    \end{minipage}
    \begin{minipage}{.5\textwidth}
      \centering
      \includegraphics[width=0.9\linewidth]{figures/results/gan_train/not_pretrained/uncertainty/max_distribution/confidence_margin/umls/uncertainty_umls_rew.png}
    \end{minipage}%
    \begin{minipage}{.5\textwidth}
      \centering
      \includegraphics[width=0.9\linewidth]{figures/results/gan_train/not_pretrained/uncertainty/max_distribution/confidence_ratio/umls/uncertainty_umls_rew.png}
    \end{minipage}%
    \caption{Rewards during adversarial training on \umls dataset. 
    Figures show training rewards of 1000 epochs with \ussoftmax with different uncertainty metrics.
    Top left shows entropy, top right least confidence, 
    bottom left confidence margin and bottom right confidence ratio.}
    \label{fig:advtrain_metrics_umls_rew}
\end{figure}
However, this decreases steadily after reaching the maximum value.
Thus, although the MRR continued to increase over the course of the training, sampling from the negative triple set did not result in a further increase in the reward.
As the reward continues to decrease, this means that the returned distance of the negative triple in the discriminator continues to increase.
Thus, the model is better and better able to recognise the sampled negative triple and to assign a high distance to it.
It can also be seen that although all the rewards of the various uncertainty metrics have the same course, the distances with a \transd model in the discriminator are significantly lower than with a \transe model.


\input{content/chapters/05_evaluation/sections/03_measure/02_measure_wn18rr}

\subsubsection{WN18 Dataset}
\label{subsubsec:measures_wn18}
%
Also for the adversarial training on the WN18 dataset, the development of the MRR value is similar for all uncertainty metrics (\Autoref{fig:advtrain_measures_wn18}).
\begin{figure}
    \centering
    \begin{minipage}{.5\textwidth}
      \centering
      \includegraphics[width=0.9\linewidth]{figures/results/gan_train/not_pretrained/uncertainty/max_distribution/entropy/wn18/1k_epochs/uncertainty_wn18_mrrs.png}
    \end{minipage}%
    \begin{minipage}{.5\textwidth}
      \centering
      \includegraphics[width=0.9\linewidth]{figures/results/gan_train/not_pretrained/uncertainty/max_distribution/least_confidence/wn18/uncertainty_wn18_mrrs.png}
    \end{minipage}
    \begin{minipage}{.5\textwidth}
      \centering
      \includegraphics[width=0.9\linewidth]{figures/results/gan_train/not_pretrained/uncertainty/max_distribution/confidence_margin/wn18/uncertainty_wn18_mrrs.png}
    \end{minipage}%
    \begin{minipage}{.5\textwidth}
      \centering
      \includegraphics[width=0.9\linewidth]{figures/results/gan_train/not_pretrained/uncertainty/max_distribution/confidence_ratio/wn18/uncertainty_wn18_mrrs.png}
    \end{minipage}%
    \caption{Evaluation of Adversarial training on \textsc{WN18} dataset. 
    Figures show validation MRRs for 1000 epochs with Uncertainty Sampling Distribution with different uncertainty metrics.
    Top left shows entropy, top right least confidence, 
    bottom left confidence margin and bottom right confidence ratio.}
    \label{fig:advtrain_measures_wn18}
\end{figure}
However, while model pairs with \transd as discriminator model stagnates at about an MRR of 44\%, model pairs with \transe  discriminator model decreases after reaching maximum of 44\% after about 300 epochs.
This can be noted for all uncertainty metrics.

If we take a closer look at rewards during training on \textsc{WN18} dataset, the course is similar to training rewards on datasets mentioned before (\Autoref{fig:advtrain_metrics_wn18}).
\begin{figure}
    \centering
    \begin{minipage}{.5\textwidth}
      \centering
      \includegraphics[width=0.9\linewidth]{figures/results/gan_train/not_pretrained/uncertainty/max_distribution/entropy/wn18/1k_epochs/uncertainty_wn18_rew.png}
    \end{minipage}%
    \begin{minipage}{.5\textwidth}
      \centering
      \includegraphics[width=0.9\linewidth]{figures/results/gan_train/not_pretrained/uncertainty/max_distribution/least_confidence/wn18/uncertainty_wn18_rew.png}
    \end{minipage}
    \begin{minipage}{.5\textwidth}
      \centering
      \includegraphics[width=0.9\linewidth]{figures/results/gan_train/not_pretrained/uncertainty/max_distribution/confidence_margin/wn18/uncertainty_wn18_rew.png}
    \end{minipage}%
    \begin{minipage}{.5\textwidth}
      \centering
      \includegraphics[width=0.9\linewidth]{figures/results/gan_train/not_pretrained/uncertainty/max_distribution/confidence_ratio/wn18/uncertainty_wn18_rew.png}
    \end{minipage}%
    \caption{Rewards during adversarial training on \textsc{WN18} dataset. 
    Figures show training rewards of 1000 epochs with \ussoftmax with different uncertainty metrics.
    Top left shows entropy, top right least confidence, 
    bottom left confidence margin and bottom right confidence ratio.}
    \label{fig:advtrain_metrics_wn18}
\end{figure}
Model pairs with \transe discriminator increase quickly to a reward of -8.2 at 50 epochs an increases only slowly afterwards.
In contrast, model pairs with \transd discriminator reach a local maximum at 50 epochs with only a reward of -8.8.
But subsequently, rewards decrease again until 200 epochs and start increasing again.

\input{content/chapters/05_evaluation/sections/03_measure/04_measure_fb15k237}

\subsection{Conclusion}
\label{subsec:measure_conclusion}
On all datasets, there is no significant difference between uncertainty metrics entropy, least confidence, confidence margin, and confidence ratio.
In most cases, the maximum is almost reached after 100 to 150 epochs, after this learning stagnates or improves slowly.
Regarding rewards, all uncertainty metrics have the same curves, but rewards are different between model pairs with \transe and \transe discriminator models.
Since none of the uncertainty metrics achieves a significantly better result than the other ones, we choose standard uncertainty metrics entropy.
Therefore, \ussoftmax with entropy metrics is compared to \origsampling in the next section.