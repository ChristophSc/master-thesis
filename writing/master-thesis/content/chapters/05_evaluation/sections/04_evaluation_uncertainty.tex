\section{Evaluation of Uncertainty Sampling}
\label{ch:evaluation:sec:evaluation_uncertainty}
%
Finally, the best sampling by uncertainty sampling method \ussoftmax with entropy uncertainty measure is evaluated by comparing it to the original sampling method \origsampling of \kbgan.
The evaluation takes place on datasets in \Autoref{subsec:results_uncertainty} and a final Conclusion of uncertainty sampling is given in \Autoref{subsec:uncertainty_conclusion}.
%
\subsection{Results on Datasets} \label{subsec:results_uncertainty}

\subsection{UMLS Dataset}

\subsubsection{Random Sampling}
\begin{figure}
    \centering
    \begin{minipage}{.5\textwidth}
      \centering
      \includegraphics[width=0.9\linewidth]{figures/results/gan_train/not_pretrained/random/umls/gan_train_random_umls_losses.png}
    \end{minipage}%
    \begin{minipage}{.5\textwidth}
      \centering
      \includegraphics[width=0.9\linewidth]{figures/results/gan_train/not_pretrained/random/umls/gan_train_random_umls_rewards.png}
    \end{minipage}
    \begin{minipage}{.5\textwidth}
      \centering
      \includegraphics[width=0.9\linewidth]{figures/results/gan_train/not_pretrained/random/umls/gan_train_random_umls_mrrs.png}
    \end{minipage}%
    \begin{minipage}{.5\textwidth}
      \centering
      \includegraphics[width=0.9\linewidth]{figures/results/gan_train/not_pretrained/random/umls/gan_train_random_umls_hit10s.png}
    \end{minipage}%
    \caption{Adversarial training without Pretraining on \umls dataset. 
    Negative Triples are sampled randomly according to score.
    Shown are the training losses and rewards and validation H@10 and MRRs for 5000 epochs.}
    \label{fig:advtrain_umls_not_pretrained_random}
\end{figure}



\subsubsection{Random Sampling vs. Uncertainty sampling}

\subsubsection{KINSHIP Dataset}
\label{subsubsec:uncertainty_kinship}

\textbf{WN18RR Dataset}
\label{subsubsec:uncertainty_wn18rr} \\
%
In comparison to learning curves on \umls dataset, on \textsc{WN18RR} MRR values increase with \origsampling to about 17.5\% at 100 epochs and continues to increase MRR slightly (\Autoref{fig:advtrain_wn18rr_random_vs_uncertainty}).
Nevertheless, after 1000 epochs MRR reaches almost 19\%.
In the beginning, the learning curve with \ussoftmax look similar to \origsampling, but after reaching an MRR of 17.5\% it almost stays on the same level.
in contrast, Hit@10 values constantly increase for both sampling methods such that a maximum of about 45\% is reached at 1000 epochs.
Therefore, even further learning can be expected with more epochs for both sampling methods.
\clearpage
\begin{figure}[H]
    \centering
    \begin{minipage}{.5\textwidth}
      \centering
      \includegraphics[width=0.9\linewidth]{figures/results/gan_train/not_pretrained/random/wn18rr/epochs1000/random_wn18rr_mrrs.png}
    \end{minipage}%
    \begin{minipage}{.5\textwidth}
      \centering
      \includegraphics[width=0.9\linewidth]{figures/results/gan_train/not_pretrained/uncertainty/max_distribution/entropy/wn18rr/1k_epochs/uncertainty_wn18rr_mrrs.png}
    \end{minipage}
    \begin{minipage}{.5\textwidth}
      \centering
      \includegraphics[width=0.9\linewidth]{figures/results/gan_train/not_pretrained/random/wn18rr/epochs1000/random_wn18rr_hit10.png}
    \end{minipage}%
    \begin{minipage}{.5\textwidth}
      \centering
      \includegraphics[width=0.9\linewidth]{figures/results/gan_train/not_pretrained/uncertainty/max_distribution/entropy/wn18rr/1k_epochs/uncertainty_wn18rr_hit10.png}
    \end{minipage}%
    \caption{Adversarial training on \textsc{WN18RR} dataset. 
    Left Figures show Sampling Negative triples with \origsampling, right Figures show \ussoftmax with entropy uncertainty measure.
    Shown are validation Hit@10 and MRRs for 1000 epochs.}
    \label{fig:advtrain_wn18rr_random_vs_uncertainty}
\end{figure}

If we take a closer look to curves of losses and rewards during training, both both sampling methods losses constantly decrease loss \Autoref{fig:advtrain_wn18rr_losses_rewards}.
Besides this, also the course of rewards look similar, even difference between model pairs with \transe and \transd exists in both sampling methods.
Also the course of rewards look similar, even difference between model pairs with \transe and \transd exists in both sampling methods.
Even if a higher reward is achieved with \origsampling from about 200 epochs, an increase in reward can be seen with both approaches.
Thus, the quality of the sampled negative triples increases with both sampling methods.
\clearpage
\begin{figure}[H]
    \centering
    \begin{minipage}{.5\textwidth}
      \centering
      \includegraphics[width=0.9\linewidth]{figures/results/gan_train/not_pretrained/random/wn18rr/epochs1000/random_wn18rr_losses.png}
    \end{minipage}%
    \begin{minipage}{.5\textwidth}
      \centering
      \includegraphics[width=0.9\linewidth]{figures/results/gan_train/not_pretrained/uncertainty/max_distribution/entropy/wn18rr/1k_epochs/uncertainty_wn18rr_losses.png}
    \end{minipage}
    \begin{minipage}{.5\textwidth}
      \centering
      \includegraphics[width=0.9\linewidth]{figures/results/gan_train/not_pretrained/random/wn18rr/epochs1000/random_wn18rr_rew.png}
    \end{minipage}%
    \begin{minipage}{.5\textwidth}
      \centering
      \includegraphics[width=0.9\linewidth]{figures/results/gan_train/not_pretrained/uncertainty/max_distribution/entropy/wn18rr/1k_epochs/uncertainty_wn18rr_rew.png}
    \end{minipage}%
    \caption{Comparison of training losses and rewards over 1000 epochs on \textsc{WN18RR} dataset.
    Left Figures show losses and rewards of \origsampling and right Figures show losses and rewards of \ussoftmax.}
    \label{fig:advtrain_wn18rr_losses_rewards}
\end{figure}

\input{content/chapters/05_evaluation/sections/04_uncertainty/04_uncertainty_wn18}

\input{content/chapters/05_evaluation/sections/04_uncertainty/05_uncertainty_fb15k237}

\textbf{FB15k Dataset}
\label{subsubsec:uncertainty_fb15k}\\
%

Looking at the results of the evaluation metrics MRR and HIt@10 on the fb15k dataset, both the learning curves and the maximum achieved values look very similar (\Autoref{fig:advtrain_fb15k_random_vs_uncertainty}).
For \usgan, only the model pairs with a \transe discriminator perform slightly worse.
\begin{figure}[H]
    \centering
    \begin{minipage}{.5\textwidth}
      \centering
      \includegraphics[width=0.9\linewidth]{figures/results/gan_train/not_pretrained/random/fb15k/1k_epochs/random_fb15k_mrrs.png}
    \end{minipage}%
    \begin{minipage}{.5\textwidth}
      \centering
      \includegraphics[width=0.9\linewidth]{figures/results/gan_train/not_pretrained/uncertainty/max_distribution/entropy/fb15k/1k_epochs/uncertainty_fb15k_mrrs.png}
    \end{minipage}
    \begin{minipage}{.5\textwidth}
      \centering
      \includegraphics[width=0.9\linewidth]{figures/results/gan_train/not_pretrained/random/fb15k/1k_epochs/random_fb15k_hit10.png}
    \end{minipage}%
    \begin{minipage}{.5\textwidth}
      \centering
      \includegraphics[width=0.9\linewidth]{figures/results/gan_train/not_pretrained/uncertainty/max_distribution/entropy/fb15k/1k_epochs/uncertainty_fb15k_hit10.png}
    \end{minipage}%
    \caption{Adversarial training on \textsc{FB15k} dataset. 
    Left Figures show sampling negative triples with \origsampling, right figures show \ussoftmax with entropy uncertainty metrics.
    Shown are validation Hit@10 and MRRs for 1000 epochs.}
    \label{fig:advtrain_fb15k_random_vs_uncertainty}
\end{figure}

A close look at the development of the rewards shows that for exactly these model pairs the rewards also decrease over time (\Autoref{fig:advtrain_fb15k_losses_rewards}).
Thus, \transe is clearly better able in this case and on this data set to recognize the negative triples and to assign correspondingly high distances to them.
For the other model pairs, in both approaches the reward increases with time and thus the quality of the sampled triples.
\begin{figure}[H]
    \centering
    \begin{minipage}{.5\textwidth}
      \centering
      \includegraphics[width=0.9\linewidth]{figures/results/gan_train/not_pretrained/random/fb15k/1k_epochs/random_fb15k_losses.png}
    \end{minipage}%
    \begin{minipage}{.5\textwidth}
      \centering
      \includegraphics[width=0.9\linewidth]{figures/results/gan_train/not_pretrained/uncertainty/max_distribution/entropy/fb15k/1k_epochs/uncertainty_fb15k_losses.png}
    \end{minipage}
    \begin{minipage}{.5\textwidth}
      \centering
      \includegraphics[width=0.9\linewidth]{figures/results/gan_train/not_pretrained/random/fb15k/1k_epochs/random_fb15k_rew.png}
    \end{minipage}%
    \begin{minipage}{.5\textwidth}
      \centering
      \includegraphics[width=0.9\linewidth]{figures/results/gan_train/not_pretrained/uncertainty/max_distribution/entropy/fb15k/1k_epochs/uncertainty_fb15k_rew.png}
    \end{minipage}%
    \caption{Comparison of training losses and rewards over 1000 epochs on \textsc{FB15k}  dataset.
    Left Figures show losses and rewards of \origsampling and right Figures show losses and rewards of \ussoftmax.}
    \label{fig:advtrain_fb15k_losses_rewards}
\end{figure}

\textbf{YAGO3-10 Dataset}
\label{subsubsec:uncertainty_yago3_10}\\
%
On the \textsc{YAGO3-10} dataset, the course of the learning curve looks relatively similar for both approaches, but the original approach achieves significantly better results for both MRR and Hit@10 values towards the end (\Autoref{fig:advtrain_yago3_10_random_vs_uncertainty}).
After a sharp increase in the values in both approaches, the slope of the learning curve is much steeper in the original \kbgan approach.

Looking at the rewards in \Autoref{fig:advtrain_yago3_10_losses_rewards}, they also decrease for the new sampling approach and thus the sampled negative triples are better in the original approach and improve steadily over time, while in the new approach a decreasing reward is obtained on this data set.
\clearpage
\begin{figure}[H]
    \centering
    \begin{minipage}{.45\textwidth}
      \centering
      \includegraphics[width=0.9\linewidth]{figures/results/gan_train/not_pretrained/random/yago3_10/1k_epochs/random_yago3_10_mrrs.png}
    \end{minipage}%
    \begin{minipage}{.45\textwidth}
      \centering
      \includegraphics[width=0.9\linewidth]{figures/results/gan_train/not_pretrained/uncertainty/max_distribution/entropy/yago3_10/1k_epochs/uncertainty_yago3_10_mrrs.png}
    \end{minipage}
    \begin{minipage}{.45\textwidth}
      \centering
      \includegraphics[width=0.9\linewidth]{figures/results/gan_train/not_pretrained/random/yago3_10/1k_epochs/random_yago3_10_hit10.png}
    \end{minipage}%
    \begin{minipage}{.45\textwidth}
      \centering
      \includegraphics[width=0.9\linewidth]{figures/results/gan_train/not_pretrained/uncertainty/max_distribution/entropy/yago3_10/1k_epochs/uncertainty_yago3_10_hit10.png}
    \end{minipage}%
    \caption{Adversarial training on \textsc{YAGO3-10} dataset. 
    Left Figures show sampling negative triples with \origsampling, right figures show \ussoftmax with entropy uncertainty measure.
    Shown are validation Hit@10 and MRRs for 1000 epochs.}
    \label{fig:advtrain_yago3_10_random_vs_uncertainty}
\end{figure}


\begin{figure}[H]
    \centering
    \begin{minipage}{.45\textwidth}
      \centering
      \includegraphics[width=0.9\linewidth]{figures/results/gan_train/not_pretrained/random/yago3_10/1k_epochs/random_yago3_10_losses.png}
    \end{minipage}%
    \begin{minipage}{.45\textwidth}
      \centering
      \includegraphics[width=0.9\linewidth]{figures/results/gan_train/not_pretrained/uncertainty/max_distribution/entropy/yago3_10/1k_epochs/uncertainty_yago3_10_losses.png}
    \end{minipage}
    \begin{minipage}{.45\textwidth}
      \centering
      \includegraphics[width=0.9\linewidth]{figures/results/gan_train/not_pretrained/random/yago3_10/1k_epochs/random_yago3_10_rew.png}
    \end{minipage}%
    \begin{minipage}{.45\textwidth}
      \centering
      \includegraphics[width=0.9\linewidth]{figures/results/gan_train/not_pretrained/uncertainty/max_distribution/entropy/yago3_10/1k_epochs/uncertainty_yago3_10_rew.png}
    \end{minipage}%
    \caption{Comparison of training losses and rewards over 1000 epochs on \textsc{YAGO3-10}  dataset.
    Left Figures show losses and rewards of \origsampling and right Figures show losses and rewards of \ussoftmax.}
    \label{fig:advtrain_yago3_10_losses_rewards}
\end{figure}

\subsection{Summary and Conclusion of Uncertainty Sampling}
\label{subsec:uncertainty_conclusion}
%
The results of the experiments are listed in tables to compare them to the original approach.
These show both the MRR and Hit@10 results achieved with the original approach \kbgan and those obtained with \usgan with \ussoftmax and entropy uncertainty measure.
\Autoref{tab:result_table1} includes all results for smaller datasets \umls and \kinship,
\Autoref{tab:result_table2} lists all results for \textsc{WN18RR} and \textsc{WN18}, and \Autoref{tab:result_table2} depicts results for \textsc{FB15k-237}, \textsc{FB15k} and \textsc{YAGO3-10} datasets.

\begin{table}[h]
    \centering
    \begin{tabular}{lllllll}
        \toprule
        \textbf{Method} &
        \multicolumn{2}{c}{\textbf{UMLS}} & 
        \multicolumn{2}{c}{\textbf{KINSHIP}} & 
        \multicolumn{2}{c}{\textbf{FBK15}}\\
        
        \cmidrule{2-6} \cmidrule{7-7} \\
        {} & MRR & H@10 & MRR & H@10 & MRR & H@10 \\
        
        \midrule

        \textsc{DistMult}  
        & 69.7 & 89.1 & 50.5 & 86.9 & 45.1 & 71.9\\
        
        \textsc{ComplEx}   
        & 85.6 & 97.7 & 77.7 & 96.7 & 45.5 & 72.8\\
        
        \textsc{TransE}    
        & 77.0 & 98.4 & 21.9 & 57.7 & 33.3 & 55.5 \\
        
        \textsc{TransD}    
        & 79.6 & 98.7 & 27.2 & 66.2 & 33.9 & 56.4 \\ 

        \midrule
        
        \textbf{Pre-trained  (5k epochs)}
        & & & & \\
        
        \kbgan (\textsc{DistMult} + \textsc{TransE})  
        & 62.9 & 81.5 & 13.3 & 29.2 & 43.2 & 66.0 \\
        
        \kbgan (\textsc{ComplEx} + \textsc{TransE})   
        & 63.9  & 94.5 & 6.7 & 13.9 & 43.4 & 66.2\\
        
        \kbgan (\textsc{DistMult} + \textsc{TransD})  
        & 67.4 & 81.2 & 11.9 & 24.9 & 43.7 & 67.1\\
        
        \kbgan (\textsc{ComplEx} + \textsc{TransD})   
        & 61.8 & 91.2 & 9.7 & 22.6 & 43.8 & 67.2\\

        \midrule
         
        \usgan (\textsc{DistMult} + \textsc{TransE}) 
        & 77.8 & 98.4 & 21.9 & 57.8 & 33.7 & 56.3\\
         
        \usgan (\textsc{ComplEx} + \textsc{TransE}) 
        & 78.4 & 98.2 & 22.0 & 58.8 & 33.6 & 56.2\\
          
        \usgan (\textsc{DistMult} + \textsc{TransD}) 
        & 79.8 & 99.2 & 34.4 & 57.4 & 34.4 & 57.4\\
        
        \usgan (\textsc{ComplEx} + \textsc{TransD}) 
        & 80.8  & 98.9 & 27.4 & 66.8 & 34.4 & 57.1\\
          
        \midrule
        
        \textbf{Not pre-trained  (1k epochs)}
        & & & & \\
        
        \kbgan (\textsc{DistMult} + \textsc{TransE})  
        & 72.7 & 94.0 & 24.0 & 50.1 & 42.1 & 65.7\\
        
        \kbgan (\textsc{ComplEx} + \textsc{TransE})   
        & 72.4 & 95.1 & 26.2 & 52.5 & 39.8 & 64.8\\
        
        \kbgan (\textsc{DistMult} + \textsc{TransD})  
        & 73.8 & 95.7 & 27.7 & 55.8 & 39.8 & 65.2\\
        
       \kbgan (\textsc{ComplEx} + \textsc{TransD})   
        & 71.4 & 94.8 & 24.0 & 51.1 & 41.4 & 66.3\\
        
         \midrule
         
        \usgan (\textsc{DistMult} + \textsc{TransE}) 
         & 78.4 & 98.7 & 22.7 & 56.7 & 33.2 & 55.2\\
         
        \usgan (\textsc{ComplEx} + \textsc{TransE}) 
          & 79.5  & 99.0 & 23.2 & 56.8 & 33.3 & 55.5\\
          
        \usgan (\textsc{DistMult} + \textsc{TransD}) 
         & 80.3 & 99.0 & 25.3 & 65.4 & 43.1 & 66.5\\
        
        \usgan (\textsc{ComplEx} + \textsc{TransD}) 
          & 80.6  & 99.1 & 27.9 & 66.5 & 40.1 & 65.8\\
        \bottomrule
    \end{tabular}
    \caption{Result table for smaller Datasets \textsc{UMLS}, \textsc{KINSHIP} and \textsc{FB15k}.
    First section includes results for different embedding models achieved with pre-training.
    Second section is the original \kbgan approach with random sampling and different embedding models. Since no results were given in paper of \kbgan \cite{cai2017kbgan}, these results were generated independently.
    Third section includes results achieved by replacing Random Sampling with Sampling by Uncertainty, which is called \usgan at this point.}
\label{tab:results_small_datasets}
\end{table}

In some cases such as training on \textsc{FB15k-237} adversarial training with \ussoftmax seems to work and on small datasets such as \umls even better than \origsampling.
Additionally, it is noticeable that training looks very different for different datasets.
Nevertheless, in most cases, adversarial training with \origsampling still performs better and reaches higher MRR and Hit@10 values.
Therefore, measuring and sampling by the uncertainty of the generator to classify triples as positives and negatives does not necessarily imply the uncertainty of the discriminator model.
Even if the generator model might be uncertain if a given triple is positive or negative, the discriminator can be more certain about this classification.
\clearpage

\begin{table}[H]
    \centering
    \begin{tabular}{lllll}
        \toprule
        \textbf{Method} &
        \multicolumn{2}{c}{\textbf{WN18RR}} & 
        \multicolumn{2}{c}{\textbf{WN18}}\\
        
        \cmidrule{2-4} \cmidrule{5-5} \\
        {} & MRR & H@10 & MRR & H@10\\
        
        \midrule
         
         \textbf{\kbgan} 
         & & & &\\
         
         \textsc{DistMult} + \textsc{TransE}
          & \underline{18.6} 
          & \underline{44.5} 
          & \underline{59.7} 
          & \underline{94.7} 
         \\
          
          \textsc{ComplEx} + \textsc{TransE} 
          & \underline{18.3} 
          & \underline{44.7}
          & \underline{58.4} 
          & \underline{94.6} 
           \\
          
          \textsc{DistMult} + \textsc{TransD}  
          & \underline{70.8} 
          & \underline{94.3} 
          & \underline{18.7} 
          & \underline{43.7} \\

          \textsc{ComplEx} + \textsc{TransD}
          & \underline{25.4} 
          & \underline{43.8}
          & \underline{64.1}
          & \underline{94.3}
           \\
          
          \midrule
          
          \textbf{\usgan} 
          & & & &  \\
         
          \textsc{DistMult} + \textsc{TransE}
           & 17.8 
          & 44.0
          & 42.0 
          & 91.8 
         \\
         
          \textsc{ComplEx} + \textsc{TransE}
           & 17.0 
          & 43.9
          & 42.0 
          & 91.6 
          \\
          
          \textsc{DistMult} + \textsc{TransD}
           & 17.9 
          & 42.2
          & 44.6 
          & 92.3
          \\
        
         \textsc{ComplEx} + \textsc{TransD}
          & 17.2 
          & 43.5
          & 44.8 
          & 92.3 
          \\
        \bottomrule
    \end{tabular}
    \caption{Experimental results on \textsc{WN18RR} and \textsc{WN18} dataset.
    The first section contains MRR and Hit@10 values of adversarial training with original \kbgan approach.
    The second section shows results achieved with new approach \usgan with \ussoftmax and entropy uncertainty measure.
    The generator and the discriminator are not pre-trained.}
\label{tab:result_table2}
\end{table}

Another reason might be that classification is not accurate enough.
It was observed that even if negative triples scoring ranges are lower than positive triple scoring ranges at the beginning of training, this sometimes changes over training time.
Therefore, the overlap of both ranges increases.
In addition, uncertainty is currently only measured based on the generator's uncertainty score and does not include any further information about the \ac{KG}, its relations, and entities.
\clearpage

\begin{table}[H]
    \centering
    \begin{tabular}{lllllll}
        \toprule
        \textbf{Method} &
        \multicolumn{2}{c}{\textbf{FB15k-237}} &
        \multicolumn{2}{c}{\textbf{FB15k}} &
        \multicolumn{2}{c}{\textbf{YAGO3-10}} \\
        
        \cmidrule{2-6} \cmidrule{7-7} \\
        {} & MRR & H@10 & MRR & H@10 & MRR & H@10\\
        
        \midrule
         
         \textbf{\kbgan} 
         & & & & & & \\
         
         \textsc{DistMult} + \textsc{TransE}
          & \underline{25.3} 
          & \underline{43.1}         
            & \underline{42.1} 
            & \underline{65.7}
            & \underline{18.7}
            & \underline{35.6}\\
          
          \textsc{ComplEx} + \textsc{TransE} 
          & \underline{25.3} 
          & \underline{43.2}  
            & \underline{39.8} 
            & \underline{64.8}
            & \underline{17.7}
            & \underline{35.0}\\
          
          \textsc{DistMult} + \textsc{TransD}  
          & \underline{25.7} 
          & \underline{43.5} 
            & 39.8 
            & 65.2
            & \underline{15.5}
            & \underline{33.0}\\

          \textsc{ComplEx} + \textsc{TransD}
          & \underline{25.4} 
          & \underline{43.8}
            & \underline{41.4} 
            & \underline{66.3}
            & \underline{17.4}
            & \underline{34.9}\\
          
          \midrule
          
          \textbf{\usgan} 
          & & & & & & \\
         
          \textsc{DistMult} + \textsc{TransE}
          & 23.4  
          & 40.9
             & 33.2 
             & 55.2
             & 11.0
             & 23.0\\
         
         
          \textsc{ComplEx} + \textsc{TransE}
          & 23.2  
          & 40.8 
             & 33.3 
             & 55.5
             & 11.2
             & 23.5\\
          
          \textsc{DistMult} + \textsc{TransD}
          & 23.3 
          & 40.9  
           & \underline{43.1} 
           & \underline{66.5}
           & 11.1
           & 23.8\\
        
         \textsc{ComplEx} + \textsc{TransD}
          & 23.0  
          & 41.1  
             & 40.1 
             & 65.8
             & 11.2
             & 22.9\\
          
        \bottomrule
    \end{tabular}
    \caption{Experimental results on \textsc{FB15k-237}, \textsc{FB15k} and \textsc{YAGO3-10} dataset.
    First section contains MRR and Hit@10 values of adversarial training with original \kbgan approach.
    The second section shows results achieved with new approach \usgan with \ussoftmax and entropy uncertainty measure.
    The generator and the discriminator are not pre-trained.}
\label{tab:result_table3}
\end{table}


% scores of positive and neative triples sometimes has a high overlap which leads sampling of negative triples with an average score
% hyperparameter optimization necessary for all datasets
% Models are very dependent on hyperparameters, small changes can already have large effects on learning behavior.
% Depending on the properties of a data set, the selected model pair for generator and discriminator and the set hyperparameters, a completely different training behavior can occur.