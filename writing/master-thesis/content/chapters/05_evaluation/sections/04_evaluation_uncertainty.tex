\section{Evaluation of Uncertainty Sampling}
\label{ch:evaluation:sec:evaluation_uncertainty}
%
Finally, the best sampling by uncertainty sampling method \ussoftmax with entropy uncertainty metric is evaluated by comparing it to the original sampling method \origsampling of \kbgan.
The evaluation takes place on datasets in \Autoref{subsec:results_uncertainty} and a final Conclusion of uncertainty sampling is given in \Autoref{subsec:uncertainty_conclusion}.
%
\subsection{Results on Datasets} \label{subsec:results_uncertainty}
\subsection{UMLS Dataset}

\subsubsection{Random Sampling}
\begin{figure}
    \centering
    \begin{minipage}{.5\textwidth}
      \centering
      \includegraphics[width=0.9\linewidth]{figures/results/gan_train/not_pretrained/random/umls/gan_train_random_umls_losses.png}
    \end{minipage}%
    \begin{minipage}{.5\textwidth}
      \centering
      \includegraphics[width=0.9\linewidth]{figures/results/gan_train/not_pretrained/random/umls/gan_train_random_umls_rewards.png}
    \end{minipage}
    \begin{minipage}{.5\textwidth}
      \centering
      \includegraphics[width=0.9\linewidth]{figures/results/gan_train/not_pretrained/random/umls/gan_train_random_umls_mrrs.png}
    \end{minipage}%
    \begin{minipage}{.5\textwidth}
      \centering
      \includegraphics[width=0.9\linewidth]{figures/results/gan_train/not_pretrained/random/umls/gan_train_random_umls_hit10s.png}
    \end{minipage}%
    \caption{Adversarial training without Pretraining on \umls dataset. 
    Negative Triples are sampled randomly according to score.
    Shown are the training losses and rewards and validation H@10 and MRRs for 5000 epochs.}
    \label{fig:advtrain_umls_not_pretrained_random}
\end{figure}



\subsubsection{Random Sampling vs. Uncertainty sampling}
\subsection{WN18RR Dataset}

\subsubsection{Random Sampling vs Best Uncertainty Sampling}


\begin{figure}
    \centering
    \begin{minipage}{.5\textwidth}
      \centering
      \includegraphics[width=0.9\linewidth]{figures/results/gan_train/not_pretrained/random/wn18rr/gan_train_random_wn18rr_mrrs.png}
    \end{minipage}%
    \begin{minipage}{.5\textwidth}
      \centering
      \includegraphics[width=0.9\linewidth]{figures/results/gan_train/not_pretrained/uncertainty/max_distribution/entropy/wn18rr/gan_train_uncertainty_wn18rr_mrrs.png}
    \end{minipage}
    \begin{minipage}{.5\textwidth}
      \centering
      \includegraphics[width=0.9\linewidth]{figures/results/gan_train/not_pretrained/random/wn18rr/gan_train_random_wn18rr_hit10s.png}
    \end{minipage}%
    \begin{minipage}{.5\textwidth}
      \centering
      \includegraphics[width=0.9\linewidth]{figures/results/gan_train/not_pretrained/uncertainty/max_distribution/entropy/wn18rr/gan_train_uncertainty_wn18rr_hit10s.png}
    \end{minipage}%
    \caption{Adversarial training on \textsc{WN18RR} dataset. 
    Left Figures show Sampling Negative triples with original Random Sampling, right Figures show Best Uncertainty Sampling (Uncertainty Distribution Entropy).
    Shown are validation H@10 and MRRs for 5000 epochs.}
    \label{fig:advtrain_wn18rr_random_vs_uncertainty}
\end{figure}
\subsection{WN18 Dataset}

\subsubsection{Random Sampling vs Best Uncertainty Sampling}


\begin{figure}
    \centering
    \begin{minipage}{.5\textwidth}
      \centering
      \includegraphics[width=0.9\linewidth]{figures/results/gan_train/not_pretrained/random/wn18/gan_train_random_wn18_mrrs.png}
    \end{minipage}%
    \begin{minipage}{.5\textwidth}
      \centering
      \includegraphics[width=0.9\linewidth]{figures/results/gan_train/not_pretrained/uncertainty/max_distribution/entropy/wn18/gan_train_uncertainty_wn18_mrrs.png}
    \end{minipage}
    \begin{minipage}{.5\textwidth}
      \centering
      \includegraphics[width=0.9\linewidth]{figures/results/gan_train/not_pretrained/random/wn18/gan_train_random_wn18_hit10s.png}
    \end{minipage}%
    \begin{minipage}{.5\textwidth}
      \centering
      \includegraphics[width=0.9\linewidth]{figures/results/gan_train/not_pretrained/uncertainty/max_distribution/entropy/wn18/gan_train_uncertainty_wn18_hit10s.png}
    \end{minipage}%
    \caption{Adversarial training on \textsc{WN18} dataset. 
    Left Figures show Sampling Negative triples with original Random Sampling, right Figures show Best Uncertainty Sampling (Uncertainty Distribution Entropy).
    Shown are validation H@10 and MRRs for 5000 epochs.}
    \label{fig:advtrain_wn18_random_vs_uncertainty}
\end{figure}
\subsection{FB15k237 Dataset}

\subsubsection{Random Sampling vs Best Uncertainty Sampling}


\begin{figure}
    \centering
    \begin{minipage}{.5\textwidth}
      \centering
      \includegraphics[width=0.9\linewidth]{figures/results/gan_train/not_pretrained/random/fb15k237/gan_train_random_fb15k237_mrrs.png}
    \end{minipage}%
    \begin{minipage}{.5\textwidth}
      \centering
      \includegraphics[width=0.9\linewidth]{figures/results/gan_train/not_pretrained/uncertainty/max_distribution/entropy/fb15k237/gan_train_uncertainty_fb15k237_mrrs.png}
    \end{minipage}
    \begin{minipage}{.5\textwidth}
      \centering
      \includegraphics[width=0.9\linewidth]{figures/results/gan_train/not_pretrained/random/fb15k237/gan_train_random_fb15k237_hit10s.png}
    \end{minipage}%
    \begin{minipage}{.5\textwidth}
      \centering
      \includegraphics[width=0.9\linewidth]{figures/results/gan_train/not_pretrained/uncertainty/max_distribution/entropy/fb15k237/gan_train_uncertainty_fb15k237_hit10s.png}
    \end{minipage}%
    \caption{Adversarial training on \textsc{FB15k237} dataset. 
    Left Figures show Sampling Negative triples with original Random Sampling, right Figures show Best Uncertainty Sampling (Uncertainty Distribution Entropy).
    Shown are validation H@10 and MRRs for 5000 epochs.}
    \label{fig:advtrain_fb15k237_random_vs_uncertainty}
\end{figure}
%
Results of our experiments on smaller datasets \umls and \kinship  are listed in \Autoref{tab:results_small_datasets} and bigger datasets \textsc{FB15k-237},  \textsc{WN18} and \textsc{WN18RR} are listed in \Autoref{tab:results}.
\begin{table}[h]
    \centering
    \begin{tabular}{lllll}
        \toprule
        \multirow{Method} &
        \multicolumn{2}{c}{\textbf{UMLS}} & 
        \multicolumn{2}{c}{\textbf{KINSHIP}}\\
        
        \cmidrule{2-4} \cmidrule{5-5} \\
        {} & MRR & H@10 & MRR & H@10 \\
        
        \midrule
        

        \textsc{DistMult}  
        & 69.7 & 89.1 & - & - \\
        
        \textsc{ComplEx}   
        & 85.6 & 97.7 & - & - \\
        
        \textsc{TransE}    
        & 77.0 & 98.4 & - & -  \\
        
        \textsc{TransD}    
        & 79.6 & 98.7 & - & - \\ 
        

        \midrule
        
        \textbf{Pre-trained  (5k epochs)}
        & & &  &  \\
        
        \kbgan (\textsc{DistMult} + \textsc{TransE})  
        & 62.9 & 81.5 & - & - \\
        
        \kbgan (\textsc{ComplEx} + \textsc{TransE})   
        & 63.9  & 94.5 & - & - \\
        
        \kbgan (\textsc{DistMult} + \textsc{TransD})  
        & 67.4 & 81.2 & - & -\\
        
        \kbgan (\textsc{ComplEx} + \textsc{TransD})   
        & 61.8 & 91.2 & - & -\\

        \midrule
         
        \ucgan (\textsc{DistMult} + \textsc{TransE}) 
        & 77.8 & 98.4 & - & - \\
         
        \ucgan (\textsc{ComplEx} + \textsc{TransE}) 
        & 78.4 & 98.2 & - & -\\
          
        \ucgan (\textsc{DistMult} + \textsc{TransD}) 
        & 79.8 & 99.2 & - & - \\
        
        \ucgan (\textsc{ComplEx} + \textsc{TransD}) 
        & 80.8  & 98.9 & - & - \\
          
          
        \midrule
        
        \textbf{Not pre-trained  (1k epochs)}
        & & & &  \\
        
        \kbgan (\textsc{DistMult} + \textsc{TransE})  
        & 72.7 & 94.0 & - & - \\
        
        \kbgan (\textsc{ComplEx} + \textsc{TransE})   
        & 72.4 & 95.1 & - & - \\
        
        \kbgan (\textsc{DistMult} + \textsc{TransD})  
        & 73.8 & 95.7 & - & -\\
        
       \kbgan (\textsc{ComplEx} + \textsc{TransD})   
        & 71.4 & 94.8 & - & -\\
        
         \midrule
         
        \ucgan (\textsc{DistMult} + \textsc{TransE}) 
         & 78.4 & 98.7 & - & - \\
         
        \ucgan (\textsc{ComplEx} + \textsc{TransE}) 
          & 79.5  & 99.0 & - & -\\
          
        \ucgan (\textsc{DistMult} + \textsc{TransD}) 
         & 80.3 & 99.0 & - & - \\
        
        \ucgan (\textsc{ComplEx} + \textsc{TransD}) 
          & 80.6  & 99.1 & - & - \\
          
          
        \bottomrule
    \end{tabular}
    \caption{Result table for smaller Datasets \textsc{UMLS} and \textsc{KINSHIP}.
    First section includes results for different embedding models achieved with pre-training.
    Second section is the original \kbgan approach with random sampling and different embedding models. Since no results were given in paper of \kbgan \cite{cai2017kbgan}, these results were generated independently.
    Third section includes results achieved by replacing Random Sampling with Sampling by Uncertainty, which is called \ucgan at this point.}
\label{tab:results_small_datasets}
\end{table}
\begin{table}[h]
    \centering
    \begin{tabular}{lllllll}
        \toprule
        \textbf{Method} &
        \multicolumn{2}{c}{\textbf{FB15k-237}} & 
        \multicolumn{2}{c}{\textbf{WN18}} & 
        \multicolumn{2}{c}{\textbf{WN18RR}}\\
        
        \cmidrule{2-6} \cmidrule{7-7} \\
        {} & MRR & H@10 & MRR & H@10 & MRR & H@10 \\
        
        \midrule
         
          \textsc{kbgan} (\textsc{DistMult} + \textsc{TransE})  
          & 25.3 & 43.1 & 59.7 & 94.7 & 18.6 & 44.5 \\
          
          \textsc{kbgan} (\textsc{ComplEx} + \textsc{TransE})  
          & 25.3 & 43.2 & 58.4 & 94.6 & 18.3 & 44.7 \\
          
          \textsc{kbgan} (\textsc{DistMult} + \textsc{TransD})  
          & 25.7 & 43.5 & 70.8 & 94.3 & 18.7 & 43.7 \\

          \textsc{kbgan} (\textsc{ComplEx} + \textsc{TransD})  
          & 25.4 & 43.8 & 64.1 & 94.3 & 25.4 & 43.8 \\
          
          \midrule
          
          \usgan (\textsc{DistMult} + \textsc{TransE}) 
          & 23.4  & 40.9 & 42.0 & 91.8 & 17.8 & 44.0\\
         
         \usgan (\textsc{ComplEx} + \textsc{TransE}) 
          & 23.2  & 40.8 & 42.0 & 91.6 & 17.0 & 43.9 \\
          
          \usgan (\textsc{DistMult} + \textsc{TransD}) 
          & 23.3 & 40.9 & 44.6 & 92.3 & 17.9 & 42.2 \\
        
        \usgan (\textsc{ComplEx} + \textsc{TransD}) 
          & 23.0  & 41.1 & 44.8 & 92.3 & 17.2 & 43.5 \\
          
        \bottomrule
    \end{tabular}
    \caption{Experimental results on \textsc{FB15k-237}, \textsc{WN18} and \textsc{WN18RR} dataset.
    First section shows best achieved MRR and Hit@10 values after 1000 epochs of pre-training.
    The second section contains results of adversarial training with original \kbgan approach and last section with \usgan.
    Models are not pre-trained. 
    The results of the pre-training are only shown for better comparability.
    Third section lists our results achieved with original \kbgan on different datasets.}
\label{tab:results}
\end{table}
These tables list both the results given in the \kbgan paper \cite{cai2017kbgan} and our achieved MRR and Hit@10 values on the different datasets.
Furthermore, results after 5000 epochs on pre-trained models and also results after 1000 epochs with non-pre-trained models are listed.
The model named first in the brackets is the generator and the model named second is the discriminator.


\subsection{Conclusion}
\label{subsec:uncertainty_conclusion}
%
In some cases like training on \textsc{FB15k-237} adversarial training with \ussoftmax seems to work and on small datasets like \umls even better than \origsampling.
Additionally, it is noticeable that training looks very different for different datasets.
Nevertheless, in most cases, adversarial training with \origsampling still performs better and reaches higher MRR and Hit@10 values.
Therefore, measuring and sampling by the uncertainty of the generator to classify triples as positives and negatives does not necessarily imply the uncertainty of the discriminator model.
Even if the generator model might be uncertain if a given triple is positive or negative, the discriminator can be more certain about this classification.
Another reason might be that classification is not accurate enough.
During training, we have observed that even if negative triples scoring ranges are lower than positive triple scoring ranges, this sometimes changes over training time.
Therefore, the overlap of both ranges increases.
In addition, uncertainty is currently only measured based on the generator's uncertainty score and does not include any further information about the \ac{KG}, its relations, and entities.