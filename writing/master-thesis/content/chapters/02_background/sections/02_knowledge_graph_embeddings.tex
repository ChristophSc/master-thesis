\section{Knowledge Graph Embeddings} 
\label{sec:knowledge_graph_embeddings}


\acp{KGE} are low-dimensional representations of entities and relations in a \ac{KG}. 
Numerous methods have been developed in the last few years to tackle various problems such as defining a different score function to measure the distance between entities relative to their relation.
Depending on the dimensionality of the embedding space and which types of relations should be embedded, a different embedding can be chosen.
While some of them support only symmetric relations, others support antisymmetric ones as well. 
Besides this, the types of relations in a \ac{KG} differ from each other.
We can differentiate between 1-to-1, 1-to-N, N-to-1, and N-to-N relations.
Overall, all the embedding methods can be separated by three different aspects \cite{electronics9050750}:
(i) How they represent entities and relations, (ii) how they define the scoring function and (iii) how they optimize the ranking criterion.
On the basis of this, two different categories are derived:
\textit{Triplet Fact-Based Representation Learning Models} \Autoref{subsec:triplet_fact_based_representation_learning_models} and \textit{Description-Based Representation Learning Models} \Autoref{subsec:description_based_representation_learning_models} \cite{electronics9050750}.

\subsection{Triplet Fact-Based Representation Learning Models} 
\label{subsec:triplet_fact_based_representation_learning_models}

Triplet Fact-Based Representation Learning Models are separated into three groups:
\begin{enumerate}
    \item 
    \textbf{Translation-Based Models} are based on word embedding algorithms and inspired by \textit{word2vec} \cite{electronics9050750}.
    Examples for \ac{KGE} models of this group are \textsc{TransE} \cite{TransE}, \textsc{TransH} \cite{TransH}, \textsc{TransR} \cite{TransR}, \textsc{TransD} \cite{TransD}
    and 
    \textsc{RotatE} \cite{RotatE}.
    The basic idea of \transe is that a functional relation \texttt{r} corresponds to a translation of embeddings, i.e. $h + r \thickapprox t$ when \triple{h}{r}{t} holds.
    If it does not hold, \texttt{t} should be far away from $h + r$ which lead to a energy-based framework $d(h+r, t)$ for some dissimilarity measure $d$ which is the $L_1$ or $L_2$ norm \cite{TransE}.
    The idea of \transe is depicted in \Autoref{fig:translationbasedmodels} (a).
    \begin{figure}[H]
      \centering
        \includegraphics[width=0.95\textwidth]{figures/Transe+TransD.pdf}
      \caption{Illustrations of (a) \transe and (b) \transd (based on: \cite{electronics9050750}).}
      \label{fig:translationbasedmodels}
    \end{figure}
    Therefore, head entity \texttt{h}, tail entity \texttt{t} and relation \texttt{r} are in the same vector space and the model tries to create the embeddings such that $h+r$ map to \texttt{t} as closely as possible.
    This leads to the scoring function
    \begin{equation}
        f_r(h,r,t) = || h + r - t ||_{l_1, l_2}
        \label{eq:transescoringfunction}
    \end{equation}
    \transd on the other hand works with two vectors for each entity and relation.
    The first one captures the meaning and the second one is used to construct the mapping matrices \cite{TransD}.
    It is illustrated in \Autoref{fig:translationbasedmodels} (b).
    
    \item 
    \textbf{Tensor Factorization-Based Models} transform triple facts into a 3D binary tensor $\mathcal{X} \in \mathbb{R}^{n \times n \times m}$ where $n$ and $m$ are the number of entities and relations \cite{electronics9050750}.
    Examples of this group are \textsc{RESCAL} \cite{RESCAL}, \textsc{DistMult} \cite{DistMult}, \textsc{ComplEx} \cite{ComplEx}, and \textsc{HolE} \cite{HolE}.
    \distmult simplifies the computational complexity of \rescal and restricts matrix $M_r \in \mathbb{R}^{d \times d}$, which is a matrix associated with the relation, to be a diagonal matrix, i.e. $M_r = diag(r), r \in \mathbb{R}^d$ \cite{electronics9050750}. 
    Therefore, its scoring function is transformed to
    \begin{equation}
        f_r(h,r) = h^{\top}diag(r)t\label{eq:distmultscoringfunction}
    \end{equation}
    Since \distmult is not able to model asymmetric relations, \complex \cite{ComplEx} extends \distmult by complex-valued embeddings \cite{electronics9050750} which leads to the scoring function
    \begin{equation}
        f_r(h,r) = Re(h^{\top}diag(r)\bar{t})
        \label{eq:complexscoringfunction}
    \end{equation}
    where $Re(\cdot)$ denotes the real part of a complex value and $\bar{t}$ is the complex conjugate of tail t \cite{electronics9050750}.
    
    \item 
    \textbf{Neural Network-Based Models} use deep learning and its representation and generalization capabilities to embed a knowledge graph by a neural network \cite{electronics9050750}.
    \textsc{ConvE} \cite{ConvE}, \textsc{HypER} \cite{HypER}, \textsc{ConEx} \cite{ConEx}, \textsc{ConvQ} and  \textsc{ConvO} \cite{demir2021convolutional} are examples for this group.
    
\end{enumerate}

\subsection{Description-Based Representation Learning Models}
\label{subsec:description_based_representation_learning_models}

Description-Based Representation Learning Models can also be broken down into three further categories:
\begin{enumerate}
    \item 
    \textbf{Textual Description-Based Models} is an extension of the traditional triplet-based model and integrates additional textual information for entities to evolve its performance.
    Examples are \ac{TKRL} \cite{TKRL} and \ac{TEKE} \cite{TEKE}.
    
    \item 
    \textbf{Relation Path-Based Models} refines the performance of an embedding model by multi-step relational paths, which reveal semantic relations between the entities.
    One example is \ac{PTransE} \cite{PTransE}.
    
     \item 
    \textbf{Other Models} that cannot be assigned to the other two categories and pay attention to further triple information such as temporal aspects.
\end{enumerate}



