\section{Uncertainty Sampling} 
\label{sec:uncertaintysampling}
%- or: select examples based on informativeness, diversity and density criteria
%- motivation behind uncertainty sampling: find some unlabeled examples near decision boundaries and use them to clarify the position of decision boundaries -> most informative instances
%- use density to determine whether an unlabeled example is highly representative
The use of uncertainty has already led to promising improvements in many areas of machine learning.
The term of uncertinty sampling is most often associated with the active learning, where labeled data for training a supervised model is obtained from a dataset of unlabeled instances.
Based on the informativeness of the unlabeled instances for the learning algorithm, a prioritization results in which they are labeled.
It is used in machine learning approaches, where unlabeled data is abundant, but it is difficult, time-consuming, or expensive to obtain labeled data \cite{Settles2009ActiveLL}.
Active learning aims for greater accuracy with fewer labeled training instances \cite{Settles2009ActiveLL}.
These selected unlabeled instances can either be generated de novo or sampled from a given distribution.
In the literature several query strategies have been proposed with different approaches how to receive informative instances, one of them is uncertainty sampling \cite{Settles2009ActiveLL}.
Other query strategies for Active Learning are Query-By-Committee, Expected Model Change, Variance Reduction, Fisher Information Ration, Estimated Error Reduction and Density-Weighted Methods \cite{Settles2009ActiveLL}.
They provide different strategies to obtain informative instances from the unlabeled dataset like voting of a committee consisting of several trained models, querying the instance that would impact the greatest change to the current model if we knew its label or queries instances which minimize the learner’s future error by minimizing its variance.

In Uncertainty Sampling, given a model $\theta$ which has been trained on labeled dataset $D$, each instance $x_j$ of the unlabeled data pool $U$ will be assigned a utility score $s(\theta, x_j)$.
Subsequently, the instance with the highest score will be sampled.
Most common used  uncertainty measures for utility score include \cite{human-in-the-loop}:
\begin{equation}
    \text{Entropy-based:}\hspace{15mm}
     s(\theta, x) = - \sum_{y \in \mathcal{Y}}{p_{\theta}(y | x) \cdot log p_{\theta}(y|x)}
\end{equation}
\begin{equation}
    \text{Least confidence:}\hspace{27mm}
     s(\theta, x) = 1 - \max_{y \in \mathcal{Y}}{p_{\theta}(y | x)}
\end{equation}
\begin{equation}
    \text{Margin of confidence:}\hspace{12mm}
    s(\theta, x) = p_{\theta}(y_m | x) - p_{\theta}(y_n|x)
\end{equation}
\begin{equation}
    \text{Ratio of confidence:}\hspace{33mm}
    s(\theta, x) = \frac{p_{\theta}(y_m | x)}{p_{\theta}(y_n|x)}
\end{equation}
where
$y_m = \argmax_{y \in \mathcal{Y}} p_{\theta}(y | x)$ 
and 
$y_n = \argmax_{y \in \mathcal{Y} \setminus y_m}{p_{\theta}(y | x)}$.

\Autoref{fig:uncertainty_sampling_heatmap} illustrates an example of target areas for the different uncertainty measures when there are three labels.
\begin{figure*}[t]
  \centering
    \includegraphics[width=0.60\textwidth]{figures/uncertainty_sampling_heatmap.PNG}
  \caption{Heat map of for a multilabel-classification with three labels for all four uncertainty measures (obtained from \cite{human-in-the-loop}).
  Each dot is an instance with a different label.}
  \label{fig:uncertainty_sampling_heatmap}
\end{figure*}
This shows, for example, that entropy measure samples a much larger area than least confidence.
The more different labels there are, the greater the difference between the different labels.

In addition, the following three different frameworks for measuring the uncertainty of a learner can be separated \cite{nguyen2021howtomeasure}.


\subsection{Evidence-based Uncertainty} 
\label{subsec:evidence_based_uncertainty}
%
\Ac{EBU} differentiates between uncertainty due to conflicting evidence and insufficient evidence \cite{nguyen2021howtomeasure}.
\ac{EBU} checks the influence of individual features $x^m$ in a feature representation 
$x = (x^1,..., x^d)$ of instances \cite{nguyen2021howtomeasure}.
For a given model $\theta$, the class-conditional probabilities, denoted as $p_{\theta}(x^m |0)$ and $p_{\theta}(x^m | 1)$ for binary classes 0 and 1, the values of the m-th feature for a given instance x are considered.
Subsequently, the set of features is partitioned into those that provide evidence for a positive class
\begin{equation}
    P_{\theta}(x) = \bigg\{ x^m \bigg| \frac{p_{\theta}(x^m|1)}{p_{\theta}(x^m|0)} > 1 \bigg\} 
\end{equation}
and those that provide evidence for a negative class \cite{nguyen2021howtomeasure}:
 \begin{equation}
    N_{\theta}(x) = \bigg\{ x^m \bigg| \frac{p_{\theta}(x^m|0)}{p_{\theta}(x^m|1)} > 1 \bigg\} 
\end{equation}
Afterward, either the instance with the highest conflicting evidence (\ac{CEU}) or where both evidences are low (\ac{IEU}) are queried (insufficient evidence).
The total evidence for positive class $E_1$ 
\begin{equation}
    E_1(x) = \prod\limits_{x^m \in P_{\theta}(x)} \frac{p_{\theta}(x^m|1)}{p_{\theta}(x^m|0)}
\end{equation}
and negative class $E_0$
\begin{equation}
    E_0(x) = \prod\limits_{x^m \in N_{\theta}(x)} \frac{p_{\theta}(x^m|0)}{p_{\theta}(x^m|1)}.
\end{equation} 
is determined \cite{nguyen2021howtomeasure}.
Therefore, conflicting evidence is present if $E_1$ and $E_0$ are high.
Likewise, if both $E_1$ and $E_0$ are low uncertainty because of insufficient evidence is measured.


\subsection{Credal Uncertainty}  
\label{subsec:credal_uncertainty}

\Ac{CU} seeks to differentiate between the reducible and irreducible part of uncertainty in a prediction.
To achieve this a credal set of models denoted by $C \subseteq \Theta$ with plausible candidate models is defined \cite{nguyen2021howtomeasure}.
With credal uncertainty sampling an instance x is sampled, if it has the least evidence for the dominance of one of the classes \cite{nguyen2021howtomeasure}.
A class $y'$ is dominated by another class $y$ if $y$ is more likely than $y'$ for each distribution in the credal set $C$ \cite{nguyen2021howtomeasure}:
\begin{equation} \label{eq:credal_uncertainty_dominance}
\gamma(y,y',x) = \inf_{\theta \in C} \frac{p_{\theta}(y | x)}{p_{\theta}(y' | x)} > 1.
\end{equation} 
Therefore, in case of binary classification with $\mathcal{Y} = \{0, 1\}$ this dominance is calculated by score
\begin{equation}
    s(x) = -max (\gamma(1,0,x), \gamma(0,1,x)).
\end{equation} 
For binary classification where $p_{\theta}(0|x) = 1 - p_{\theta}(1|x)$ inserted in \Autoref{eq:credal_uncertainty_dominance}, results in  
\begin{equation}
\gamma(1,0,x) = \inf_{\theta \in C} \frac{p_{\theta}(1 | x)}{1 - p_{\theta}(1 | x)}
\end{equation} 
and
\begin{equation}
\gamma(0,1,x) = \inf_{\theta \in C} \frac{1 - p_{\theta}(1 | x)}{p_{\theta}(1 | x)}
\end{equation} 

\subsection{Epistemic and Aleatoric Uncertainty}
\label{subsec:epistemic_and_alreatoric_uncertainty}

\ac{EAU} are based on the use of relative likelihoods. 
\ac{EU} samples instance for which both the positive and the negative class appear to be plausible, while \ac{AU} samples instances where none of the classes is supported.
For a given instance x, the degrees of support or plausibility of the two classes are defined as \cite{nguyen2021howtomeasure}:
\begin{equation}
\pi(1 | x) = \sup_{\theta \in \Theta} \min \bigg[ \pi_{\Theta}(\theta), p_{\theta}(1 | x) - p_{\theta}(0 | x)\bigg]
\end{equation} 
\begin{equation}
\pi(0 | x) = \sup_{\theta \in \Theta} \min \bigg[ \pi_{\Theta}(\theta), p_{\theta}(0 | x) - p_{\theta}(1 | x)\bigg]
\end{equation} 
Accordingly, $\pi(1 | x)$ is high if and only if a highly plausible model supports the positive class much stronger as the negative class and vice versa for $\pi(0 | x)$.
Therefore, the uncertainty due to either influence of the classes or lack of knowledge. 

With this definition for degrees of support for positive and negative classes, epistemic uncertainty $u_e$ and aleatoric uncertainty $u_a$ are defined as \cite{nguyen2021howtomeasure}:
\begin{equation}
u_e = \min \bigg[ \pi(1 | x), \pi(0 | x) \bigg]
\end{equation}
\begin{equation}
u_a = 1 - \max \bigg[ \pi(1 | x), \pi(0 | x) \bigg]
\end{equation}

In general, it can be said that epistemic uncertainty returns uncertainty within a single model's prediction and aleatoric uncertainty the uncertainty across multiple predictions.
An example is illustrated in \Autoref{fig:differences_aleatoric_epistemic}.
\begin{figure*}[t]
  \centering
    \includegraphics[width=0.90\textwidth]{figures/uncertainty_differences.PNG}
  \caption{Difference between aleatoric and epistemic uncertainty (obtained from \cite{human-in-the-loop}).}
  \label{fig:differences_aleatoric_epistemic}
\end{figure*}
Two different classes with label A and label B are shown in the figure.
Some of the instances have already been labelled, others not yet.
Furthermore, a total of five different decision boundaries from five different predictions can be identified.
The first marked instances is close to decision boundaries of all five predictions.
Therefore, we have high epistemic uncertainty since all models are uncertain how to label the first instance.
Thus, we have low aleatoric uncertainty since decision boundaries of all models are close together.
For second instance all models are certain how to label it so epistemic uncertainty is low.
In contrast, the variance of distance to decision boundaries is very high, which results in high aleatoric uncertainty.
Consequently, for third instance both, aleatoric and epistemic uncertainty is high.
