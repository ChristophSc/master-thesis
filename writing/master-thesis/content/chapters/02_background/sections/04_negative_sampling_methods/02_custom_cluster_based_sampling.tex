\subsection{Custom Cluster-Based Sampling}
\label{subsec:custom_cluster_based_sampling}


Custom Cluster-Based Sampling samples negative triples from small clusters which are based on closeness between entities.
Instead of sampling from the whole set of entities, they a are divided into a number of groups and randomly sampled to create negative triples. Examples are TransE-\ac{SNS} \cite{TransE-SNS} or \ac{NSCaching} \cite{zhang2019nscaching} which are based on K-Means clustering algorithm or caching techniques. 


\textbf{Evaluation}:\\
% CUSTOM CLUSTER-BASED SAMPLING
To get more efficiency in the training process and to search for suitable entities in a more targeted way is aimed with  \textit{Custom Cluster-Based Sampling} methods (\autoref{sec:negativesamplingmethods}).
By selecting negative samples only from a handful of candidates and not from the entire entity set, a better correlation between the positive and corresponding negative triple is expected.
This negative triple should be closer to the original positive triple and thus provide more valuable information for the \ac{KGE} model.
For example, Domain Sampling \cite{domainSampling} is to sample from the same domain.
For example, if it is recognized that 'Germany' is a country and that 'France' is a nearby country, the much more valuable negative triple (Paderborn, locatedIn, France) could be sampled from the  positive triple (Paderborn, locatedIn, Germany).
However, as \acp{KG} grow rapidly and are updated frequently, continuous renewing custom clusters is essential and difficult \cite{qianunderstanding}.

With these approaches of Negative Sampling from a fixed distribution two more different problems arise:
Since they ignore changes in the distribution of negative triples, they suffer from the vanishing gradient and biased estimation problem \cite{zhang2021efficient}.
The scoring functions tend to give observed (positive) triplets large values and most of the non-observed (probably negative) triplets will have smaller values during the training.
Therefore, if negative triplets are uniformly sampled, it is very likely to pick up one with zero gradients.
This leads to the following main challenges for Negative Sampling \cite{zhang2021efficient}: 
(i) The negative triple's dynamic distribution has to be captured and 
(ii) triples have to be effectively sampled from this distribution.