\subsection{Custom Cluster-Based Sampling}
\label{subsec:custom_cluster_based_sampling}


Custom Cluster-Based Sampling samples negative triples from small clusters which are based on closeness between entities.
Instead of sampling from the whole set of entities, they a are divided into a number of groups and randomly sampled to create negative triples. 
Examples are TransE-\ac{SNS} \cite{TransE-SNS} or \ac{NSCaching} \cite{zhang2019nscaching} which are based on K-Means clustering algorithm or caching techniques. 

% TransE-\ac{SNS}

% \ac{NSCaching} 

\textbf{Evaluation}:\\
% CUSTOM CLUSTER-BASED SAMPLING
With custom cluster-based sampling methods more efficiency in the training process and suitable entities are found in a more targeted way.
By selecting negative samples only from a handful of candidates and not from the entire entity set, a better correlation between the positive and corresponding negative triple is expected.
This negative triple should be closer to the original positive triple and provide more valuable information for the \ac{KGE} model.
For instance, domain sampling \cite{domainSampling} aims to sample entities within the same domain.
Therefore, if in a given positive triple \triple{Paderborn}{locatedIn}{Germany} the entity \texttt{Germany} is recognized as a country and that \texttt{France} as a country nearby \texttt{Germany}, the much more valuable negative triple \triple{Paderborn}{locatedIn}{France} can be sampled.
However, as \acp{KG} grow rapidly and are updated frequently, continuous renewing custom clusters is essential and difficult \cite{qianunderstanding}