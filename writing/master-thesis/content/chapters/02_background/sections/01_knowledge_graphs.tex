\section{Knowledge Graphs} 
\label{sec:knowledge_graphs}

There are several definitions of a \acp{KG} in the literature. 
In the scope of this work, we work with the following definition given by \cite{ConEx, RotatE}.
Let the set of entities be represented by \entities and relations by \relations.
Then, a \ac{KG} $\kg= \{\triple{h}{r}{t} \}  \subseteq \entities \times \relations \times \entities$ can be formalised as a set of triples where each triple contains a head entity \texttt{h} and a tail entity \texttt{t} with \texttt{h}, \texttt{t} $\in$ \entities and a relation \texttt{r} $\in$ \relations which can be \cite{ConEx}
\begin{itemize}
    \item 
    \emph{symmetric} if $\triple{h}{r}{t} \iff \triple{t}{r}{h}$ for all pairs of entities $\texttt{h},\texttt{t}\in \entities$, 
   
   \item 
   \emph{anti-symmetric} if $\triple{h}{r}{t} \in \kg \Rightarrow \triple{t}{r}{h} \not \in \kg$ for all $\texttt{h} \not= \texttt{t}$, and
    
    \item 
    \emph{transitive}/\emph{composite} if $\triple{h}{r}{t}\in\kg \wedge \triple{t}{r}{y} \in \kg  \Rightarrow \triple{h}{r}{y} \in \kg$ for all $\texttt{h},\texttt{t},\texttt{y}\in \entities$.
\end{itemize}
In addition, $\texttt{r}^{-1}$ denotes the inverse of a relation \texttt{r} where for any two entities \texttt{h} and \texttt{t}, \triple{h}{r}{t} $\in$ \kg $\iff (\texttt{t},\texttt{r}^{-1},\texttt{h}) \in \kg$ \cite{ConEx}.
