\subsection{Evidence-based Uncertainty} 
\label{subsec:evidence_based_uncertainty}

\ac{EBU} differentiates between uncertainty due to conflicting evidence and insufficient evidence \cite{nguyen2021howtomeasure}.
\ac{EBU} checks the influence of individual features in the feature representation.
Therefore, it partitions them into those that provide evidence for the positive and for the negative class \cite{nguyen2021howtomeasure}:
\begin{equation}
    P_{\theta}(x) = \bigg\{ x^m \bigg| \frac{p_{\theta}(x^m|1)}{p_{\theta}(x^m|0)} > 1 \bigg\} 
\end{equation}
 \begin{equation}
    N_{\theta}(x) = \bigg\{ x^m \bigg| \frac{p_{\theta}(x^m|0)}{p_{\theta}(x^m|1)} > 1 \bigg\} 
\end{equation}
Subsequently, either the instance with the highest conflicting evidence (\ac{CEU}) or where both evidences are low (\ac{IEU}) are queried (insufficient evidence).
The total evidence is for positive class $E_1$ and negative class $E_0$ is determined as \cite{nguyen2021howtomeasure}:
\begin{equation}
    E_1(x) = \prod\limits_{x^m \in P_{\theta}(x)} \frac{p_{\theta}(x^m|1)}{p_{\theta}(x^m|0)}
\end{equation}
\begin{equation}
    E_0(x) = \prod\limits_{x^m \in N_{\theta}(x)} \frac{p_{\theta}(x^m|0)}{p_{\theta}(x^m|1)}
\end{equation} 
Therefore, conflicting evidence is present if $E_1$ and $E_0$ are high.
Likewise, if both $E_1$ and $E_0$ are low we measured uncertainty because of insufficient evidence.
