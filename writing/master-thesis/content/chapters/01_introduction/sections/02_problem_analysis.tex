\section{Problem Analysis}
\label{sec:problem_analysis}

Negative sampling plays an important role in embedding learning.
However, since there are no negative triples in a \ac{KG}, the generation of negative triples in current approaches poses some problems.
First of all, it can be said that while many negative sampling methods currently demonstrate high performance, the sampled negative triples are often too simple and represent a trivial solution. 
As a result, embedding models do not learn or learn less from the provided negative triples and therefore, they do not improve the embedding.
Instead, they suffer from the vanishing gradient or biased estimation problem \cite{zhang2021efficient}.
The vanishing gradient problem is present when the gradients of the loss functions approach zero and consequently, the model is unable to learn during the training process.
This results from the fact that most \ac{KGE} models, due to simplicity and efficiency, use Uniform Negative Random Sampling.

% UNIFORM RANDOM SAMPLING
Uniform Negative Random Sampling is a common technique of negative sampling where either the head or the tail entity in a given positive triple \triple{h}{r}{t} is randomly replaced by any other entity of the \ac{KG} which remains in the new negative triple \triple{h’}{r}{t} or \triple{h}{r}{t’}. 
Therefore, it is very likely to pick an entity which results in a zero gradient because the negative triple can be easily discriminated from the positive one \cite{cai2017kbgan}.
For example, by replacing the head entity of the positive triple \triple{h}{r}{t} = \triple{Joe Biden}{bornIn}{USA} with head entity \texttt{h'} = \texttt{Paderborn} would result in the negative triple \triple{h'}{r}{t} =  \triple{Paderborn}{bornIn}{USA} which is not very informative for the embedding.
By simply replacing the randomly selected head or tail entity of an again randomly selected entity of the \ac{KG} does not use any further information.
For example, it would have been useful if either negative sampling had recognized that the head entity \texttt{Joe Biden} is a person and to replace it with another person.
Moreover, recognizing that the tail entity as well as the sampled entity \texttt{Paderborn} is a location and its replacement would have led to the much more meaningful negative triple \triple{Joe Biden}{bornIn}{Paderborn}.  
Thus, while this approach is a fast and effective way to generate negative triples, it leads to a low learning factor in the embedding model.

% BERNOULLI SAMPLING
More useful negative examples are created by Bernoulli Sampling, which notes more information about a \ac{KG} and its individual entities and relations.
In comparison to Uniform Negative Random Sampling, it considers types of relations between entities (one-to-many, many-to-one and many-to-many) \cite{zhang2021efficient}.
These relation types are an indicator for the sampling approach if it is better to replace the head or the tail entity.
From the example above, it would have been recognized that the relation \texttt{bornIn} is a many-to-one relation.
Therefore, the head entity cannot have this relation to multiple entities, making each replaced tail entity a more useful negative triple.
Even though this is still a very fast and effective way to create negative triples, they are still easy to distinguish from positive ones.

% OTHER INFORMATION USED
In addition to these most commonly used methods, there are others which leverage external constraints such as entity types.
However, this resource does not always exist or is accessible \cite{cai2017kbgan}.
Instead of sampling from all entities in a \ac{KG}, other negative sampling methods take the approach of sampling only from a handful of selected entities.
For example, by sampling entities within the same domain, they hope to increase efficiency \cite{qiannegative}.
However, due to the rapid growth and frequent updating of \acp{KG}, constantly updating custom clusters is essential and skilled \cite{qiannegative}. 
Additionally, the creation of subsets of entities leads to a degradation of sampling performance and the information needed for this is not always available or is very difficult to derive from a \ac{KG}.


%Many of these approaches aim to find hard negative examples that are close to positive facts from a given \ac{KG} and thus have a positive effect on the embedding learning process.

% CHALLENGES
%Consequently, there are two main challenges for sampling negative triples \cite{zhang2021efficient}:
%At first, it is necessary to capture and model the dynamic distribution if negative triples to sample informative and useful negative triples with high gradients which help the model during the embedding learning process.
%Secondly, these negative triples have to be sampled effectively  so that the Negative Sampling does not negatively affect the performance of embedding learning.
 
