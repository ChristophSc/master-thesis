\section{Objectives}
\label{sec:objectives}

% OBJECTIVE + HYPHOTETHESIS
Inspired by uncertainty sampling in active learning, we want to incorporate uncertainty information in an existing negative sampling process from embedding learning.
Unlike previous approaches, we thus do not sample a negative triple which is closest to a positive one, but the one where the embedding model is most uncertain about.
Therefore, the sampled instances can be hard negative examples, but also other negative triples which are valuable for the embedding model.
By implementing a new sampling method, we expect to generate more informative negative triples and therefore, have a positive impact on a negative sampling process.
This could be an acceleration of learning process like achieving same accuracies with less epochs, same accuracies with less training time or achieving a better overall accuracy.

% CONTRIBUTIONS
Our thesis will make the following contributions:
First, we will look at existing approaches and analyze how uncertainty is measured in uncertainty sampling of active learning and which of these approaches is applicable for negative sampling.
After an implementation of uncertainty sampling is done, 
we will perform an evaluation by testing our approach on different \ac{KG} datasets.
Due to the partly large amount of data and correspondingly longer execution times, \ac{PC2}\footnote{\url{https://pc2.uni-paderborn.de/}} will be used for this purpose,
Subsequently, the results are compared with existing methods to conclude our approach \textbf{Sampling of Negative Triples for Knowledge Graph Embeddings by Uncertainty}.








