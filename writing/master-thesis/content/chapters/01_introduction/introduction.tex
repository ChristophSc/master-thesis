\chapter{Introduction}
\label{ch:introduction}

% general introduction to the topic, definition of important terms
\acp{KG} represent structured collections of facts describing the world   \cite{hogan2020knowledge}.
These collections of facts have been used in a wide range of application, e.g., question answering, structured search \cite{zhang2019nscaching}, and link prediction \cite{cai2017kbgan, Alam2020AffinityDN}.
In recent years, numerous \acp{KG} such as \textsc{FB15K}, \textsc{WN18}, \textsc{YAGO} \cite{ConEx} and \textsc{WikiData} \cite{arnaoutwikinegata} have been released.
% TODO: Why are there several ones, which kind of data to they contain?
% TODO: Mention RDF and other format of triples?
Within a \ac{KG}, entities are stored as nodes and relations as directed edges \cite{zhang2019nscaching}.
Facts are represented as a triple in the form of (head entity, relation, tail entity), denoted as \triple{h}{r}{t}.
These triples indicate that the head entity \texttt{h} (subject) is connected with the tail entity \texttt{t} (object) by a specific relation \texttt{r} (predicate) \cite{zhang2019nscaching, Alam2020AffinityDN}.
% TODO: advantages and possibilities of KG graphs in comparison to (relational) databases
% https://www.ontotext.com/knowledgehub/fundamentals/what-is-a-knowledge-graph/
One of the main challenges for \acp{KG} is to find a Knowledge Graph Representation which encodes similarities and differences among  entities and relations. 
\ac{KRL} is a critical research issue and forms the basis for many knowledge acquisition tasks and applications.
For this reason, for example, simple approaches like one-hot encoding does not provide promising results.
Early works in this area used the symbolic triplet data for statistical relational learning, but this approach neither has good generalization performance, nor it can be applied for large scale \acp{KG} \cite{zhang2021efficient}.
However, \ac{KGE} is a renowned area of research in recent years which transforms a \acp{KG} into a low-dimensional vector space using embedding models \cite{Alam2020AffinityDN}.
Many approaches learn vector representations for \acp{KG} while learning a parametrized scoring function that assigns scores to input triples.
A score of a triple is expected to reflect the likelihood that the input triple is true \cite{ConvE, qiannegative}.
To learn these low-dimensional \acp{KGE} for entities and relations, several training approaches are available \cite{Ruffinelli2020You}.
During the training process of these approaches positive and negative triples are discriminated.
However, most known \acp{KG} contain only positive instances for space efficiency \cite{qiannegative} and not all positive information are represented in \acp{KG} either.
Accordingly, \acp{KG} are an incomplete picture of the reality.
Nevertheless, missing negative examples are needed to learn the \ac{KGE}.
Thus, different \ac{KGE} models have developed which are using different methods to generate the negative triples by Negative Sampling.
It is an essential part of distinguishing the models \cite{Ruffinelli2020You} and impacts the \ac{KGRL} and the performance of subsequent tasks which are using \acp{KGE}. 
Therefore, generating negative examples by Negative Sampling is an essential and important part of learning \acp{KGE}.





%- embedding based models:
%    - have better generalization ability
%    - better inference efficiency
%    - scalable
%    - shown promising performance in basic KG tasks
%\cite{qianunderstanding}:
%- One-hot encoding is broadly used to convert features or instances into vectors,
%-> great interpretability but incapable of capturing latent semantics



\section{Negative Sampling}
\label{sec:negative_sampling}

%Negative sampling was originally used for neuronal probabilistic language models and was referred to as importance sampling \cite{qiannegative, qianunderstanding}.
%Mikolov et al. \cite{MikolovSCCD13} refer to negative sampling as a simplified version of \ac{NCE} to overcome the computational difficulty which was associated with probabilistic language models \cite{qianunderstanding}.
%This was due to their partition functions summing over all words which is expensive if the amount of vocabulary is high \cite{qianunderstanding}.
%In comparison, negative sampling transforms the difficult density estimation problem into a binary classification problem.
%Consequently,  true samples are distinguished from noise samples to simplify the computation as well as accelerating training \cite{qianunderstanding}.
%Originally, the partition function was normalized into a probability distribution based on the entire vocabulary.
%Instead, true samples from a \ac{KG} are separated from noise distribution samples to asymptotically estimate the true distribution, which is highly efficient with low computational cost \cite{qianunderstanding}.

% HISTORY + GENERAL
Negative sampling originates from neuronal probabilistic language models where it was introduced to overcome the computational difficulty by transforming the difficult density estimation problem into a binary classification problem \cite{qianunderstanding}.
Considering nodes as words and the neighbors of the nodes as the context of a word, graph representation learning of \acp{KGE} is similar to language modeling \cite{qianunderstanding}.
Therefore, the idea was applied to \ac{KGE} learning and has become common practice over the past several years.
Instead of discriminating true samples from noise samples, positive triples are discriminated from negative ones.
Since bad or too obviously incorrect negative triples fail to capture the latent semantics in a \ac{KG} and lead to a zero loss problem, several negative sampling methods have been developed and used in different \ac{KGE} models \cite{qiannegative}.
Therefore, generated negative triples with a high quality ensure successful training, and the learned embedding performs better in downstream tasks, negative sampling becomes a very important part in \ac{KGE} learning.

% ASSUMPTIONS FOR LEARNING EMBEDDINGS
Since there are only positive triples in a \ac{KG} and thus no information is available about the distribution of negative triples or their quality, negative sampling is a challenging task for embedding models.
There are two different ways of looking at \acp{KG} at hand for negative sampling:
Negative triples can be generated either under the \ac{CWA} or the \ac{OWA} \cite{qiannegative}.
Both assumptions consider statements stored in the \ac{KG} as true, but differ on unobserved facts which are not present in a \ac{KG}.
With the \ac{CWA} they are considered as false, but the \ac{OWA} assumes that missing triples are simply unknown and consequently can be either true or false
\cite{qiannegative}.
The \ac{CWA} has two main drawbacks.
On the one hand, it has worse performance in downstream tasks \cite{qiannegative} since false-negative triples can be generated.
False-negative triples are defined as triples that are assumed not to be true, but actually, reflect true facts of the reality \cite{qianunderstanding}.
On the other hand, the \ac{CWA} has scalability issues due to a large number of negative samples \cite{qiannegative}.
Therefore, generating more informative negative triples by negative sampling with good performance represents a non-trivial step in \ac{KGE} learning.
Especially, because the quality of these generated negative triples has a direct impact on the \acp{KGE} \cite{qiannegative}.

% GENERAL PROCESS OF NEGATIVE SAMPLING
Overall, there are several variations for embedding learning with negative sampling in the literature.
However, they have the following two steps in common.
At first, for a given positive triple, a negative triple is generated.
Subsequently, positive triples from a \ac{KG} as well as generated negative triples from negative sampling are given to an embedding model.
Finally, embeddings are updated in the direction of the negative gradient of the loss function.
The following section elucidates this process and analyzes its problem.

\section{Problem Analysis}
\label{sec:problem_analysis}

Negative sampling plays an important role in embedding learning.
However, since there are no negative triples in a \ac{KG}, the generation of negative triples in current approaches poses some problems.
First of all, it can be said that while many negative sampling methods currently demonstrate high performance, the sampled negative triples are often too simple and represent a trivial solution. 
As a result, embedding models do not learn or learn less from the provided negative triples and therefore, they do not improve the embedding.
Instead, they suffer from the vanishing gradient or biased estimation problem \cite{zhang2021efficient}.
The vanishing gradient problem is present when the gradients of the loss functions approach zero and consequently, the model is unable to learn during the training process.
This results from the fact that most \ac{KGE} models, due to simplicity and efficiency, use Uniform Negative Random Sampling.

% UNIFORM RANDOM SAMPLING
Uniform Negative Random Sampling is a common technique of negative sampling where either the head or the tail entity in a given positive triple \triple{h}{r}{t} is randomly replaced by any other entity of the \ac{KG} which remains in the new negative triple \triple{h’}{r}{t} or \triple{h}{r}{t’}. 
Therefore, it is very likely to pick an entity which results in a zero gradient because the negative triple can be easily discriminated from the positive one \cite{cai2017kbgan}.
For example, by replacing the head entity of the positive triple \triple{h}{r}{t} = \triple{Joe Biden}{bornIn}{USA} with head entity \texttt{h'} = \texttt{Paderborn} would result in the negative triple \triple{h'}{r}{t} =  \triple{Paderborn}{bornIn}{USA} which is not very informative for the embedding.
By simply replacing the randomly selected head or tail entity of an again randomly selected entity of the \ac{KG} does not use any further information.
For example, it would have been useful if either negative sampling had recognized that the head entity \texttt{Joe Biden} is a person and to replace it with another person.
Moreover, recognizing that the tail entity as well as the sampled entity \texttt{Paderborn} is a location and its replacement would have led to the much more meaningful negative triple \triple{Joe Biden}{bornIn}{Paderborn}.  
Thus, while this approach is a fast and effective way to generate negative triples, it leads to a low learning factor in the embedding model.

% BERNOULLI SAMPLING
More useful negative examples are created by Bernoulli Sampling, which notes more information about a \ac{KG} and its individual entities and relations.
In comparison to Uniform Negative Random Sampling, it considers types of relations between entities (one-to-many, many-to-one and many-to-many) \cite{zhang2021efficient}.
These relation types are an indicator for the sampling approach if it is better to replace the head or the tail entity.
From the example above, it would have been recognized that the relation \texttt{bornIn} is a many-to-one relation.
Therefore, the head entity cannot have this relation to multiple entities, making each replaced tail entity a more useful negative triple.
Even though this is still a very fast and effective way to create negative triples, they are still easy to distinguish from positive ones.

% OTHER INFORMATION USED
In addition to these most commonly used methods, there are others which leverage external constraints such as entity types.
However, this resource does not always exist or is accessible \cite{cai2017kbgan}.
Instead of sampling from all entities in a \ac{KG}, other negative sampling methods take the approach of sampling only from a handful of selected entities.
For example, by sampling entities within the same domain, they hope to increase efficiency \cite{qiannegative}.
However, due to the rapid growth and frequent updating of \acp{KG}, constantly updating custom clusters is essential and skilled \cite{qiannegative}. 
Additionally, the creation of subsets of entities leads to a degradation of sampling performance and the information needed for this is not always available or is very difficult to derive from a \ac{KG}.


%Many of these approaches aim to find hard negative examples that are close to positive facts from a given \ac{KG} and thus have a positive effect on the embedding learning process.

% CHALLENGES
%Consequently, there are two main challenges for sampling negative triples \cite{zhang2021efficient}:
%At first, it is necessary to capture and model the dynamic distribution if negative triples to sample informative and useful negative triples with high gradients which help the model during the embedding learning process.
%Secondly, these negative triples have to be sampled effectively  so that the Negative Sampling does not negatively affect the performance of embedding learning.
 


\section{Objectives}
\label{sec:objectives}

% OBJECTIVE + HYPHOTETHESIS
Inspired by uncertainty sampling in active learning, we want to incorporate uncertainty information in an existing negative sampling process from embedding learning.
Unlike previous approaches, we thus do not sample a negative triple which is closest to a positive one, but the one where the embedding model is most uncertain about.
Therefore, the sampled instances can be hard negative examples, but also other negative triples which are valuable for the embedding model.
By implementing a new sampling method, we expect to generate more informative negative triples and therefore, have a positive impact on a negative sampling process.
This could be an acceleration of learning process like achieving same accuracies with less epochs, same accuracies with less training time or achieving a better overall accuracy.

% CONTRIBUTIONS
This thesis will make the following contributions:
First, we will look at existing approaches and analyze how uncertainty is measured in uncertainty sampling of active learning.
Subsequently, these approaches are reviewed for their applicability so that a new negative sampling approach can be created.
After an implementation of uncertainty sampling is carried out, 
an evaluation is performed by testing the new approach on different \ac{KG} datasets.
Due to the partly large amount of data and correspondingly longer execution times, \ac{PC2}\footnote{\url{https://pc2.uni-paderborn.de/}} will be used for this purpose.
Finally, the results are compared with existing methods to conclude the approach approach of \textbf{Sampling of Negative Triples for Knowledge Graph Embeddings by Uncertainty}.










\section{Related Work} 
\label{sec:relatedwork}

% EMBEDDING MODELS
The history of \acp{KG} goes back several years, but in recent years there has been a lot of research in this area, especially in the context of \acp{KGE}.  
Dai et. al \cite{electronics9050750} created an overview of several different and most common used embedding models, their approaches and application possibilities.
Among many different embedding models, the distance-based models \transe \cite{TransE} and \transd \cite{TransD} can be highlighted, which will also take a more important part in the course of this work.
Other embedding models represent \distmult \cite{DistMult} and \complex \cite{ComplEx}, which are based on semantic matching. 

% NEGATIVE SAMPLING
Due to the importance of sampling good negative triples, much research has also been done in this area in recent years.
However, standard techniques continue to be Uniform Random Sampling \cite{TransE} and Bernoulli Sampling \cite{TransH}, which do not produce high quality negative triples, but are used in many models due to their simplicity and high performance.  
A more complex approach to generate negative samples is, for example, domain sampling \cite{domainSampling}, which samples only entities from a subset of the entities of a \ac{KG}.
Pioneers of dynamic negative sampling are \kbgan \cite{cai2017kbgan} and \igan \cite{IGAN}, which attempt to estimate the distribution of negative triples by constructing a \ac{GAN}.
Inspired by \acp{GAN}, which were proposed for generating samples in a continuous space such as images, pre-trained models are improved through an adversarial learning process.
An overview over all the different negative sampling techniques is given for example in \cite{qiannegative} or \cite{MCNS}.

% ACTIVE LEARNING + UNCERTAINTY SAMPLING
In addition to this topic of \acp{KGE}, our approach includes other work in the literature regarding uncertainty sampling of active learning.
Originally, uncertainty sampling comes from active learning which supports supervised learning systems where unlabeled data is abundant, but it is difficult, time-consuming, or expensive to obtain labeled instances \cite{Settles2009ActiveLL}.
Several different approaches are available to select the instances to be labeled.
Uncertainty sampling is one of them and uses a classifier to identify unlabeled examples with the least confidence \cite{5272205}.
Therefore, the most informative unlabeled examples are selected for human annotation.
In turn, in the uncertainty sampling, several measures are available on how to obtain the uncertain cases of the classifier \cite{nguyen2021howtomeasure}.

\section{Structure of the Thesis}
\label{sec:structure_of_thesis}

This thesis is structured as follows:
At first in \textbf{\Autoref{ch:introduction}} a general introduction to the topic is given, the problem is analyzed, related work is mentioned and the objectives of this thesis are presented.
\textbf{\Autoref{ch:background}} establishes the important definitions of terms and presents the state of the art approaches and methods that exist in research.
In \textbf{\Autoref{ch:approach}} the motivation and a detailed description of the approach is given.
\textbf{\Autoref{ch:implementation}} deals with the practical realization and implementation of the approach.
Subsequently, \textbf{\Autoref{ch:evaluation}} evaluates our approach and compares it to existing methods using various datasets and metrics, and draws conclusions.
Lastly, \textbf{\Autoref{ch:summaryanddiscussion}} summarizes the work and addresses discussions for future work.

