\chapter{Approach}
\label{ch:approach}

In this chapter our approach of Sampling of Negative Triples for Knowledge Graph Embeddings by Uncertainty is described.
At first, in \Autoref{sec:idea} the general idea of Sampling by Uncertainty is described and how this idea came about.
First of all, as described in \Autoref{sec:uncertaintysampling}, we have various metrics available for Uncertainty Sampling.
These are discussed in \Autoref{sec:selection_of_an_uncertainty_sampling_metric} and a suitable metric for our problem is determined.
Subsequently, to measure uncertainty and sample negative triples a classification problem needs to be defined which takes place in \Autoref{sec:definition_of_a_classification_problem}.
In \Autoref{sec:calculation_of_probabilitie_for_triple_classification}, different possibilities to calculate the individual probabilities of whether a given triple is positive or negative can be calculated.
From these probabilities uncertainty scores can be derived to finally sample negative triples by uncertainty (\Autoref{sec:calculation_of_uncertainty_scores}).
The individual steps are then reflected in the modified procedure of our approach (\Autoref{sec:procedure}).
At the end of the chapter, various possibilities for improving the classification are listed (\Autoref{sec:improving_classification}).

\section{Idea} 
\label{sec:idea}

% 1) Presentation of my idea
% - origin of my idea
% - Reasons for Uncertainty - advantages and expected results
% - Why did I choose KBGAN (Sampling Negatives from Dynamic Distribution) 
%   as basis for my approach
% - disadvantages of KBGAN that I want to solve
% - why uncertainty sampling can solve them


% description and general introduction to solution
With our approach, we aim to improve the training process of embedding models by incorporating more information into the Negative Sampling process and thus selecting negative examples that are more informative and more valuable for the embedding model.
This is achieved by including Uncertainty Sampling in state-of-the-art approaches.
Uncertainty Sampling is the most commonly used query framework among all the other strategies frameworks (\autoref{sec:uncertaintysampling}) from Active Learning.
In addition, it is easier to implement than some other frameworks, because, for example, it does not require to train multiple models as is the case with the Query-by-Committee approach.
Nevertheless, it provides promising results for many Active Learning scenarios and is therefore also used for our approach to sample negative triples.

Since \ac{KBGAN} captures the dynamic distribution of negative examples, but suffers particularly from non-efficient Sampling, our first considerations are based on this approach.
This addresses the problem of instability and degeneracy of the training process, so that it is more likely to sample negative examples with a higher gradient.
However, it is also conceivable for other existing approaches with an inefficient Negative Sampling process by replacing it with Uncertainty Sampling.

% ways to improve KBGAN
In general, in \ac{KBGAN} there are two different ways to improve the approach and to allow Sampling of negative triples with a higher gradient:
First, this can be achieved by improving the general quality of the negative examples in subset $Neg$.
In the original approach, $Neg$ is generated only by Uniform Sampling, so that with high probability there are many useless negative triples.
Second, the adversarial learning process can be optimized by learning faster with less Sampling which is addressed by our approach.
To achieve this, we want to sample more informative negative examples.
Informative at this point means particularly interesting triples for the embedding model because those that are difficult to classify as positive or negative triple.
In many other negative sampling approaches, the goal is to create so-called hard negative examples and use them for the training process of the embedding model. 
In our approach, the result of the sampling process can also be such a hard negative example, but this need not necessarily be the case.
Now, what 
In \autoref{fig:informativeinstances} the distinction between informative and hard negative examples is illustrated.
\begin{figure*}[t]
  \centering
    \includegraphics[width=0.75\textwidth]{figures/informative_instances.PNG}
  \caption{Example of Uncertainty Sampling for positive and negative instances. Simplified to two-dimensional embedding space d1 and d2 and borders b1 and b2 between positive and negative instances.}
  \label{fig:informativeinstances}
\end{figure*}
For simplicity, positive and negative instances are shown here in two-dimensional embedding space with dimensions d1 and d2.
The embedding models sets the border between positive and negative triples at b1.
For the approach of sampling hard negatives the embedding model should choose negative instance n1 because it is the closest one to the border b1 and therefore, it is the most difficult one to classify as positive or negative.
Instead, Uncertainty Sampling aims to sample the most informative negative triple.
But what is the most informative triple for an embedding model?
The most informative triple is the one that the model helps the most to distinguish between positive and negative triples.
For this reason, the negative triple n2 would be much more informative, since its classification would indicate whether the boundary between positive and negative triples should rather be set at b1 or at b2.
Thus, by setting the new boundary b2 would contribute to a more accurate embedding of the data.

\section{Selection of an Uncertainty Sampling Framework} 
\label{sec:selection_of_an_uncertainty_sampling_type}

% review of all frameworks
In the literature we have three different frameworks of uncertainty sampling at our disposal:
\ac{EBU}, \ac{EAU} and \ac{CU}.
These frameworks of uncertainty sampling have different viewpoints on uncertainty and how to measure it, but all of them are based on a classification problem of two or more classes.
In \ac{EBU} we consider uncertainty within a model by looking at the individual features of a model and if they provide evidence for classes.
This can be either very many (\textit{conflicting-evidence}) or only a few features (\textit{insufficient-evidence}) which are indicative for respective classes.
In contrast, \ac{EAU} considers \textit{epistemic uncertainty} within a single model’s prediction, and \textit{aleatoric uncertainty} across multiple model predictions \cite{human-in-the-loop}.
\Ac{CU} differentiates between the reducible and irreducible part of the uncertainty in a prediction and the domination of one class over the other.

% determination of one framework
In an adversarial training process, negative triples are first passed to the generator, which calculates scores for them based on current embeddings.
subsequently, based on these scores, one of these negative triples is sampled and given to the discriminator.
Therefore, we only have information about the embeddings of the generator model and all negative triple scores.
We do not have any additional features for the sampling process.
For this reason, we will first take framework of epistemic uncertainty from \ac{EAU}.
The other frameworks can be considered as well, but for them additional models or model predictons or additional features are needed.

%- literature on aleatoric uncertainty tends to focus on the optimal types of ensembles and dropouts \cite{human-in-the-loop}

%for epistemic uncertainty the literature focuses
%on improve accuracy on probability distributions within a single model \cite{human-in-the-loop}

%epistemic
%-> philosophy: 
%due to limited data and knowledge. 
%Given enough training samples, epistemic uncertainty will decrease.
%historically: meant lack of knowledge
%model uncertainty
%-> machine learning:
%-> literature on epistemic uncertainty tends to focus
%on getting more accurate probability distributions from within a single model

%aleatoric:
%-> philosophy:
%arising from the natural stochasticity of observations
%cannot be reduced even when more data is provided
%historically meant inherent randomness
%data uncertainty
%-> machine learning:

\section{Definition of a Classification Problem}
\label{sec:definition_of_a_classification_problem}
%
% JUSTIFICATION FOR A CLASSIFICATION
Since all uncertainty measures are based on a classification problem, this needs to be defined first.
Before this definition, we take a closer look at an example of uncertainty sampling in a \ac{SVM} (\Autoref{fig:svm}).
\begin{figure}[H]
  \centering
    \includegraphics[width=0.45\textwidth]{figures/SVM.pdf}
  \caption{Example of uncertainty sampling in a \ac{SVM} classification. Instances with least distance to linear plane are sampled.
  The width of boundary is maximized  (figure based on \cite{human-in-the-loop}).}
  \label{fig:svm}
\end{figure}
The figure shows a binary classification of labels A and B in a two-dimensional space.
Instances of classes A and B are separated as best as possible by a boundary.
Some of the instances are still unlabelled.
Now the instances that are most informative for the model and most helpful in defining the boundary between instances of class A and class B are sampled by uncertainty sampling.
The labeling of these instances in an active learning process helps to define a more precise boundary and thus enables a better distinction between instances with label A and label B.

In contrast to \acp{SVM}, \ac{KGE} models try to find the best possible embedding for a given set of entities and relations by distinguishing positive triples from the \ac{KG} and negative triples sampled by negative sampling.
What is the boundary between two labels in an \ac{SVM} is the scoring function in an embedding model.
In the case of a tensor factorization-based model, true triples achieve higher scores, and negative triples lower scores.
Therefore, a separation between positive and negative triples in a \ac{KGE} is achieved.
An optimal embedding would be able to perfectly distinguish between positive and negative triples.
However, since such embedding does not exist, the scoring function returns high scoring values for positive triples and low scores for negative triples.
Therefore, there is no clear 'decision boundary' between positive and negative instances.
To figure out uncertain cases of negative triples and apply uncertainty measures based on probabilities for different classes, the scoring values of the generator need to be converted into a classification problem.
Therefore, if the embedding model is uncertain about the classification of a negative triple, its uncertainty score will be high.
By sampling these uncertain triples to the discriminator it learns to distinguish between positive and uncertain negative triples.
Thus, uncertainty sampling is used to select instances based on a classification of positive and negative triples.
Since the generator is a tensor factorization-based model and their score indicates the plausibility of a triple to be true, the area of positive triple scores is slightly higher than the area of negative scores.
Nevertheless, since a perfect embedding model is unknown, there is an overlap of positive and negative triple scores.
Based on these different areas of the positive and negative triple scores, two different uncertainty areas can be defined.

The first possible definition of classification is depicted in \Autoref{fig:badVsGoodApproach}.
According to this definition, the classification is based only on low and high negative triple scores and therefore, uncertainty is only measured among all negative triple scores.
This way of defining the classification problem results in a sampling of instances that are easily distinguishable from the positive triples and therefore, less helpful to learn a good embedding.
For this reason, information about positive triple scores needs to be included.
\clearpage
\begin{figure}[H]
  \centering
    \includegraphics[width=0.9\textwidth]{figures/badVsGoodApproach.pdf}
  \caption{Illustration of sampled instances by uncertainty according to negative triple scores.
  Range of negative triple scores is in $[a, b]$ and range of positive triple scores is in  $[c, d]$ with $a,b,c,d \in \mathbb{R}$.}
  \label{fig:badVsGoodApproach}
\end{figure}

Therefore, the second option of the definition for a classification problem includes positive triple scores as well.
The resulting uncertainty area is illustrated in \Autoref{fig:positiveVsNegativeApproach}.
According to this approach, by defining class $y = 0$ for negative triples and class $y = 1$ for positive triples, each triple $l = (h,r,t)$ can be classified in a binary classification problem.
Accordingly, we can calculate the probability of being either a negative or a positive triple for each triple $\mathbb{P}(y| f_G(l)) \in [0,1]$ for $y \in \{0,1\}$.
Since we want to have as good negative triples as possible and therefore, triples that can be classified either positive or negative, we look for negative triples $l' = (h',r,t')$ that are close to the probability of $\mathbb{P}(y = 0| f_G(l')) = \mathbb{P}(y = 1| f_G(l')) = 0.5$.
Since this means the triple can be assigned to both the positive and the negative class.
Accordingly, the model is uncertain about how to classify the triple.
\begin{figure}[H]
  \centering
    \includegraphics[width=0.9\textwidth]{figures/positiveVsNegativeApproach.pdf}
  \caption{Illustration of the sampled instances by uncertainty according to negative and positive triple scores.
  Range of negative triple scores is in $[a, b]$ and range of positive triple scores is in  $[c, d]$ with $a,b,c,d \in \mathbb{R}$.}
  \label{fig:positiveVsNegativeApproach}
\end{figure}
% Now uncertainty sampling measures high uncertainty for triples which are close to positive ones, so which could be either a negative or a positive one.
After setting up the definition of the classification, the next step is to calculate the probabilities with which a triple belongs to a class.
\clearpage

\section{Calculation of Probabilities for Triple Classification} \label{sec:calculation_of_probabilities_for_triple_classification}
%
Having established the binary classification problem between positive and negative triples, it is necessary to determine how to calculate the probability for a triple to belong to one of these classes.
Assuming we have a generator scoring function $f_G$ which returns scoring values for each triple $l = (h,r,t)$ such that $f_G(l) \in \mathbb{R}$.
Therefore, to provide probabilities belonging to positive or negative triple class, two boundaries need to be defined:
First, $score_{min}$ which defines at which scoring value a triple has the probability of zero to be a positive triple: 
\begin{equation} \label{eqn:prob_score_min}
    \mathds{P}(y = 1 | f_G(l) \leq score_{min}) = 0
\end{equation}
and second $score_{max}$ at which scoring value a triple has the probability 1 to be positive one 
\begin{equation} \label{eqn:prob_score_max}
    \mathds{P}(y = 1 | f_G(l) \geq score_{max}) = 1
\end{equation}
Accordingly, for all other scoring values $score_{min} \leq f_G(l) \leq score_{max}$
\begin{equation} \label{eqn:prob_score_all}
    \mathds{P}(y = 1 | score_{min} \leq f_G(l) \leq score_{max}) \in [0, 1]
\end{equation}
Since we have a binary classification, the counter probabilities are given by:
\begin{equation} \label{eqn:counter_prob}
    \mathds{P}(y = 0 | f_G(l)) = 1 - \mathds{P}(y = 1 | f_G(l)) \in \mathbb{R}
\end{equation}

\textbf{Determination of Boundaries $\textbf{score}_{\textbf{min}}$ and $\textbf{score}_{\textbf{max}}$}\\
For the determination of scoring boundaries $score_{min}$ and $score_{max}$, the following two options are available.
The first option to define these borders is depicted in \Autoref{fig:positives_negatives1}.
\begin{figure}[H]
  \centering
    \includegraphics[width=0.9\textwidth]{figures/positives_negatives1.pdf}
  \caption{First option to define probabilities of triples to be positive ($\mathds{P}(y = 1 | f_G(l))$ ) or negative ($\mathds{P}(y = 0 | f_G(l))$). 
  $score_{min}$ at the left border of all positive triple scores and $score_{max}$ at the right border of all negative triples.}
  \label{fig:positives_negatives1}
\end{figure}
and results in the following definitions of  $score_{min}$ and 
$score_{max}$:
\begin{equation} \label{eqn:opt2_score_min}
    score_{min} = \argmin_{(h,r,t) \in \kg}{f_G(h,r,t)}
\end{equation}
\begin{equation} \label{eqn:opt2_score_max}
    score_{max} = \argmax_{(h',r,t') \in Neg}{f_G(h',r,t')}
\end{equation}
This option has the advantage of focusing only on the overlapping part of the model's prediction.
However, the drawback is that several triples of the negative triple set $Neg$ are disregarded.
If negative and positive scoring ranges do not overlap, all negative triples from $Neg$ would get a probability of zero to be a positive one, and therefore, according to this option, the model would not be uncertain about the classification of any negative triples in $Neg$.
Consequently, it is not suitable for our approach.

The other option for the definition of $score_{min}$ and $score_{max}$ is shown in \Autoref{fig:positives_negatives2}.
In comparison to the first option, this one provides a probability of being positive to every negative triple, no matter how the ranges of negative and positive triple scores are arranged.
\begin{figure}[H]
  \centering
    \includegraphics[width=0.9\textwidth]{figures/positives_negatives2.pdf}
  \caption{Second option to define probabilities of triples to be positive ($\mathds{P}(y = 1 | f_G(l))$) or negative ($\mathds{P}(y = 0 | f_G(l))$). 
  $score_{min}$ at the left border of all negative triple scores and $score_{max}$ at the right border of all positive triple scores.}
  \label{fig:positives_negatives2}
\end{figure}
With the scoring function $f_G$ of the generator $G$, this option results in the following definitions of $score_{min}$ and $score_{max}$:
\begin{equation} \label{eqn:opt1_score_min}
    score_{min} := \argmin_{(h',r,t') \in Neg}{f_G(h',r,t')}
\end{equation}
\begin{equation} \label{eqn:opt1_score_max}
    score_{max} := \argmax_{(h,r,t) \in \kg}{f_G(h,r,t)}
\end{equation}
\clearpage
Therefore, this option is more suitable for our approach and is implemented for uncertainty sampling.
Consequently, the probability of a triple $l = (h, r, t)$ to be a positive ($y = 1$) can be calculated with following \Autoref{eqn:positive_probability}.
\begin{equation}  \label{eqn:positive_probability}
    \mathds{P}(y = 1|f_G(l)) =
    \begin{cases}
        0 \ \ \ \ \ \ \ \ \ \ \ \ \ \ \ \ \ \ \ \ \ \ \ \ \ \ \  \ \   
         \text{if} \  f_G(h,r,t) \leq score_{min}
         
        \\ \\
        1 \ \ \ \ \ \ \ \ \ \ \ \ \ \ \ \ \ \ \ \ \ \ \ \ \ \ \   \ \ \
        \text{if} \ f_G(h,r,t) \geq score_{max}
         
        \\ \\
        \frac{f_G(h,r,t) - score_{min}}{score_{max} - score_{min}}
        \ \ \ \ \ \ 
         \text{else}
        \\
    \end{cases} 
\end{equation}
while $\mathds{P}(y = 1|f_G(l)) \in [0, 1]$.
Since we have a binary classification problem for positive and negative triples, the probability of a triple to be negative ($y=0$) is the counter probability
\begin{equation} \label{eqn:negative_probability}
    \mathds{P}(y = 0 | f_G(h, r, t)) = 1 - \mathds{P}(y = 1| f_G(h, r, t)) \in [0,1].
\end{equation}

\section{Calculation of Uncertainty Scores}
\label{sec:calculation_of_uncertainty_scores}
%
Uncertainty sampling takes place based on probabilities with which the instances belong to the respective classes.
After determining the calculation of probabilities in the previous section, the calculation of uncertainty scores can take place at this point.
For this, a transformation function is used to map from the generator scores to the uncertainty scores of the individual triples.
This transformation function $\phi$ uses the individual class probabilities.
Therefore, generator scores obtained by $f_G(l)$ for a triple $l = (h,r,t)$ are  mapped to uncertainty score $u(l)$ with following definition.
\begin{equation} \label{eqn:uncertainty_function}
    u(l) = \phi(f_G(x)) \in  [0,1].
\end{equation}
For this transformation function, several uncertainty measures are available.
These are entropy, least confidence, margin of confidence and ratio of confidence (\Autoref{sec:uncertaintysampling}).
They provide distributions of uncertainty scores $u(l) \in [0, 1]$ for probabilities $\mathbb{P}(y = 1 | f_G(l))$ of a given triple $l = (h, r, t)$ (\Autoref{fig:sampling_distributions}).
\clearpage
\begin{figure}[H]
    \centering
    \begin{minipage}{.4\textwidth}
      \centering
      \includegraphics[width=\linewidth]{figures/entropy_graph.PNG}
    \end{minipage}%
    \begin{minipage}{.4\textwidth}
      \centering
      \includegraphics[width=\linewidth]{figures/least_confident_graph.PNG}
    \end{minipage}
    \begin{minipage}{.4\textwidth}
      \centering
      \includegraphics[width=\linewidth]{figures/smallest_margin_graph.PNG}
    \end{minipage}%
    \begin{minipage}{.4\textwidth}
      \centering
      \includegraphics[width=\linewidth]{figures/smallest_ratio.PNG}
    \end{minipage}%
    \caption{Uncertainty sampling distributions for (a) entropy, (b) least confidence, (c) margin of confidence, and (d) ratio of confidence in a binary classification problem.
    The x-axis marks the probability of a triple belonging to the positive class $\mathbb{P}(y = 1 | f_G(l))$ and the y-axis the uncertainty score $u(l)$ of the corresponding uncertainty measure.}
    \label{fig:sampling_distributions}
\end{figure}

% describe sampling methods
% 1) always sample maximum uncertainty score
Based on the calculated probabilities, uncertainty scores are assigned to the triples.
Therefore, two different \textbf{Uncertainty Sampling Methods} are derived based on these distributions:
\begin{itemize}
    \item 
    % sampling according to max distribution
    \textbf{\underline{U}ncertainty \underline{S}ampling with \underline{Max}imum Distribution (\textsc{USMax})}:\\
    The first option is to always sample the triple with the maximum uncertainty score.
    As can be seen from \autoref{fig:sampling_distributions} all uncertainty measures have a maximum uncertainty score of 1 which is located at $P(y = 1 | f_G(l)) = 0.5$.
    Therefore, with \usmax all triples that are closest to this probability of 0.5 of being a positive triple get a sampling probability.
    If only one negative triple reaches a maximum uncertainty score among all negative triples, it will be sampled.
    If there are several triples with a maximum uncertainty score, one of these triples is sampled with an equal probability according to a uniform distribution.
    This results in the following definition of sampling method \usmax with the  sampling probability $\mathbb{P}_{\usmax}(l) \in [0,1]$ for $l = (h,r,t)$:
    \begin{equation} 
        \mathbb{P}_{\usmax}(l) =
        \begin{cases}
             0 \ \ \ \ \ \ \ \ \ \ \ \ \ \  \ \ \ \ \ \ \ \ \ \ \ \ \ \ \ \ \ \ \ \ 
             \text{if}\ \ \ u(l) < \max_{l \in Neg}(u(l)) 
             \\ \\
            \frac{1}{|\{l | u(l) = \max_{l \in Neg}(u(l)\}|} 
            \ \ \ \
            \text{else} 
             \\
        \end{cases}
         \label{eq:usmax}
    \end{equation}
    
    \item
    \textbf{\underline{U}ncertainty \underline{S}ampling with \underline{Softmax} Distribution (\textsc{USSoftmax})}:\\    
    % 
    The second sampling method is to sample according to a softmax probability distribution based on uncertainty scores.
    Thus, triples close to $P(y = 1 | f_G(l)) = 0.5$ still have the highest probability of being sampled, but other triples for which the model is more certain are also assigned a probability of being sampled.
    \ussoftmax is defined as follows:
    \begin{equation}
        \label{eq:ussoftmax}
        \mathbb{P}_{\ussoftmax}(l) = \frac{e^{u(l)}}{\sum_{l = \in Neg}{e^{u(l)}}} \in [0,1]
    \end{equation}
\end{itemize}
In the original \kbgan approach, triples were given a sampling probability according to their achieved score in the generator model.
Thus, negative triples with the highest score also received the highest sampling probability.
This sampling method is referred to as \textsc{Original Sampling} (\origsampling) in the following.
For the individual uncertainty measures, the following definitions for $\phi$ and considerations for the two sampling methods \usmax and \ussoftmax result.

\textbf{Entropy} 
considers the differences between all predictions.
According to the definition of entropy (\Autoref{eqn:entropy_def}), the mapping function $\phi$ is defined as
\begin{equation}
    \phi(x) = - \sum_{y \in \mathcal{Y}}{\mathds{P}(y | x) \cdot log \mathds{P}(y | x)}.
\end{equation}
In this case $x = f_G(l)$ can be inserted for a triple $l = (h,r,t)$.
Additionally, we have a binary classification problem with $\mathcal{Y} = \{0,1\}$.
Inserting $x$ and $\mathcal{Y}$ leads to
\begin{equation}
= - \mathds{P}(y = 1| f_G(l)) log \mathds{P}(y = 1 | f_G(l))
- \mathds{P}(y = 0| f_G(l)) log \mathds{P}(y = 0 | f_G(l)).
\end{equation}
And since in a binary classification the probability of one class can be expressed by the counter probability of the other class 
$\mathds{P}(y = 0| f_G(l)) = 1 - \mathds{P}(y = 1 | f_G(l))$.
\begin{align*} 
&= - \mathds{P}(y = 1| f_G(l)) log \mathds{P}(y = 1 | f_G(l))\\
  &\hspace{4mm}- ((1 - \mathds{P}(y = 1 | f_G(l))) log(1 - \mathds{P}(y = 1 | f_G(l)))) \\
&= - \mathds{P}(y = 1| f_G(l)) log \mathds{P}(y = 1 | f_G(l)) \\
   &\hspace{4mm}- (log(1 - \mathds{P}(y = 1 | f_G(l))) - \mathds{P}(y = 1 | f_G(l)) log(1 - \mathds{P}(y = 1 | f_G(l)))) \\
&= - \mathds{P}(y = 1| f_G(l)) log \mathds{P}(y = 1 | f_G(l)) \\
   &\hspace{4mm}- log(1 - \mathds{P}(y = 1 | f_G(l))) + \mathds{P}(y = 1 | f_G(l)) log(1 - \mathds{P}(y = 1 | f_G(l))) 
\end{align*}
As can be seen from the graph of the entropy function (\Autoref{fig:sampling_distributions}) it has the high point of the function at $P(y = 1 | f_G(l)) = 0.5$ like the other uncertainty measures.
Therefore, at this point the model is most uncertain about the classification between negative ($y = 0$) and positive triple($y = 1$).
However, it differs to the other metrics since it slopes flatter in both directions ($P(y = 1 | f_G(l)) = 0$ and $P(y = 1 | f_G(l)) = 1$).
Therefore, if triples are sampled by sampling method \ussoftmax, it is more likely that triples will be sampled that are further away from $P(y = 1 | f_G(l)) = 0.5$ than with the other uncertainty measures.

\textbf{Least Confidence} 
measures the distance between the most confident prediction and 100\% confidence.
According to the definition of least confidence (\Autoref{eqn:least_confidence_def}) the transformation function $\phi$ for this uncertainty measure is defined as
\begin{equation}
    \phi(x) = (1 - \max_{y \in \mathcal{Y}}{\mathds{P}(y | x)}) \cdot \left(\frac{n}{n-1}\right).
\end{equation}
It is normalized for $n$ classes to an output between 0 and 1.
Again, $x = f_G(l)$ for a triple $l = (h,r,t)$ can be inserted and since we have a binary classification with $n = 2$ classes and $\mathcal{Y} = \{0,1\}$, the equation can be rewritten to
\begin{equation} \label{eq:leastconfidence}
\begin{split}
\phi_1(l) 
& = (1 - \max({\mathds{P}(y = 1| f_G(l)), \mathds{P}(y = 0| f_G(l))})) \cdot 2 \\
&= 2 - 2 \max({\mathds{P}(y = 1| f_G(l)), \mathds{P}(y = 0| f_G(l))}).
\end{split}
\end{equation}
In contrast to entropy, the graph slopes much faster in the direction of $\mathds{P}(y | f_G(l)) = 0$ and $\mathds{P}(y | f_G(l)) = 1$, so that triples with lower uncertainty of the model are also sampled with lower probability with sampling method \textsc{USSoftmax}.
\clearpage

\textbf{Margin of Confidence}
deals with the two most confident predictions $y_m = \argmax_{y \in \mathcal{Y}} \mathds{P}(y | x)$ and $y_n = \argmax_{y \in \mathcal{Y} \setminus y_m}{\mathds{P}(y | x)}$.
Given the definition of margin of confidence (\Autoref{eqn:margin_of_confidence_def}), transformation function $\phi$ for this uncertainty measure results in
\begin{equation}
    \phi(x) = 1 - (\mathds{P}(y_m |x) - \mathds{P}(y_n | x)).
\end{equation}
To unify it with the transformation functions of the other uncertainty measures, their difference is subtracted from 1.
For triple $l = (h,r,t)$ and $x = f_G(l)$ and due to the fact that we have a binary classification problem with either $y_m = 1$ and $y_n = 0$ or vice versa, there are two different cases.
For first case of $y_m = 1$ and $y_n = 0$ this results in
\begin{equation}
\begin{split}
\phi_1(l) 
&= 1 - (\mathds{P}(y = 1 |f_G(l)) - \mathds{P}(y = 0 | f_G(l))) \\
&= 1 - (\mathds{P}(y = 1 |f_G(l)) -  (1 - \mathds{P}(y = 1 | f_G(l))) \\
&= 1 - (2 \mathds{P}(y = 1 |f_G(l)) - 1) \\
&= 2 - 2 \mathds{P}(y = 1 |f_G(l))
\end{split}
\end{equation}
For second case of $y_m = 0$ and $y_n = 1$ we have  $\phi_2 = 2 - 2 \mathds{P}(y = 0 |f_G(l))$. 
And therefore this leads to overall definition of $\phi(x)$ for margin of confidence
\begin{equation} \label{eq:marginofconfidence}
    \phi(l) = 2 - 2 \max(\mathds{P}(y = 0 |f_G(l)), \mathds{P}(y = 1 |f_G(l))).
\end{equation}
Since the equations of least confidence (\Autoref{eq:leastconfidence}) and margin of confidence (\ref{eq:marginofconfidence}) are equal, sampling according to these uncertainty measures in a binary classification is equal.
This is also evident when looking at the function graphs depicted in \Autoref{fig:sampling_distributions}.
Therefore, for this case of binary classification, the same uncertainty scores are assigned and \usmax as well as \ussoftmax sample triples with the same probability.

\textbf{Ratio of Confidence}
is quite similar to margin of confidence, but instead of the difference it compares the ratio of the two most confident predictions
\begin{equation}
    \phi(x) = \frac{\mathds{P}(y_m | x)}{\mathds{P}(y_n |x)}
\end{equation}
where $y_m$ and $y_n$ are equally defined as in margin of confidence.
Inserted $x = f_G(l)$ for a triple $l = (h,r,t)$ leads to
\begin{equation}
    \phi(l) = \frac{\mathds{P}(y_m |  f_G(l))}{\mathds{P}(y_n |f_G(l))}.
\end{equation}
Looking at the function graph of ratio of confidence in \Autoref{fig:sampling_distributions}, due to the ratio between the two most confident predictions $y_m$ and $y_n$, there is an even stronger focus on sampling only uncertain triples with \ussoftmax around $\mathds{P}(y = 1 | f_G(l)) = 0.5$.

\section{Procedure of Adversarial Training with Sampling by Uncertainty}
\label{sec:procedure}

In the original \kbgan approach two embedding models are available in form of the generator and discriminator.
Therefore, we could either sample by the uncertainty of the generator or the discriminator embedding model.
In \kbgan approach the negative triple set $Neg$ is created before they are given to the generator.
Subsequently they are sampled by probability distribution and given to discriminator.
Therefore, since it corresponds to the original course, we choose the uncertainty in the prediction of the generator and the original sampling by probabilities calculated by the scores of the generator is replaced by sampling from uncertainty.
The correspondingly modified architecture and the sequence of the sampling process is shown in \Autoref{fig:uncertainty_sampling_architecture}.
\begin{figure}[H]
  \centering
    \includegraphics[width=0.90\textwidth]{figures/ucgan_architecture.pdf}
  \caption{Uncertainty Sampling replaces the original sampling in \ac{KBGAN}. 
  The new uncertainty sampling component is highlighted in green color.}
  \label{fig:uncertainty_sampling_architecture}
\end{figure}
The changed elements are marked in green.
Accordingly, these elements change the adversarial learning procedure.
As in the original \kbgan our approach includes two components of a generator and a discriminator which are \ac{KGE} models that can be either pre-trained or not.
In \Autoref{fig:uncertainty_sampling_architecture} steps 1 to 8 are marked which will be described in the following 
The training process can be described in the following steps
which are described in more detail below.
To make the difference between the original model and our approach clear, we will use an example to explain this process.

\textbf{1. Creation of Negative Triple Set Neg}\\
For each epoch in the training process a set of negative triples is created.
This set, which is defined by $Neg(h,r,t)=\{(h_i',r,t_i')\}_{i=1\dots N_s}$, is created by Bernoulli sampling of $N_s$ negative triples.
Therefore, either head or tail entity is replaced with different probabilities according to the mapping property of relations and entities.
In comparison to Random Sampling, it applies higher probabilities to head entity replacement in one-to-many relations and tail in many-to-one relations \cite{qiannegative}.
For each epoch and for each positive triple $(h,r,t) \in KG$ a negative triple set is created.
In \kbgan as well as our approach the size of $Neg$ is set to twenty.

In our example for illustration purposes, we only consider a negative triple set $Neg$ of size five.
\Autoref{tab:neg_example} shows five different negative triples $l_i$ with $i \in [1,5]$.
For simplicity for a positive triples $(h,r,t) = (\text{})$
only tail entities are replaced to obtain $l_i$. 
\begin{table}[h]
    \centering
    \begin{tabular}{llllll}
        \toprule
        
        \textbf{Tail entities}
        & \textbf{$t_1$} & \textbf{$t_2$} & \textbf{$t_3$} & \textbf{$t_4$} & \textbf{$t_5$} \\
         
        & Florida
        & China
        & BarackObama
        & StarTrek
        & Google  \\

        \bottomrule
    \end{tabular}
    \caption{Example: For a given positive triple $(h,r,t)$ five different negative triples were created by replacing the tail entities to obtain negative triple set 
    $Neg = \{(h,r,t_i')\}$ for $i \in [1,5]$.}
\label{tab:neg_example}
\end{table}

\textbf{2. Calculating Minimum and Maximum Score for Positive and Negative Triples}\\
%
To be able to carry out the classification in the later uncertainty sampling process, the scores of all positive triples of the \ac{KG} as well as the negative triples of $Neg$ must be calculated before the actual adversarial training.
For this purpose, the corresponding triples are passed to the generator model, and, based on the embeddings trained so far, a score for all triples is calculated with the scoring function $f_G$.
Assuming the \ac{KG} consists of N positive triples and for each triple, a negative triple set of size $N_s$ was first created. 
Consequently, scores of N positive triples and $N \cdot N_s$ negative triples are calculated to obtain the minima and maxima of both positive and negative triples.
Our example results in the scoring ranges shown in \autoref{fig:scoring_ranges}.
\begin{figure*}[t]
  \centering
    \includegraphics[width=0.75\textwidth]{figures/scoremin_scoremax_example.pdf}
  \caption{Score range of positive and negative triples.
  The uncertainty of a model is in the scoring range where the 
  model found scores of both positive and negative triples.}
  \label{fig:scoring_ranges}
\end{figure*}
For negative triples, we obtained a score range of $[-9.1, 2.2]$, for positive triples a score range of $[-4, 10.7]$ and overlap between $[-4, 2.2]$.
Therefore, $score_{min} = -9.1$ and $score_{max} = 10.7$.
Since we have a fixed negative triple set for each epoch, they only need to be calculated once before each epoch.

\textbf{3. Calculate Score for all Negative Triples in Neg}\\
% DESCRIPTION
At this point, the adversarial training process between generator and discriminator starts.
The goal of this step is to calculate the scoring values for all negative triples.
Thus, in this step, all negative triples are passed to the generator scoring function $f_G$.
Since the generator is a tensor factorization-based model, the score obtained reflects the plausibility of each negative triple to be true.
After all scoring values of the negative triples have been calculated, they are given to the Uncertainty Sampling component.

% EXAMPLE - ORIGINAL APPRObACH FOR COMPARISON
In comparison, in the original \kbgan approach the sampling probabilities are directly derived from the scoring values. 
According to the example, a score for each triple $l_i' = (h,r,t_i')$ is calculated first.
Since all scores $f_G(l) \in \mathbb{R}$, they to not sum up to one.
Therefore, the softmax function is used to obtain sampling probabilities $p_i$ for each negative triple $l_i$, where each $p_i$ is defined as
\begin{equation}
    p_i = \frac{\exp{f_G(l_i)}}{\sum_{j=1}^{N_s}{\exp{f_G(l_j)}}}.
\end{equation}
Consequently, negative triples with high scores get a high probability to be sampled.
For our examples, this results in the values shown in \Autoref{tab:generator_scores}.
\begin{table}[h]
    \centering
    \begin{tabular}{llllll}
        \toprule
        
        \textbf{Tail entities}
        & \textbf{$t_1$} & \textbf{$t_2$} & \textbf{$t_3$} & \textbf{$t_4$} & \textbf{$t_5$} \\
         
        & Florida
        & China
        & BarackObama
        & StarTrek
        & Google  \\

        \midrule
        
        \textbf{Generator scores for $l_i = (h_i, r_i, t_i)$}
        & -5.916 
        & -2.249  
        & 3.719 
        & -0.761 
        & -6.942 \\
        
         \midrule
        
        \textbf{Original probabilities of $l_i$ to be sampled}
        & 1.05\%
        & 0.25\% 
        & 98.62\%  
        & 0.01\% 
        & 0.07\%
        \\
        \bottomrule
    \end{tabular}
    \caption{Example: For a given positive triple $(h,r,t)$ five different negative triples were created by replacing the tail entities to obtain negative triple set 
    $Neg = \{(h,r,t_i')\}$ for $i \in [1,5]$.
    For all negative triples a score $s_i$ in calculated by generator G scoring function $s_G$: $s_G(h,r,t_i') = s_i$.
    Therefore, tail entities $t_i$ for negative triples $(h_i, r_i, t_i)$ with $i \in [0,5]$ given by the generator model G with scoring function $s_G$ }
\label{tab:generator_scores}
\end{table}
According to our example, the negative triple $l_3' = (h,r,t_3)$ achieves the highest score and, therefore, obtains the highest probability to be sampled.


% EXAMPLE NEW APPROACH
Since our approach needs the scores to classify the triples, instead of probabilities  the output of generator G are scores $s_i \in \mathbb{R}$ for each negative triple in $Neg$.
Therefore, the original sampling technique is replaced by an Uncertainty Sampler which samples one triple based on the model's uncertainty.




\textbf{4. Sampling of an Uncertain Negative Triple}\\

All calculated scores for negative triples from generator are handed to the Uncertainty Sampler to calculate the uncertainty of the model and sample a triple where the model is uncertain how to classify it.
As shown in \Autoref{fig:uncertainty_sampling_component}, the Uncertainty Sampler is divided into three sub-components and steps.
\begin{figure*}[t]
  \centering
    \includegraphics[width=0.90\textwidth]{figures/uncertainty_sampling_component.pdf}
  \caption{Uncertainty Sampling component which contains a Classifier, an Uncertainty Scorer and an Uncertainty Sampler. Uncertainty Sampler can sample either according to Uncertainty Max or Uncertainty Distribution. Input of the Uncertainty Sampling component are scores of triples and the output is a negative triple sampled by uncertainty.}
  \label{fig:uncertainty_sampling_component}
\end{figure*}
It contains a \textbf{Classifier}, an \textbf{Uncertainty Scorer} and a \textbf{Sampler} component.
These are explained in more detail below.

\textbf{Classifier} \\
The Classifier's task is to classify the negative triples by assigning each negative triple a probability of being a positive triple.
It receives the negative triple scores $s_i \in \mathbb{R}$, which are either very high or very low logit values from the generator as input and returns a probability for each negative triple $(h,r,t) \in Neg$ to be positive  ($\mathbb{P}(y = 1| (h,r,t))$) (\Autoref{eqn:positive_probability}) as output. 

For our example, in \Autoref{tab:positive_probabilities} generator scores for all triples $l_i = (h_i, r_i, t_i)$ are listed which are the input of the Classifier.
For these triples and their generator score probabilities are derived for being a positive triple.
\begin{table}[H]
    \centering
    \begin{tabular}{llllll}
        \toprule
        
        \textbf{Tail entities}
        &  \textbf{$t_1$} & \textbf{$t_2$} & \textbf{$t_3$} & \textbf{$t_4$} & \textbf{$t_5$} \\
         
        \midrule
        
        \textbf{Generator scores}
         & $s_1$ & $s_2$ & $s_3$ & $s_4$ & $s_5$ \\
        
        \textbf{for $l_i' = (h_i, r_i, t_i)$}
        & -5.916 
        & -2.249  
        & 3.719 
        & -0.761 
        & -6.942 \\
        
        \midrule
                        
        \textbf{Probabilities of being} 
        & $p_1$ & $p_2$ & $p_3$ & $p_4$ & $p_5$ \\
        
        \textbf{a positive triple}
        & 16.09\% 
        & 34.54\% 
        & 64.57\% 
        & 42.03\% 
        & 10.93\%  \\

        \bottomrule
    \end{tabular}
    \caption{Example: For all negative triples $l_i' = (h, r, t_i)$ scores $s_i$ for $i \in [1,5]$ are calculated by generator G and given to the Classifier component.
    From these scores probabilities $p_i$ for $i \in [1, 5]$ for being a positive one are calculated.}
\label{tab:positive_probabilities}
\end{table}


\textbf{Uncertainty Scorer} \\
The task of the Uncertainty Scorer is to assign uncertainty values for all negative triples.
As input it receives the probabilties from the Classifier of each triple to be a positive one.
As output it returns uncertainty scores $u_i \in [0,1]$ for $i \in [1, N_s]$ for all negative triples $l_i = (h_i, r_i, t_i)$ where $u_i = 0$ indicates low and $u_i = 1$ high uncertainty.
As mentioned in \Autoref{sec:calculation_of_uncertainty_scores},  there are several Uncertainty Metrics at our disposal:
Entropy, Least Confidence, Margin of Confidence and Ratio of Confidence.
One of these Uncertainty metrics is chosen according to configuration and for each triple the model´s uncertainty is measured.
The closer a probability of a negative triple to be a positive one is to 0.5, the higher is the uncertainty score. 
Uncertainty scores $u(l)$ are calculated with \Autoref{eqn:uncertainty_function}.
For our example, this means that the Uncertainty Scorer calculates the uncertainties listed in \Autoref{tab:uncertainty_scores} for the individual uncertainty metrics.
\begin{table}[H]
    \centering
    \begin{tabular}{llllll}
        \toprule
        
        &  \textbf{$t_1$} & \textbf{$t_2$} & \textbf{$t_3$} & \textbf{$t_4$} & \textbf{$t_5$} \\
         
        \midrule
        
        \textbf{Probabilities of being}
         & $p_1$ & $p_2$ & $p_3$ & $p_4$ & $p_5$   \\
         
        \textbf{a positive triple}
        & 16.09\% 
        & 34.54\% 
        & 64.57\%
        & \underline{42.03\%} 
        & 10.93\%  \\
        
        \midrule
        \textbf{Uncertainty scores}
        & $u_1$ & $u_2$ & $u_3$ & $u_4$ & $u_5$ \\
        
        Entropy 
        & 0.6364 & 0.9299 & 0.9378 & \underline{0.9816} & 0.4977 \\
        
        Least confidence 
        & 0.3218 & 0.6909 & 0.7086 & \underline{0.8406} & 0.2185 \\
        
        Confidence margin
        & 0.3218 & 0.6909 & 0.7086 & \underline{0.8406} & 0.2185 \\
        
        Confidence ratio
        & 0.19174 & 0.5277 & 0.5487 & \underline{0.7250} & 0.1227\\
        
        \bottomrule
    \end{tabular}
    \caption{Example: For all negative triples $l_i' = (h, r, t_i)$ an probability $p_i$ to be positive was calculated. Uncertainty Scorer component retrieves these probabilities as input and calculates uncertainty scores $u_i$ for $i \in [1, 5]$ with different uncertainty measures.}
\label{tab:uncertainty_scores}
\end{table}
As you can see from the example, $l_4 = (h_4, r_4, t_4)$ has a probability of 42.03\% which is closest to 0.5 among all triples.
For each triple the uncertainty scores are calculated for all uncertainty metrics and the highest score is marked in bold.
As you can see, although the uncertainty scores are different, $l_4$ receives the highest value for all four uncertainty metrics.
These uncertainty values $u_i$ are now passed to the Sampler component.


\textbf{Sampler} \\
The actual sampling of a negative triple takes place within this component, which is then passed on to the discriminator.
It receives uncertainty scores for all negative triples from Uncertainty Scorer as input and returns one negative triple as output.
As mentioned in \Autoref{sec:calculation_of_uncertainty_scores} we have two methods at our disposal how to sample an uncertain triple.
Either Uncertainty Max which always samples the triple where model is most uncertain or Uncertainty Distribution which samples according to a distribution based on uncertainty scores.
One triple is sampled according to configured sampling method and returned to discriminator where the next step takes place.

If we sample by Uncertainty Max this means selecting only the triple with the maximum uncertainty score.
In our example, this would always be the triple $l_4$ for all uncertainty metrics.
If we sample by Uncertainty Distribution method, we have to calculate sampling probabilities based on uncertainty scores first and then pick one triple according the probability distribution.
In \Autoref{fig:uncertainty_metrics_example_distribution}, for all triples and all uncertainty metrics the sampling probabilities are displayed.
\input{content/chapters/03_approach/sections/table_uncertainty_metrics_example_distribution}
For all different metrics $t_4$ is still the most probable one to be sampled, but the uncertainty sampling metrics differ in their probability to sample $t_4$.
E.g. while based on confidence ratio, $t_4$ is sampled with a probability of 26.36\% it is only sampled with probability of 23.63\% for entropy uncertainty metrics.
The more negative triples $t_i$ exist, the more these sampling probabilities differ.






\textbf{5. Passing Original Positive Triple and Sampled Negative Triple to Discriminator Model D}\\
Subsequently, negative triples are sampled according to the calculated uncertainty scores, either the one with maximum uncertainty score or its distribution and given to the discriminator.


\textbf{6. Calculation of Distances}\\
From this step on, everything is the same as the original \ac{KBGAN} approach:
The generated negative triple $(h',r,t')$ as well as the positive triple $(h, r, t)$ are sent to the discriminator.

\textbf{7. Calculation of Loss between Positive and Negative Triple}\\

The aim is to improve the discriminator model, which is achieved by minimising the marginal loss.
The marginal loss is defined as in \Autoref{eq:marginalloss2}.
Therefore, all positive triples $(h,r,t) \in \mathcal{T}$, where $\mathcal{T}$ is the set of all positive triples, and their sampled negative triple $(h',r', t')$  are passed to the marginal loss function $L_D$ of the discriminator
\begin{multline} \label{eq:marginalloss2}.
    L_D =\sum_{(h,r,t)\in\mathcal{T}}[f(h,r,t)-f(h',r,t')+\gamma]_+\\
    (h',r,t') \sim p_G(h,r,t|h,r,t) 
\end{multline}
while $p_G(h', r, t'|h, r, t)$ is the probability distribution on negative triples and defined as \cite{cai2017kbgan}
\begin{multline}
    p_G(h',r,t'|h,r,t)=\frac{\exp f_G(h',r,t')}{\sum\exp f_G(h^*,r,t^*)} \\
    (h^*,r,t^*)\in Neg(h,r,t)
\end{multline}
\textcolor{red}{UPDATE: EITHER UNCERTAINTY MAX OR UNCERTAINTY DISTRIBUTION PROBABILITY DISTRIBUTION}
Therefore, discriminator loss is minimized if difference between positive and negative triple score is low and in optimal case zero.
The sampled negative triple $(h',r,t')$ was picked based on the uncertainty of the generator model.
By sampling such an uncertain negative triple from generator, we hope to catch an uncertain triple for the discriminator as well.


\textbf{8. Passing Reward to Discriminator based on Distance of Negative Triple}\\
The reward defined by $r = - f_D(h',r,t')$ of the current negative triple is calculated and returned to the Generator and the next training iteration begins.

- reward is used for training of Generator
- 

These adversarial training steps are repeated until determined number of epochs is reached, such that the generator improves the quality of sampled negative triples and discriminator improves embedding over time.
Output of our algorithm is the adversarial trained discriminator and its embedding.





\section{Improving Classification by Adding More Information} 
\label{sec:improving_classification}

Until now, only information of the scoring function of the generator is used for the classification of the triple as either positive or negative.
However, since this only contains implicit information about, for example, the structure of the \ac{KG} and its entities and relations, the idea is to supplement this classification with further, explicit information.
Therefore, the idea is to include additional functions for classification in the form of the following functions: 

Assuming we have the function $peergroup(e) \subseteq \mathcal{KG}$ which returns a set of peers for given entity $e \in \entities_{\mathcal{T}}$ by its embedding. 
Then the $Generator\_Score$ function contains the following features from \cite{arnaout2020enriching}:
\begin{itemize}
    \item 
    \emph{\ac{PEER}:} 
    Relative frequency of peers within a peer group that is related to different objects, e.g. 0.9 of persons are married. 
    \begin{equation}
        PEER(h,r) = \frac{|\{p | p \in peergroup(h), (h, r, t) \in \mathcal{T}\}|}{|peergroup(h)|}
    \end{equation}

    \item
    \emph{\ac{POP}:} 
    The popularity of the tail entity $t$ in $\mathcal{T}$. 
    \begin{equation}
        POP(t) = \frac{|\{t | (h, r , t) \in \mathcal{T}\}|}{|\mathcal{T}|}
    \end{equation}

    \item 
    \emph{\ac{FRQ}:} 
    Frequency of a relation/predicate $r$ in $\mathcal{T}$. 
    \begin{equation}
        FRQ(r) = \frac{|\{r | (h, r, t) \in \mathcal{T}\}|}{|\mathcal{T}|}
    \end{equation}
    
    \item 
    \emph{\ac{PIVO}:} 
    Textual background information about an entities $h$ and $t$ and is a pivoting classifier like in \cite{arnaout2020enriching}.
    
    \item 
    \emph{$f_G$:} 
    $Generator\_Score$ function of the \ac{KGE} model from generator like in the original \ac{KBGAN} approach (e.g. \textsc{DistMult}  or \textsc{ComplEx}).
\end{itemize}


To incorporate these information in the score function than just the embedding of a triple, we want to extend the original score function $f_G$ by several information about entities and relations in the \ac{KG}.
Therefore, our new score function $Classification\_Score$ is defined as follows:
\begin{equation}
    Classification\_Score(h, r, t)=
    \begin{cases}
         \lambda_1 \text{PEER(\textit{h, r})} + \lambda_2 \text{POP(\textit{t})} + \lambda_3 \text{PIVO(\textit{h, t})} + \lambda_4 f_G(\textit{h, r, t})
         \\ \ \ 
         if\ \ \ \neg(\textit{h, r, t})
         \\ \\
         \lambda_1 \text{PEER(\textit{h, r})} + \lambda_2 \text{FRQ(\textit{r})} + \lambda_3 \text{PIVO(\textit{h, t})} + \lambda_4 f_G(h, r, t)
         \\ \ \ 
         if\ \ \ \neg \exists (\textit{h, r, \_})
         \\
    \end{cases}
\end{equation}
where $\lambda_i \in [0, 1]$ for $i \in [1,4]$  are hyperparameters and $\sum_{i=1}^{4}\lambda_i = 1$.
Initially they are set to randomly generated values but can be optimized at a later stage.
Furthermore, $\neg (h, r, t)$ is a \textit{grounded negative statement} and is satisfied if $(h, r, t) \notin$ \ac{KG} and $\neg\exists(h, r, \_)$ is a \textit{universally negative statement} which is satisfied if there exists no $t$ such that $(h, r, t) \in KG$ \cite{arnaout2020enriching}.
\textit{PEER}, \textit{POP}, \textit{PIVO} and \textit{FRQ} are functions that provide additional information about the \ac{KG}, its entities and relations such that the $Generator\_Score$ can make more reliable statements about whether the given triple $(h, r, t)$ is positive or not.
However, these components of the function can also be replaced or extended by further components known from for example Cluster-Based Sampling methods that give more information about a \ac{KG} and its structure.
For example, the K-Means algorithm \cite{qianunderstanding} or other techniques known from Relational Sampling, Nearest Neighbor Sampling or Near miss Sampling can be considered \cite{kotnis2017analysis}.




\textcolor{red}{TODO: filtered negatives which appear in the KG}