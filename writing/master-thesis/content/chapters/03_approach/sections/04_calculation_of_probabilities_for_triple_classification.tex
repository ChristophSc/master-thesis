\section{Calculation of Probabilities for Triple Classification}
\label{sec:calculation_of_probabilities_for_triple_classification}
Having established the binary classification problem between positive and negative triples, it is necessary to determine how to calculate for a triple the probability of belonging to one of these classes.
Based on \Autoref{fig:positiveVsNegativeApproach} with ranges of positive and negative triple scores, we want to determine two bounds.
Assuming we have a generator scoring function $f_G$ which retrieves scoring values for each triple $l = (h,r,t)$ such that $f_G(l) \in \mathbb{R}$.
Therefore, to provide probabilities belonging to positive or negative triple class. we need to define two boundaries:
First, $score_{min}$ which defines at which scoring value a triple has the probability 0 to be a positive triple: 
\begin{equation} \label{eqn:prob_score_min}
    \mathds{P}(y = 1 | f_G(l) \leq score_{min}) = 0
\end{equation}
and second $score_{max}$ at which scoring value a triple has the probability 1 to be positive one 
\begin{equation} \label{eqn:prob_score_max}
    \mathds{P}(y = 1 | f_G(l) \geq score_{max}) = 1
\end{equation}
Accordingly, for all other scoring values $score_{min} \leq f_G(l) \leq score_{max}$
\begin{equation} \label{eqn:prob_score_all}
    \mathds{P}(y = 1 | score_{min} \leq f_G(l) \leq score_{max}) \in [0, 1]
\end{equation}
Since we have a binary classification, counter probabilities are given by:
\begin{equation} \label{eqn:counter_prob}
    \mathds{P}(y = 0 | f_G(l)) = 1 - \mathds{P}(y = 1 | f_G(l)) \in \mathbb{R}
\end{equation}

\textbf{Determination of boundaries $\textbf{score}_{\textbf{min}}$ and $\textbf{score}_{\textbf{max}}$}\\
Based on \Autoref{fig:positiveVsNegativeApproach}, we have following two options for determination of scoring boundaries.

% use borders of negatives/positives for classification
First option to define these borders is depicted in \Autoref{fig:positives_negatives1}.
\begin{figure*}[t]
  \centering
    \includegraphics[width=0.75\textwidth]{figures/positives_negatives1.pdf}
  \caption{First option to define probabilities of triples to be positive ($\mathds{P}(y = 1 | f_G(l))$ ) or negative ($\mathds{P}(y = 0 | f_G(l))$). 
  $score_{min}$ at the left border of all positive triple scores and $score_{max}$ at the right border of all negative triples.}
  \label{fig:positives_negatives1}
\end{figure*}
and results in the following definitions of  $score_{min}$ and 
$score_{max}$:
\begin{equation} \label{eqn:opt2_score_min}
    score_{min} = \argmin_{(h,r,t) \in \mathcal{KG}}{f_G(h,r,t)}
\end{equation}
\begin{equation} \label{eqn:opt2_score_max}
    score_{max} = \argmax_{(h',r,t') \in Neg}{f_G(h',r,t')}
\end{equation}
This option has the advantage of focusing only on the overlapping part of the model's prediction.
However, the drawback is that several triples of the negative triple set $Neg$ are disregarded.
If negative and positive scoring ranges do not overlap, all negative triples from $Neg$ would get a probability of zero to be a positive one and therefore, according to this option the model would not be uncertain about classification of all negative triples in $Neg$ and is therefore not considered for our approach.

The other option is showed in \Autoref{fig:positives_negatives2}.
\begin{figure*}[t]
  \centering
    \includegraphics[width=0.75\textwidth]{figures/positives_negatives2.pdf}
  \caption{Second option to define probabilities of triples to be positive ($\mathds{P}(y = 1 | f_G(l))$) or negative ($\mathds{P}(y = 0 | f_G(l))$). 
  $score_{min}$ at the left border of all negative triple scores and $score_{max}$ at the right border of all positive ones.}
  \label{fig:positives_negatives2}
\end{figure*}
In comparison to the first option, this one provides a probability of being positive to almost every negative triple,
no matter how the ranges of negative and positive triple scores are arranged.
With $f_G$ the scoring function of the generator, this option results in the following definitions of $score_{min}$ and $score_{max}$:
\begin{equation} \label{eqn:opt1_score_min}
    score_{min} := \argmin_{(h',r,t') \in Neg}{f_G(h',r,t')}
\end{equation}
\begin{equation} \label{eqn:opt1_score_max}
    score_{max} := \argmax_{(h,r,t) \in \mathcal{KG}}{f_G(h,r,t)}
\end{equation}

Since $score_{min}$ and $score_{max}$ are now defined by option two, the probability of a triple to be positive ($y = 1$) can be calculated with following \Autoref{eqn:positive_probability}.
\begin{equation}  \label{eqn:positive_probability}
    \mathds{P}(y = 1|(h, r, t)) =
    \begin{cases}
        0 \ \ \ \ \ \ \ \ \ \ \ \ \ \ \ \ \ \ \ \ \ \ \ \ \ \ \  \ \   
         \text{if} \  f_G(h,r,t) \leq score_{min}
         
        \\ \\
        1 \ \ \ \ \ \ \ \ \ \ \ \ \ \ \ \ \ \ \ \ \ \ \ \ \ \ \   \ \ \
        \text{if} \ f_G(h,r,t) \geq score_{max}
         
        \\ \\
        \frac{f_G(h,r,t) - score_{min}}{score_{max} - score_{min}}
        \ \ \ \ \ \ 
         \text{else}
        \\
    \end{cases} 
\end{equation}
while $\mathds{P}(y = 1|(h, r, t))  \in [0, 1]$.
Since we have a binary classification problem for positive and negative triples, the probability of a triple to be negative ($y=0$) is
\begin{equation} \label{eqn:negative_probability}
    \mathds{P}(y = 0 | f_G(h, r, t)) = 1 - \mathds{P}(y = 1| f_G(h, r, t)) \in [0,1]
\end{equation}


