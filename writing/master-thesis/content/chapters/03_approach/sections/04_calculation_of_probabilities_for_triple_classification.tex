\section{Calculation of Probabilities for Triple Classification}
\label{sec:calculation_of_probabilitie_for_triple_Classification}
% how to measure the probability of a triple to be positive
Before we can calculate and sample negative triples from Neg by Uncertainty, we have to calculate the probability of a triple being positive.
Based on \autoref{fig:uncertainty} with ranges of positive and negative triple scores, we want to determine two bounds: 
First, $score_{min}$ which defines at which scoring value a triple has the probability 0 to be a positive triple ($\mathds{P}(y = 1 | l = (h,r,t)) = 0$) and second $score_{max}$ at which scoring value a triple has the probability 1 to be positive one ($\mathds{P}(y = 1 | (h,r,t)) = 1$) . 
All scores lower than $score_{min}$ get the probability 0 and all scores higher than $score_{max}$ get the probability 1.

% use borders of negatives/positives for classification
First option to define these borders is depicted in \autoref{fig:positives_negatives1}.
\begin{figure*}[t]
  \centering
    \includegraphics[width=0.75\textwidth]{figures/positives_negatives1.PNG}
  \caption{First option to define probabilities of triples to be positive ($\mathds{P}(y = 1 | (h,r,t))$ ) or negative ($\mathds{P}(y = 0 | (h,r,t))$). 
  $score_{min}$ at the left border of all negative triple scores and $score_{max}$ at the right border of all positive triples.}
  \label{fig:positives_negatives1}
\end{figure*}
The advantage of this option is, that almost every negative triple is assigned a probability between 0 and 1 and therefore, an uncertainty score higher 0.
This results in a probability for almost all negative triples to be sampled.
The disadvantage is that depending on the shift of the positive and negative triple scoring ranges this would result in bad negative triples or too good negative triples being sampled.
With $f_G$ the scoring function of the generator, the first option results in the following definitions:
\begin{equation} \label{eqn:opt1_score_min}
    score_{min} := \argmin_{(h',r,t') \in Neg}{f_G(h',r,t')}
\end{equation}
\begin{equation} \label{eqn:opt1_score_max}
    score_{max} := \argmax_{(h,r,t) \in \mathcal{KG}}{f_G(h,r,t)}
\end{equation}

The other option is showed in \autoref{fig:positives_negatives2}.
\begin{figure*}[t]
  \centering
    \includegraphics[width=0.75\textwidth]{figures/positives_negatives2.PNG}
  \caption{Second option to define probabilities of triples to be positive ($\mathds{P}(y = 1 | (h,r,t))$) or negative ($\mathds{P}(y = 0 | (h,r,t))$). 
  $score_{min}$ at the left border of all positive triple scores and $score_{max}$ at the right border of all negative.}
  \label{fig:positives_negatives2}
\end{figure*}
In comparison to the first option, this one has the advantage of focusing only on the overlapping part of the model's prediction.
However, the drawback is that this static view becomes problematic when the scoring range of positive and negative triples does not overlap.
This option results in the following definitions:
\begin{equation} \label{eqn:opt2_score_min}
    score_{min} := \argmin_{(h,r,t) \in \mathcal{KG}}{f_G(h,r,t)}
\end{equation}
\begin{equation} \label{eqn:opt2_score_max}
    score_{max} := \argmax_{(h',r,t') \in Neg}{f_G(h',r,t')}
\end{equation}

Since $score_{min}$ and $score_{max}$ are now defined either with first option of 
\autoref{eqn:opt1_score_min} and \ref{eqn:opt1_score_max} or according the second option with 
\autoref{eqn:opt2_score_min} and \ref{eqn:opt2_score_max}, the probability of a triple to be positive ($y = 1$) can be calculated with the following \autoref{eqn:positive_probability}
\begin{equation}  \label{eqn:positive_probability}
    \mathds{P}(y = 1|(h, r, t)) =
    \begin{cases}
        0 \ \ \ \ \ \ \ \ \ \ \ \ \ \ \ \ \ \ \ \ \ \ \ \ \ \ \  
        if \  f_G(h,r,t) < score_{min}
         
        \\ \\
        1 \ \ \ \ \ \ \ \ \ \ \ \ \ \ \ \ \ \ \ \ \ \ \ \ \ \ \ 
        if \ f_G(h,r,t) > score_{max}
         
        \\ \\
        \frac{f_G(h,r,t) - score_{min}}{score_{max} - score_{min}}
        \ \ \ \ \ \ \ 
        else
        \\
    \end{cases}  \in [0, 1]
\end{equation}
and accordingly, since we have a binary classification problem for positive and negative triples, the probability of a triple to be negative ($y=0$) is
\begin{equation} \label{eqn:negative_probability}
    \mathds{P}(y = 0|(h, r, t)) := 1 - \mathds{P}(y = 1|(h, r, t)) \in [0,1]
\end{equation}