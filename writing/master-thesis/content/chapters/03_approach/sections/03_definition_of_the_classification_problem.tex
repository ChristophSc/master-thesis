\section{Definition of a Classification Problem}
\label{sec:definition_of_a_classification_problem}
% uncertainty sampling approaches require a classification problem
% we can either: 
% 1) good <-> bad negative triple (like original approach)
% 2) provide prob of triple to be true 
Next, since Uncertainty Sampling needs a classification problem, there are two options available for the definition:
First, like in the original \kbgan approach, we could distinguish between \textit{good} and \textit{bad} negative triples and try to provide the best ones to the discriminator.
Since the generator of the adversarial training is a tensor factorization-based model, it provides scores which indicate plausibility of a triple.
According to this plausibility scores a sampling distribution among all negative triples is created. 
This has the advantage that we only have to calculate the scores of negative triples, but we lose the information about positive triples and their scores.
Therefore, negative triples are sampled according to their calculated scores with either linearly increasing probability as in \cite{cai2017kbgan} or as in \cite{UKGE} with logistic function or a bounded rectifier.
To follow the open-world assumption and reduce the probability of sampling false-negatives,
\kbgan limits the number of negative triples to twenty \cite{cai2017kbgan}.

Another approach is to not only look at the quality of a negative triple, but instead distinguish between negative and positive triples and their scores.
In contrast to the first option in this one also information about positive triples and their scores is preserved.
If we define the classes $y = 0$  for negative triples and $y = 1$ for positive triples, we can classify each triple $(h,r,t)$ in a binary classification problem.
Accordingly, we can calculate the probability of being either a negative or a positive triple for each triple $\mathbb{P}(y| (h,r,t)) \in [0,1]$ for $y \in \{0,1\}$.
Since we want to have as good negative triples as possible and therefore, triples that can be classified either positive or negative, we look for negative triples $(h',r,t')$ close to the probability of $\mathbb{P}(y = 0| (h',r,t')) = \mathbb{P}(y = 1| (h',r,t')) = 0.5$.
Since this means the triple can be assigned to both the positive and the negative class, 
the model is uncertain how to classify the triple.

The difference between the two approaches can be explained by the following example and \autoref{fig:uncertainty}, which illustrates scoring values of positive and negative triples and the uncertainty of a model´s prediction.
\begin{figure*}[t]
  \centering
    \includegraphics[width=0.75\textwidth]{figures/uncertainty.PNG}
  \caption{Score range of positive and negative triples.
  The uncertainty of a model is in the scoring range where the model found scores of both positive and negative triples.}
  \label{fig:uncertainty}
\end{figure*}
Suppose we have the positive triple $l_p=(h,r,t)$ (t marked as blue dot) in our KG and some other entities $t^* \in \entities$ (green dots) from positive triples $(h,r,t^*)$ from the \ac{KG} .
After we obtain the negative triple set $Neg$ through the Negative Sampling process and scored them by the generator embedding model, also a range of scoring values for negative triples exists. 
Therefore, for entities within the overlap of positive and negative scoring values, the model is uncertain whether they are positive or negative.
The original sampling approach of \kbgan assigns now ascending probabilities for increasing scoring values.
Therefore, it is most likely to sample $t'_3$, because the negative triple $(h,r,t'_3)$ achieves the highest score and, accordingly, has the highest probability to be positive.
Although, the entities $t'_1$ and $t'_2$ are also very interesting for the model, because a classification based on the achieved scoring values is not easy.
Consequently, instead of only including negative triple scoring information by differentiating between \textit{good} and \textit{bad} negative triples, the second approach incorporates information of positive triples.
Thus, this more detailed distinction allows us to avoid sampling false negative triples, since, for example, sampling $t'_3$ can be particularly problematic if it yields an even higher score and should actually be classified as a positive triple because it is a false negative triple.