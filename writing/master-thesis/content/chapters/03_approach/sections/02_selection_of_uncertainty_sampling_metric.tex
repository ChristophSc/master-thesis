\section{Selection of an Uncertainty Sampling Framework} 
\label{sec:selection_of_an_uncertainty_sampling_type}
%
According to the literature, three different frameworks of uncertainty sampling are at our disposal:
\ac{EBU}, \ac{EAU} and \ac{CU}.
These frameworks of uncertainty sampling have different viewpoints on uncertainty and how to measure it, but all of them are based on a classification problem of two or more classes.
In \ac{EBU} we consider uncertainty within a model by looking at the individual features of a model and if they provide evidence for classes.
This can be either very many (\textit{conflicting-evidence}) or only a few features (\textit{insufficient-evidence}) which are indicative for respective classes.
In contrast, \ac{EAU} considers \textit{epistemic uncertainty} within a single model’s prediction, and \textit{aleatoric uncertainty} across multiple model predictions \cite{human-in-the-loop}.
\Ac{CU} differentiates between the reducible and irreducible part of the uncertainty in a prediction and the domination of one class over the other.

% determination of one framework
To determine one of the uncertainty sampling frameworks for our approach, the available information in the original \kbgan approach is considered.
In the adversarial training process, the negative triple set $Neg$ is created and passed to the generator.
The generator scores all negative triples based on current embeddings.
Based on these scores, sampling probabilities are calculated.
According to this probability distribution, one of these negative triples is sampled and given to the discriminator.
Therefore, only information about the embeddings of the generator as well as the discriminator model, all negative triples, and their scores is available.
Consequently, it can be sampled either by the uncertainty of the generator or the discriminator embedding model.
Therefore, since it corresponds to the original course, we choose the uncertainty of the generator model.
The original sampling method according to scores is replaced by sampling with uncertainty in the new approach of \textsc{USGAN}.
For this reason, at first, the framework of epistemic uncertainty from \ac{EAU} is taken.
The other frameworks can be considered as well, but additional models, model predictions, or additional features are needed.
%
%- literature on aleatoric uncertainty tends to focus on the optimal types of ensembles and dropouts \cite{human-in-the-loop}
%
%for epistemic uncertainty the literature focuses
%on improve accuracy on probability distributions within a single model \cite{human-in-the-loop}
%
%epistemic
%-> philosophy: 
%due to limited data and knowledge. 
%Given enough training samples, epistemic uncertainty will decrease.
%historically: meant lack of knowledge
%model uncertainty
%-> machine learning:
%-> literature on epistemic uncertainty tends to focus
%on getting more accurate probability distributions from within a single model
%
%aleatoric:
%-> philosophy:
%arising from the natural stochasticity of observations
%cannot be reduced even when more data is provided
%historically meant inherent randomness
%data uncertainty
%-> machine learning: