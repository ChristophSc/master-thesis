\section{Improving Classification by Adding More Information} 
\label{sec:improving_classification}

Until now, only information of the scoring function of the generator is used for the classification of the triple as either positive or negative.
However, since this only contains implicit information about, for example, the structure of the \ac{KG} and its entities and relations, the idea is to supplement this classification with further, explicit information.
Therefore, the idea is to include additional functions for classification in the form of the following functions: 

Assuming we have the function $peergroup(e) \subseteq \mathcal{KG}$ which returns a set of peers for given entity $e \in \entities_{\mathcal{T}}$ by its embedding. 
Then the $Generator\_Score$ function contains the following features from \cite{arnaout2020enriching}:
\begin{itemize}
    \item 
    \emph{\ac{PEER}:} 
    Relative frequency of peers within a peer group that is related to different objects, e.g. 0.9 of persons are married. 
    \begin{equation}
        PEER(h,r) = \frac{|\{p | p \in peergroup(h), (h, r, t) \in \mathcal{T}\}|}{|peergroup(h)|}
    \end{equation}

    \item
    \emph{\ac{POP}:} 
    The popularity of the tail entity $t$ in $\mathcal{T}$. 
    \begin{equation}
        POP(t) = \frac{|\{t | (h, r , t) \in \mathcal{T}\}|}{|\mathcal{T}|}
    \end{equation}

    \item 
    \emph{\ac{FRQ}:} 
    Frequency of a relation/predicate $r$ in $\mathcal{T}$. 
    \begin{equation}
        FRQ(r) = \frac{|\{r | (h, r, t) \in \mathcal{T}\}|}{|\mathcal{T}|}
    \end{equation}
    
    \item 
    \emph{\ac{PIVO}:} 
    Textual background information about an entities $h$ and $t$ and is a pivoting classifier like in \cite{arnaout2020enriching}.
    
    \item 
    \emph{$f_G$:} 
    $Generator\_Score$ function of the \ac{KGE} model from generator like in the original \ac{KBGAN} approach (e.g. \textsc{DistMult}  or \textsc{ComplEx}).
\end{itemize}


To incorporate these information in the score function than just the embedding of a triple, we want to extend the original score function $f_G$ by several information about entities and relations in the \ac{KG}.
Therefore, our new score function $Classification\_Score$ is defined as follows:
\begin{equation}
    Classification\_Score(h, r, t)=
    \begin{cases}
         \lambda_1 \text{PEER(\textit{h, r})} + \lambda_2 \text{POP(\textit{t})} + \lambda_3 \text{PIVO(\textit{h, t})} + \lambda_4 f_G(\textit{h, r, t})
         \\ \ \ 
         if\ \ \ \neg(\textit{h, r, t})
         \\ \\
         \lambda_1 \text{PEER(\textit{h, r})} + \lambda_2 \text{FRQ(\textit{r})} + \lambda_3 \text{PIVO(\textit{h, t})} + \lambda_4 f_G(h, r, t)
         \\ \ \ 
         if\ \ \ \neg \exists (\textit{h, r, \_})
         \\
    \end{cases}
\end{equation}
where $\lambda_i \in [0, 1]$ for $i \in [1,4]$  are hyperparameters and $\sum_{i=1}^{4}\lambda_i = 1$.
Initially they are set to randomly generated values but can be optimized at a later stage.
Furthermore, $\neg (h, r, t)$ is a \textit{grounded negative statement} and is satisfied if $(h, r, t) \notin$ \ac{KG} and $\neg\exists(h, r, \_)$ is a \textit{universally negative statement} which is satisfied if there exists no $t$ such that $(h, r, t) \in KG$ \cite{arnaout2020enriching}.
\textit{PEER}, \textit{POP}, \textit{PIVO} and \textit{FRQ} are functions that provide additional information about the \ac{KG}, its entities and relations such that the $Generator\_Score$ can make more reliable statements about whether the given triple $(h, r, t)$ is positive or not.
However, these components of the function can also be replaced or extended by further components known from for example Cluster-Based Sampling methods that give more information about a \ac{KG} and its structure.
For example, the K-Means algorithm \cite{qianunderstanding} or other techniques known from Relational Sampling, Nearest Neighbor Sampling or Near miss Sampling can be considered \cite{kotnis2017analysis}.




\textcolor{red}{TODO: filtered negatives which appear in the KG}