\section{Procedure of Adversarial Training with Sampling by Uncertainty}
\label{sec:procedure}
%
The correspondingly modified architecture and the sequence of the sampling process are shown in \Autoref{fig:uncertainty_sampling_architecture}.
\begin{figure}[H]
  \centering
    \includegraphics[width=0.90\textwidth]{figures/ucgan_architecture.pdf}
  \caption{Uncertainty Sampling replaces the original sampling in \ac{KBGAN}. 
  The new uncertainty sampling component is highlighted in green color.}
  \label{fig:uncertainty_sampling_architecture}
\end{figure}
The changed elements are marked in green.
Accordingly, these elements change the adversarial learning procedure.
As in the original \kbgan, our approach includes two components of a generator and a discriminator which are \ac{KGE} models that can be either pre-trained or not.
In \Autoref{fig:uncertainty_sampling_architecture} steps 1 to 8 are marked which will be described in the following 
The training process can be described in the following steps
which are described in more detail below.
To make the difference between the original model and our approach clear, we will use an example to explain this process.

\textbf{1. Creation of Negative Triple Set Neg}\\
For each epoch in the training process a set of negative triples is created.
This set, which is defined by $Neg(h,r,t)=\{(h_i',r,t_i')\}_{i=1\dots N_s}$, is created by Bernoulli sampling of $N_s$ negative triples.
Therefore, either head or tail entity is replaced with different probabilities according to the mapping property of relations and entities.
In comparison to Random Sampling, it applies higher probabilities to head entity replacement in one-to-many relations and tail in many-to-one relations \cite{qiannegative}.
For each epoch and for each positive triple $(h,r,t) \in KG$ a negative triple set is created.
In \kbgan as well as our approach the size of $Neg$ is set to twenty.

In our example for illustration purposes, we only consider a negative triple set $Neg$ of size five.
\Autoref{tab:neg_example} shows five different negative triples $l_i$ with $i \in [1,5]$.
For simplicity for a positive triples $(h,r,t) = (\text{})$
only tail entities are replaced to obtain $l_i$. 
\begin{table}[h]
    \centering
    \begin{tabular}{llllll}
        \toprule
        
        \textbf{Tail entities}
        & \textbf{$t_1$} & \textbf{$t_2$} & \textbf{$t_3$} & \textbf{$t_4$} & \textbf{$t_5$} \\
         
        & Florida
        & China
        & BarackObama
        & StarTrek
        & Google  \\

        \bottomrule
    \end{tabular}
    \caption{Example: For a given positive triple $(h,r,t)$ five different negative triples were created by replacing the tail entities to obtain negative triple set 
    $Neg = \{(h,r,t_i')\}$ for $i \in [1,5]$.}
\label{tab:neg_example}
\end{table}
\textbf{2. Calculating Minimum and Maximum Score for Positive and Negative Triples}\\
%
To be able to carry out the classification in the later uncertainty sampling process, the scores of all positive triples of the \ac{KG} as well as the negative triples of $Neg$ must be calculated before the actual adversarial training.
For this purpose, the corresponding triples are passed to the generator model, and, based on the embeddings trained so far, a score for all triples is calculated with the scoring function $f_G$.
Assuming the \ac{KG} consists of N positive triples and for each triple, a negative triple set of size $N_s$ was first created. 
Consequently, scores of N positive triples and $N \cdot N_s$ negative triples are calculated to obtain the minima and maxima of both positive and negative triples.
Our example results in the scoring ranges shown in \autoref{fig:scoring_ranges}.
\begin{figure*}[t]
  \centering
    \includegraphics[width=0.75\textwidth]{figures/scoremin_scoremax_example.pdf}
  \caption{Score range of positive and negative triples.
  The uncertainty of a model is in the scoring range where the 
  model found scores of both positive and negative triples.}
  \label{fig:scoring_ranges}
\end{figure*}
For negative triples, we obtained a score range of $[-9.1, 2.2]$, for positive triples a score range of $[-4, 10.7]$ and overlap between $[-4, 2.2]$.
Therefore, $score_{min} = -9.1$ and $score_{max} = 10.7$.
Since we have a fixed negative triple set for each epoch, they only need to be calculated once before each epoch.
\textbf{3. Calculate Score for all Negative Triples in Neg}\\
% DESCRIPTION
At this point, the adversarial training process between generator and discriminator starts.
The goal of this step is to calculate the scoring values for all negative triples.
Thus, in this step, all negative triples are passed to the generator scoring function $f_G$.
Since the generator is a tensor factorization-based model, the score obtained reflects the plausibility of each negative triple to be true.
After all scoring values of the negative triples have been calculated, they are given to the Uncertainty Sampling component.

% EXAMPLE - ORIGINAL APPRObACH FOR COMPARISON
In comparison, in the original \kbgan approach the sampling probabilities are directly derived from the scoring values. 
According to the example, a score for each triple $l_i' = (h,r,t_i')$ is calculated first.
Since all scores $f_G(l) \in \mathbb{R}$, they to not sum up to one.
Therefore, the softmax function is used to obtain sampling probabilities $p_i$ for each negative triple $l_i$, where each $p_i$ is defined as
\begin{equation}
    p_i = \frac{\exp{f_G(l_i)}}{\sum_{j=1}^{N_s}{\exp{f_G(l_j)}}}.
\end{equation}
Consequently, negative triples with high scores get a high probability to be sampled.
For our examples, this results in the values shown in \Autoref{tab:generator_scores}.
\begin{table}[h]
    \centering
    \begin{tabular}{llllll}
        \toprule
        
        \textbf{Tail entities}
        & \textbf{$t_1$} & \textbf{$t_2$} & \textbf{$t_3$} & \textbf{$t_4$} & \textbf{$t_5$} \\
         
        & Florida
        & China
        & BarackObama
        & StarTrek
        & Google  \\

        \midrule
        
        \textbf{Generator scores for $l_i = (h_i, r_i, t_i)$}
        & -5.916 
        & -2.249  
        & 3.719 
        & -0.761 
        & -6.942 \\
        
         \midrule
        
        \textbf{Original probabilities of $l_i$ to be sampled}
        & 1.05\%
        & 0.25\% 
        & 98.62\%  
        & 0.01\% 
        & 0.07\%
        \\
        \bottomrule
    \end{tabular}
    \caption{Example: For a given positive triple $(h,r,t)$ five different negative triples were created by replacing the tail entities to obtain negative triple set 
    $Neg = \{(h,r,t_i')\}$ for $i \in [1,5]$.
    For all negative triples a score $s_i$ in calculated by generator G scoring function $s_G$: $s_G(h,r,t_i') = s_i$.
    Therefore, tail entities $t_i$ for negative triples $(h_i, r_i, t_i)$ with $i \in [0,5]$ given by the generator model G with scoring function $s_G$ }
\label{tab:generator_scores}
\end{table}
According to our example, the negative triple $l_3' = (h,r,t_3)$ achieves the highest score and, therefore, obtains the highest probability to be sampled.


% EXAMPLE NEW APPROACH
Since our approach needs the scores to classify the triples, instead of probabilities  the output of generator G are scores $s_i \in \mathbb{R}$ for each negative triple in $Neg$.
Therefore, the original sampling technique is replaced by an Uncertainty Sampler which samples one triple based on the model's uncertainty.



\textbf{4. Sampling of an Uncertain Negative Triple}\\

All calculated scores for negative triples from generator are handed to the Uncertainty Sampler to calculate the uncertainty of the model and sample a triple where the model is uncertain how to classify it.
As shown in \Autoref{fig:uncertainty_sampling_component}, the Uncertainty Sampler is divided into three sub-components and steps.
\begin{figure*}[t]
  \centering
    \includegraphics[width=0.90\textwidth]{figures/uncertainty_sampling_component.pdf}
  \caption{Uncertainty Sampling component which contains a Classifier, an Uncertainty Scorer and an Uncertainty Sampler. Uncertainty Sampler can sample either according to Uncertainty Max or Uncertainty Distribution. Input of the Uncertainty Sampling component are scores of triples and the output is a negative triple sampled by uncertainty.}
  \label{fig:uncertainty_sampling_component}
\end{figure*}
It contains a \textbf{Classifier}, an \textbf{Uncertainty Scorer} and a \textbf{Sampler} component.
These are explained in more detail below.

\textbf{Classifier} \\
The Classifier's task is to classify the negative triples by assigning each negative triple a probability of being a positive triple.
It receives the negative triple scores $s_i \in \mathbb{R}$, which are either very high or very low logit values from the generator as input and returns a probability for each negative triple $(h,r,t) \in Neg$ to be positive  ($\mathbb{P}(y = 1| (h,r,t))$) (\Autoref{eqn:positive_probability}) as output. 

For our example, in \Autoref{tab:positive_probabilities} generator scores for all triples $l_i = (h_i, r_i, t_i)$ are listed which are the input of the Classifier.
For these triples and their generator score probabilities are derived for being a positive triple.
\begin{table}[H]
    \centering
    \begin{tabular}{llllll}
        \toprule
        
        \textbf{Tail entities}
        &  \textbf{$t_1$} & \textbf{$t_2$} & \textbf{$t_3$} & \textbf{$t_4$} & \textbf{$t_5$} \\
         
        \midrule
        
        \textbf{Generator scores}
         & $s_1$ & $s_2$ & $s_3$ & $s_4$ & $s_5$ \\
        
        \textbf{for $l_i' = (h_i, r_i, t_i)$}
        & -5.916 
        & -2.249  
        & 3.719 
        & -0.761 
        & -6.942 \\
        
        \midrule
                        
        \textbf{Probabilities of being} 
        & $p_1$ & $p_2$ & $p_3$ & $p_4$ & $p_5$ \\
        
        \textbf{a positive triple}
        & 16.09\% 
        & 34.54\% 
        & 64.57\% 
        & 42.03\% 
        & 10.93\%  \\

        \bottomrule
    \end{tabular}
    \caption{Example: For all negative triples $l_i' = (h, r, t_i)$ scores $s_i$ for $i \in [1,5]$ are calculated by generator G and given to the Classifier component.
    From these scores probabilities $p_i$ for $i \in [1, 5]$ for being a positive one are calculated.}
\label{tab:positive_probabilities}
\end{table}


\textbf{Uncertainty Scorer} \\
The task of the Uncertainty Scorer is to assign uncertainty values for all negative triples.
As input it receives the probabilties from the Classifier of each triple to be a positive one.
As output it returns uncertainty scores $u_i \in [0,1]$ for $i \in [1, N_s]$ for all negative triples $l_i = (h_i, r_i, t_i)$ where $u_i = 0$ indicates low and $u_i = 1$ high uncertainty.
As mentioned in \Autoref{sec:calculation_of_uncertainty_scores},  there are several Uncertainty Metrics at our disposal:
Entropy, Least Confidence, Margin of Confidence and Ratio of Confidence.
One of these Uncertainty metrics is chosen according to configuration and for each triple the model´s uncertainty is measured.
The closer a probability of a negative triple to be a positive one is to 0.5, the higher is the uncertainty score. 
Uncertainty scores $u(l)$ are calculated with \Autoref{eqn:uncertainty_function}.
For our example, this means that the Uncertainty Scorer calculates the uncertainties listed in \Autoref{tab:uncertainty_scores} for the individual uncertainty metrics.
\begin{table}[H]
    \centering
    \begin{tabular}{llllll}
        \toprule
        
        &  \textbf{$t_1$} & \textbf{$t_2$} & \textbf{$t_3$} & \textbf{$t_4$} & \textbf{$t_5$} \\
         
        \midrule
        
        \textbf{Probabilities of being}
         & $p_1$ & $p_2$ & $p_3$ & $p_4$ & $p_5$   \\
         
        \textbf{a positive triple}
        & 16.09\% 
        & 34.54\% 
        & 64.57\%
        & \underline{42.03\%} 
        & 10.93\%  \\
        
        \midrule
        \textbf{Uncertainty scores}
        & $u_1$ & $u_2$ & $u_3$ & $u_4$ & $u_5$ \\
        
        Entropy 
        & 0.6364 & 0.9299 & 0.9378 & \underline{0.9816} & 0.4977 \\
        
        Least confidence 
        & 0.3218 & 0.6909 & 0.7086 & \underline{0.8406} & 0.2185 \\
        
        Confidence margin
        & 0.3218 & 0.6909 & 0.7086 & \underline{0.8406} & 0.2185 \\
        
        Confidence ratio
        & 0.19174 & 0.5277 & 0.5487 & \underline{0.7250} & 0.1227\\
        
        \bottomrule
    \end{tabular}
    \caption{Example: For all negative triples $l_i' = (h, r, t_i)$ an probability $p_i$ to be positive was calculated. Uncertainty Scorer component retrieves these probabilities as input and calculates uncertainty scores $u_i$ for $i \in [1, 5]$ with different uncertainty measures.}
\label{tab:uncertainty_scores}
\end{table}
As you can see from the example, $l_4 = (h_4, r_4, t_4)$ has a probability of 42.03\% which is closest to 0.5 among all triples.
For each triple the uncertainty scores are calculated for all uncertainty metrics and the highest score is marked in bold.
As you can see, although the uncertainty scores are different, $l_4$ receives the highest value for all four uncertainty metrics.
These uncertainty values $u_i$ are now passed to the Sampler component.


\textbf{Sampler} \\
The actual sampling of a negative triple takes place within this component, which is then passed on to the discriminator.
It receives uncertainty scores for all negative triples from Uncertainty Scorer as input and returns one negative triple as output.
As mentioned in \Autoref{sec:calculation_of_uncertainty_scores} we have two methods at our disposal how to sample an uncertain triple.
Either Uncertainty Max which always samples the triple where model is most uncertain or Uncertainty Distribution which samples according to a distribution based on uncertainty scores.
One triple is sampled according to configured sampling method and returned to discriminator where the next step takes place.

If we sample by Uncertainty Max this means selecting only the triple with the maximum uncertainty score.
In our example, this would always be the triple $l_4$ for all uncertainty metrics.
If we sample by Uncertainty Distribution method, we have to calculate sampling probabilities based on uncertainty scores first and then pick one triple according the probability distribution.
In \Autoref{fig:uncertainty_metrics_example_distribution}, for all triples and all uncertainty metrics the sampling probabilities are displayed.
\input{content/chapters/03_approach/sections/table_uncertainty_metrics_example_distribution}
For all different metrics $t_4$ is still the most probable one to be sampled, but the uncertainty sampling metrics differ in their probability to sample $t_4$.
E.g. while based on confidence ratio, $t_4$ is sampled with a probability of 26.36\% it is only sampled with probability of 23.63\% for entropy uncertainty metrics.
The more negative triples $t_i$ exist, the more these sampling probabilities differ.





\textbf{5. Passing Original Positive Triple and Sampled Negative Triple to Discriminator Model D}\\
Subsequently, negative triples are sampled according to the calculated uncertainty scores, either the one with maximum uncertainty score or its distribution and given to the discriminator.

\textbf{6. Calculation of Distances}\\
From this step on, everything is the same as the original \ac{KBGAN} approach:
The generated negative triple $(h',r,t')$ as well as the positive triple $(h, r, t)$ are sent to the discriminator.
\textbf{7. Calculation of Loss between Positive and Negative Triple}\\

The aim is to improve the discriminator model, which is achieved by minimising the marginal loss.
The marginal loss is defined as in \Autoref{eq:marginalloss2}.
Therefore, all positive triples $(h,r,t) \in \mathcal{T}$, where $\mathcal{T}$ is the set of all positive triples, and their sampled negative triple $(h',r', t')$  are passed to the marginal loss function $L_D$ of the discriminator
\begin{multline} \label{eq:marginalloss2}.
    L_D =\sum_{(h,r,t)\in\mathcal{T}}[f(h,r,t)-f(h',r,t')+\gamma]_+\\
    (h',r,t') \sim p_G(h,r,t|h,r,t) 
\end{multline}
while $p_G(h', r, t'|h, r, t)$ is the probability distribution on negative triples and defined as \cite{cai2017kbgan}
\begin{multline}
    p_G(h',r,t'|h,r,t)=\frac{\exp f_G(h',r,t')}{\sum\exp f_G(h^*,r,t^*)} \\
    (h^*,r,t^*)\in Neg(h,r,t)
\end{multline}
\textcolor{red}{UPDATE: EITHER UNCERTAINTY MAX OR UNCERTAINTY DISTRIBUTION PROBABILITY DISTRIBUTION}
Therefore, discriminator loss is minimized if difference between positive and negative triple score is low and in optimal case zero.
The sampled negative triple $(h',r,t')$ was picked based on the uncertainty of the generator model.
By sampling such an uncertain negative triple from generator, we hope to catch an uncertain triple for the discriminator as well.

\textbf{8. Passing Reward to Discriminator based on Distance of Negative Triple}\\
The reward defined by $r = - f_D(h',r,t')$ of the current negative triple is calculated and returned to the Generator and the next training iteration begins.

- reward is used for training of Generator
- 

These adversarial training steps are repeated until a determined number of epochs is reached, such that the generator improves the quality of sampled negative triples and the discriminator improves embedding over time.
The output of our algorithm is the adversarially trained discriminator and its embedding.



