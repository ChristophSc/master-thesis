\textbf{2. Calculating Minimum and Maximum Score for Positive and Negative Triples}\\
%
To be able to carry out the classification in the later uncertainty sampling process, the scores of all positive triples of the \ac{KG} as well as the negative triples of $Neg$ must be calculated before the actual adversarial training.
For this purpose, the corresponding triples are passed to the generator model, and, based on the embeddings trained so far, a score for all triples is calculated with the scoring function $f_G$.
Assuming the \ac{KG} consists of N positive triples and for each triple, a negative triple set of size $N_s$ was first created. 
Consequently, scores of N positive triples and $N \cdot N_s$ negative triples are calculated to obtain the minima and maxima of both positive and negative triples.
Our example results in the scoring ranges shown in \autoref{fig:scoring_ranges}.
\begin{figure*}[t]
  \centering
    \includegraphics[width=0.75\textwidth]{figures/scoremin_scoremax_example.pdf}
  \caption{Score range of positive and negative triples.
  The uncertainty of a model is in the scoring range where the 
  model found scores of both positive and negative triples.}
  \label{fig:scoring_ranges}
\end{figure*}
For negative triples, we obtained a score range of $[-9.1, 2.2]$, for positive triples a score range of $[-4, 10.7]$ and overlap between $[-4, 2.2]$.
Therefore, $score_{min} = -9.1$ and $score_{max} = 10.7$.
Since we have a fixed negative triple set for each epoch, they only need to be calculated once before each epoch.