\textbf{3. Calculate Score for all Negative Triples in Neg}\\
% DESCRIPTION
At this point, the adversarial training process between generator and discriminator starts.
The goal of this step is to calculate the scoring values for all negative triples.
Thus, in this step, all negative triples are passed to the generator scoring function $f_G$.
Since the generator is a tensor factorization-based model, the score obtained reflects the plausibility of each negative triple to be true.
After all scoring values of the negative triples have been calculated, they are given to the Uncertainty Sampling component.

% EXAMPLE - ORIGINAL APPRObACH FOR COMPARISON
In comparison, in the original \kbgan approach the sampling probabilities are directly derived from the scoring values. 
According to the example, a score for each triple $l_i' = (h,r,t_i')$ is calculated first.
Since all scores $f_G(l) \in \mathbb{R}$, they to not sum up to one.
Therefore, the softmax function is used to obtain sampling probabilities $p_i$ for each negative triple $l_i$, where each $p_i$ is defined as
\begin{equation}
    p_i = \frac{\exp{f_G(l_i)}}{\sum_{j=1}^{N_s}{\exp{f_G(l_j)}}}.
\end{equation}
Consequently, negative triples with high scores get a high probability to be sampled.
For our examples, this results in the values shown in \Autoref{tab:generator_scores}.
\begin{table}[h]
    \centering
    \begin{tabular}{llllll}
        \toprule
        
        \textbf{Tail entities}
        & \textbf{$t_1$} & \textbf{$t_2$} & \textbf{$t_3$} & \textbf{$t_4$} & \textbf{$t_5$} \\
         
        & Florida
        & China
        & BarackObama
        & StarTrek
        & Google  \\

        \midrule
        
        \textbf{Generator scores for $l_i = (h_i, r_i, t_i)$}
        & -5.916 
        & -2.249  
        & 3.719 
        & -0.761 
        & -6.942 \\
        
         \midrule
        
        \textbf{Original probabilities of $l_i$ to be sampled}
        & 1.05\%
        & 0.25\% 
        & 98.62\%  
        & 0.01\% 
        & 0.07\%
        \\
        \bottomrule
    \end{tabular}
    \caption{Example: For a given positive triple $(h,r,t)$ five different negative triples were created by replacing the tail entities to obtain negative triple set 
    $Neg = \{(h,r,t_i')\}$ for $i \in [1,5]$.
    For all negative triples a score $s_i$ in calculated by generator G scoring function $s_G$: $s_G(h,r,t_i') = s_i$.
    Therefore, tail entities $t_i$ for negative triples $(h_i, r_i, t_i)$ with $i \in [0,5]$ given by the generator model G with scoring function $s_G$ }
\label{tab:generator_scores}
\end{table}
According to our example, the negative triple $l_3' = (h,r,t_3)$ achieves the highest score and, therefore, obtains the highest probability to be sampled.


% EXAMPLE NEW APPROACH
Since our approach needs the scores to classify the triples, instead of probabilities  the output of generator G are scores $s_i \in \mathbb{R}$ for each negative triple in $Neg$.
Therefore, the original sampling technique is replaced by an Uncertainty Sampler which samples one triple based on the model's uncertainty.


