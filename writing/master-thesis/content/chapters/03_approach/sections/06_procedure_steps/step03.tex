\textbf{3. Calculate Score for all Negative Triples in $Neg$}\\

At this point, the adversarial training between generator and discriminator starts.
In this training, the negative triples from the set Neg are passed on to the generator model G.
Since the generator is a tensor factorisation-based model, the score and thus the plausibility of each triple $l$ is calculated with the scoring function $f_G$.
Therefore, with this step a score $s_i$ for each triple $l_i = (h',r,t') \in Neg$ is calculated.

In the original \kbgan approach the sampling probabilities are directly calculated from scores and given to the Sampler.
Since score $s_i \in \mathbb{R}$, sampling probabilities are calculated which sum up to 1 for all 
With the softmax function, the individual negative triples of the set Neg are assigned such sampling probabilities that their sum is 1.
The softmax Function which is defined as 
\begin{equation}
    s(x_i) = \frac{x_i}{\sum_{j=1}^{n}{e^{x_j}}}
\end{equation}
and therefore, in the  original approach we obtain sampling probability $p_i$ for each negative triple $l_i$:
\begin{equation}
    p_i = s(f_G(l_i)) = \frac{f_G(l_i)}{\sum_{j=1}^{N_s}{e^{f_G(l_j)}}}
\end{equation}
Therefore, we obtain high probabilities to sample negative triples with high scores.
In the example, this results in the values shown in \Autoref{tab:generator_scores}.
\begin{table}[h]
    \centering
    \begin{tabular}{llllll}
        \toprule
        
        \textbf{Tail entities}
        & \textbf{$t_1$} & \textbf{$t_2$} & \textbf{$t_3$} & \textbf{$t_4$} & \textbf{$t_5$} \\
         
        & Florida
        & China
        & BarackObama
        & StarTrek
        & Google  \\

        \midrule
        
        \textbf{Generator scores for $l_i = (h_i, r_i, t_i)$}
        & -5.916 
        & -2.249  
        & 3.719 
        & -0.761 
        & -6.942 \\
        
         \midrule
        
        \textbf{Original probabilities of $l_i$ to be sampled}
        & 1.05\%
        & 0.25\% 
        & 98.62\%  
        & 0.01\% 
        & 0.07\%
        \\
        \bottomrule
    \end{tabular}
    \caption{Example: For a given positive triple $(h,r,t)$ five different negative triples were created by replacing the tail entities to obtain negative triple set 
    $Neg = \{(h,r,t_i')\}$ for $i \in [1,5]$.
    For all negative triples a score $s_i$ in calculated by generator G scoring function $s_G$: $s_G(h,r,t_i') = s_i$.
    Therefore, tail entities $t_i$ for negative triples $(h_i, r_i, t_i)$ with $i \in [0,5]$ given by the generator model G with scoring function $s_G$ }
\label{tab:generator_scores}
\end{table}

Since our approach needs the scores to classify the triples, instead the output of generator G are scores $s_i \in \mathbb{R}$ for each negative triple in $Neg$.
Therefore, the original Sampling technique is replaced by an Uncertainty Sampler which samples one triple based on the model´s uncertainty.


