\textbf{6. Calculation of Distances}\\

After both the positive and the negative triple have been passed to the discriminator, the scoring function $f_D$ is used to calculate their score.
Since discriminator model is a Translation-Based Model, these scores are distances between head, relation and tail en the embedding space.
In the original \kbgan and also our approach we used \textsc{TransE} and \textsc{TransD}.
For example, the scoring function of \textsc{TransE} is defined as 
\begin{equation}
    f_{TransE}(h,r,t) = || h + r - t ||_{l_1, l_2}
\end{equation}
such that if entities and relations are closer together in embedding space, the triple is considered to be true and vice versa.
So for our example the positive triple $l_p$ as well as sampled negative triple $l_4$ are scored by discriminator with scoring function $f_D$.
Therefore, we obtain two distances for positive triple $d_1 = f_D(l_p)$ and negative triple $d_2 = f_D(l_4)$
Since low distances indicate positive triple its distance should be lower the negative triple distance ($d_1 < d_2)$.
Nevertheless, since the generator tries to trick the discriminator with triples where it is most uncertain if there are true or not, $d_2$ should not be too high or even close do $d_1$.





