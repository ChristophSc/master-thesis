\textbf{4. Sampling of One Most Uncertain Negative Triple}\\
All of these calculates scores from generator are handed to the Uncertainty Sampler to calculate the uncertainty of the model.
For this, each negative triple in $Neg$ is classified first by assigning a probability $\mathds{P}(y = 1 | (h,r,t)) \forall (h,r,t) \in Neg$ with \autoref{eqn:positive_probability}.
These probabilities are used to calculate uncertainty scores using function $u(l)$ defined in \autoref{eqn:uncertainty_function}.


\begin{figure*}[t]
  \centering
    \includegraphics[width=0.90\textwidth]{figures/uncertainty_sampling_component.pdf}
  \caption{Uncertainty Sampling component which contains a Classifier, an Uncertainty Scorer and an Uncertainty Sampler. Uncertainty Sampler can sample either according to Uncertainty Max or Uncertainty Distribution. Input of the Uncertainty Sampling component are scores of triples and the output is a negative triple sampled by uncertainty.}
  \label{fig:uncertainty_sampling_component}
\end{figure*}

\textbf{Classifier} \\
Input: Scores of generator
scores = logits, which can very low and ver high values
Output: Probabilities of being true

\textbf{Uncertainty Scorer}  \\
Input: Positive Probabilities
Output: uncertainty scores which are between 0 and 1
0 is not uncertain, 1 is very uncertain

\textbf{Uncertainty Sampler} \\
input: uncertainty scores
Output: Either 1) Uncertainty Max with triple with highest uncertainty score or 2) triple sampled by probability distribution based on probability scores
-> highest uncertainty = highest probability and vice versa










% --------> 

% example 
% illustrates that sampling the negative triple with maximum uncertainty score results in same triple for all uncertainty measures:
The different Uncertainty Sampling methods based on the different Uncertainty Metrics can be illustrated by the following example which is depicted in \Autoref{tab:uncertainty_metrics_example_max}.
\begin{table}[h]
    \centering
    \begin{tabular}{llllll}
        \toprule
        
        &  \textbf{$t_1$} & \textbf{$t_2$} & \textbf{$t_3$} & \textbf{$t_4$} & \textbf{$t_5$} \\
         
        \midrule
        
         \textbf{Probabilities of being a positive triple}
        & 0.6285 & 0.7498 & 0.8297 & \textbf{0.6207} & 0.8297   \\
        
        \midrule
        \textbf{Uncertainty scores}
        & & & & & \\
        
        Entropy 
        & 0.6364 & 0.9299 & 0.9378 & \textbf{0.9816} & 0.4977 \\
        
        Least confidence 
        & 0.3218 & 0.6909 & 0.7086 & \textbf{0.8406} & 0.2185 \\
        
        Confidence margin
        & 0.3218 & 0.6909 & 0.7086 & \textbf{0.8406} & 0.2185 \\
        
        Confidence ratio
        & 0.19174 & 0.5277 & 0.5487 & \textbf{0.7250} & 0.1227\\
        
        \bottomrule
    \end{tabular}
    \caption{Result}
\label{tab:uncertainty_metrics_example_max}
\end{table}
Suppose we have five different negative triples $l_i = (h_i, r_i, t_i)$ where $i \in [1,5]$.
At first, for each of them probabilities of being a positive triple is calculated.
As you can see from the example, $t_4$ has a probability of 0.6207 which is closest to 0.5.
For each triple the  uncertainty scores are calculated for all uncertainty metrics and the highest score is marked in bold.
As you can see, although the uncertainty scores are different, $t_4$ receives the highest value for all four uncertainty metrics.
For this reason, when sampling the triple with maximum uncertainty score in a binary classification, the same triple is chosen for all metrics.

On the other hand, sampling can be done using a probability distribution based on the uncertainty scores.
In the following we refer to this Uncertainty Sampling Method as \textbf{Uncertainty Sampling Distribution}.
By normalizing the Least Confident function, this results in an equal distribution only for this and Margin of Confidence.    
Due to the stronger drop in the direction of $\mathds{P}(y = 1 | (h,r,t)) = 0$ and $\mathds{P}(y = 1 | (h,r,t)) = 1$ respectively, this causes an increase of the sampling probability of triples close to an uncertainty score of 1 and vice versa.
Therefore, negative triples with a high probability of being a positive triple are assigned a low uncertainty score according to this principle.
This also reduces the probability that false negative triples in the negative triple set $Neg$ will be sampled.
Accordingly, it is possible to generate a larger set than twenty negative triples in the negative triple set $Neg$.\\
An example is depicted in \autoref{fig:uncertainty_metrics_example_distribution}.
\input{content/chapters/03_approach/sections/table_uncertainty_metrics_example_distribution}
For all different metrics $t_4$ is still the most probable one to be sampled, but the uncertainty sampling metrics differ in their probability to sample $t_4$.
E.g. while based on confidence ratio, $t_4$ is sampled with a probability of 31.39\% it is only sampled with probability of 23.71\% for entropy uncertainty metrics.
The more negative triples $t_i$ exist, the more these sampling probabilities differ.





