\textbf{4. Sampling of an Uncertain Negative Triple}\\

All calculated scores for negative triples from generator are handed to the Uncertainty Sampler to calculate the uncertainty of the model and sample a triple where the model is uncertain how to classify it.
As shown in \Autoref{fig:uncertainty_sampling_component}, the Uncertainty Sampler is divided into three sub-components and steps.
\begin{figure*}[t]
  \centering
    \includegraphics[width=0.90\textwidth]{figures/uncertainty_sampling_component.pdf}
  \caption{Uncertainty Sampling component which contains a Classifier, an Uncertainty Scorer and an Uncertainty Sampler. Uncertainty Sampler can sample either according to Uncertainty Max or Uncertainty Distribution. Input of the Uncertainty Sampling component are scores of triples and the output is a negative triple sampled by uncertainty.}
  \label{fig:uncertainty_sampling_component}
\end{figure*}
It contains a \textbf{Classifier}, an \textbf{Uncertainty Scorer} and a \textbf{Sampler} component.
These are explained in more detail below.

\textbf{Classifier} \\
The Classifier's task is to classify the negative triples by assigning each negative triple a probability of being a positive triple.
It receives the negative triple scores $s_i \in \mathbb{R}$, which are either very high or very low logit values from the generator as input and returns a probability for each negative triple $(h,r,t) \in Neg$ to be positive  ($\mathbb{P}(y = 1| (h,r,t))$) (\Autoref{eqn:positive_probability}) as output. 

For our example, in \Autoref{tab:positive_probabilities} generator scores for all triples $l_i = (h_i, r_i, t_i)$ are listed which are the input of the Classifier.
For these triples and their generator score probabilities are derived for being a positive triple.
\begin{table}[H]
    \centering
    \begin{tabular}{llllll}
        \toprule
        
        \textbf{Tail entities}
        &  \textbf{$t_1$} & \textbf{$t_2$} & \textbf{$t_3$} & \textbf{$t_4$} & \textbf{$t_5$} \\
         
        \midrule
        
        \textbf{Generator scores}
         & $s_1$ & $s_2$ & $s_3$ & $s_4$ & $s_5$ \\
        
        \textbf{for $l_i' = (h_i, r_i, t_i)$}
        & -5.916 
        & -2.249  
        & 3.719 
        & -0.761 
        & -6.942 \\
        
        \midrule
                        
        \textbf{Probabilities of being} 
        & $p_1$ & $p_2$ & $p_3$ & $p_4$ & $p_5$ \\
        
        \textbf{a positive triple}
        & 16.09\% 
        & 34.54\% 
        & 64.57\% 
        & 42.03\% 
        & 10.93\%  \\

        \bottomrule
    \end{tabular}
    \caption{Example: For all negative triples $l_i' = (h, r, t_i)$ scores $s_i$ for $i \in [1,5]$ are calculated by generator G and given to the Classifier component.
    From these scores probabilities $p_i$ for $i \in [1, 5]$ for being a positive one are calculated.}
\label{tab:positive_probabilities}
\end{table}


\textbf{Uncertainty Scorer} \\
The task of the Uncertainty Scorer is to assign uncertainty values for all negative triples.
As input it receives the probabilties from the Classifier of each triple to be a positive one.
As output it returns uncertainty scores $u_i \in [0,1]$ for $i \in [1, N_s]$ for all negative triples $l_i = (h_i, r_i, t_i)$ where $u_i = 0$ indicates low and $u_i = 1$ high uncertainty.
As mentioned in \Autoref{sec:calculation_of_uncertainty_scores},  there are several Uncertainty Metrics at our disposal:
Entropy, Least Confidence, Margin of Confidence and Ratio of Confidence.
One of these Uncertainty metrics is chosen according to configuration and for each triple the model´s uncertainty is measured.
The closer a probability of a negative triple to be a positive one is to 0.5, the higher is the uncertainty score. 
Uncertainty scores $u(l)$ are calculated with \Autoref{eqn:uncertainty_function}.
For our example, this means that the Uncertainty Scorer calculates the uncertainties listed in \Autoref{tab:uncertainty_scores} for the individual uncertainty metrics.
\begin{table}[H]
    \centering
    \begin{tabular}{llllll}
        \toprule
        
        &  \textbf{$t_1$} & \textbf{$t_2$} & \textbf{$t_3$} & \textbf{$t_4$} & \textbf{$t_5$} \\
         
        \midrule
        
        \textbf{Probabilities of being}
         & $p_1$ & $p_2$ & $p_3$ & $p_4$ & $p_5$   \\
         
        \textbf{a positive triple}
        & 16.09\% 
        & 34.54\% 
        & 64.57\%
        & \underline{42.03\%} 
        & 10.93\%  \\
        
        \midrule
        \textbf{Uncertainty scores}
        & $u_1$ & $u_2$ & $u_3$ & $u_4$ & $u_5$ \\
        
        Entropy 
        & 0.6364 & 0.9299 & 0.9378 & \underline{0.9816} & 0.4977 \\
        
        Least confidence 
        & 0.3218 & 0.6909 & 0.7086 & \underline{0.8406} & 0.2185 \\
        
        Confidence margin
        & 0.3218 & 0.6909 & 0.7086 & \underline{0.8406} & 0.2185 \\
        
        Confidence ratio
        & 0.19174 & 0.5277 & 0.5487 & \underline{0.7250} & 0.1227\\
        
        \bottomrule
    \end{tabular}
    \caption{Example: For all negative triples $l_i' = (h, r, t_i)$ an probability $p_i$ to be positive was calculated. Uncertainty Scorer component retrieves these probabilities as input and calculates uncertainty scores $u_i$ for $i \in [1, 5]$ with different uncertainty measures.}
\label{tab:uncertainty_scores}
\end{table}
As you can see from the example, $l_4 = (h_4, r_4, t_4)$ has a probability of 42.03\% which is closest to 0.5 among all triples.
For each triple the uncertainty scores are calculated for all uncertainty metrics and the highest score is marked in bold.
As you can see, although the uncertainty scores are different, $l_4$ receives the highest value for all four uncertainty metrics.
These uncertainty values $u_i$ are now passed to the Sampler component.


\textbf{Sampler} \\
The actual sampling of a negative triple takes place within this component, which is then passed on to the discriminator.
It receives uncertainty scores for all negative triples from Uncertainty Scorer as input and returns one negative triple as output.
As mentioned in \Autoref{sec:calculation_of_uncertainty_scores} we have two methods at our disposal how to sample an uncertain triple.
Either Uncertainty Max which always samples the triple where model is most uncertain or Uncertainty Distribution which samples according to a distribution based on uncertainty scores.
One triple is sampled according to configured sampling method and returned to discriminator where the next step takes place.

If we sample by Uncertainty Max this means selecting only the triple with the maximum uncertainty score.
In our example, this would always be the triple $l_4$ for all uncertainty metrics.
If we sample by Uncertainty Distribution method, we have to calculate sampling probabilities based on uncertainty scores first and then pick one triple according the probability distribution.
In \Autoref{fig:uncertainty_metrics_example_distribution}, for all triples and all uncertainty metrics the sampling probabilities are displayed.
\input{content/chapters/03_approach/sections/table_uncertainty_metrics_example_distribution}
For all different metrics $t_4$ is still the most probable one to be sampled, but the uncertainty sampling metrics differ in their probability to sample $t_4$.
E.g. while based on confidence ratio, $t_4$ is sampled with a probability of 26.36\% it is only sampled with probability of 23.63\% for entropy uncertainty metrics.
The more negative triples $t_i$ exist, the more these sampling probabilities differ.




