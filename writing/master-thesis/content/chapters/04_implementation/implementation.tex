\chapter{Implementation}
\label{ch:implementation}


After explaining our approach, we will now go into more detail about the implementation of \ucgan.
For this purpose, \autoref{sec:algorithm} first presents the \ac{GAN} training process in pseudo code.
Since some components were already implemented in the original \kbgan approach, in this chapter we focus on the differences between the new \ucgan and the original approach.
Already implemented components are only discussed to the extent that the adversarial training can be understood.
We also focus on describing the implementation of Adversarial Training.
The pre-training takes place separately, but does not differ from the basic training of the original individual models like \distmult, \complex, \transe and \transd.
Nevertheless, some methods for pre-training can be found in the individual class diagrams.
\Autoref{sec:adv_training}  first introduces the most important classes and methods for the functionality of Adversarial Training.
This also includes information on how to start the training and how and which configurations can be made.
In the following \Autoref{sec:sampling_by_uncertainty}, the new changes in the implementation for the new sampling method based on uncertainty are presented.
At the end of this chapter a hyperparameter optimization is described in \Autoref{sec:hyperparameter_optimization} and configurations to run training on $PC^2$ is reported in \Autoref{sec:processing_on_pc2}.

\section{Algorithm}
\label{sec:algorithm}
%
The original algorithm of \kbgan was modified for \usgan by replacing the original sampling by sampling with uncertainty.
\Autoref{alg:ucgan} shows the individual steps of \kbgan in pseudo-code.
In it, replaced steps of \kbgan are marked in red and the new steps are marked in green.
The steps marked in black have not changed from the original approach.
\begin{algorithm}[H]
\small
 \KwData{training set of positive fact triples $\mathcal{T}=\{(h,r,t)\}$}
 \KwIn{(Pre-trained) generator G with parameters $\theta_G$ and score function $f_G(h,r,t)$, and (pre-trained) discriminator D with parameters $\theta_D$ and score function $f_D(h,r,t)$}
 \KwOut{Adversarially trained discriminator}
 $b \longleftarrow 0$\tcp*[l]{baseline for policy gradient}
 \Repeat{convergence}{
  Sample a mini-batch of data $\mathcal{T}_{batch}$ from $\mathcal{T}$ \;
  
  $G_G \longleftarrow 0$, $G_D \longleftarrow 0$\tcp*[l]{gradients of parameters of G and D}
  $r_{sum} \longleftarrow 0$\tcp*[l]{for calculating the baseline}
  Uniformly randomly sample $N_s$ negative triples for each positive triple $(h,r,t) \in \kg$: $Neg(h,r,t)=\{(h_i',r,t_i')\}_{i=1\dots N_s}$\;
  
  $score_{min} \longleftarrow \argmin_{(h',r,t')\in Neg}{score(h',r,t')}$\;
  
  $score_{max} \longleftarrow \argmax_{(h,r,t)\in \kg}{score(h,r,t)}$ \;
   
  \For{$(h,r,t)\in \mathcal{T}_{batch}$}{
    Classify all negative triples by calculating their probability of being a positive triple:  $\mathds{P}(y = 1|f_G(h, r, t)) = \frac{f_G(h,r,t) - score_{min}}{score_{max} - score_{min}}$\;
    
    Calculate uncertainty scores $u_i \in Neg$ with $i \in [1, N_s]$ of all negative triples\;
    
    Obtain their probability of being sampled: $p_i=\frac{exp^{u_i}}{\sum_{j=1}^{N_s} exp^{u_j}}$\;
    
    Sample one negative triple $(h_s',r,t_s')$ from $Neg(h,r,t)$  based on uncertainty scores  according to $\{p_i\}_{i=1\dots N_s}$. Assume its probability to be $p_s$\;

    $G_D \longleftarrow G_D + \nabla_{\theta_D}[f_D(h,r,t)-f_D(h_s',r,t_s')+\gamma]_+$\tcp*[l]{accumulate gradients for D}
    
    $r \longleftarrow -f_D(h_s',r,t_s'), r_{sum} \longleftarrow r_{sum} + r$\tcp*[l]{$r$ is the reward}
    
    $G_G \longleftarrow G_G + (r-b)\nabla_{\theta_G} \log p_s$\tcp*[l]{accumulate gradients for G}
  }
  $\theta_G \longleftarrow \theta_G+\eta_G G_G, \theta_D \longleftarrow \theta_D-\eta_D G_D$\tcp*[l]{update parameters}
  $b \leftarrow r_{sum} / |\mathcal{T}_{batch}|$\tcp*[l]{update baseline}
 }
 \caption{\usgan algorithm with sampling method \ussoftmax.}
 \label{alg:ucgan}
\end{algorithm}
It can be seen that the original sampling steps have been replaced and that sampling is no longer based on generator scores but uncertainty scores.
For the calculation of this uncertainty, further steps such as the calculation of $score_{min}$ and $score_{max}$ are necessary before the actual sampling to classify and then sample them.

\section{Adversarial Training}
\label{sec:adv_training}
%
The adversarial training starts via the script \textit{gan\_train.py}.
Before the training begins, the configurations are loaded from the \textit{config.yaml} file where all the settings and parameters of the training are defined.
An example configuration file is shown in the appendix in \Autoref{app:config_yaml}.
The most important settings for adversarial training such as the definition of the generator and discriminator model are defined under \textit{adv}.
These settings can also be overwritten via parameters when starting the script.
Then, training, validation, and test data are loaded and the adversarial training for the specified number of epochs begins.

The adversarial training runs via the \texttt{BaseModel} class depicted in \Autoref{fig:basemodel_classdiagram}.
\begin{figure*}[t]
  \centering
    \includegraphics[width=0.95\textwidth]{figures/BaseModel.png}
  \caption{\ac{UML}-Classdiagram of \texttt{BaseModel} and inherited Models  \texttt{DistMult},  \texttt{ComplEx},  \texttt{TransE} and  \texttt{TransD}}
  \label{fig:basemodel_classdiagram}
\end{figure*}
It implements methods for loading and saving the learned models in specified files, a \texttt{pretrain} method which is overwritten by the inherited classes, and a function called \texttt{test\_link} which performs an evaluation for a given data set and calculates the MRR and Hit@10 values.
However, the most important methods of this class are \texttt{gen\_step} and \texttt{dis\_step}, which represent an iteration step of the generator and the discriminator respectively.
Therefore, the scores of the negative triples are calculated in the \texttt{gen\_step} and one of the negative triples is sampled for each positive triple.
Furthermore, \texttt{gen\_step} has several parameters.
The negative triples are passed to the generator via the parameters \texttt{head}, \texttt{rel}, and \texttt{tail}.
Also, \texttt{n\_sample} negative triples are sampled for each positive using the specified \texttt{sampler}.
The different samplers are described in more detail in the next \Autoref{sec:sampling_by_uncertainty}.
This sampled negative triple and the underlying positive triple are then passed to the \texttt{dis\_step} function.
Since the generator and discriminator can be different KGE models, they are each specified using the \texttt{mdl} variable of the inherited class.
It is an object of the class \texttt{BaseModule} and its inheriting classes.
This and inherited classes are depicted in \Autoref{fig:basemodule_classdiagram}
\begin{figure*}[t]
  \centering
    \includegraphics[width=0.95\textwidth]{figures/BaseModule.PNG}
  \caption{\ac{UML}-Classdiagram of \texttt{BaseModule} which inherits from the class \texttt{Module} from the package \texttt{torch.nn}.
  For each implemented model for generator and discriminator there are inherited modules \texttt{DistMultModule},  \texttt{ComplExModule},  \texttt{TransEModule} and  \texttt{TransDModule}}
  \label{fig:basemodule_classdiagram}
\end{figure*}
These represent a \texttt{torch.nn.Module} and contain the embeddings for entities and relations.
Each class implements a \texttt{score} function which calculates a score for a given triple.
The current embeddings of the entities and relations are also reflected in the local variables of the \texttt{torch.nn.Embedding} type.
For the calculation of the losses during training, a \texttt{softmax\_loss} method was implemented for the generator models and a \texttt{pair\_loss} method for the discriminator models.
The forward functions are inherited from the class \texttt{torch.nn.Module}, which defines the computation performed at every call.
This represents a simple call to the respective scoring function.

By implementing these models and passing negative and positive triples, it is now possible to perform an adversarial training.
Originally, the sampling of a negative triple was carried out in the generator.
Since in our approach there is now an extension of the original model to include a sampling based on uncertainty scores, the sampling functionality is outsourced to additional classes.
Sampling is now done by an additional class \texttt{BaseSampler}, of which our \texttt{BaseModel} has an instance named \texttt{smpl}.
\clearpage

 \section{Sampling by Uncertainty}
 \label{sec:sampling_by_uncertainty}
 
For the implementation of Sampling by Uncertainty, the existing approach was adapted and supplemented with further classes and functionalities.
To keep the functionality of the original sampling, it was moved to a class \texttt{OriginalSampler} and the new sampling method to \texttt{UncertaintySampler} which are depicted in \Autoref{fig:base_sampler}.
\begin{figure*}[t]
  \centering
    \includegraphics[width=0.95\textwidth]{figures/classdiagrams/BaseSampler.pdf}
    \caption{Classdiagram \texttt{BaseSampler} with inherited classes \texttt{OriginalSampler} and \texttt{UncertaintySampler}.}
  \label{fig:base_sampler}
\end{figure*}
This allows to determine the sampling method in the configuration file and dynamically create an object of the corresponding class during runtime.
\texttt{OriginalSampler} only overrides the \texttt{sample} method of 
the inheriting class \texttt{BaseSampler} which takes several parameters which are defined as follows.
\begin{itemize}
    \item 
    \texttt{n}: number of negative triple sets
    
    \item 
    \texttt{n\_sample}: 
    number of negative triples which need to be sampled from each negative triple set.
    Usually it is set to 1.
    
    \item 
    \texttt{logits}:
    logit values for all negative triples returned from the generator model
    
    \item 
    \texttt{min\_score}:
    minimum score of all negative triples from all negative triple sets.
    
    \item 
    \texttt{max\_score}:
    maximum score of all positive triples in the \ac{KG}
\end{itemize}
Besides overriding the \texttt{sample} method, \texttt{UncertaintySampler} additionally contains attributes with objects of classes \texttt{Classifier} and \texttt{UncertaintyScorer}.
These are depicted in \Autoref{fig:uncertainty_sampler}.
\begin{figure*}[t]
  \centering
    \includegraphics[width=0.95\textwidth]{figures/classdiagrams/UncertaintySampler.pdf}
    \caption{Classdiagram \texttt{UncertaintySampler} and its components.
    Inherited class \texttt{UncertaintySamplerMax} implements functionality to always a negative triple with maximum uncertainty score and \texttt{UncertaintySamplerSoftmax} implements functionality to sample a negative triple according to a softmax probability distribution.}
  \label{fig:uncertainty_sampler}
\end{figure*}
In addition, the different sampling methods were implemented in two inheriting classes \texttt{UncertaintySamplerSoftmax} and \texttt{UncertaintySamplerMax} which are using a softmax and a max probability distribution to sample.

As described in \Autoref{ch:approach}, the \texttt{Classifier} contains the additional functionality of classifying negative triples.
Therefore, for a given set of \texttt{logits} given by the generator, it calculates the probability that they are a negative triple (output = 0) or a positive triple (output = 1).
Accordingly, the function \texttt{classify} returns values $\in [0,1]$ for all negative triples.
This probability is determined on basis of given \texttt{min\_score} and \texttt{max\_score}.

The \texttt{UncertaintyScorer} is responsible for measuring and scoring the uncertainty of the generator model for given negative triples.
For this purpose, it has various types of an \texttt{UncertaintyMeasure} at its disposal, as shown in \Autoref{fig:uncertainty_measures}.
\begin{figure*}[t]
  \centering
    \includegraphics[width=0.95\textwidth]{figures/classdiagrams/UncertaintyScorer.pdf}
  \caption{Classdiagram with \texttt{UncertaintyScorer} and its \texttt{UncertaintyMeasure}.
  \texttt{UncertaintyMeasure} has several inherited classes depending on its implementation of how to measure uncertainty.}
  \label{fig:uncertainty_measures}
\end{figure*}
Thus, for the different uncertainty metrics, classes were implemented that override the \texttt{measure\_uncertainty} method and, for a given set of probabilities \texttt{probs}, assign an uncertainty score $u_i \in [0,1]$ for $i \in [1, |Neg|]$.

\section{Hyperparameter Optimization}
\label{sec:hyperparameter_optimization}
%
As in \cite{TuckER}, all hyperparameters were chosen by random search based on validation set performance.
Since the processing time on the \umls dataset is low and can also be performed on the local computer, it was initially mainly worked with.
Accordingly, the adaptation of the model and the hyperparameter tuning was based on the results on the \umls data set.
This was subsequently applied to the other datasets.
This resulted in the hyperparameters for the different models shown in \Autoref{tab:hyperparams}.

\begin{table}[H]
    \centering
    \begin{tabular}{lll}
        \toprule
        
        Model & 
        Hyperparameters & 
        Constraints or Regularizations \\
    
        \midrule
        
        \textsc{DistMult} & $k=50, \lambda=0.1$ & L2 regularization: $L_{reg}=L+\lambda||\Theta||_2^2$ \\ 
        
        \textsc{ComplEx}  & $k=25, \lambda=0.1$ & L2 regularization: $L_{reg}=L+\lambda||\Theta||_2^2$ \\
        
        \textsc{TransE} & $L_1$ distance, $k=50, \gamma=3$ & $||\mathbf{e}||_2\leq 1,||\mathbf{r}||_2\leq 1$ \\ 
        
        \textsc{TransD} & $L_1$ distance, $k=50, \gamma=3$ & $||\mathbf{e}||_2\leq 1,||\mathbf{r}||_2\leq 1,||\mathbf{e_p}||_2\leq 1,||\mathbf{r_p}||_2\leq 1$  \\ 
        
        \bottomrule
    \end{tabular}
    \caption{Hyperparameter settings for generator and discriminator models.
    For generator models a learning rate of $\lambda=0.1$ is used. 
    Constraints and regularizations are taken from \cite{cai2017kbgan}.
    All hyperparameters are shared among all datasets. 
    $\Theta$ represents all parameters in the model.}
    \label{tab:hyperparams}
\end{table}
The contraints and regularizations shown in the table are taken from the original paper \cite{cai2017kbgan}.
The hyperparameters include the dimension of the embeddings $k$, the learning rate $\lambda$, the metric for the distance-based embedding models, and their margin $\gamma$.
With these hyperparameters settings, an evaluation of our sampling method \usgan was carried out which is described in more detail in the next \Autoref{ch:evaluation}.

\section{Processing on $PC^2$}
\label{sec:processing_on_pc2}
%
Since most of the evaluated datasets are large and running algorithms on them are computationally expensive, the servers of \ac{PC2} \footnote{https://pc2.uni-paderborn.de/} were used to ensure faster processing of the learning process.
The \ac{OCuLUS} cluster was used for this because of its GPU support.
For this, a \ac{VPN} connection was created with the server and the corresponding files were uploaded to the user directory.
Subsequently, a setup of the configuration files for job manager CCS has been executed to start jobs with a \texttt{ccsalloc} command.
