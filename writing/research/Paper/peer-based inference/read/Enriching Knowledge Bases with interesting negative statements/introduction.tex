\section{Introduction}
\label{sec:intro}
\noindent
\textbf{Motivation and Problem.\ }
Structured knowledge is crucial in a range of applications like question answering, dialogue agents, and recommendation systems. The required knowledge is usually stored in KBs, and recent years have seen a rise of interest in KB construction, querying and maintenance, with notable projects being Wikidata~\cite{WD}, DBpedia~\cite{DBPEDIA}, Yago~\cite{YAGO}, or the Google Knowledge Graph~\cite{GKG}. These KBs store positive statements such as \textit{``Ren\'{e}e Zellweger won the 2020 Oscar for the best actress''}, and are a key asset for many knowledge-intensive AI applications.

A major limitation of all these KBs is their inability to deal with negative information~\cite{FHPPW2006}. At present, most major KBs only contain positive statements, whereas statements such as that \textit{``Tom Cruise did not win an Oscar''} could only be inferred with the major assumption that the KB is complete - the so-called \textit{closed-world assumption} (CWA). Yet as KBs are only pragmatic collections of positive statements, the CWA is not realistic to assume, and there remains uncertainty whether statements not contained in a KBs are false, or truth is merely unknown to the KB. 

Not being able to formally distinguish whether a statement is false or unknown poses challenges in a variety of applications. In medicine, for instance, it is important to distinguish between knowing about the absence of a biochemical reaction between substances, and not knowing about its existence at all. In corporate integrity, it is important to know whether a person was never employed by a certain competitor, while in anti-corruption investigations, absence of family relations needs to be ascertained. In data science and machine learning, on-the-spot counterexamples are important to ensure the correctness of learned extraction patterns and associations.

\noindent
\textbf{State of the Art and its Limitations.\ }
Absence of explicit negative knowledge has consequences for usage of KBs: for instance, today's \textit{question answering} (QA) systems are well
geared for positive questions, and questions where exactly one answer should be returned (e.g., quiz questions or reading comprehension tasks) \cite{Fader2014,WIKIQA}. In contrast, for answering negative questions like \emph{``Actors without Oscars''}, QA systems lack a data basis. Similarly,
they struggle with positive questions that have no answer, like \emph{``Children of Emmanuel Macron''},
too often still returning a best-effort answer even if
it is incorrect. Materialized negative information would allow a better treatment of both cases. 


\noindent
\textbf{Approach and Contribution.\ }
In this paper, we make the case that important negative knowledge should be explicitly materialized. We motivate this selective materialization with the challenge of overseeing a near-infinite space of false statements,
%\\
%\GW{what is ``near-infinite'' ?????}\ha{**}\\
 and with the importance of explicit negation in search and question answering. %We 
%then 
%develop two complementary approaches towards generating negative statements derived based on related entities.\\ 

We consider three classes of negative statements: (i) groun\-ded negative statements \textit{``Tom Cruise is not a British citizen''}, (ii) conditional negative statements \textit{``Tom Cruise has not won an award from the Oscar categories''} and (iii) universal negative statements \textit{``Tom Cruise is not member of any political party''}. In a nutshell, given a KB and an entity e, we select highly related entities to e (we call them \textit{peers}). We then use these peers to derive positive expectations about e, where the absence of these expectations might be interesting for e. In this approach, we are assuming completeness within a group of peers. More precisely, if the KB does not mention the \textit{Nobel Prize in Physics} as an award won by \textit{Stephen Hawking}, but does mention it for at least one of his peers, it is assumed to be false for \textit{Hawking}, and not a missing statement. This is followed by a ranking step where we use predicate and object prominence, frequency, and textual context in a  learning-to-rank model.


%Our 
%%%GW: avoid frequent use of first person plural
The salient contributions of this paper are:

\begin{enumerate}
%[noitemsep,topsep=0pt,parsep=0pt,partopsep=0pt]
%GW: journal paper has no hard space limit ==> no unnecessary fiddling with latex env settings
    \item We make the first comprehensive case for materializing \textit{useful} negative statements, and formalize important classes of such statements.
    \item We present a judiciously designed method for collecting and ranking negative statements based on knowledge about related entities.
    \item We show the usefulness of our models in use cases like entity summarization, decision support, and question answering.\\
%    and make public a number of resources (data, open-code, and web tool) to encourage future research on this topic.}
Experimental datasets and code are released as resources for further research\footnote{\url{https://www.mpi-inf.mpg.de/departments/databases-and-information-systems/research/knowledge-base-recall/interesting-negations-in-kbs/}}.
\end{enumerate}


The present article extends the earlier conference publication~\cite{negationakbc} in several directions:

\begin{enumerate}
    \item We extend the statistical inference to ordered sets of related entities, thereby removing the need to select a single peer set, and obtaining finer-grained contextualizations of negative statements (Section~\ref{sec:temporal}); 
    \item To bridge the gap between overly fine-grained groun\-ded negative statements and coarse universal negative statements, %\GW{this is incomprehensible without explicitly introducing these two categories earlier!} \ha{**how can we introduce them without being formal here, too soon?Can we be vague(we propose a new kind and show its necessity)}\\
    we introduce a third notion of negative statement, \emph{conditional negative statements}, and show how to compute them post-hoc (Section~\ref{sec:restricted});
    \item We evaluate the value of negative statements in an additional use case, with hotels from Booking.com (Section~\ref{sec:extrinsicevaluation}).
\end{enumerate}
