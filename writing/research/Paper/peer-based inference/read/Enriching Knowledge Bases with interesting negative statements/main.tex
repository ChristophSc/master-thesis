\documentclass[5p]{elsarticle} 

\usepackage{algpseudocode}
\usepackage{multirow}
\usepackage{footnote}
\usepackage{tabularx}
\usepackage{booktabs}
\usepackage{graphicx}
\usepackage{flushend}
\usepackage{amsmath}
\usepackage{tablefootnote}
\usepackage{hyperref}
\usepackage{enumitem}
\usepackage[toc,page]{appendix}
\usepackage[dvipsnames]{xcolor}
\usepackage{comment}
\usepackage[utf8]{inputenc}
\usepackage[linesnumbered,ruled]{algorithm2e}
\newcommand{\newtext}[1]{\textcolor{blue}{#1}}
\newcommand{\revisiontwo}[1]{\textcolor{purple}{#1}}
\newcommand{\ha}[1]{\textcolor{green}{Hiba: #1}}
\newcommand{\jeff}[1]{\textcolor{orange}{[Jeff: #1]}}
\newcommand{\term}[1]{\begin{small}\texttt{#1}\end{small}}
\newtheorem{defn}{Definition}
\newtheorem{problem}{Problem}
\newcounter{example}
\newenvironment{example}[1][]{\refstepcounter{example}\par\medskip
   \noindent \textbf{Example~\theexample. #1} \rmfamily}{\medskip}
\usepackage{textcomp}
\newcommand{\textapprox}{\raisebox{0.5ex}{\texttildelow}}   


\begin{document}

\newcommand{\sr}[1]{{\textcolor{red}{SR: #1}}}
\newcommand{\GW}[1]{{\color{purple}{GW: #1}}}

\title{Negative Statements Considered Useful}

\author[add1]{Hiba Arnaout\corref{cor1}}
\ead{harnaout@mpi-inf.mpg.de}
\author[add1]{Simon Razniewski}
\ead{srazniew@mpi-inf.mpg.de}
\author[add1]{Gerhard Weikum}
\ead{weikum@mpi-inf.mpg.de}
 \author[add2]{Jeff Z. Pan}
\ead{http://knowledge-representation.org/j.z.pan/}

\cortext[cor1]{Corresponding author}
\address[add1]{Max Planck Institute for Informatics, Saarland Informatics Campus, Saarbr{\"u}cken 66123, Germany}
\address[add2]{School of Informatics, The University of Edinburgh, Informatics Forum, Edinburgh EH8 9AB, Scotland}


\begin{abstract}


%
Knowledge bases (KBs)
%pragmatic collections of knowledge 
about notable entities and their properties
are an important asset in applications such as search, question answering and dialogue. 
%Rooted in a long tradition in knowledge representation, 
%GW: the fact that KBs ignore this KR work means they are not rooted in ...
All popular KBs capture virtually only positive statements, 
and abstain from taking any stance on statements not 
stored in the KB.
%  
This paper makes the case for explicitly stating salient statements 
%which 
that
%%%GW: there is a semantic diff between which and that
do \textit{not} hold.
Negative statements are useful to overcome
 limitations of question answering systems that are mainly geared for positive questions;
 they  can also contribute to informative summaries of entities.
 Due to the abundance of such invalid statements, any effort to compile them needs to address ranking by saliency. We present a statistical inference method for 
 compiling and ranking negative statements, based on expectations from positive statements of related  entities
 in peer groups. %which we then rank using supervised and unsupervised features.} 
 Experimental results, with a variety of datasets,
 show that 
 %this approach hold promising potential to 
 the method can effectively
 discover notable negative statements,
 and extrinsic studies underline their usefulness for
 entity summarization.
 %%%GW: negation means negating s.t.; negative statements is a self-contained noun phrase
 %Along with this paper, \newtext{we make the first resources on useful negative statements available, along with a web-based browsing interface for Wikidata entities, and code that allows researchers to discover useful negations in their own data.}
 Datasets and code 
 %for these experiments 
 are released as resources for further research.

%%% old abstract
\begin{comment}
Knowledge bases (KBs), pragmatic collections of knowledge about notable entities, are an important asset in applications such as search, question answering and dialogue. Rooted in a long tradition in knowledge representation, all popular KBs only store positive information, but abstain from taking any stance towards statements not contained in them.
% 
  In this paper, we make the case for explicitly stating useful statements which are \textit{not} true. Negative statements would be important to overcome current limitations of question answering, yet due to their potential abundance, any effort towards compiling them needs a tight coupling with ranking. We introduce an approach towards automatically compiling negative statements, based on expectations set from positive statements of related entities, which we then rank using supervised and unsupervised features. Experimental results show that this approach hold promising potential to discover notable negations. Along with this paper, we make the first resources on useful negative statements available, along with a web-based browsing interface for Wikidata entities, and code that allows researchers to discover useful negations in their own data.
\end{comment} 

\end{abstract}



\begin{keyword}
%knowledge bases \sep logics \sep negation \sep information retrieval\sep RDF \sep temporal data \sep information extraction \sep ranking
knowledge bases, negative knowledge, 
%temporal ordering, 
information extraction, 
statistical inference,
ranking
\end{keyword}

\maketitle    

\chapter{Introduction}
\label{ch:introduction}

% general introduction to the topic, definition of important terms
\acp{KG} represent structured collections of facts describing the world   \cite{hogan2020knowledge}.
These collections of facts have been used in a wide range of application, e.g., question answering, structured search \cite{zhang2019nscaching}, and link prediction \cite{cai2017kbgan, Alam2020AffinityDN}.
In recent years, numerous \acp{KG} such as \textsc{FB15K}, \textsc{WN18}, \textsc{YAGO} \cite{ConEx} and \textsc{WikiData} \cite{arnaoutwikinegata} have been released.
% TODO: Why are there several ones, which kind of data to they contain?
% TODO: Mention RDF and other format of triples?
Within a \ac{KG}, entities are stored as nodes and relations as directed edges \cite{zhang2019nscaching}.
Facts are represented as a triple in the form of (head entity, relation, tail entity), denoted as \triple{h}{r}{t}.
These triples indicate that the head entity \texttt{h} (subject) is connected with the tail entity \texttt{t} (object) by a specific relation \texttt{r} (predicate) \cite{zhang2019nscaching, Alam2020AffinityDN}.
% TODO: advantages and possibilities of KG graphs in comparison to (relational) databases
% https://www.ontotext.com/knowledgehub/fundamentals/what-is-a-knowledge-graph/
One of the main challenges for \acp{KG} is to find a Knowledge Graph Representation which encodes similarities and differences among  entities and relations. 
\ac{KRL} is a critical research issue and forms the basis for many knowledge acquisition tasks and applications.
For this reason, for example, simple approaches like one-hot encoding does not provide promising results.
Early works in this area used the symbolic triplet data for statistical relational learning, but this approach neither has good generalization performance, nor it can be applied for large scale \acp{KG} \cite{zhang2021efficient}.
However, \ac{KGE} is a renowned area of research in recent years which transforms a \acp{KG} into a low-dimensional vector space using embedding models \cite{Alam2020AffinityDN}.
Many approaches learn vector representations for \acp{KG} while learning a parametrized scoring function that assigns scores to input triples.
A score of a triple is expected to reflect the likelihood that the input triple is true \cite{ConvE, qiannegative}.
To learn these low-dimensional \acp{KGE} for entities and relations, several training approaches are available \cite{Ruffinelli2020You}.
During the training process of these approaches positive and negative triples are discriminated.
However, most known \acp{KG} contain only positive instances for space efficiency \cite{qiannegative} and not all positive information are represented in \acp{KG} either.
Accordingly, \acp{KG} are an incomplete picture of the reality.
Nevertheless, missing negative examples are needed to learn the \ac{KGE}.
Thus, different \ac{KGE} models have developed which are using different methods to generate the negative triples by Negative Sampling.
It is an essential part of distinguishing the models \cite{Ruffinelli2020You} and impacts the \ac{KGRL} and the performance of subsequent tasks which are using \acp{KGE}. 
Therefore, generating negative examples by Negative Sampling is an essential and important part of learning \acp{KGE}.





%- embedding based models:
%    - have better generalization ability
%    - better inference efficiency
%    - scalable
%    - shown promising performance in basic KG tasks
%\cite{qianunderstanding}:
%- One-hot encoding is broadly used to convert features or instances into vectors,
%-> great interpretability but incapable of capturing latent semantics



\section{Negative Sampling}
\label{sec:negative_sampling}

%Negative sampling was originally used for neuronal probabilistic language models and was referred to as importance sampling \cite{qiannegative, qianunderstanding}.
%Mikolov et al. \cite{MikolovSCCD13} refer to negative sampling as a simplified version of \ac{NCE} to overcome the computational difficulty which was associated with probabilistic language models \cite{qianunderstanding}.
%This was due to their partition functions summing over all words which is expensive if the amount of vocabulary is high \cite{qianunderstanding}.
%In comparison, negative sampling transforms the difficult density estimation problem into a binary classification problem.
%Consequently,  true samples are distinguished from noise samples to simplify the computation as well as accelerating training \cite{qianunderstanding}.
%Originally, the partition function was normalized into a probability distribution based on the entire vocabulary.
%Instead, true samples from a \ac{KG} are separated from noise distribution samples to asymptotically estimate the true distribution, which is highly efficient with low computational cost \cite{qianunderstanding}.

% HISTORY + GENERAL
Negative sampling originates from neuronal probabilistic language models where it was introduced to overcome the computational difficulty by transforming the difficult density estimation problem into a binary classification problem \cite{qianunderstanding}.
Considering nodes as words and the neighbors of the nodes as the context of a word, graph representation learning of \acp{KGE} is similar to language modeling \cite{qianunderstanding}.
Therefore, the idea was applied to \ac{KGE} learning and has become common practice over the past several years.
Instead of discriminating true samples from noise samples, positive triples are discriminated from negative ones.
Since bad or too obviously incorrect negative triples fail to capture the latent semantics in a \ac{KG} and lead to a zero loss problem, several negative sampling methods have been developed and used in different \ac{KGE} models \cite{qiannegative}.
Therefore, generated negative triples with a high quality ensure successful training, and the learned embedding performs better in downstream tasks, negative sampling becomes a very important part in \ac{KGE} learning.

% ASSUMPTIONS FOR LEARNING EMBEDDINGS
Since there are only positive triples in a \ac{KG} and thus no information is available about the distribution of negative triples or their quality, negative sampling is a challenging task for embedding models.
There are two different ways of looking at \acp{KG} at hand for negative sampling:
Negative triples can be generated either under the \ac{CWA} or the \ac{OWA} \cite{qiannegative}.
Both assumptions consider statements stored in the \ac{KG} as true, but differ on unobserved facts which are not present in a \ac{KG}.
With the \ac{CWA} they are considered as false, but the \ac{OWA} assumes that missing triples are simply unknown and consequently can be either true or false
\cite{qiannegative}.
The \ac{CWA} has two main drawbacks.
On the one hand, it has worse performance in downstream tasks \cite{qiannegative} since false-negative triples can be generated.
False-negative triples are defined as triples that are assumed not to be true, but actually, reflect true facts of the reality \cite{qianunderstanding}.
On the other hand, the \ac{CWA} has scalability issues due to a large number of negative samples \cite{qiannegative}.
Therefore, generating more informative negative triples by negative sampling with good performance represents a non-trivial step in \ac{KGE} learning.
Especially, because the quality of these generated negative triples has a direct impact on the \acp{KGE} \cite{qiannegative}.

% GENERAL PROCESS OF NEGATIVE SAMPLING
Overall, there are several variations for embedding learning with negative sampling in the literature.
However, they have the following two steps in common.
At first, for a given positive triple, a negative triple is generated.
Subsequently, positive triples from a \ac{KG} as well as generated negative triples from negative sampling are given to an embedding model.
Finally, embeddings are updated in the direction of the negative gradient of the loss function.
The following section elucidates this process and analyzes its problem.

\section{Problem Analysis}
\label{sec:problem_analysis}

Negative sampling plays an important role in embedding learning.
However, since there are no negative triples in a \ac{KG}, the generation of negative triples in current approaches poses some problems.
First of all, it can be said that while many negative sampling methods currently demonstrate high performance, the sampled negative triples are often too simple and represent a trivial solution. 
As a result, embedding models do not learn or learn less from the provided negative triples and therefore, they do not improve the embedding.
Instead, they suffer from the vanishing gradient or biased estimation problem \cite{zhang2021efficient}.
The vanishing gradient problem is present when the gradients of the loss functions approach zero and consequently, the model is unable to learn during the training process.
This results from the fact that most \ac{KGE} models, due to simplicity and efficiency, use Uniform Negative Random Sampling.

% UNIFORM RANDOM SAMPLING
Uniform Negative Random Sampling is a common technique of negative sampling where either the head or the tail entity in a given positive triple \triple{h}{r}{t} is randomly replaced by any other entity of the \ac{KG} which remains in the new negative triple \triple{h’}{r}{t} or \triple{h}{r}{t’}. 
Therefore, it is very likely to pick an entity which results in a zero gradient because the negative triple can be easily discriminated from the positive one \cite{cai2017kbgan}.
For example, by replacing the head entity of the positive triple \triple{h}{r}{t} = \triple{Joe Biden}{bornIn}{USA} with head entity \texttt{h'} = \texttt{Paderborn} would result in the negative triple \triple{h'}{r}{t} =  \triple{Paderborn}{bornIn}{USA} which is not very informative for the embedding.
By simply replacing the randomly selected head or tail entity of an again randomly selected entity of the \ac{KG} does not use any further information.
For example, it would have been useful if either negative sampling had recognized that the head entity \texttt{Joe Biden} is a person and to replace it with another person.
Moreover, recognizing that the tail entity as well as the sampled entity \texttt{Paderborn} is a location and its replacement would have led to the much more meaningful negative triple \triple{Joe Biden}{bornIn}{Paderborn}.  
Thus, while this approach is a fast and effective way to generate negative triples, it leads to a low learning factor in the embedding model.

% BERNOULLI SAMPLING
More useful negative examples are created by Bernoulli Sampling, which notes more information about a \ac{KG} and its individual entities and relations.
In comparison to Uniform Negative Random Sampling, it considers types of relations between entities (one-to-many, many-to-one and many-to-many) \cite{zhang2021efficient}.
These relation types are an indicator for the sampling approach if it is better to replace the head or the tail entity.
From the example above, it would have been recognized that the relation \texttt{bornIn} is a many-to-one relation.
Therefore, the head entity cannot have this relation to multiple entities, making each replaced tail entity a more useful negative triple.
Even though this is still a very fast and effective way to create negative triples, they are still easy to distinguish from positive ones.

% OTHER INFORMATION USED
In addition to these most commonly used methods, there are others which leverage external constraints such as entity types.
However, this resource does not always exist or is accessible \cite{cai2017kbgan}.
Instead of sampling from all entities in a \ac{KG}, other negative sampling methods take the approach of sampling only from a handful of selected entities.
For example, by sampling entities within the same domain, they hope to increase efficiency \cite{qiannegative}.
However, due to the rapid growth and frequent updating of \acp{KG}, constantly updating custom clusters is essential and skilled \cite{qiannegative}. 
Additionally, the creation of subsets of entities leads to a degradation of sampling performance and the information needed for this is not always available or is very difficult to derive from a \ac{KG}.


%Many of these approaches aim to find hard negative examples that are close to positive facts from a given \ac{KG} and thus have a positive effect on the embedding learning process.

% CHALLENGES
%Consequently, there are two main challenges for sampling negative triples \cite{zhang2021efficient}:
%At first, it is necessary to capture and model the dynamic distribution if negative triples to sample informative and useful negative triples with high gradients which help the model during the embedding learning process.
%Secondly, these negative triples have to be sampled effectively  so that the Negative Sampling does not negatively affect the performance of embedding learning.
 


\section{Objectives}
\label{sec:objectives}

% OBJECTIVE + HYPHOTETHESIS
Inspired by uncertainty sampling in active learning, we want to incorporate uncertainty information in an existing negative sampling process from embedding learning.
Unlike previous approaches, we thus do not sample a negative triple which is closest to a positive one, but the one where the embedding model is most uncertain about.
Therefore, the sampled instances can be hard negative examples, but also other negative triples which are valuable for the embedding model.
By implementing a new sampling method, we expect to generate more informative negative triples and therefore, have a positive impact on a negative sampling process.
This could be an acceleration of learning process like achieving same accuracies with less epochs, same accuracies with less training time or achieving a better overall accuracy.

% CONTRIBUTIONS
This thesis will make the following contributions:
First, we will look at existing approaches and analyze how uncertainty is measured in uncertainty sampling of active learning.
Subsequently, these approaches are reviewed for their applicability so that a new negative sampling approach can be created.
After an implementation of uncertainty sampling is carried out, 
an evaluation is performed by testing the new approach on different \ac{KG} datasets.
Due to the partly large amount of data and correspondingly longer execution times, \ac{PC2}\footnote{\url{https://pc2.uni-paderborn.de/}} will be used for this purpose.
Finally, the results are compared with existing methods to conclude the approach approach of \textbf{Sampling of Negative Triples for Knowledge Graph Embeddings by Uncertainty}.










\section{Related Work} 
\label{sec:relatedwork}

% EMBEDDING MODELS
The history of \acp{KG} goes back several years, but in recent years there has been a lot of research in this area, especially in the context of \acp{KGE}.  
Dai et. al \cite{electronics9050750} created an overview of several different and most common used embedding models, their approaches and application possibilities.
Among many different embedding models, the distance-based models \transe \cite{TransE} and \transd \cite{TransD} can be highlighted, which will also take a more important part in the course of this work.
Other embedding models represent \distmult \cite{DistMult} and \complex \cite{ComplEx}, which are based on semantic matching. 

% NEGATIVE SAMPLING
Due to the importance of sampling good negative triples, much research has also been done in this area in recent years.
However, standard techniques continue to be Uniform Random Sampling \cite{TransE} and Bernoulli Sampling \cite{TransH}, which do not produce high quality negative triples, but are used in many models due to their simplicity and high performance.  
A more complex approach to generate negative samples is, for example, domain sampling \cite{domainSampling}, which samples only entities from a subset of the entities of a \ac{KG}.
Pioneers of dynamic negative sampling are \kbgan \cite{cai2017kbgan} and \igan \cite{IGAN}, which attempt to estimate the distribution of negative triples by constructing a \ac{GAN}.
Inspired by \acp{GAN}, which were proposed for generating samples in a continuous space such as images, pre-trained models are improved through an adversarial learning process.
An overview over all the different negative sampling techniques is given for example in \cite{qiannegative} or \cite{MCNS}.

% ACTIVE LEARNING + UNCERTAINTY SAMPLING
In addition to this topic of \acp{KGE}, our approach includes other work in the literature regarding uncertainty sampling of active learning.
Originally, uncertainty sampling comes from active learning which supports supervised learning systems where unlabeled data is abundant, but it is difficult, time-consuming, or expensive to obtain labeled instances \cite{Settles2009ActiveLL}.
Several different approaches are available to select the instances to be labeled.
Uncertainty sampling is one of them and uses a classifier to identify unlabeled examples with the least confidence \cite{5272205}.
Therefore, the most informative unlabeled examples are selected for human annotation.
In turn, in the uncertainty sampling, several measures are available on how to obtain the uncertain cases of the classifier \cite{nguyen2021howtomeasure}.

\section{Structure of the Thesis}
\label{sec:structure_of_thesis}

This thesis is structured as follows:
At first in \textbf{\Autoref{ch:introduction}} a general introduction to the topic is given, the problem is analyzed, related work is mentioned and the objectives of this thesis are presented.
\textbf{\Autoref{ch:background}} establishes the important definitions of terms and presents the state of the art approaches and methods that exist in research.
In \textbf{\Autoref{ch:approach}} the motivation and a detailed description of the approach is given.
\textbf{\Autoref{ch:implementation}} deals with the practical realization and implementation of the approach.
Subsequently, \textbf{\Autoref{ch:evaluation}} evaluates our approach and compares it to existing methods using various datasets and metrics, and draws conclusions.
Lastly, \textbf{\Autoref{ch:summaryanddiscussion}} summarizes the work and addresses discussions for future work.


\section{State of the Art}
\label{sec:related}

%################################################
%################################################

\subsection{Negation in Existing Knowledge Bases}
\label{sec:existing}

%Before discussing research works, we review the state of the art of negation representation in existing popular KBs.
%GW: unnecessary intro, the headings tell everything already

%\GW{This entire subsection is too long and too verbose. The emphasis is on one example after the other -- the general point are not made concisely. Usually, I am all in favor of illustrative examples, but in the Related Work section they should be kept to a minimum.}\ha{**I'll work on shortening it, but should I keep the paragraphs and shorten their content or merge them together?}\\

%SR: Done, I removed many examples

\noindent
\textbf{Deleted Statements.\ }
Statements that were once part of a KB but got subsequently deleted are promising candidates for negative information~\cite{edithistory2019}. As an example, we studied deleted statements between two Wikidata versions from 1/2017 and 1/2018, focusing in particular on statements for people (close to 0.5m deleted statements). On a random sample of 1k deleted statements, we found that over 82\% were just caused by ontology modifications, granularity changes, rewordings, or prefix modifications. %, such as: \term{(Ghandi; lifes\-tyle; Vegetarian)} changed to \term{(Ghandi; lifestyle; Veget\-arianism)}, \term{(Ghandi; place of death; New Delhi)} changed to \term{(Ghandi; place of death; Gandhi Smriti)}, and \term{(James Green; oxfordID; 101011386)} to \term{(James Green; oxfordID; 11386)}. 
Another 15\% were statements that were actually restored a year later, so presumably reflected erroneous deletions. The remaining 3\% represented actual negation, yet we found them to be rarely noteworthy, i.e., presenting mostly things like corrections of birth dates or location updates reflecting geopolitical changes.

In Wikidata, erroneous changes can also be directly recorded via the deprecated rank feature~\cite{MKGGB2018}. Yet again we found that this mostly relates to errors coming from various import sources, and did not concern the active collection of interesting negations, as advocated in this article.%, like that ``Stephen Hawking did not win the Nobel Prize in Physics.''} 

\noindent
\textbf{Count and Negated Predicates.\ }
Another way of expressing negation is via counts matching with instances, for instance, storing 5 children statements for \textit{Trump} and numerical statement \term{(number of children; 5)} allow to infer that anyone else is not a child of \textit{Trump}. Yet while such count predicates exist in popular KBs, none of them has a formal way of dealing with these, especially concerning linking them to instance-based predicates~\cite{ghoshSWJ}.

Moreover, some KBs contain relations that carry a negative meaning. For example, DBpedia has predicates like \emph{carrier never available} (for phones), %, e.g., \term{(LG Shine; carrier never available; --02-07)} and 
or \emph{never exceed alt} (for airplanes), %, e.g., \term{(Piper PA 48 Enforcer; never exceed alt; 350)}. Another example is the biomedical KB 
Knowlife~\cite{ernst2015knowlife} contains medical predicates like \emph{is not caused by} %, e.g., term{(asthma; is not caused by; exercise pain management)} 
and \emph{is not healed by}, %, e.g., term{(asthma; is not healed by; brovana)}. A third example is Wikidata, for instance 
and Wikidata contains \emph{does not have part} and % (243 statements), or 
\emph{different from}. % (492K statements). 
Yet these present very specific pieces of knowledge, and do not generalize. %, e.g., \term{(arm; does not have part; hand)}, \term{(Hover Church; does not hav\-e part; bell tower)}, and \term{(brain death; different from; d\-eath)} which do not generalize to other Wikidata properties. 
Although there have been discussions to extend the Wikidata data model to allow generic opposites\footnote{\url{https://www.wikidata.org/wiki/Wikidata:Property_proposal/fails_compliance_with}}, these have not been worked out so far.

\noindent
\textbf{Wikidata No-Values.\ }
Wikidata can capture statements about \textit{universal absence} via the ``no-value'' symbol~\cite{erxleben2014introducing}. This allows KB editors to add a statement where the object is empty. For example, what we express as \term{$\neg \exists x$(Angela Merkel; child; x)}, the current version of Wikidata allows to be expressed as \term{(Angela Merkel; child; no-value)}\footnote{\url{https://www.wikidata.org/wiki/Q567}}. As of 8/2021, there exist 135k of such ``no-value'' statements, yet only used in narrow domains. For instance, 53\% of these statements come for just two properties \textit{country} (used almost exclusively for geographic features in Antarctica), and \textit{follows} (indicating that an artwork is not a sequel).


%\noindent
%\newtext{\textbf{Ongoing discussions.\ }
%An interesting discussion took place on the \textit{Wikidata's Project Chat} webpage\footnote{\url{https://www.wikidata.org/wiki/Wikidata:Property_proposal/fails_compliance_with}} about the need for an ``opposite'' to a property. More particularly, the opposite of the property \emph{complies with (P5009)} to state when an entity does not comply with the criterion associated with an entity. There is a way of stating that the film \term{Beauty and the Beast; complies with; the Bechdel test}. However, stating that the film ``Hackers'' fails to comply with the ``Bechdel test'' cannot be stated by simply negating the property \emph{complies with}, but through a workaround that introduced the negative entity (object) \emph{``fails the Bechdel Test (Q45172088)''}, and then stating that ``Hackers'' \emph{has quality} \emph{``fails the Bech\-del Test (Q45172088)''}. This, however, is not a practical nor generalizable way to deal with every possible negation that the KB presents.}






\subsection{Negation in Logics and Data Management}

%\GW{you need to spell out CWA and OWA upon first mention;
%you need to define and explain PCA;
%I prefer LCA (local closed world assumption) over the ambiguous acronym PCA)}\ha{** wait, aren't we confusing two different concepts here? the first is when your peer has it, you should have it. So if Einstein has the nobel, Hawking should. Otherwise it's negative, and that's what we describe as LCWA. The other concept is in order to go forward with a candidate negation, the entity should have at least one other object for the same predicate. If Hawking has won at least one other award, then we proceed with the candidate he has not won the nobel. We are referring to this as PCA.}\\

Negation has a long history in logics and data management. Early database paradigms usually employed the closed-world assumption (CWA), i.e., assumed that all statements not stated to be true were false \cite{Reiter78}, \cite{ICWA}. On the Semantic Web and for KBs, in contrast, the open-world assumption (OWA) has become the standard. The OWA asserts that the truth of statements not stated explicitly is unknown. Both semantics represent somewhat extreme positions, as in practice it is neither conceivable that all statements not contained in a KB are false, nor is it useful to consider the truth of all of them as unknown, since in many cases statements not contained in KBs are indeed not there because they are known to be false~\cite{razniewskilimits}. Between these two assumptions, there is also the so-called local (partial) closed-world assumption~\cite{RPZ2010c}, where the open-world assumption is used in general, while the  closed-world assumption can be applied to some predicates (classes or properties).

In limited domains, logical rules and constraints, such as Description Logics \cite{logichandbook}, \cite{Calvanese2007} or OWL, can be used to derive negative statements. An example is the statement that every person has only one birth place, which allows to deduce with certainty that a given person who was born in \textit{France} was not born in \textit{Italy}. OWL also allows to explicitly assert negative statements \cite{mcguinness2004owl}, yet so far is predominantly used as ontology description language and for inferring intensional knowledge, not for extensional information (i.e., instances of classes and relations), with a few exceptions, like the rewriting based approach to instance retrieval for negated concepts, based on the notion of inconsistency-based first-order-rewritability~\cite{DuPa2015}. Different levels of negations and inconsistencies in Description Logic-based ontologies are proposed in a general framework~\cite{FHPPW2006}.

In~\cite{DBLP:journals/amai/AnalytiADP13,Analyti04negationand}, a thorough study on negative information in the Resource Description Framework (RDF) argues in favor of explicit negation. In particular, it makes the point that any knowledge representation formalism must be able to deal with \textit{informative} negative information, on top of informative positive information. The authors then propose ERDF (extended RDF), where an ERDF triple can be either positive or negative. The framework also distinguishes between two kinds of negation: weak (``she doesn't like snow'') and strong (``she dislikes snow''). The former is denoted using the \textapprox \   symbol, and the latter using the $\neg$ symbol.

The notion of \texttt{noValue} in RDF was introduced in~\cite{DBLP:books/aw/AbiteboulHV95}. It has been recently adapted in~\cite{DBLP:conf/semweb/DarariPN15} for representing no-value information in RDF and incorporating such information into query answering. The intuition behind it is to distinguish whether a result set of a SPARQL query is empty due to lack of information or actual negation.

The AMIE framework \cite{galarraga2017predicting} employed rule mining to predict the completeness of properties for given entities. This corresponds to learning  whether the CWA holds in a local part of the KB, inferring that all absent values for a subject-predicate pair are false. For our task, this could be a building block, but it does not address the inference of {\em useful} negative statements.

RuDiK~\cite{ortona2018rudik} is a rule mining system that can learn rules with negative atoms in rule heads (e.g., people born in \textit{Germany} cannot be
\textit{U.S.} president). This could be utilized towards predicting negative statements. %Unfortunately, the mining also discovers many convoluted and exotic rules (e.g., people whose body weight is less than their birth year cannot win the Nobel prize), often with  a large number of atoms in the rule body, and such rules are among the top-ranked ones. 
%Even good rules, such as ``people with birth year after 2000 do not win the Nobel prize'', are not that useful for our task. 
Unfortunately, such rules predict way too many -- correct, but uninformative -- negative statements, essentially enumerating a huge set of people who are not \textit{U.S.} presidents. 
%\GW{do we really need this lengthy example? what is the key point that matters for the paper's big picture? is there any?}\ha{**}\\
The same work also proposed a precision-oriented variant of the CWA that assumes negation only if subject and object are connected by at least one other relation. Unfortunately, this condition is rarely met in interesting cases. For instance, most of the negative statements in Table~\ref{tbl:may:einstein} have alternative connections between subject and object in Wikidata.




\subsection{Related Areas}

\noindent
\textbf{\small Linguistics and Textual Information Extraction (IE).\ } Negation is an important feature of human language \cite{Morante2012}. While there exists a variety of ways to express negation, state-of-the-art methods are able to detect quite reliably whether a segment of text is negated or not \cite{extendingnegex}, \cite{wu2014}.  There is also work on using knowledge graphs to help detect false statements in texts, such as news~\cite{PPLL+2018}. %Yet theories of conversational schemes indicate that negative statements can also be inferred from sentences that do not contain explicit negation: For instance, following Grice's maxims of cooperative communication \cite{grice1975logic}, a reasonable conclusion from the sentence ``John has two children, Mary and Bob'' is that nobody else is a child of John. Such inferences are called scalar implicatures, and they play a considerable role in language pragmatics~\cite{carston1998informativeness}.
%

A body of work targets negation in medical data and health records. In \cite{cruzdiaz}, a supervised system for detecting negation, speculation and their scope in biomedical data is developed, based on the annotated BioScope corpus \cite{bioscope}.
In \cite{Goldin03learningto}, the focus is on negations via the keyword ``not''. The challenge here is the right scoping, e.g., ``Examination could not be performed due to the Aphasia'' does not negate the medical observation that the patient has Aphasia.
In \cite{KEhealth}, a rule-based approach based on NegEx~\cite{CHAPMAN}, and a vocabulary-based approach for prefix detection were introduced.
PreNex \cite{PRENEX} also deals with negation prefixes. The authors propose to break terms into prefixes and root words to identify this kind of negation. They rely on a pattern matching approach over medical documents. 



In \cite{AKB}, an anti-knowledge base containing negations is mined from Wikipedia change logs, with the focus however being again on factual mistakes, and precision, not interestingness, is employed as main evaluation metric. In~\cite{CSKB}, the focus is to obtain meaningful negative samples for augmenting commonsense KBs.
We explore text extraction in more details in the proposed \emph{pattern-based query log extraction} method in our earlier conference publication~\cite{negationakbc}.


\noindent
\textbf{Statistical Inference and KB Completion.\ } As text extraction often has limitations, data mining and machine learning are frequently used on top of extracted or user-built KBs, in order to detect interesting patterns in existing data, or in order to predict statements not yet contained in a KB. There exist at least three popular approaches, rule mining, tensor factorization, and vector space embeddings \cite{KGembsurvey}. Rule mining is an established, interpretable technique for pattern discovery in structured data, and has been successfully applied to KBs for instance by the AMIE system \cite{AMIE3}. Tensor factorization and vector space embeddings are latent models, i.e., they discover hidden commonalities by learning low-dimensional feature vectors \cite{global2014}. To date, all these approaches only discover positive statements. On the other hand, if one considers logical entailments as a means to enhance such rule mining and latent model based approaches, such as in an iterative manner~\cite{WPKD2020}, negative statements in theory can be discovered with the help of disjoint axioms; however, the quality of knowledge graph completion methods still have room for improvement. Recently, an inference model has been proposed to build a knowledge graph with commonsense contradictions~\cite{ANION}, like ``Wearing a mask is seen as responsible'' is the contradiction of ``Not wearing a mask is seen as carefree''.

\noindent
\textbf{Ranking KB Statements.\ } In applications such as entity summarization over web-scale KBs, returned result sets are often very large. Ranking statements is a core task in managing access to KBs, with techniques often combining generative language-models for queries on weighted and labeled graphs~\cite{NAGA,Yahya2016,arnaoutjws2018}. In \cite{Bast}, the authors propose a variety of functions to rank values of type-like predicates. These algorithms include retrieving entity-related texts, binary classifiers with textual features, and counting word occurrences.  In \cite{huang2019contextual}, the focus is on identifying the informativeness of statements within the context of the query, by exploiting deep learning techniques. In this work, applications such as entity summarization returns a set of \textit{negative} statements. To assign each statement a relevance score, we use a mixture of the metrics that are usually used for ranking positive statements (e.g., frequency of property), and metrics that are specific for negative statements (e.g., unexpectedness).


%\newtext{Although these approaches has not tackled the specifics of negative statements, we use some of the proven useful features in our work, namely frequency and popularity of entities and relations. On top of that, we introduce additional ranking metrics.}
%%%GW: unnecessary generic text, no extra information




\section{Model}
\label{sec:formalization}

%\GW{problem statements and research problems come a little late here, all this should be clear already by now;
%hence rename the section into "Model and Design Space",
%and drop repetitive "problem formalizations";\\
%the entire section does not look and work like a design space discussion, most is about the model, and the alternative designs are just deleted statements and something vague about PCA, plus there is some repetition of related work;\\
%either beef this up with systematic discussion, or change the heading and trim the section}\ha{**}\\

For the remainder we assume that a KB is a set of statements, each being a triple $(s;p;o)$ of subject $s$, property $p$ and object $o$.

%\noindent
%\textbf{Formalization.\ }
%For the remainder we assume that a KB$\doteq (\mathcal{T}, \mathcal{S})$ consists of $\mathcal T$ is a set of %statements, each being a\ triples \term{(s; p; o)} of subject \term{s}, property \term{p} and object \term{o}, while $\mathcal S$ is the schema, or TBox in the terminology of Description Logics (DLs). We refer the reader to \cite{logichandbook,Pan2016, Pan2017} for a more detailed introduction of description logics and knowledge graphs. In this paper, we assume that the KB might have very simple schema axioms, such as domain axioms for properties.  \\
%\GW{unnecessarily wide: we only need SPO statements, can't we define this in one sentence?}\ha{**}\\

Let $K^i$ be an (imaginary) ideal KB that perfectly represents reality, i.e., contains exactly those statements that hold in reality. Under the OWA, (practically) available KBs, $K^a$ contains correct statements, but may be incomplete, so the condition $K^a \subseteq K^i$ holds, but not the converse \cite{razniewski2011completeness}.
We distinguish two forms of negative statements.


\begin{defn}[Negative Statements] \mbox{ }
 % Let $s,o$ be  entities,   and $p$ a property:
\begin{enumerate}[noitemsep,topsep=0pt,parsep=0pt,partopsep=0pt]
\item A grounded negative statement $\neg (s, p, o)$ is satisfied if $(s, p, o) \notin K^i$.
 %   \item A \textit{grounded negative statement} {\normalfont \term{$\neg$(s; p; o)}} is  satisfied if $K^i \ \cup \ \{ ${\normalfont \term{(s; p; o)}}$\} \models \bot$ and $K^{i} \models s:domain(p)$, % is not in $K^i$.
%    \item A \textit{restricted negative statement} {\normalfont \term{$\neg\exists$(s; p; \_), (\_, p', o')}} is satisfied if there exists no {\normalfont \term{o}} such that {\normalfont \term{(s; p; o) and (o, p', o')}} are in $K^i$.
\item A universally  negative statement $\neg\exists o: (s, p, o)$ is satisfied if there exists no $o$ such that $(s; p; o) \in K^i$.
%    \item A \textit{universally negative statement} {\normalfont \term{$\neg$(s; p; \_)} is satisfied if $K^i \ \cup \ $ \{\term s: $  \exists \term p.\top)\}   \models \bot$, where domain(\term p) is the domain of the property \term p.}%; 
 %   \item  A \textit{scoped universally negative statement} {\normalfont \term{$\neg$(s; p; \_)$_{Sc}$  } is satisfied if $K^i \cup $ \{\term s$: %domain(\term p)
   % Sc \sqcap \forall \term p.\bot\} \not \models \bot$, where $Sc$ is the scope class.%;
\end{enumerate} 
\end{defn}

%\sr{Thanks making this more formal! One question on syntax though, if I instantiate the second definition with (Merkel, child, \_), I obtain that this statement is satisfied, if $K^i \ \cup \ $ \{\term Merkel: $\neg($domain(\term child)$\ \sqcap \ \forall \term child.\bot)\}   \models \bot$, where the second part of the union, I presume, corresponds to something like \{\term Merkel: $\neg(Merkel, Putin, Pope Francis, ...)\}$ - which looks syntactically not quite right - is there perhaps something missing?}

%\GW{here you need examples!!!!!}

%\jeff{Thanks for checking. In the case of (Markel, child, \_), we have \{Markel: $\neg (Person \sqcap \forall child.\bot$)\} union with K$^i$ should be inconsistent. Note that \{Markel: $\neg (Person \sqcap \forall child.\bot$)\} can be transformed into \{Markel: $\neg Person \sqcup \exists child.\top\}$ }

An example of a grounded negative statement is that \textit{``Bruce Willis was not born in the U.S.''}, and is expressed as \term{$\neg$(Bruce Willis; born in; U.S.)}. An example of a universally negative statement is that \textit{``Leonardo DiCaprio has never been married''}, expressed as \term{$\neg\exists o:$(Leonardo DiC\-aprio; spouse; o)}. Both types of negative statements represent standard logical constructs, and could also be expressed in the OWL ontology language. Grounded negative statements could be expressed via negative property statements (e.g., {\small{\texttt{NegativeObjectPropertyAssertion }}{\small\texttt{(:born In :Bruce Willis :U.S.)}}}), while universally negative statements could be expressed via \texttt{ObjectAllValuesFrom} or \texttt{owl:complementOf} \cite{erxleben2014introducing} (e.g., {\small{\texttt{ ClassAssertion (ObjectAl\-lValuesFrom (:spouse owl:Nothing) :Leonardo Dicaprio)}}}). Without further constraints, for these classes of negative statements, checking that there is no conflict with a positive statement is trivial. In the presence of further constraints or entailment regimes, one could resort to (in)cons\-istency checking services \cite{logichandbook,Pan2017,gadwww}.

% SR: Reformulated a bit, note that consistency is not the same as correctness. A statement like ``Simon has no siblings'' is consistent with a KB, until someone asserts a sibling. Whereas correctness is a somewhat higher goal, of not just asking whether one can state something contradiction-free, but whether the statement is also true in reality.


Yet compiling negative statements faces two other challenges.
First, being not in conflict with positive statements is a necessary but not a sufficient condition for correctness of negation, due to the OWA. In particular, $K^i$ is only a virtual construct, so methods to derive correct negative statements have to rely on the limited positive information contained in $K^a$, or utilize external evidence, e.g., from text. Second, the set of correct negative statements is near-infinite, especially for grounded negative statements. Thus, unlike for positive statements, negative statement construction/extraction needs a tight coupling with ranking methods.

\noindent
\textbf{Research Problem 1.\ }
Given an entity $e$, compile a ranked list of useful grounded negative and universally negative statements.


%\noindent
%\textbf{Design Space.\ }
%A first thought is that \textit{deletions from time-variant KBs} are a natural source. For instance, in Wikidata, for subjects of type
%person
%within the year 2018, more than 500K triples have been deleted. Yet on careful inspection we found that most of these concern ontology restructuring, granularity refinements, or blatant typos, thus do not give rise to important negation.\\
%\GW{this is mere repetition of related work}\ha{**}\\


%A second conceivable approach is to leverage the CWA, or its relaxed variant PCA (Partial Completeness Assumption, aka. LCWA for Local CWA) \cite{AMIEP}, to generate negative statements. \newtext{Informally, PCA says if an entity $s$ has some value for a property  $p$, then we can assume that  all of its values for $p$ are in $KB^a$}
%\newtext{
%\begin{defn}[Partial Completeness Assumption] \mbox{ }
% The PCA is the assumption that if
%Given 
%\term{(s; p; o)} $\in K^a$ %for 
%some entities $s,o$ and property $p$, 
%then: %
%the partial completeness assumption states that
%\small
%\begin{equation}
%\label{eqn:pca}
%\forall (s; p; o'): (s; p; o') \in K^i \Longleftrightarrow (s; p; o') \in K^a.
%\end{equation}
%\normalsize
%\end{defn}
%}
%\GW{no OWL here: good; the variables in the universal quantifier do not make sense, since you already said that this is for a given spo $\in$ Ka -- so the universal quantifier should be forall o' only; alternatively you can drop the prefix "if spo $\in$ Ka", but then the entire formula needs to be written differently}\ha{**}\\

%Using just the active domain of Wikidata for grounding, the CWA would give rise to about $6.4\times10^{18}$ negative statements\footnote{80 Million subjects times 1000 properties times 80 Million objects.}. Assuming that Wikidata covers 10\% of all true statements per entity, more than 99.999\% of the negative statements would be correct, but hardly noteworthy. For the PCA, the total would be about $3.2\times10^{16}$ negative statements (assuming an average of 5 populated properties per entity), and almost all of these would be correct.
%But these approaches would miss the true issue: merely enumerating huge sets of negative statements is not insightful even with (trivially) high precision.
%The key challenge rather is to identify useful statements that users find noteworthy.

%Statistical inference methods, ranging from association rule mining such as AMIE and RuDiK~\cite{AMIE,ortona2018rudik} to embedding models such as TransE and HolE~\cite{transE,holE} can predict positive statements and provide ranked lists of role fillers for KB relations. In Section~\ref{sec:inference}, we develop a statistical inference method for negative statements, which generates candidate sets from related entities, and uses a set of popularity and probability heuristics in order to rank these statements. 
\section{Peer-based Statistical Inference}
\label{sec:inference}
%\ha{we need to rename our methods, emphasize the peer group (but they both include peers so I am not sure; we can call the peer groups something else in the temporal method}
We next present a method to derive useful negative statements by combining information from similar entities (``peers'') with supervised calibration of ranking heuristics. The idea is that peers that are similar to a given entity can give expectations on relevant statements that \textit{should} hold for the entity. For instance, several entities similar to the physicist \emph{Stephen Hawking} have won the \textit{Nobel in Physics}. We may thus conclude that him not winning this prize could be an especially useful statement. Yet related entities also share other traits, e.g., many famous physicists are \textit{U.S.} citizens, while \textit{Hawking} is \textit{British}. We thus need to devise ranking methods that take into account various clues such as frequency, importance, unexpectedness, etc.\\

\noindent
\textbf{Peer-based Candidate Retrieval.\ }
To scale the method to web-scale KBs, in the first stage, we compute a candidate set of negative statements using the CWA on certain parts of the KB, to be ranked in the second stage. Given a subject $e$, we proceed in three steps:
\begin{enumerate}[noitemsep,topsep=0pt,parsep=0pt,partopsep=0pt]
    \item \textit{Obtain peers:} We collect entities that set expectations for statements that $e$ could have, the so-called \textit{peer groups} of $e$. These groups can be based on (i) structured facets of the subject~\cite{RECOIN}, such as \textit{occupation, nationality}, or \emph{field of work} for people, or classes/types for other entities, (ii) graph-based measures such as distance or connectivity~\cite{ponza}, or (iii) entity embeddings such as TransE~\cite{transE}, possibly in combination with clustering, thus reflecting latent similarity. 
    \item \textit{Count statements:} We count the relative frequency of all predicate-object pairs (i.e., \term{(\_,p,o)}) and predicates (i.e., \term{(\_,p,\_)}) within the peer groups, and retain the maxima, if candidates occur in several groups. This way, statements are retained if they occur frequently in at least one of the possibly orthogonal peer groups.
    \item \textit{Subtract positives:} We remove those predicate-object pairs and predicates that exist for $e$.
\end{enumerate}
\  \\
Algorithm~\ref{alg:peer} shows the full procedure of the peer-based inference method. 
In line 2, groups of peers $P[]$ are selected based on some blackbox function \textit{peer\_groups}.
\begin{equation*}
P = [P_1, ... P_n] \text{, with } n>=1.
\end{equation*}
\noindent
Every group $P_i$ is a set of peers, defined as follows.
\begin{equation*}
P_i = \{pe_1, ..., pe_m\} \text{, with } m<=s.
\end{equation*}
Subsequently, for each peer group, it collects all the positive information that these peers have (line 6 and 7), and stores them as a list of candidate statements.
\begin{equation*}
candidates = \{st_1, ..., st_w\}.
\end{equation*}
A statement $st_j$ in $candidates$ is either a predicate P or a predicate-object pair PO.
After collecting information about the peers, the loop at line 10 iterates over the list of unique statements $ucandidates$, computes their relative frequency, and stores them in the final list of negations $N$. $N$ is a list of negation objects~\footnote{Here, object is meant as a data type and not a KB-triple object.}, where every object consists of a negation statement and its score.
\begin{equation*}
N = [(\neg st_1, sc_1), ..., (\neg st_r, sc_r)].
\end{equation*}
Across peer groups, it retains the maximum relative frequencies (hence, line 13), if a property or statement occurs across several. Before returning the top $k$ results as output (line 18), it subtracts those already possessed by entity $e$ (line 17).

\begin{algorithm*}[t!]
    \SetKwInOut{Input}{Input}
    \SetKwInOut{Output}{Output}
    \Input{\small knowledge base $\mathit{KB}$, entity $e$, peer collection function \textit{peer\_groups}, \small max. size of a peer group $s$, \small number of results $k$}
    \Output{ \small $k$-most frequent negative statement candidates for $e$}
        \textit{\textbf{P}}$[]$= \textit{peer\_groups}$(e, s)$ \Comment{List of peer group(s); Group $P_i$ at position $i$ is one group (set) with at most $s$ peers.}\\
         \textit{\textbf{N}}$[]$= $\emptyset$ \Comment{\small Ranked list of negative statements about $e$.}\\
    \For{$P_i$ $\in$ \textbf{P}}{ 
        $candidates$ = [] \Comment{\small Positive statements (i.e., predicate and predicate-object pairs) of $P_i$ members.}\\
        \For{$pe$ $\in$ $P_i$}{ 
            $candidates$+=$collectP(pe)$ \Comment{\small Collecting predicates that hold for one peer (pe).}\\
            $candidates$+=$collectPO(pe)$ \Comment{\small Collecting predicate-object pairs that hold for pe.}\\
        }
        $ucandidates$ = $unique(candidates)$\Comment{\small List of unique statements in $candidates$.}\\
        \For{$st$ $\in$ $ucandidates$}{
            $sc$ = $\frac{count(st, candidates)}{s}$ \Comment{\small sc computes how many peers share the statement st, normalized by $s$.}\\
            \If{$getnegation(N, st).score$ $<$ $sc$}{ $setscore(N, st, sc)$}  
        }
     }
        $N$-=$\mathit{inKB}(e,N)$ \Comment{\small Remove statements $e$ already has.}\\
        return $max(N,k)$
\caption{Peer-based candidate retrieval algorithm.}
\label{alg:peer}
\end{algorithm*}


\begin{example}
Consider the entity $e$=\textit{Brad Pitt}. Table \ref{tab:brad} shows a few examples of his peers and candidate negative statements.
We instantiate the peer group choice to be based on structured information, in particular, shared occupations with the subject, as in Recoin~\cite{RECOIN}. In Wikidata, \textit{Pitt} has 9 occupations, thus we would obtain 9 peer groups of entities sharing one of these with \textit{Pitt}.
\begin{equation*}
P = [\text{actors, film directors, ..., models}] \text{, with } n=9.
\end{equation*}
For readability, let us consider statements derived from only one of these peer groups, \emph{actor}. Let us assume 3 entities in that peer group.
\begin{equation*}
P_\text{actor} = \{\text{Russel Crowe, Tom Hanks,  Denzel Washington}\}
\end{equation*}
The list of negative candidates, $candidates$, are all the predicate and predicate-object pairs shown in the columns of the 3 actors. And in this particular example, $N$ is just $ucandidates$ with scores for only the \textit{actor} group.
\begin{equation*}
\begin{split}
N = [(\neg (\text{award; Oscar for Best Actor}), 1.0),\\  (\neg \exists x(\text{instagram; }x\text), 0.67),\\ 
(\neg (\text{citizen; New Zealand}), 0.33),\\ 
(\neg \exists x(\text{convicted; }x), 0.33),\\ 
(\neg \exists x(\text{child; }x), 1.0),\\  (\neg(\text{occupation; screenwriter}), 1.0),\\  (\neg(\text{citizen; U.S.}), 0.67)].
\end{split}
\end{equation*}
Candidates that \textit{hold} for \textit{Pitt} are then dropped.
\begin{equation*}
\begin{split}
N = [(\neg (\text{award; Oscar for Best Actor}), 1.0),\\  (\neg \exists x(\text{instagram; }x), 0.67),\\ 
(\neg (\text{citizen; New Zealand}), 0.33),\\ 
(\neg \exists x(\text{convicted;} x), 0.33),\\  (\neg(\text{occupation; screenwriter}), 1.0)].
\end{split}
\end{equation*}
 The top-k of the rest of candidates in $N$ are finally returned. The top-3 negative statements, for this example, are \term{$\neg$(award; Oscar for Best Actor)}, \term{$\neg$(occupation; screenwriter)}, and \term{$\neg \exists x$(instagram; x)}. 

The ``if'' statement at line 12 is only needed when multiple peer groups are considered for an entity. In the case where a negative statement is inferred from more than 1 group, only the version with the highest score is added to the final set. In the original (\textit{full}) example, \textit{Pitt} belongs to the group \textit{actor} and the group \textit{model}. The negation \term{$\neg$(occupation; screenwriter)} was inferred twice, once from each group, with a relative frequency of 0.9 from the \textit{actor} group and 0.2 from the \textit{model} group. We add the one with the higher score to the final set and disregard the other one. An alternative is to combine or compute the average of the scores across groups.

Note that without proper thresholding, the candidate set grows very quickly, for instance, if using only 30 peers, the candidate set for \textit{Pitt} on Wikidata is already about 1500 statements.
\end{example}\\

\begin{table*}
  \caption{Discovering candidate statements for \textit{Brad Pitt} from one peer group with 3 peers.}
  \label{tab:brad}
   \resizebox{\textwidth}{!}{\begin{tabular}{lll|l|l}
    \toprule
 \multicolumn{1}{c}{\bf{Russel Crowe}} &  \multicolumn{1}{c}{\bf{Tom Hanks}} &  \multicolumn{1}{c}{\bf{Denzel Washington}} &  \multicolumn{1}{|c}{\bf{Brad Pitt}} & \multicolumn{1}{|c}{\bf{Candidate statements}}\\
    \midrule
(award; Oscar for Best Actor) & (award; Oscar for Best Actor) & (award; Oscar for Best Actor) & (citizen; U.S.) & $\neg$(award; Oscar for Best Actor), 1.0\\
(citizen; New Zealand) & (citizen; U.S.) & (citizen; U.S.) & (child; $x$)& $\neg$(occup.; screenwriter), 1.0\\
(child; $y$) & (child; $z$) & (child; $u$) &  & $\neg \exists l $(instagram; $l$), 0.67\\
(occup.; screenwriter) & (occup.; screenwriter) & (occup.; screenwriter) & & $\neg w$(convicted; $w$), 0.33\\
(convicted; $v$)  & (instagram; $r$) & (instagram; $f$) & & $\neg$(citizen; New Zealand), 0.33\\
(instagram; $t$) & & & & \\
    \bottomrule
  \end{tabular}}
\end{table*}

\noindent
\textbf{Ranking Negative Statements.\ }
Given potentially large candidate sets, in a second step, ranking methods are needed. Our rationale in the design of the following four ranking metrics is to combine frequency signals with popularity and probabilistic likelihoods in a \emph{learning-to-rank model}.
\begin{enumerate}[noitemsep,topsep=0pt,parsep=0pt,partopsep=0pt]
\item \emph{Peer frequency (PEER):} The statement discovery procedure already provides a relative frequency, e.g., 0.9 of a given actor's peers are married, but only 0.1 are political activists. The former is an immediate candidate for ranking.
    \item \emph{Object popularity (POP):} When the discovered statement is of the form  $\neg$\term{(s; p; o)}, its relevance might be reflected by the popularity\footnote{Wikipedia page views.} of the \term{Object}. For example, $\neg$\term{(Brad Pitt; award; Oscar for Best Actor)} would get a higher score than $\neg$\term{(Brad Pitt; award; London Film Critics' Circle Award)}, because of the high popularity of the \textit{Academy Awards} over the \textit{London Film Awards}.
    \item \emph{Frequency of the Property (FRQ):} When the discovered statement has an empty \term{Object} \term{$\neg \exists x$(s; p; x)}, the frequency of the \term{Property} will reflect the authority of the statement. To compute the frequency of a \term{Property}, we refer to its frequency in the KB. For example, \term{$\neg \exists x$(Joel Slater; citizen; x)}  will get a higher score (4.1m citizenships in Wikidata) than \term{$\neg \exists x$(Joel Slater; twitter; x)} (294k Twitter usern\-ames).
    \item \emph{Pivoting likelihood (PIVO):} In addition to these \linebreak frequency/view-based metrics, we propose to consider textual background information about $e$ in order to better decide whether a negative statement is relevant. To this end, we build a set of statement pivoting classifier~\cite{razniewski2017doctoral}, i.e., classifiers that decide whether an entity has a certain statement (or property), each trained on the Wikipedia embeddings~\cite{wikipedia2vec} of 100 entities that have a certain statement (or property), and 100 that do not\footnote{On withheld data, linear regression classifiers achieve 74\% avg.\ accuracy on this task.}. To score a new statement (or property) candidate, we then use the pivoting score of the respective classifier, i.e., the likelihood of the classifier to assign the entity to the group of entities having that statement (or property).
\end{enumerate}

\noindent
The final score of a candidate statement is then computed as follows.

\begin{defn}[Ensemble Ranking Score]
\label{def:ensemble}
{\small
\begin{equation*}
Score=\begin{cases}
 \lambda_1 \text{PEER} + \lambda_2 \text{POP(\textit{o})} + \lambda_3 \text{PIVO}\\ \ \ if\ \ \ \neg(\textit{s; p; o}) \ is \ \mathit{satisfied}\\ \\
 \lambda_1 \text{PEER} + \lambda_4 \text{FRQ(\textit{p})} + \lambda_3 \text{PIVO}\\ \ \ if\ \ \ \neg \exists x (\textit{s; p; x}) \ is \ \mathit{satisfied}\\
\end{cases}
\end{equation*}}
\end{defn}


Hereby $\lambda_1$, $\lambda_2$, $\lambda_3$, and $\lambda_4$ are parameters to be tuned on data withheld from training.
\section{Order-oriented Peer-based Inference}
\label{sec:temporal}


In the previous section,  we assume a binary peer relation as the basis of peer group computation. In other words,  for each entity, any other entity is either a peer, or is not. Yet in expressive KBs, relatedness is typically graded and multifaceted, thus reducing this to a binary notion risks losing valuable information. We therefore investigate, in this section, how negative statements can be computed while using ordered peer set.

Orders on peers arise naturally when using real-valued similarity functions, such as Jaccard-similarity, or cosine distance of embedding vectors. An order also naturally arises when one uses temporal or spatial features for peering. Here are some examples:
\begin{enumerate}
    \item \emph{Spatial:} Considering the class \emph{national capital}, the peers closest to \textit{London} are \textit{Brussels} (199 miles), \textit{Paris} (213 miles), \textit{Amsterdam} (223 miles), etc.
    \item \emph{Temporal:} The same holds for temporal orders on attributes, e.g., via his role as president, the entities most related to \textit{Biden} are \textit{Trump} (predecessor), \textit{Obama} (pre-predecessor), \textit{Bush} (pre-pre-predecessor), etc.
\end{enumerate}

\noindent
\textbf{Formalization.\ }
%Let $\textit{sim}(e_1,e_2) \rightarrow \mathcal{R}$ be a real-valued similarity function over a pair of entities $e_1,e_2$. 
Given a target entity $e_0$, a similarity function $\textit{sim}(e_a,e_b) \rightarrow \mathcal{R}$, and a set of candidate peers $E=\{e_1,...,e_n\}$, we can sort $E$ by $\textit{sim}$ to derive an ordered list of sets $L=[S_1, ..., S_n]$, where each $S_i$ is a subset of $E$ that consists of highly related entities to $e_0$.
%and for any two members of $S$, $S_i$ and $S_j$ ($i\leq i < j \leq n$), we have 
%$S_i \cap S_j=\emptyset$.
%; that is,  the list is ordered by decreasing similarity
%of peers by assigning all $e\in E$ into sets ordered by decreasing similarity.}
%\GW{this is a little awkward and could be avoided/simplified; why don't you just say that the order induces a list of entities L = [$e_{i_1}, e_{i_2}$ ...].
%then you can say that because of ties this could be a partial order and could then be grouped into a list of lists S ...; but I doubt that you really need this list of lists S; I tend to think that all of the subsequent materials can also be cast into operating over L (with ties randomly broken)!!!}\ha{**}\\
% SR: Simplified it, but note that we do not want to break ties at random, as the boundaries carry important information (winners of the last 3 years), which are lost if we allow arbitrary sizes (last 2, 3, 4, 5, 6, 7 winners)
\begin{example}
Let us consider temporal recency of having won the \textit{Oscars for Best Actor/Actress} as similarity function w.r.t. the target entity %... (winner of the Physics Nobel Prize in ???). Then, the ordered list of peers would be: \term{[\{Wilhelm Röntgen\}, ..., \{Emilio Segrè, Owen Chamberlain\}, \{Don\-ald Glaser\}, \{Robert Hofstadter, Rudolf Mössbauer\}, \{L\-ev Landau\}, \{Maria Mayer, J. Hans Jensen, Eugene Wign\-er\}..., \{James Peebles, Michel Mayor, Di\-dier Queloz\}].}
\emph{Olivia Colman}.   The ordered list of closest peer sets $S$ is 
\term{[\{Frances McDormand, Gary Oldman\}, \{Emma Stone, Casey Affleck\}, \{Brie Larson, L\-eonardo DiCaprio\}, \{Julianne Moore, Eddie Redmayne\}.., \{Janet Gaynor, Emil Jannings\}].}
\end{example}


Given an index of interest $m$ ($m \leq n)$, we have a prefix list $S_{[1,m]}$   of such an ordered peer set list $L$. For any negative statement candidate $\textit{stmt}$, we can     compute two ranking features:

\begin{enumerate}
    \item \emph{Prefix-volume (VOL)}: The prefix volume denotes the size of the prefix in terms of peer entities considered, i.e., $\textit{VOL}=|S_1 \cup ... \cup S_m|$. Note that the volume should not be mixed with the length $m$ of the prefix, which does not allow easy comparison, as sets may contain very different numbers of members.
    \item \emph{Peer frequency (PEER)}: As in Section~\ref{sec:inference}, \textit{PEER} denotes the fraction of entities in $S_1 \cup ... \cup S_m$ for which \textit{stmt} holds, i.e., $\textit{FRQ}$ / $\textit{VOL}$, where $\textit{FRQ}$ is the number of entities sharing the statement.
\end{enumerate}

Note that these two ranking features change values with prefix length. In addition, we can also consider static features like \textit{POP} and \textit{PIVO}, as introduced before.\\



Consider the entity $e$=\textit{Olivia Colman} from our example, with prefix length 3. For the statement \term{(citizen of; U.S.)}, $\textit{FRQ}$ is 5 and $\textit{VOL}$ is 6, i.e., unlike \emph{Olivia Colman}, 5 out of the 6 winners of the previous 3 years are U.S. citizens. Now considering prefix length 2, for the statement \term{(occupation;  director)}, $\textit{FRQ}$ is 1 and $\textit{VOL}$ is 4, i.e., unlike \emph{Olivia Colman}, 1 out of the 4 winners of the previous 2 years are directors.

We can now proceed to the actual problem of this section.

\medskip

\noindent
\textbf{Research Problem 2.\ }
Given an entity $e$ and an ordered set of peers, compile a ranked list of useful %explanation-augmented 
negative statements.\\

% explanations need to be introduced before we can use them in the definition!!

\noindent
\textbf{Ranking.\ }
What makes a negative statement from an ordered peer set \textit{informative}? It is easy to see that a statement is preferred over another, if it has both a higher peer frequency (\textit{PEER}) and prefix volume (\textit{VOL}). For example, the statement $\neg$\term{(citizen of; U.S.)} above is preferable over $\neg$\term{(occupation; director)}, due to it being both reported on a larger set of peers, and with higher relative frequency.  Yet statements can be incomparable along these two metrics, and this problem even arises when comparing a statement with itself over different prefixes: Is it more helpful if 3 out of the previous 4 winners are \textit{U.S.} citizens, or 7 out of the previous 10?

To resolve such situations, we propose to map the two features into a single one as follows:
\begin{equation}
\label{eqn:context}
score(\textit{stmt},L,m) = \lambda \cdot \textit{PEER} + (1-\lambda) \cdot log(\textit{FRQ})
\end{equation}
%\end{defn}}

where $\lambda$ is again a parameter allowing to trade off the effects of the two variables. Note that we propose a logarithmic contribution of \textit{FRQ} - this is based on the rationale that larger number of peers is preferable. For example, for the same \textit{PEER} value 0.5, we can have a statement with 5 peers out of 10 and 1 peer out of 2.

%\newtext{The first part of the equation, $\frac{k}{l}$, is fraction of the number of entities $k$ that have $s$ in their list of statements and the number of read entities (in other words, the length of the prefix) $l$. The second part, $log(k)$ is added to give ensure higher scores for more frequent statements. For instance, with two statements $s_{1}$ and $s_{2}$, with $s_{1}$ appearing for 1 of the last 2 peers, and $s_{2}$ appearing for 10 of the last 20 peer. They will both receive the same $\frac{k}{l}$ = 0.5. However $s_{2}$ will get a higher overall score because it is more frequent.}
Given the above example, the score for \emph{Olivia Colman}'s negative statement $\neg$\term{(citizen of; U.S.)} at prefix length 3 and $\alpha=0.5$ is $0.76$, with verbalization as ``unlike 5 of the previous 6 winners''. The same statement with prefix length 2 will receive a score of $0.61$, with verbalization as ``unlike 3 of the previous 4 winners''. As for $\neg$\term{(occupation; director)} at prefix length 3 and $\alpha=0.5$ is $0.08$, with verbalization as ``unlike 1 of the previous 6 winners''.  The same statement with prefix length 2 will receive a score of $0.13$, with verbalization as ``unlike 1 of the previous 4 winners''. This example is illustrated in Figure~\ref{fig:diagram}.\\

\noindent
\textbf{Computation.\ }
Having defined how statements over ordered peer sets can be ranked, we now present an efficient algorithm, Algorithm~\ref{alg:contextualizations}, to compute the optimal prefix length per statement candidate, based on a single pass over the prefix. Given the entity $e$=\textit{Olivia Colman}, ordered sets of her peers are collected in line 2.
\begin{equation*}
\begin{split}
L = [\text{winners of Oscar, winners of BAFTA,}\\ \text{..., recipients of CBE}].
\end{split}
\end{equation*}
For readability, we proceed with one ordered peer group, namely the winners of Oscar for Best Actor/Actress. The group contains ordered winners prior to $e$.
\begin{equation*}
\begin{split}
L_\text{winners of Oscar} = [\{\text{Frances McDormand, Gary Oldman}\},\\ \{\text{Emma Stone, Casey Affleck}\},\\ \{\text{Brie Larson, Leonardo DiCaprio}\},\\ \{\text{Julianne Moore, Eddie Redmayne}\}\\ \text{..,}\\ \{\text{Janet Gaynor, Emil Jannings}\}].
\end{split}
\end{equation*}
Similar to the previous algorithm, all statements of the peers are then retrieved from the KB (line 11 and 12). For every candidate statement $st$, the score(s) of the statement is computed with different prefix lengths (loop at line 27), starting with $pos$ (position of $e$ in the ordered set) and stopping at the start position 1. The maximum score is then returned with its corresponding values of \textit{FRQ} and \textit{VOL}, i.e., $max\_\mathit{frq}$ and $max\_\mathit{vol}$ (line 37). The returned candidate statement with its highest score (within one ordered group of peers $L_i$) is compared across many ordered groups of peers (i.e., other groups in $L$), to be either replaced or disregarded from the final list of negations $N$.

%\newtext{Our second statistical inference method introduces two key differences. First, it proposes the term \textit{``contextualization''} of negative statements. We implicitly used contextualizations already in the previous method, for \textit{Einstein not being Communist}, the contextualization was for instance ``40 out of 80 similar entities are Communists''. In the present method, where we group using temporal signals, this gets much more \emph{concrete}. Now a precondition is required so that we can infer useful negations about $e$, namely that $e$ should be a member of at least one \textit{time-based peer group}. For example if $e$ is \term{Barack Obama}, this condition is fulfilled because he is a member of several time-based peer groups: presidents of the United States (44th), member of the State Senate of Illinois (between 1997 and 2004), and United States senator (between 2005 and 2008).}

%\newtext{Second, unlike in the similarity-based method where the peer group size is fixed and where peers are not ordered, in this method, different peers have different importance depending on their position in the time-based peer group. We consider more dynamic peer groups by introducing an additional mode for peering, shown in Algorithm~\ref{alg:contextualizations} and discussed later in more details. 


%The idea behind it is that a negative statement might be more useful in specific time frames, where a portion of the peer group is considered. 

%For example, given the time-based peer group of presidents of the United States, the negation stating that the 22nd president, \textit{Grover Cleveland}, was \textit{not} Republican might not be as useful considering the full group of presidents, as when pointing out that he was the first non-Republican president after 4 consecutive Republican presidents. \ha{To do: I will change examples to fit one theme: scientists and Nobel winners but let's talk big picture first and whether we need to change the way we're presenting this section.}}

%\ha{We need a clearer definition of the time-based group, we can specify what do they have in common (Case 1: a 'category' or a group title which is mainly a PO they share, like award; Nobel in Physics and with this ePO a specific time signal .. could be a from/to or a point in time; Case 2: no specific title but a property indicating time order 'follows' and for this I use the type of the entities as the title of the group; for example 'songs')}
%\newtext{We introduce the \emph{temporal statistical inference} method formally.}

%\textbf{Formalization.\ }
%\newtext{\begin{defn}[Time-based peer group]
%A time-based peer group is an ordered list $[S_1, \ldots, S_n]$ of sets $S_i$ of entities. Given an entity $e$ and a time-based peer group $t$ in which $e$ occurs first in $S_i$, the prefix for $e$ is the list $[S_1, \ldots, S_{i-1}]$. \footnote{For sake of simplicity, in the rare case of later occurrences of an entity that appears multiple times, with a gap, in a time series, this definition considers only the prefix before the first occurrence.}

%\end{defn}}



%\newtext{\begin{defn}[Temporal contextualizations] \mbox{ }\\
%Given negative statement $s$ for an entity $e$, a temporal contextualization is a triple ($t$,$k$,$l$), where $t$ is a time-based peer group which $e$ is a member of, and $k$ and $l$ are integers. A temporal contextualization ($t$,$k$,$l$) for $s$ is satisfied, if it is true that among the $l$ entities before $e$ in $t$, for at least $k$, s is true. Additionally, $l$ must exactly match the sum of the sizes of the last sets in the prefix of $e$.
%\end{defn}}
%\jeff{We might need to think again about the use of definitions; some reviewers tend to think that definitions are necessary only when there are theorems using them; others are a bit relax and also accept algorithms that are using them too} \ha{Same comment by Gerhard}
%\newtext{In our example, for \term{Maria Mayer}, $l$ must be either 1, 3, 4, 6, etc, as there is no order inside sets, so a statement with $l=2$ - ``among the past 3 winners'' is not meaningful, as there is 1 winner in the previous year, and then 3 previous winners considering the past two years, and then 3 winners considering the past three years.}\sr{The previous example is a negative one (i.e., what is not permitted) - better to start with an example of something permitted} \ha{I'm not sure I understand. I'm trying to show what l means and why do we have l to begin with.}
%\newtext{\begin{defn}[\small Maximal temporal contextualization] \mbox{ }
%A contextualization is maximal, if it is not dominated by any other satisfied contextualization. 
%\end{defn}}
%\newtext{\begin{defn}[Scoring contextualizations] \mbox{ }

%\textbf{Ranking time-based negative statements.\ }


%\newtext{\begin{problem}
%Given an entity $e$ that participates in a time-based peer group $t$, compile a ranked list of useful grounded negative statements, with time-based explanations.
%\end{problem}}
%\ha{This is different from problem 1 which is specific to baseline method (then we might want to move problem 1 to Section 4. In problem 2, an entity has to be a member of at least one time-based peer group and we're only considering grounded negative statements for higher correctness.}
%\sr{You are right, either we have a problem statement in either section or in none, I will rethink this.}




\begin{figure}
 \caption{Retrieving useful negative statements about \textit{Olivia Colman}, using an ordered peer group.}
\includegraphics[width=\linewidth]{figures/diagram}
\label{fig:diagram}
\end{figure}


\begin{algorithm*}[t!]
    \SetKwInOut{Input}{Input}
    \SetKwInOut{Output}{Output}
    \Input{\small knowledge base $\mathit{KB}$, entity $e$, ordered peer collection function \textit{ordered\_peers}, \small number of results $k$, \small hypeparameter of scoring function $\alpha$}
    \Output{ \small top-$k$ negative statement candidates for $e$}
        
        \textit{\textbf{L}}$[]$= \textit{ordered\_peers}$(e)$ \Comment{List of ordered peer group(s); Group $L_i$ at position $i$ is one ordered group (list).}\\
         \textit{\textbf{N}}$[]$= $\emptyset$ \Comment{\small Ranked list of negative statements about $e$.}\\
    \For{$L_i$ $\in$ \textbf{L}}{
        $candidates$ = [] \\
        $pos$=position($L_i$, $e$) \Comment{\small Position of $e$ in the ordered set.}\\
        \For{$pe$ $\in$ $L_i$}{ 
            \If{$pe$ == $p$}{continue}
            $candidates$+=$collectP(pe)$\\
            $candidates$+=$collectPO(pe)$
        }
        $ucandidates$ = $unique(candidates)$\\
        \For{$st$ $\in$ $ucandidates$}{
            $sc$ = $scoring(st, L_i, e, pos, \alpha)$ \Comment{\small Dynamic scoring of every statement st with different prefix lengths.}\\
            \If{$getnegation(N, st).score$ $<$ $sc$}{ $setscore(N, st, sc)$}
        }
     }
        $N$-=$\mathit{inKB}(e,N)$\\
        return $max(N,k)$
    \texttt{\\}
    \texttt{\\}
    \SetKwFunction{FMain}{scoring}
    \SetKwProg{Fn}{Function}{:}{}
    \Fn{\FMain{$st$, $S$, $e$, $pos$, $\alpha$}}{
        max\_sc = - $\inf$; max\_frq = - $\inf$; max\_vol = - $\inf$; \Comment{Initializing the maximum score, frequency, and volume for statement $st$.}\\
        frq = 0; vol=0; \Comment{Initializing the frequency and volume of statement $st$.}\\
         \For{$j$ = $pos$; $j$ $>=$ 1; $j{-}{-}$}
             {
                vol += countentities($S[j]$) \Comment{\small Computing number of entities at position $j$.}\\
                frq += countif($st$, $candidates$, $S[j]$) \Comment{\small Computing number of entities at position $j$ that share $st$.}\\
                sc = $\alpha * \frac{frq}{vol} + (1-\alpha)* log(frq)$ \Comment{Computing the score of $st$ at position $j$.}\\
                 \If{sc $>$ max\_sc}{
                 max\_sc = sc;\\
                 max\_frq = frq;\\
                 max\_vol = vol\; 
                }
         }}
         \textbf{return} max\_sc, max\_frq, max\_vol\\
\caption{Order-oriented peer-based candidate retrieval algorithm.}
\label{alg:contextualizations}
\end{algorithm*}
\section{Conditional Negative Statements}
\label{sec:restricted}

In our negation inference methods, we generate two classes of negative statements, grounded negative statements, and universally negative statements. These two classes represent extreme cases: each grounded statement negates just a single assertion, while each universally negative statement negates all possible assertions for a property. Consequently, grounded statements may make it difficult to be concise, while universally negative statements do not apply whenever at least one positive statement exists for a property.
A compromise between these extremes is to restrict the scope of universal negation. For example, it is cumbersome to list all major universities that \textit{Einstein} did not study at, and it is not true that he did not study at any university. However, salient statements are that he \textit{did not study at any U.S. university}, or that he \textit{did not study at any private university}.
We call these statements \emph{conditional negative statements}, as they represent a conditional case of universal negation. In principle, the conditions used to constrain the object could take the form of arbitrary logical formulas. For proof of concept, we focus here on conditions that take the form of a single triple pattern.

\begin{defn}
A conditional negative statement takes the form $\neg \exists o$: (s; p; $o$), (o; p'; o'). It is satisfied if there exists no $o$ such that (s; p; $o$) and  (o; p'; o') are in $K^i$.%, where C is a condition of form ($o$; $p_{1}^{'}$; $o_{1}^{'}$), ($o_{1}^{'}$; $p_{2}^{'}$; $o_{2}^{'}$), .., ($o_{n-1}^{'}$; $p_{n}^{'}$; $o_{n}^{'}$).
\end{defn}
%\sr{Needs alignment with Definition 1 Description Logic style}
%\GW{the notation is a wild hybrid, looks like logics, but misses quantifiers and instead uses wildcards \_;
%change to standard logics (not OWL):
%e.g. not exists x: educated(Einstein,x) and locatedIn(x,USA),
%along these lines}\ha{**but I can also take care of this based on what we will have for the first two types}\\

%$p_{1}^{'}$, $p_{2}^{'}$, .., $p_{n}^{'}$ are aspects of the statement. With n=1, the condition $C$ becomes simply a single triple pattern.

In the following, we call the property $p'$ the \textit{aspect} of the conditional negative statement.

\begin{example}
Consider the statement that Einstein did not study at any \textit{U.S.} university. It could be written as $\neg\exists o:$ \term{(Einstein; education; o)}, \term{(o; located in; U.S.)}. It is true, as \textit{Einstein} only studied at \textit{ETH Zurich, Luitpold-Gymnasium, Alte Kantonsschule Aarau}, and \textit{University of Zurich}, located in \textit{Switzerland} and \textit{Germany}. Another possible conditional negative statement is $\neg\exists o:$ \term{(Einstein; education; o)}, \term{(o; type; private University)}, as none of these schools are private.  %Now consider the statement that \textit{Einstein} did not study at any \textit{North American} university. The condition $C$ could be written by joining two triple patterns (i.e., $n=2$), as $\neg\exists o:$ \term{(Einstein; educated at; o)}, \term{(o; located in; x)}, \term{(x; continent; North America)}.
\end{example}

As before, the challenge is that there is a near-infinite set of true conditional negative statements, so a way to identify interesting ones is needed. For example, \textit{Einstein} also did not study at any \textit{Jamaican} university, nor did he study at any university that \textit{Richard Feynman} studied at, etc. 
One way to proceed would be to traverse the space of possible conditional negative statements, and score them with another set of metrics. Yet compared to universally negative statements, the search space is considerably larger, as for every property, there is a large set of possible conditions via novel properties and constants (e.g., \textit{``that was located in Armenia/Brazil/China/Denmark/...''}, \textit{``that was attended by Abraham/Beethoven/Cleopatra/...''}). 
So instead, for efficiency, we propose to make use of previously generated grounded negative statements: In a nutshell, the idea is first to generate grounded negative statements, then in a second step, to \emph{lift} subsets of these into more expressive conditional negative statements. 
A crucial step is to define this lifting operation, and what the search space for this operation is.

With the \textit{Einstein} example, shown in Table~\ref{tab:einsteinlifting}, we could start from three relevant grounded negative statements that \textit{Einstein} did not study at \textit{MIT, Stanford}, and \textit{Harvard}. One option is to lift them based on aspects they all share: their locations, their types, or their memberships. The values for these aspects are then automatically retrieved: they are all located in the \textit{U.S.}, they are all private universities, they are all members of the \textit{Digital Library Federation}, etc., however, not all of these may be interesting. So instead we propose to \emph{pre-define} possible aspects for lifting, either using manual definition, or using methods for facet discovery, e.g., for faceted interfaces~\cite{oren2006extending}. For manual definition, we assume the condition to be in the form of a single triple pattern. A few samples are shown in Table~\ref{tab:aspects}. For \term{educated at}, it would result in statements like ``e was not educated in the \textit{U.K.}'' or ``e was not educated at a public university''; for \term{award received}, like ``e did not win any category of \textit{Nobel Prize}''; and for \term{position held}, like ``e did not hold any position in the \textit{House of Representatives}''.\\


\noindent
\textbf{Research Problem 3.\ }
Given a set of grounded negative statements about an entity $e$, compile a ranked list of useful conditional negative statements.\\

We propose an approach with Algorithm~\ref{alg:restricted}. Consider $e$=\emph{Einstein}, and the set of possible aspects $\mathit{ASP}$ for lifting containing only two aspects about \term{educated at}, for readability. 
\begin{equation*}
ASP = [(\text{educated at: located in, instance of})].
\end{equation*}
The three grounded negative statements about \emph{Einstein} with \term{educated at} property are:
\begin{equation*}
\mathit{NEG}=[\neg(\text{educated at: MIT, Stanford, Harvard})].
\end{equation*}
The loop at line 2 considers every property ($neg.p$) in $\mathit{NEG}$ (e.g., \term{educated at}), and collect its aspects at line 3. For this example, the list of aspects $asp$ for this predicate consists of the location and the type of the educational institution.
\begin{equation*}
asp=[\text{located in, instance of}].
\end{equation*}
At line 4, the loop visits every aspect $a$ in $asp$ and look for aspect values (i.e., the locations and types of Einstein's schools). $neg.o$ are the objects that share the same predicate in the grounded negative statements list.
\begin{equation*}
neg.o=[\text{MIT, Stanford, Harvard}].
\end{equation*}
For every object $o$, aspect values are collected and their relative frequencies are stored. For readability, line 6 is only a high level version of this step. As mentioned before, the aspects are manually pre-defined and their values are automatically retrieved.
\begin{equation*}
 \begin{aligned}
getaspvalues(\text{Wikidata, located in, MIT}) &= [\text{U.S}].\\ 
getaspvalues(\text{Wikidata, located in, Stanford}) &= [U.S].\\
getaspvalues(\text{Wikidata, located in, Harvard}) &= [U.S].\\
getaspvalues(\text{Wikidata, instance of, MIT}) &=\\ [\text{institute of technology, private university}].\\
getaspvalues(\text{Wikidata, instance of, Stanford}) &=\\ [\text{research university, private university}].\\
getaspvalues(\text{Wikidata, instance of, Harvard}) &=\\ [\text{research university, private university}].\\
 \end{aligned}
\end{equation*}
Hence the aspect value for \term{educated at}, namely \term{(located in; U.S.)} receives a score of 3, and is added to the conditional negation list $\mathit{cond\_NEG}$. After retrieving and scoring all the aspect values, the top-2 (with $k$ =2) conditional negative statements are returned. In this example, the final results are $\mathit{cond\_NEG}$ = \term{[($\neg\exists o$(Einstein; educated at; $o$) ($o$; located in; U.S.), 3)}, (\term{$\neg\exists o$(Einstein; educa\-ted at; $o$) ($o$, instance of; private university), 3)}].

\begin{table*}
  \caption{Negative statements about \textit{Einstein}, before and after lifting.}
  \label{tab:einsteinlifting}
  \centering
  \scalebox{0.9}{
   \begin{tabular}{l|l}
    \toprule
    \multicolumn{1}{c}{\textbf{Grounded negative statements}} & \multicolumn{1}{c}{\textbf{Conditional negative statements}}\\
    \midrule
$\neg$(educated at; MIT) & $\neg\exists o$(educated at; $o$) ($o$; located in; U.S.)\\
$\neg$(educated at; Stanford) & $\neg\exists o$(educated at; $o$) ($o$, instance of; private university)\\
$\neg$(educated at; Harvard) & \\
    \bottomrule
  \end{tabular}
  }
  \end{table*}
  
\begin{algorithm*}[t!]
    \SetKwInOut{Input}{Input}
    \SetKwInOut{Output}{Output}
    \Input{\small knowledge base $\mathit{KB}$, entity $e$, aspects $\mathit{ASP}$ = [($x_{1}$: $y_{1}$, $y_{2}$, ..), ..., ($x_{n}$: $y_{1}$, $y_{2}$, ..)], \small grounded negative statements about $e$ $\mathit{NEG}$ = [$\neg$($p_{1}$: $o_{1}$, $o_{2}$, ..), ..., $\neg$($p_{m}$: $o_{1}$, $o_{2}$, ..)], \small number of results $k$}
    \Output{\small $k$-most frequent conditional negative statements for $e$}
    \textbf{$\mathit{cond\_NEG}$}= $\emptyset$ \Comment{\small Ranked list of conditional negations about $e$.}\\
           \For{$neg.p$ $\in$ $\mathit{NEG}$}{ 
           $\mathit{asp}$ = $getspects(neg.p, ASP)$ \Comment{Retrieving aspects of predicate $neg.p$.}\\
           \For{$a$ $\in$ $asp$}{
           \For{$o$ $\in$ $neg.o$}{ 
            $\mathit{cond\_NEG}$ += getaspvalues($\mathit{KB}$, $a$, $o$) \Comment{Collecting aspect values about $o$.}
           }
           }
        }
        $\mathit{cond\_NEG}$-=$inKB(e,\mathit{cond\_NEG})$\\
        return $max(\mathit{cond\_NEG},k)$
\caption{Lifting grounded negative statements algorithm.}
\label{alg:restricted}
\end{algorithm*}

\begin{table}
  \caption{A few samples of property aspects.}
  \label{tab:aspects}
  \centering
  \scalebox{0.8}{
   \begin{tabular}{l|l}
    \toprule
    \multicolumn{1}{c}{\textbf{Property}} & \multicolumn{1}{c}{\textbf{Aspect(s)}}\\
    \midrule
educated at & located in; instance of;\\
award received & subclass of;\\
position held & part of;\\
    \bottomrule
  \end{tabular}
  }
  \end{table}
\section{Experimental Evaluation}
\label{sec:experiments}

\subsection{Peer-based Inference}
\label{subsec:similarityexp}
\noindent
\textbf{Setup.\ }We instantiated the peer-based inference method with 30 peers, popularity based on Wikipedia page views, and peer groups based on entity occupations.
The choice of this simple peering function was inspired by Recoin~\cite{RECOIN}.
In order to further ensure relevant peering, we also only considered entities as candidates for peers, if their Wikipedia viewcount was at least a quarter of that of the subject entity.
We randomly sampled 100 popular Wikidata people. For each of them, we collected 20 negative statement candidates: 10 with the highest \textit{PEER} score, 10 being chosen at random from the rest of retrieved candidates. We then used crowdsourcing\footnote{\url{https://www.mturk.com}} to annotate each of these 2000 statements on whether it was interesting enough to be added to a biographic summary text (Yes/Maybe/No). Each task was given to 3 annotators. Interpreting the answers as numeric scores (1/0.5/0), we found a standard deviation of 0.29, and full agreement of the 3 annotators on 25\% of the questions. Our final labels are the numeric averages among the 3 annotations.

\noindent
\textbf{Parameter Tuning.\ } To learn optimal parameters for the ensemble ranking function (Definition~\ref{def:ensemble}), we trained a linear regression model using 5-fold cross validation on the 2k labels for usefulness. Four example rows are shown in Table~\ref{tab:training}. Note that the ranking metrics were normalized using a ranked transformation to obtain a uniform distribution for every feature.

The average obtained optimal parameter values were -0.03 for \textit{PEER}, 0.09 for \textit{FRQ(p)}, -0.04 for \textit{POP(o)}, and 0.13 for \textit{PIVO},  and a constant value of 0.3., with a 71\% out-of-sample precision.

\begin{table*}
  \caption{Data samples for illustrating parameter tuning.}
  \label{tab:training}
 \resizebox{\textwidth}{!}{\begin{tabular}{llllll}
    \toprule
    \multicolumn{1}{c}{\bf{Statement}} & \multicolumn{1}{c}{\bf{PEER}} & \multicolumn{1}{c}{\bf{FRQ(p)}}& \multicolumn{1}{c}{\bf{POP(o)}}& \multicolumn{1}{c}{\bf{PIVO}}&\multicolumn{1}{c}{\bf{Label}}\\
    \midrule
$\neg$(Bruce Springsteen; award; Grammy Lifetime Achievement Award) & 0.8 & 0.8 & 0.55 & 0.25 & 0.83\\
$\neg$(Gordon Ramsay; lifestyle; mysticism) & 0.3 & 0.8 & 0.8 & 0.65 & 0.33\\
$\neg \exists x$(Albert Einstein; doctoral student; x) & 0.85 & 0.9 & 0.15 & 0.4 & 0.66\\
$\neg \exists x$(Celine Dion; educated at; x) & 0.95 & 0.95 & 0.25 & 0.95 & 0.5\\
    \bottomrule
  \end{tabular}}
  \end{table*}
\noindent
\textbf{Ranking Metric.\ }To compute the ranking quality of our method against a number of baselines, we used the Discounted Cumulative Gain (DCG)~\cite{NDCG}, which is a measure that takes into consideration the rank of relevant statements and can incorporate different relevance levels. DCG is defined as follows:
\begin{equation*}
DCG(i)=\begin{cases}
& G(1)\ \ \ if\ $i=1$\\
 & DCG(i-1)+\frac{G(i)}{log(i)} \ \ \ \text{\textit{otherwise}}
\end{cases}
\end{equation*}
where i is the rank of the result within the result set, and $G(i)$ is the relevance level of the result. We set $G(i)$ to a value between 1 and 3, depending on the annotator's assessment. We then averaged, for each result (statement), the ratings given by all annotators and used it as the relevance level for the result. Dividing the obtained DCG by the DCG of the ideal ranking, we obtained the normalized DCG (nDCG), which accounts for the variance in performance among queries (entities).

\noindent
\textbf{Baselines.\ }We used three \textit{baselines}: As a naive baseline, we randomly ordered the 20 statements per entity. This baseline gives a lower bound on what any ranking model should exceed. We also used two competitive embedding-based baselines, TransE~\cite{transE} and HolE~\cite{holE}. For these two, we used pretrained models, from~\cite{ho2018rule}, on Wikidata (300k statements) containing prominent entities of different types, which we enriched with all the statements about the sampled entities. We plugged their prediction score for each candidate grounded negative statement.\footnote{Note that both models are not able to score statements about universal absence, a trait shared with the object popularity heuristic in our ensemble.}

\begin{table*}
  \caption{Ranking metrics evaluation results for peer-based inference.}
  \label{tab:rankingNDCG}
  \centering
  \resizebox{0.8\textwidth}{!}{\begin{tabular}{llllll}
    \toprule
    \multicolumn{1}{l}{\bf{Ranking Model}} &\multicolumn{1}{c}{\bf{Coverage(\%)}} & \multicolumn{1}{c}{$\boldsymbol{nDCG_3}$} & \multicolumn{1}{c}{$\boldsymbol{nDCG_5}$}& \multicolumn{1}{c}{$\boldsymbol{nDCG_{10}}$}& \multicolumn{1}{c}{$\boldsymbol{nDCG_{20}}$}\\
    \midrule
Random & 100 & 0.37 & 0.41 & 0.50 & 0.73\\
TransE~\cite{transE} & 31 & 0.43 & 0.47 & 0.55 & 0.76\\
HolE~\cite{holE} & 12 &	0.44 & 0.48 & 0.57 & 0.76\\
\midrule
Property Frequency & 11 & \bf{0.61} & \bf{0.61} & \bf{0.66} & \bf{0.82}\\
Object Popularity & 89 & 0.39 & 0.43 & 0.52 & 0.74\\
Pivoting Score & 78 & 0.41 & 0.45 & 0.54 & 0.75\\
Peer Frequency & 100 &	\bf{0.54} &	\bf{0.57} &	\bf{0.63} &	\bf{0.80}\\
\midrule
Ensemble & 100 &	\bf{0.60} &	\bf{0.61} &	\bf{0.67} & \bf{0.82}\\
    \bottomrule
  \end{tabular}}
  \end{table*}
  \noindent
  \textbf{Results.\ }Table \ref{tab:rankingNDCG} shows the average $nDCG$ over the 100 entities for top-k negative statements for k equals 3, 5, 10, and 20. As one can see, our ensemble outperforms the best baseline by 6 to 16\% in $nDCG$. The coverage column reflects the percentage of statements that this model was able to score. For example, for the \textit{Popularity of Object}, $POP(o)$ metric, a universally negative statement will not be scored. The same applies to TransE and HolE.
  
 Ranking with the \textit{Ensemble} and ranking using the \textit{Frequency of Property} outperforms all other ranking metrics and the three baselines, with an improvement over the random baseline of 20\% for k=3 and k=5. Examples of ranked top-3 negative statements for \textit{Albert Einstein} are shown in Table \ref{tab:rank_qualitative}. The random rank basically display any candidate negation if it holds for at least one peer. For instance, \textit{Omar Sharif} is \textit{Einstein}'s peer under the \textit{non-fiction writer} group. This makes the negation ``Tarek Sharif not a child of Einstein'' possible, hence, the necessity for a ranking step. Moreover, \textit{Omar Sharif} is also an actor, which brings other topics to the result set of \textit{Einstein}, such as not winning \textit{film awards}. This is where peer frequency makes a difference, i.e., most of \textit{Einstein}'s peers are \textit{not} actors. By relying on the property frequency for ranking, we can see that only universally absent statements get the highest scores. Even though it displays interesting negations (e.g., despite his status as famous researcher, \textit{Einstein} truly never formally supervised any PhD student), the top-k result set lacks grounded negative statements. Ensemble ranking, on the other hand, takes into consideration several features simultaneously, and covers both classes of negation. It returns interesting statements such as that \textit{Einstein} notably refused to work on the \textit{Manhattan} project, and was suspected of communist sympathies. 
 

\begin{table*}
  \caption{Top-3 results for \textit{Albert Einstein} using 3 ranking metrics.}
  \label{tab:rank_qualitative}
  \centering 
   \resizebox{0.8\textwidth}{!}{\begin{tabular}{l|l|l}
    \toprule
    \multicolumn{1}{l}{\textbf{Random rank}} &    \multicolumn{1}{l}{\textbf{Property frequency}} & \multicolumn{1}{l}{\textbf{Ensemble}}\\
\toprule
\multicolumn{1}{l}{$\neg \exists x$(instagram; x)} &    \multicolumn{1}{l}{$\neg \exists x$(doctoral student; x)} & \multicolumn{1}{l}{$\neg$(occup.; astrophysicist)}\\
 \multicolumn{1}{l}{$\neg$(child; Tarek Sharif)} & \multicolumn{1}{l}{$\neg \exists x$(candidacy in election; x)} & \multicolumn{1}{l}{$\neg$(party; Communist Party USA)}\\
  \multicolumn{1}{l}{$\neg$(award; BAFTA)} & \multicolumn{1}{l}{$\neg \exists x$(noble title; x)} & \multicolumn{1}{l}{$\neg \exists x$(doctoral student; x)}\\
    \bottomrule
  \end{tabular}}
  \label{tbl:may:einstein}
\end{table*}


\noindent
\textbf{Correctness Evaluation.\ } We used crowdsourcing to assess the correctness of results from the peer-based method. We collected 1k negative statements belonging to the three types, namely people, literature work, and organizations. Every statement was annotated 3 times as either correct, incorrect, or ambiguous. 63\% of the statements were found to be correct, 31\% were incorrect, and 6\% were ambiguous. Most incorrect statements are due to KB completion issues. Interpreting the scores numerically (0/0.5/1), annotations showed a standard deviation of 0.23. 

\noindent
\textbf{PCA (Partial Completeness Assumption) vs.\ CWA\ }
For a sample of 200 statements about people (10 each for 20 entities), half generated only relying on the CWA, half additionally filtered to satisfy the PCA (subject has at least one other object for that property~\cite{AMIEP}), we manually checked correctness.
We observed  84\% accuracy
for PCA-based statements, and 57\% for CWA-based statements. So the PCA yields significantly more correct negative statements, though losing the ability to predict universally negative statements.

\noindent
\textbf{Subject coverage.} Our peer-based inference method offers a very high subject coverage and is able to discover negative statements about almost any existing entity in a given KB, whereas for pre-trained embedding-based baselines, many subjects are out-of-vocabulary, or come with too little information to predict statements.






%################################################
%################################################
%################################################


\subsection{Inference with Ordered Peers}
\label{sub:temporalexperiments}

In the following, we used temporal order on specific roles, or on specific attribute values, to compute ordered peer sets. In particular, we used two common forms of temporal information in Wikidata to compute such peer groups: 
\begin{itemize}
    \item \textbf{Time-based Qualifiers (TQ)}: Temporal qualifiers are time signals associated with statements about entities. In Wikidata, some of those qualifiers are \textit{point in time} (P585), \textit{start time} (P580), and \textit{end time} (P582). A few samples are shown in Table~\ref{tab:qualifiers}.
    \item \textbf{Time-based Properties (TP)}: Temporal properties are properties like \textit{follows} (P155) and \textit{followed by} (P156) indicating a chain of entities, ordered from oldest to newest, or from newest to oldest. For instance, \term{[The Cossacks; followed by; War and Peace; followed by; Anna Karenina; ..]}\footnote{Novels of by Leo Tolstoy.}
\end{itemize}

We created TQ groups from aggregating information about people sharing the same statements. For example, \term{position held; President of the U.S.} is one TQ group, where members will have a \textit{start time} for this position, as well as an \textit{end time}. In case of absence of an \textit{end time}, this implies that the statement holds to this day (\term{Donald Trump}'s statement in Table~\ref{tab:qualifiers}). In other words, we aggregated entities sharing the same predicate-object pair, which will be treated as the peer group's title, and ranked them in ascending order of time qualifiers. For the \textit{point in time} qualifier, we simply ranked the dates from oldest to newest, and for the \textit{start/end date}, we ranked the end date from oldest to newest.
If the \textit{end date} is missing, the entity will be moved to the newest slot.

We collected a total of 19.6k TQ groups (13.6k using the \textit{start/end date} qualifier and 6k using the \textit{point in time} qualifier). Based on a manual analysis of a random sample of 100 groups of different sizes, we only considered time series with at least 10 entities\footnote{This variable can be easily adjusted depending on the preference of the developers and/or the purpose of the application.}.

We created TP groups by first collecting all entities reachable by one of the transitive properties, \textit{follows} (P155) and \textit{followed by} (P156). Considering each of the collected entities as a source entity, we computed the longest possible path of entities with only transitive properties. This path consists in an ordered set of peers. To avoid the problem of double-branching (one entity followed by two entities), we considered the two directions separately. Again, one path will be chosen at the end; the one with maximum length. 
The total number of TP groups is 19.7k groups. We limited the size of the groups to at least 10 and at most 150\footnote{We did not truncate the groups, we simply disregarded any group smaller or larger than the thresholds.}. 

\begin{table*}
  \caption{Samples of temporal information in Wikidata.}
  \label{tab:qualifiers}
  \centering
  \scalebox{0.9}{
   \begin{tabular}{l|l}
    \toprule
    \multicolumn{1}{c}{\textbf{Statement}} & \multicolumn{1}{c}{\textbf{Time-based qualifier(s)}}\\
    \midrule
(Barack Obama; position held; U.S. senator) & {\small \textit{start time}}: 3 January 2005; {\small \textit{end time}}: 16 November 2008\\
(Maya Angelou; award received; Presidential Medal of Freedom) & {\small \textit{point in time}}: 2010\\
(Donald Trump; spouse; Melania Trump) & {\small \textit{start time}}: 22 January 2005\\
    \bottomrule
  \end{tabular}
  }
  \end{table*}
  
\noindent
\textbf{Setup and Baseline.} We chose 100 entities, that belongs to at least one ordered set of peers, from Wikidata: 50 people and 50 literature works. We collected top-5 negative statements for each of those entities (for people, we consider TQ groups, and for literature works, TP groups). We made this choice because of the lack of entities of type person with transitive properties. In case an entity belongs to several groups, we merged all the results it is receiving from different groups, ranked them, and retrieved the top-5 statements. Similarly, as a baseline, using the peer-based inference method of Section~\ref{sec:inference}, instantiated with cosine similarity on Wikipedia embeddings~\cite{wikipedia2vec} as similarity function, we collected the top-5 negative statements for the same entities. We ended up with 1k statements, 500 inferred by each model.

\noindent
\textbf{Correctness Evaluation.} We randomly retrieved 400 negative statements from the 1k statements collected above, 200 from each model (100 about people, and 100 about literature works). We then assessed the correctness of each method using crowdsourcing. We showed each statement to 3 annotators, asking them to choose whether this statement is correct, incorrect, or ambiguous. Results are shown in Table~\ref{tab:simvstempcorrec}. Our order-oriented inference method clearly infers less incorrect statements by 9 percentage points for people, and 5 for literature works. It also produces more correct statements for people by 10 percentage points, and literature work by 3. The percentage of queries with full agreement in this task is 37\%. Also, annotations show a standard deviation of 0.17.
\begin{table}
\centering
  \caption{Correctness of order-oriented and peer-based methods.}
  \label{tab:simvstempcorrec}
  \begin{tabular}{llc}
    \toprule
   & \multicolumn{1}{l}{\bf{People}}  & \multicolumn{1}{c}{\bf{Literature Work}}  \\
    \midrule
    \multicolumn{3}{c}{Peer-based inference}\\
    \midrule
           & \multicolumn{1}{l}{\bf{\%}}  & \multicolumn{1}{c}{\bf{\%}}  \\
    \midrule
Correct & 81 & 88\\
Incorrect &  18 & 12\\
Ambiguous & 1 & 0\\
\midrule
\multicolumn{3}{c}{Order-oriented inference}\\
\midrule
       & \multicolumn{1}{l}{\bf{\%}}  & \multicolumn{1}{c}{\bf{\%}}  \\
    \midrule
Correct & \bf{91} & \bf{91}\\
Incorrect & 9 & 7\\
Ambiguous & 0 & 2 \\
    \bottomrule
  \end{tabular}
  
  \end{table}

\noindent
\textbf{Subject Coverage.} To assess the subject coverage of the order-oriented method, we randomly sampled 1k entities from each dataset, and tested whether it is a member of at least one ordered set, thus the ability to infer useful negative statements about it. For TQ groups, we randomly sampled 1k people, which results in a  coverage of 54\%. And for TP groups, we randomly sampled 1k literature works, and also received a coverage of 54\%. Although the order-oriented method produces better negative statements on both notions of correctness and usefulness (as we will see next), it does not outperform the baseline on subject coverage. However, using a different function to order peers might affect this drastically (e.g., using real-valued similarity functions like cosine distance of embeddings).

\noindent
\textbf{Usefulness.} To assess the quality of our inferred statements from the order-oriented inference method against the baseline (the peer-based inference method), we presented to the annotators two sets of top-5 negative statements about a given entity, and asked them to choose the more interesting set. The total number of opinions collected, given 100 entities, 3 annotations each, is 300. To avoid biases, we repeatedly switched the position of the sets. Results are shown in Table~\ref{tab:WDinterestingness}. Overall results show that our method is preferred by 10\% of the entities for both domains. The standard deviation of this task is 0.24 and the percentage of queries with full agreement is 18\%. We observe two advantages of the ordered set of peers over the previous method: i) it gives better interpretations of what a peer is, by automatically producing labels for peer groups (e.g., Presidents of the U.S., Winners of the Best Actor Academy Award); and ii) it maximizes the \textit{peerness} within a group. For instance, with Wikipedia embedding~\cite{wikipedia2vec}, closest peers to \textit{Donald Trump} are \textit{Hillary Clinton} and \textit{Donald Trump Jr.}. While the peerness with the input entity is obvious, there is not much similarity between the peers themselves, hence, very sparse candidate negations. However, with the order-oriented peering, \textit{Trump}'s peers include \textit{Barack Obama} and \textit{George W. Bush}, who are also peers of each other. 

\noindent
\textbf{Evaluation of Verbalizations.} One main contribution that our order-oriented inference method offers are \textit{verbalizations} produced with every inferred negative statement. In other words, it can, unlike the peer-based inference method, produce more concrete explanations of the usefulness of the inferred negations. For example, the inferred negative statement \term{$\neg$(Abraham Lincoln; cause of death; natural} \term{causes)} was inferred by both of our methods. However, each method offers a different verbalization. For the peer-based method, the verbalization is ``unlike 10 of 30 similar people'', and for the order-oriented method is ``unlike 12 of the previous 12 presidents of the U.S.''. To assess the quality of the verbalizations more formally, we conducted a crowdsourcing task with 100 useful negations that were inferred by both methods from our previous experiment. For every negative statement, the annotator was shown two different verbalizations on ``why is this negative statement noteworthy''. We asked the annotator to choose the better verbalization, she can choose Verbalization1, Verbalization2, or Either/Neither. Results show that verbalizations produced by our order-oriented inference method were chosen 76\% of the time, by the peer-based inference method 23\% of the time, and the either or neither option only 1\% of the time. The standard deviation is 0.23, and the percentage of queries with full agreement is 20\%. Table~\ref{tab:explanations} shows a number of examples, using different grouping functions for the peer-based method.

\begin{table*}
  \caption{Negative statements and their verbalizations using peer-based and order-oriented methods.}
  \label{tab:explanations}
 \resizebox{\textwidth}{!}{\begin{tabular}{llll}
    \toprule
    \multicolumn{1}{c}{\bf{Statement}} & \multicolumn{1}{c}{\bf{Order-oriented}} & \multicolumn{1}{c}{\bf{Peer-based}}& \multicolumn{1}{c}{\bf{Peering}}\\
      \multicolumn{1}{c}{\bf{}} & \multicolumn{1}{c}{\textit{Unlike..}} & \multicolumn{1}{c}{\textit{Unlike..}}& \multicolumn{1}{c}{}\\
    \midrule
   $\neg$(Emmanuel Macron; member;  National Assembly)	& 29 of 36 members of La République En Marche party & 70 of 100 similar people &	WP embed.~\cite{wikipedia2vec}\\
   $\neg$(Tim Berners-Lee; citizenship; U.S.) & 101 of previous 115 winners of the MacArthur Fellowship & 53 of 100 sim. comp. scientists &	Structured facets\\
 $\neg$(Michael Jordan; occupation; basketball coach) & 27 of prev. 49 winners of the NBA All-Defensive Team & 31 of 100 sim. people & WP embed.~\cite{wikipedia2vec}\\
 $\neg$(Theresa May; position; Opposition Leader) & 11 of prev. 14 Leaders of the Conservative Party & 10 of 100 sim. people &	WP embed.~\cite{wikipedia2vec}\\
 $\neg$(Cristiano Ronaldo; citizenship; Brazil) & 4 of prev. 7 winners of the Ballon d'Or & 20 of 100 sim. football players & Structured facets\\
    \bottomrule
  \end{tabular}}
  \end{table*}

\begin{table}
\centering
  \caption{Usefulness of order-oriented and peer-based methods.}
  \label{tab:WDinterestingness}
  \begin{tabular}{llc}
    \toprule
    & \multicolumn{1}{l}{\bf{People}}  & \multicolumn{1}{c}{\bf{Literature Work}}  \\
    \midrule
           &     \multicolumn{1}{l}{\bf{\%}}  & \multicolumn{1}{c}{\bf{\%}}\\
           \midrule
    Peer-based inference & 42 & 44\\
Order-oriented inference  & \bf{52} & \bf{54} \\
Both & 6 & 2\\
    \bottomrule
  \end{tabular}
  
  \end{table}







%#####################################
\subsection{Conditional Negative Statements Evaluation}
\label{subsec:restrictedexp}

We evaluated our lifting technique to retrieve useful conditional negative statements, based on three criteria: (i) compression, (ii) correctness, and (iii) usefulness. We collected the top-200 negative statements about 100 entities (people, organizations, and art work), and then applied lifting on them.

\noindent
\textbf{Compression.}   On average, 200 statements are reduced to 33, which means that lifting compresses the result set by a factor of 6.

\noindent
\textbf{Correctness.} We asked the crowd to assess the correctness of 100 conditional negative statements (3 annotations per statement), chosen randomly. To make it easier for annotators who are unfamiliar with RDF  triples\footnote{Especially because of the triple-pattern condition.}, we manually converted them into natural language statements, for example ``\textit{Bing Crosby did not play any keyboard instruments}''. Results show that 57\% were correct, 23\% incorrect, and 20\% were uncertain. The standard deviation of this task is 0.24 and the percentage of queries with full agreement is 18\%.

\noindent
\textbf{Usefulness.} For every entity, we showed 3 annotators 2 sets of top-3 negative statements: a grounded and universally negative statements set and a conditional negative statement set, and asked them to choose the one with more interesting information. Results are shown in Table~\ref{tab:conditionaluse}. The conditional statements were chosen 45 percentage points more than the grounded and universally negative statements. The standard deviation of this task is 0.22 and the percentage of queries with full agreement is 21\%. The significant out-performance of the conditional class over the other two classes is that it encapsulates them. Without losing the information from the original result set, lifting summarizes negations in meaningful manner, at the same time, allowing more diverse statements to be displayed in a top-k set. An example is shown in Table~\ref{tab:leolifting}, with entity $e=$\textit{Leonardo Dicaprio}, and its top-3 results. Even though he is one of the most accomplished actors in the world, unlike many of his peers, he never attempted directing any kind of creative work (films, plays, television shows, etc..).

\begin{table}
  \caption{Usefulness of conditional negative statements.}
  \centering
  \label{tab:conditionaluse}
  \begin{tabular}{l|l}
    \toprule
        \multicolumn{1}{c}{\textbf{Preferred}} & 
        \multicolumn{1}{c}{\textbf{(\%)}}\\
            \midrule
    \multicolumn{1}{l}{Conditional negative statements} & \multicolumn{1}{c}{\textbf{70}}\\
    \multicolumn{1}{l}{Grounded and universally negative statements} & \multicolumn{1}{c}{25}\\
    \multicolumn{1}{l}{Either or neither} & \multicolumn{1}{c}{5}\\
    \bottomrule
  \end{tabular}
\end{table}

\begin{table*}
  \caption{Top-3 negative statements about \textit{Leonardo Dicaprio}, before and after lifting.}
  \label{tab:leolifting}
  \centering
  \scalebox{0.9}{
  \begin{tabular}{l|l}
    \toprule
    \multicolumn{1}{c}{\textbf{Negative statements}} & \multicolumn{1}{c}{\textbf{Conditional negative statements}}\\
    \midrule
$\neg$(occupation; film director) & $\neg\exists o$ (occupation; $o$) ($o$; subclass of; director)\\
$\neg$(occupation; theater director) & $\neg \exists x$(spouse; x) \\
$\neg$(occupation; television director) & $\neg \exists x$(child; x) \\
    \bottomrule
  \end{tabular}}
  \end{table*}










%######################################

\section{Extrinsic Evaluation}
\label{sec:extrinsicevaluation}
We highlight the relevance of negative statements for:
\begin{itemize}
    \item Entity summarization on Wikidata.
    \item Decision support with hotel data from Booking.com.
    \item Question answering on various structured search engines.
\end{itemize}

\subsection{Entity Summarization}
In this experiment we analyze whether mixed positive-negative statement set can compete with standard positive-only statement sets in the task of entity summarization. In particular, we want to show that the addition of negative statements will \textit{increase the descriptive power} of structured summaries.

% 
We collected 100 Wikidata entities from 3 diverse types: 40 people, 30 organizations (including publishers, financial institutions, academic institutions, cultural centers, businesses, and more), and 30 literary works (including creative work like poems, songs, novels, religious texts, theses, book reviews, and more). On top of the negative statements that we infered, we collected relevant positive statements about those entities.\footnote{We defined a number of common/useful properties to each of  type, e.g., for people, ``position held''is a relevant property for positive statements.} We then computed for each entity \textit{e} a sample of 10 positive-only statements, and a mixed set of 7 positive and 3 \textit{correct}\footnote{We manually checked the correctness of these negative statements.} negative statements, produced by the peer-based method. We relied on peering using Wikipedia embeddings~\cite{wikipedia2vec}. Annotators were then asked to decide which set contains more new or unexpected information about \textit{e}. More particularly, for every entity, we asked workers to assess the sets (flipping the position of our set to avoid biases), leading to a total number of 100 tasks for 100 entities. We collected 3 opinions per task. Overall results show that mixed sets with negative information were preferred for 72\% of the entities, sets with positive-only statements were preferred for 17\% of the entities, and the option ``both or neither'' was chosen for 11\% of the entities. Table~\ref{tab:posneg} shows results per each considered type.  The standard deviation is 0.24, and the percentage of queries with full agreement is 22\%.
Table~\ref{tab:won} shows three diverse examples. The first one is \textit{Daily Mirror}. One particular noteworthy negative statement in this case is that the newspaper is not owned by the ``\textit{News U.K.}'' publisher which owns a number of other \textit{British} newspapers like \textit{The Times, The Sunday Times, and The Sun}. The second entity is \textit{Peter the Great} who died in \textit{Saint Petersburg} and not \textit{Moscow}, and who did not receive the \textit{Order of St Alexander Nevsky} which was first established by his wife, a few months after his death. And the third entity is \textit{Twist and Shout}. Although it is a known song by \textit{The Beatles}, they were \textit{not} its composers, writers, nor original performers.

\begin{table*}
  \caption{Results for the entities \textit{Daily Mirror}, \textit{Peter the Great}, and \textit{Twist and Shout}.}
  \label{tab:won}
    \center
    \begin{centering}
  \scalebox{0.9}{
  \begin{tabular}{c|c}
    \toprule
      \multicolumn{2}{c}{\textbf{Daily Mirror}}\\
          \midrule
    \multicolumn{1}{c}{\textbf{Pos-only}} & \multicolumn{1}{c}{\textbf{Pos-and-neg}}\\
    \midrule 
(owned by; Reach plc) & \bf{\textit{$\neg$(newspaper format; broadsheet)}}\\
(newspaper format; tabloid) & (newspaper format; tabloid)\\
(country; United Kingdom) & \bf{\textit{$\neg$(country; U.S.)}}\\
(language of work or name; English) & (language of work or name; English)\\
(instance of; newspaper) & \bf{\textit{$\neg$(owned by; News U.K.)}}\\
... & ...\\
\midrule
      \multicolumn{2}{c}{\textbf{Peter the Great}}\\
          \midrule
    \multicolumn{1}{c}{\textbf{Pos-only}} & \multicolumn{1}{c}{\textbf{Pos-and-neg}}\\
    \midrule 
(military rank; general officer) & (military rank; general officer)\\
(owner of; Kadriorg Palace) & (owner of; Kadriorg Palace)\\
(award; Order of the Elephant) & \bf{\textit{$\neg$(place of death; Moscow)}}\\
(award; Order of St. Andrew) & (award; Order of St. Andrew)\\
(father; Alexis of Russia) & \bf{\textit{$\neg$(award; Knight of the Order of St. Alexander Nevsky)}}\\
... & ...\\
\midrule
      \multicolumn{2}{c}{\textbf{Twist And Shout}}\\
          \midrule
    \multicolumn{1}{c}{\textbf{Pos-only}} & \multicolumn{1}{c}{\textbf{Pos-and-neg}}\\
    \midrule 
(composer; Phil Medley)& \bf{\textit{$\neg$(composer; Paul McCartney)}}\\
(performer; The Beatles) & (performer; The Beatles)\\
(producer; George Martin) & \bf{\textit{$\neg$(composer; John Lennon)}}\\
(instance of; musical composition) & (instance of; musical composition)\\
(lyrics by; Phil Medley) & \bf{\textit{$\neg$(lyrics by; Paul McCartney)}}\\
... & ...\\
    \bottomrule
  \end{tabular}
  }
  \end{centering}
  \end{table*}
  
\begin{table*}
  \caption{Positive-only vs.\ positive and negative statements.}
  \label{tab:posneg}
  \center
  \begin{centering}
  \scalebox{0.99}{
  \begin{tabular}{l|l|l|l}
    \toprule
        \multicolumn{1}{c}{\textbf{Preferred Choice}} & 
        \multicolumn{1}{c}{\bf{Person} \textbf{(\%)}} & \multicolumn{1}{c}{\bf{Organization} \textbf{(\%)}} & \multicolumn{1}{c}{\bf{Literary work} \textbf{(\%)}}\\
            \midrule
    \multicolumn{1}{l}{Pos-and-neg} & \multicolumn{1}{c}{\textbf{71}} &\multicolumn{1}{c}{\textbf{77}}&\multicolumn{1}{c}{\textbf{66}}\\
    \multicolumn{1}{l}{Pos-only} & \multicolumn{1}{c}{22} &  \multicolumn{1}{c}{10} & \multicolumn{1}{c}{17} \\
    \multicolumn{1}{l}{Both or neither} & \multicolumn{1}{c}{7} & \multicolumn{1}{c}{13} & \multicolumn{1}{c}{17}\\
    \bottomrule
  \end{tabular}
  }
  \end{centering}
\end{table*}


In this experiment, we showed that adding negative statements to a set of positive statements increases its quality, and for that, we chose a split of 7 positive and 3 negative statements for top-10 results. One may wonder whether that is actually the best proportion. This motivates another analysis, \textit{finding out the portion of negative statements to be added to a positive top-k set of statements that maximizes the relevance gain} (i.e., nDCG). We used the annotators' assessment of relevancy of individual positive and negative statements. 
 We then compiled them as sets of top-k results with different k values and different portions of negative statements. The decision of adding a certain negative statement should respect the constraint of not decreasing the relevance gain (i.e., nDCG) of the currently chosen top-k results. We calculated the ideal ratio of positive to negative statements for k results. The ideal portion of negative statements within top-k statements about entity \textit{e} was obtained for k=3, 5, 10, and 20. For a set of top-3 or top-5 statements, 1 negative statement is ideal, for 10 statements, 2 are ideal, and for 20, 5 are ideal.

\subsection{Decision Support}
Negative statements are highly important also in specific domains. In online shopping, characteristics not possessed by a product, such as the \textit{IPhone 7} not having a headphone jack, are a frequent topic highly relevant for decision making. The same applies to the hospitality domain: the absence of features such as free WiFi or gym rooms are important criteria for hotel bookers, although portals like Booking.com currently only show (sometimes overwhelming) positive feature sets.

To illustrate this, based on a comparison of 1.8k hotels in India, as per their listing on Booking.com, using the peer-based method, we inferred useful negative features. For peering, we considered all other hotels in India, and for ranking, we computed peer frequencies (\textit{PEER}). We then used crowdsourcing over the results of 100 hotels. We asked annotators to check two sets of features about a given hotel, one set containing 5 random %\sr{I added ``random'' here - is that correct?? This may be a point of criticism - random positives versus top negatives, the negatives are better, but that may just be the ranking? Perhaps tone down/shorten this first evaluation, the second one on decision making is anyway much more interesting} \ha{Yes, random. But random here, unlike with Wikidata, doesn't necessarily mean useless. Like another label for an entity or the weight of a person. Most of the features are common between hotels. But I see your point.} 
positive-only features, and one set containing a mix of 3 positive and 2 negative features. Their task was to choose which set of features will help them more in deciding whether to stay in this hotel or not. They can choose one of the sets, or both. For every hotel, we request 3 annotators.

Table~\ref{tab:hotelsnumber} shows that sets with negative features were chosen 16 percentage points more than the positive-only sets. The standard deviation of this task is 0.22 and the percentage of queries with full agreement is 28\%. Table~\ref{tab:hotels} shows three hotels with useful negative features. Although the \textit{Hotel Asia The Dawn} lists 64 positive features, negative information such as that it does not offer air conditioning and free Wifi may give important clues for decision making.
\begin{table}
  \caption{Usefulness of hotel features.}
  \centering
  \label{tab:hotelsnumber}
  \begin{tabular}{l|l}
    \toprule
        \multicolumn{1}{c}{\textbf{Preferred Choice}} & 
        \multicolumn{1}{c}{\textbf{(\%)}}\\
            \midrule
    \multicolumn{1}{l}{Pos-and-neg} & \multicolumn{1}{c}{\textbf{54}}\\
    \multicolumn{1}{l}{Pos-only} & \multicolumn{1}{c}{38}\\
    \multicolumn{1}{l}{Either or neither} & \multicolumn{1}{c}{8}\\
    \bottomrule
  \end{tabular}
\end{table}

\begin{table*}
\caption{Negative statements for hotels in India.}
\centering
\scalebox{0.9}{\begin{tabular}{lcl}
\toprule
\bf{Hotel} & \bf{Number of positive features} & \bf{Top-3 negative features} \\ \midrule
The Sultan Resort & 106 & $\neg$ Parking; $\neg$ Fan; $\neg$ Newspapers\\
Vista Rooms at Mount Road & 28 & $\neg$ Room service; $\neg$ Food \& Drink; $\neg$ 24-hour front desk\\
Hotel Asia The Dawn & 64 & $\neg$ Air conditioning; $\neg$ Free Wifi; $\neg$ Free private parking\\
\bottomrule
\end{tabular}}
\label{tab:hotels}
\end{table*}

Moreover, we collected 20 pairs of hotels from the same dataset, and showed every pair's Booking.com pages to 3 annotators. We asked them to choose the better hotel for them. Then we showed them negative features about the pair, and asked them whether this new information would change their mind on their initial decision. A screenshot of the task is shown in Figure~\ref{fig:hotel}. 42\% changed their pick after negative features were revealed. The standard deviation on this task is 0.15. The full agreement of the 3 annotators on \textit{changing the hotel after negative features were revealed} is 35\%. The full agreement of annotators \textit{choosing the same hotel at the end of the task} is 30\%. The latter agreement measure disregard whether they have changed their decision or stayed with their initial choice.

\begin{figure*}
 \caption{Extrinsic use-case: decision support on hotel data.}
 \centering
\includegraphics[width=0.85\textwidth]{figures/hotel.png}
\label{fig:hotel}
\end{figure*}

\subsection{Question Answering}
In this experiment, we compared the results to negative questions over a diverse set of sources. We manually compiled 20 questions that involve negation, such as \emph{``Actors without Oscars''}\footnote{Sample textual queries: ``actors with no Oscars'', ``actors with no spouses'', ``film actors who are not film directors'', ``football players with no Ballon d'Or'', ``politicians who are not lawyers''.}. We compared them over four highly diverse sources: Google Web Search (increasingly returning structured answers from the Google knowledge graph~\cite{GKG}), WDAqua~\cite{Diefenbach2017} (an academic state-of-the-art KBQA system), the Wikidata SPARQL endpoint~\footnote{\url{https://query.wikidata.org/}} (direct access to structured data), and our peer-based method. For Google Web Search and WD\-Aqua, we submitted the queries in their textual form, and considered answers from Google if they come as structured knowledge panels. For Wikidata and peer-based inference, we transform the queries into SPARQL queries\footnote{sample SPARQL queries: \url{https://w.wiki/A6r}, \url{https://w.wiki/9yk}, \url{https://w.wiki/9yn}, \url{https://w.wiki/9yp}, \url{https://w.wiki/9yq}}, which we either fully executed over the Wikidata endpoint, or executed the positive part over the Wikidata endpoint, while evaluating the negative part over a dataset produced by our peer-based inference method. Note that all queries were safe, since they were designed to always asks for a class of entities (e.g., entities of occupation actor) that do not satisfy a certain property (e.g., having won the Oscar), which was captured via SPARQL MINUS with a shared variable. For each method, we then self-evaluated the number of results, the correctness and relevance of the (top-5) results. All methods were able to return highly correct statements, yet Google Web Search and WDAqua return no results for 18 and 16 of the queries, respectively.

We continued the assessment over a sample of 5 queries. Wikidata SPARQL returned by far the highest number of results, 250k on average, yet did not perform ranking, thus returned results that are hardly relevant (e.g., a local Latvian actor to the Oscar question). The peer-based inference outperforms it by far in terms of relevance (72\% vs.\ 44\% for Wikidata SPARQL). We point out that although Wikidata SPARQL results appear highly correct, this has no formal foundation, due to the absence of a stance of OWA KBs towards negative knowledge. For example, most actors or people did \textit{not} win Oscars, which makes 99.99\% of the entities returned by Wikidata's SPARQL query correct, even under the OWA.

%\section{Discussion}
%\label{sec:disc}
%\ha{Do we still need this section? Since we dropped the text method and our statistical inference methods are now NOT complementary as much as the second is an refined version of the first}
%\newtext{The two presented approaches, namely \textit{peer-based} and \textit{temporal} statistical inference, are believed to be complementary for the following reasons:
%\begin{enumerate}
%    \item Relevance: While there is a sizeable overlap between the negative statements inferred in both methods, the temporal statistical inference is superior with providing more interesting explanations for the inferred statements, as shown in Table~\ref{tab:explanations}. 
%    \item Coverage: For an entity to be considered in the temporal method, it has to be a member of at least one ordered set of peers, which gives the subject coverage advantage in this case to the peer-based method.
%    \item Correctness: Because of its more concrete way of grouping highly related entities, the correctness of the negative statements in the temporal method were shown to be more correct than the similarity based method, as shown in Table~\ref{tab:simvstempcorrec}.
%\end{enumerate}}

\section{Resources}
\label{sec:datasets}
\noindent
\textbf{Negative Statement Datasets for Wikidata.\ }
\label{sec:dataset}
We publish the first 
datasets that contain dedicated \emph{useful} negative statements about entities in Wikidata: (i) Peer-based and order-oriented inference data: 14m negative statements about popular 600k entities from various types, (ii) release the mturk-annotated on the correctness of 1k negative statements of Section~\ref{subsec:similarityexp}, and (iii) 40k ordered set of peers introduced in Section~\ref{sub:temporalexperiments}.

\noindent
\textbf{Open-source Code.\ } We publish our code for peer-based inference, so others can execute it on their own datasets \footnote{\url{https://github.com/HibaArnaout/usefulnegations}}.

\noindent
\textbf{Demo.\ }
\label{sec:demo}
A web-based platform, Wikinegata~\cite{ArnaoutRWP21,arnaout2021negative} for browsing useful negations about Wikidata entities, is available at: \url{https://d5demos.mpi-inf.mpg.de/negation/}.

\noindent
A screenshot is shown in Figure~\ref{fig:demo}. 

All experimental material related to this paper can be found on a dedicated webpage\footnote{\url{https://www.mpi-inf.mpg.de/departments/databases-and-information-systems/research/knowledge-base-recall/interesting-negations-in-kbs}}.



\begin{figure*}
 \caption{Interface for Wikinegata - useful negative statements about \textit{Leonardo DiCaprio}.}
 \centering
\includegraphics[width=0.9\textwidth]{figures/wikinegata.png}
\label{fig:demo}
\end{figure*}


\section{Discussion}
\label{sec:discussion}
\subsection{Quality Considerations}

\noindent
\textbf{The CWA on the Semantic Web.\ }
Negation has traditionally been avoided on the Semantic Web, as it challenges the vision that anyone can state anything, without risking logical conflicts. In the present work, we showed that enriching KBs with useful negative statements is beneficial in use cases such as entity summarization and consumer decision making. In order to compile a set of likely correct negative statements about an entity, we assumed the CWA in parts of the KBs, namely within peer groups, and in the case of 
%ground 
grounded
%%%GW: check the whole paper: ground --> grounded
%
negative statements, with the additional requirement that there is at least one other positive statement for the same entity-property pair. Although this approach outperforms other techniques, like embedding-based KB completion, inferences may still be incorrect. While correctness can be tuned to some extend by sacrificing recall (e.g., requiring very high thresholds on \textit{PEER} and \textit{PIVO}), errors are still possible. It is therefore advised to show 
%our automated inferences to end-users as ``truths'', but rather to KB curators as suggestions~\cite{RECOIN}. 
candidate statements from automatic inference to KB curators
for final assessment~\cite{RECOIN}.
%Alternatively, they may be used at aggregate level for training ML models~\cite{CSKB}.
%They could nevertheless be used also as noisy samples for training ML models
%at an aggregate level~\cite{CSKB}.


\noindent
\textbf{Real-world Changes and KB Maintenance.\ } 
Due to real-world changes or new information added to the KB, some of the negative statements already inferred might become incorrect. For instance, \textit{DiCaprio} has won his first \textit{Oscar} in 2016. After the year 2016, the negative statement \term{$\neg$(DiCaprio; award; Oscar)} is no longer correct. Negative statements should therefore be timestamped, and ideally, additions of positive statements should automatically trigger updates of validity end-point timestamps.

\noindent
\textbf{Class Hierarchies.\ }
Some incorrect negations can be detected by help of subsumption checks (rdfs:subClassOf). For example, the presented method might incorrectly infer the statement \term{$\neg$(Douglas Adams; occupation; author)}, which contradicts the two positive assertions that \textit{Douglas Adams is a writer}, and \textit{writer} is a subclass of \textit{author}. One could detect such contradictions by use of a generic ontology reasoner like Protégé, or implement custom checks. For our specific use case of negative inference at scale, we found that checks focused on one or two hops in the class hierarchy capture a significant proportion of these errors. For KBs at the scale of Wikidata, one could precompute prominent subsumptions, and build these checks into the methodology (e.g., triggering a check for presence of ``occupation-writer'' whenever ``occupation-not author'' is inferred).

Subsumption similarly also affects properties: the relation \textit{CEO} (between a company and a person) is a subproperty of \textit{employee}, and as such, subject-object-pairs present for the former should not appear as negations for the latter.


%To tackle incorrect negations due to problems with class hierarchies, at inference time, one can perform a number of hierarchical checks, using the property \textit{subclass of} for $n$ levels upward and $n$ downward. For n=1, an upward check would be ``director is a subclass of artist'' and a downward check would be ``film-director is a subclass of director''. For example, we consider the statement \term{$\neg$(Douglas Adams; occupation; author)} \textit{incorrect} because he is a writer, and writer is a subclass of author.
%In an interactive retrieval system, these checks can be computationally expensive. The precomputation of inferences as well as hierarchical checks are recommended. Another way of exploiting the class hierarchy is to involve the subsumption relation between classes earlier in the methodology, for instance, when collecting positive statements about the input entity and its peers.}

%\noindent
%\newtext{\textbf{Type Relations.\ }
%Similarly to classes, type relations can contribute to the correctness of the inferred negations. For example, for awards and their instances, a winner of an \textit{Academy Award for Best Actor} is a winner of an %\textit{Academy Award} (because the former is an ``instance of'' the latter).}
%newtext


\noindent
\textbf{Modelling and Constraint Enforcement.\ } 
Some inferred negations are incorrect due to modelling issues, resulting in inconsistencies. An example is \textit{Dijkstra} and the negative statement that his field of work is \textit{not} \textit{Computer Science}, and \textit{not} \textit{Information Technology}, while he has the positive value \textit{Informatics}, which is arguably near-synonymous, yet in the Wikidata taxonomy, the two represent independent concepts, two hops apart. Some other incorrect negative statements could be due to a lack of constraints. For instance, for most businesses, the \textit{headquarters location} property is completed using \textit{cities}, but for \textit{Siemens} in Wikidata, the \textit{building} is added instead (\textit{Palais Ludwig Ferdinand}), making our inferred statement \term{$\neg$(Siemens; headquarters location; Munich)} incorrect. Although Wikidata encourages editors to use cities for the  \textit{headquarters location} property and advise them to use another property for specific buildings, it has not been automatically enforced yet.\footnote{\url{https://www.wikidata.org/wiki/Property:P159}}
%newtext

\subsection{Discovering Relevant Lifting Aspects}
For inferring conditional negative statements, the lifting aspects we used in this paper have been manually defined (see Table~\ref{tab:aspects}). For instance, if the grounded negative statements to be lifted describe educational institutions, then the aspects that make sense are the location of the institution (\textit{U.S., Germany, Japan, etc.}) and its type (\textit{public, private, research, etc.}). This 
%for sure, 
does not scale well when the KB contains thousands of properties with thousands of possible aspects. Automatically discovering these aspects would improve the quality of conditional negative statements. A good start is the work in~\cite{oren2006extending}. 
An aspect is described as an important characteristic of an entity. For example, for a book, the number of pages is not an important aspect, but genre is.
This work introduces aspect ranking metrics such as object cardinality: a \textit{good} predicate (e.g., genre) has a finite list of values to choose from (e.g., comedy, thriller, romance). Unsuitable predicates using this metric would be the predicate \textit{number of pages} or \textit{publication date}. In addition, the AMIE system~\cite{AMIE,AMIEP} mines rules on millions of triples, and is specifically tailored to support open-world KBs. It can discover, for example, that \textit{musicians that are influenced by each other often play the same instrument.} The \textit{instrument} can be directly used as an aspect for lifting grounded negative statements (with predicate \textit{influenced by}) about a \textit{musician-entity}. In particular, \textit{musician x} (a pianist), is not influenced by anyone who plays the guitar, or more surprisingly not influenced by anyone who plays the piano (if that is the case). We consider this to be a promising research direction. It is worth exploring and improving the ideas in Section~\ref{sec:restricted} further.%newtext

\subsection{Entity Prominence and Class Specificity}

\noindent
\textbf{Negations in the Long Tail.\ }
Our method builds on the assumption that peer entities are available, 
%for which data is 
%sufficiently complete. 
for which we have sufficient data.
%%%GW: complete sounds like a technically crisp term, and so does sufficiently complete. if this is meant informally here, then use more liberal wording.
%
For long-tail entities, both assumptions may be challenged. For entities with extremely little positive information (e.g., \url{https://www.wikidata.org/wiki/Q97355589}, for which only first name, last name, and gender are known), it is not possible to identify relevant peers, and hence, our method is not applicable. Low amounts of positive information on peers, in contrast, can be better compensated.
Since our method is mainly concerned with finding the most interesting candidates for negation, absolute frequencies are not important, as long as it is possible to find a reasonable difference in frequencies among peers (i.e., not every positive statement appearing only once). If there is interest to put emphasis on specific facts, one could also 
%manually tweak 
adjust
the ranking algorithms, e.g., giving ``citizenship'' negations a 
%static 
boost in the ranking.
%newtext



\begin{table*}
  \caption{Negations across classes of Wikidata entities.}
  \centering
  \label{tab:differenttypes}
    \scalebox{0.8}{\begin{tabular}{l|l|l|l}
    \toprule
        \multicolumn{1}{c}{\textbf{Class}} & 
        \multicolumn{1}{c}{\textbf{Number of entities}} &
        \multicolumn{1}{c}{\textbf{3 most frequent negated properties}} &  \multicolumn{1}{c}{\textbf{Sample entities}}\\
            \midrule
    \multicolumn{1}{l}{Book} & \multicolumn{1}{c}{8k} & \multicolumn{1}{l}{author, genre, publisher} & Fahrenheit 451, Little Birds\\
    \multicolumn{1}{l}{Person} & \multicolumn{1}{c}{500k} & \multicolumn{1}{l}{spouse, child, occupation} & Elon Musk, Oprah Winfrey\\
    \multicolumn{1}{l}{Country} & \multicolumn{1}{c}{199} & \multicolumn{1}{l}{diplomatic relation, member of, language used} & Germany, China\\
     \multicolumn{1}{l}{Primary school} & \multicolumn{1}{c}{14k} & \multicolumn{1}{l}{instance of, heritage designation, country} & Deutsche Schule Helsinki, Saint Joseph school\\
     \multicolumn{1}{l}{Film} & \multicolumn{1}{c}{26k} & \multicolumn{1}{l}{cast member, genre, screenwriter} & Taxi Driver, Inception\\    
     \multicolumn{1}{l}{Building} & \multicolumn{1}{c}{28k} & \multicolumn{1}{l}{architect, instance of, heritage designation} & NY Times Building, White House\\ 
     \multicolumn{1}{l}{Organizations} & \multicolumn{1}{c}{22k} & \multicolumn{1}{l}{headquarters location, instance of, country} & World Trade Organization, BBC\\
     \multicolumn{1}{l}{Musical group} & \multicolumn{1}{c}{8k} & \multicolumn{1}{l}{instance of, record label, genre} & Coldplay, Jonas Brothers\\ 
     \multicolumn{1}{l}{Business} & \multicolumn{1}{c}{20k} & \multicolumn{1}{l}{parent organization, headquarters location, industry} & Nokia, Facebook\\ 
     \multicolumn{1}{l}{Scientific journal} & \multicolumn{1}{c}{5k} & \multicolumn{1}{l}{main subject, editor, publisher} & Journal of Web Semantics, Nature\\
     \multicolumn{1}{l}{Literary work} & \multicolumn{1}{c}{24k} & \multicolumn{1}{l}{author, composer, lyrics by} & Diary of Anne Frank, Don Quixote\\
    \bottomrule
  \end{tabular}}
\end{table*}


\noindent
\textbf{Negations for Different Classes.\ }
In practice, we have applied our method on 11 diverse classes of entities: people, literary works, organizations, businesses, scientific journals, countries, buildings, musical groups, primary schools, books, and films. We have observed that within each class, interesting negations often cover the same properties. For instance, for people, interesting negative knowledge is mostly about awards, occupations, education, and 
%spouses. 
family.
%
We show 
%some 
statistics on frequent properties for every class of entities in Table~\ref{tab:differenttypes}. We do \textit{not} filter nor assign weights for certain properties per class. The relative frequency metric takes care of prioritizing which property's negation \textit{makes sense} in every class. For \textit{people}, the reported properties are fairly {general} and not %reflective of 
tied to
specific subsets of this very large class. For instance, for sports figures, \textit{member of sports team} is the most frequent property, and for politicians, \textit{position held} is the dominating property.

We notice that negations for small classes, such as buildings and literary works,
%are often more \textit{correct} 
have a higher correctness ratio
than larger classes, such as people. Entities of type \textit{person} have 3 times more possible properties to fill than entities of type \textit{book}. Given a book (e.g., \textit{Orientalism}), a handful of properties and property-object pairs could be added and the information about the entity is considered near-complete (e.g., \textit{main subject}, \textit{author}, \textit{genres}, \textit{publisher}, and \textit{language}). 
%Whereas 
In contrast,
for a person (e.g., \textit{Joe Rogan}), the entity requires a greater effort and/or larger information sources to be considered complete (e.g.,  \textit{occupation}, \textit{education}, \textit{residence}, \textit{birth place}, \textit{citizenship}, \textit{sport}, \textit{religion}, and many more). On the other hand, larger classes offer richer and more diverse possibilities for \textit{interesting} negations. A result set for a person often covers a wider range of topics, such as personal information, professional achievements, relations with other people. A result set for a book is less diverse, often negating the same property repeatedly with different objects.\\



\section{Conclusion \& Future Directions}

This article has made the first comprehensive case for explicitly materializing useful negative statements in KBs. We have introduced a statistical inference approach on retrieving and ranking candidate negative statements, based on expectations set by highly related peers. We have also released several resources to encourage further research.

In future work we would like to explore a number of research directions:
\begin{enumerate}
    \item Missing vs negative statements: How to maximize trade-ability between fewer highly correct statements, and larger sets of interesting negation candidates.
    \item Mining complex negations: Our focus was on simple - grounded and universal - negation, with a hint at more complex conditional statements. It is open to extend that to (i) automatically finding aspects, (ii) further joins \textit{``did not study at a university which was graduating any Nobel prize winner''}, (iii) negation on sets of entities instead of entity-centric \textit{``no African country has hosted any Olympic games''}, etc.
    \item Exploring how textual information extraction of implicit negations can boost negation coverage, e.g., statements like \textit{``Theresa May is an only child.''} (corresponding to \term{$\neg \exists x$(sibling; x)}), or \textit{``George Washington had no formal education.''} (corresponding to \term{$\neg \exists x$(educated at; x)}).
    \item Exploiting the ontology that comes with the KB to improve the correctness of inferred negations by making use of constraints like class and property subsumption.
\end{enumerate}


\section*{Acknowledgement}
\noindent This work is supported by the German Science Foundation (DFG: Deutsche Forschungsgemeinschaft) by grant 4530095897: ``Negative Knowledge at Web Scale''.


\bibliographystyle{splncs04}
\bibliography{refs}


\end{document}