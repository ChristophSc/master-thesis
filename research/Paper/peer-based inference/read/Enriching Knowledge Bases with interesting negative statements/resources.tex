\section{Resources}
\label{sec:datasets}
\noindent
\textbf{Negative Statement Datasets for Wikidata.\ }
\label{sec:dataset}
We publish the first 
datasets that contain dedicated \emph{useful} negative statements about entities in Wikidata: (i) Peer-based and order-oriented inference data: 14m negative statements about popular 600k entities from various types, (ii) release the mturk-annotated on the correctness of 1k negative statements of Section~\ref{subsec:similarityexp}, and (iii) 40k ordered set of peers introduced in Section~\ref{sub:temporalexperiments}.

\noindent
\textbf{Open-source Code.\ } We publish our code for peer-based inference, so others can execute it on their own datasets \footnote{\url{https://github.com/HibaArnaout/usefulnegations}}.

\noindent
\textbf{Demo.\ }
\label{sec:demo}
A web-based platform, Wikinegata~\cite{ArnaoutRWP21,arnaout2021negative} for browsing useful negations about Wikidata entities, is available at: \url{https://d5demos.mpi-inf.mpg.de/negation/}.

\noindent
A screenshot is shown in Figure~\ref{fig:demo}. 

All experimental material related to this paper can be found on a dedicated webpage\footnote{\url{https://www.mpi-inf.mpg.de/departments/databases-and-information-systems/research/knowledge-base-recall/interesting-negations-in-kbs}}.



\begin{figure*}
 \caption{Interface for Wikinegata - useful negative statements about \textit{Leonardo DiCaprio}.}
 \centering
\includegraphics[width=0.9\textwidth]{figures/wikinegata.png}
\label{fig:demo}
\end{figure*}

