\section{Conclusion \& Future Directions}

This article has made the first comprehensive case for explicitly materializing useful negative statements in KBs. We have introduced a statistical inference approach on retrieving and ranking candidate negative statements, based on expectations set by highly related peers. We have also released several resources to encourage further research.

In future work we would like to explore a number of research directions:
\begin{enumerate}
    \item Missing vs negative statements: How to maximize trade-ability between fewer highly correct statements, and larger sets of interesting negation candidates.
    \item Mining complex negations: Our focus was on simple - grounded and universal - negation, with a hint at more complex conditional statements. It is open to extend that to (i) automatically finding aspects, (ii) further joins \textit{``did not study at a university which was graduating any Nobel prize winner''}, (iii) negation on sets of entities instead of entity-centric \textit{``no African country has hosted any Olympic games''}, etc.
    \item Exploring how textual information extraction of implicit negations can boost negation coverage, e.g., statements like \textit{``Theresa May is an only child.''} (corresponding to \term{$\neg \exists x$(sibling; x)}), or \textit{``George Washington had no formal education.''} (corresponding to \term{$\neg \exists x$(educated at; x)}).
    \item Exploiting the ontology that comes with the KB to improve the correctness of inferred negations by making use of constraints like class and property subsumption.
\end{enumerate}


\section*{Acknowledgement}
\noindent This work is supported by the German Science Foundation (DFG: Deutsche Forschungsgemeinschaft) by grant 4530095897: ``Negative Knowledge at Web Scale''.

