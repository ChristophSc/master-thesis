\chapter{Timeline}
\label{ch:timeline}

% general description of the schedule
The schedule shown in Figure \ref{fig:timeline} is established for the master thesis.
\begin{sidewaysfigure}
  \centering
    \includegraphics[width=0.95\textwidth]{figures/gantt_chart.png}
  \caption{Gantt-Chart with tasks and milestones of master thesis time plan.}
  \label{fig:timeline}
\end{sidewaysfigure}
The start of the thesis is scheduled for the beginning of November.
The work is basically divided into a practical and a theoretical part, which should take place at the same time and both use multiple milestones.

% Theoretical part
The theoretical part contains three milestones:
First, further research work follows so that the Introduction, Related work and Background for this Thesis can be given and the first milestone is reached.
Subsequently, our \ac{UCGAN} approach is described, evaluated and summarized, so that the next milestone of a first version of the thesis is reached.
After further corrections and revisions, the final version of the implementation will be handed in at the end of March.

%Practical Part
Simultaneously during the writing of the thesis the practical implementation of the \ac{UCGAN} approach takes place.
First, the individual components of the score function are implemented, which includes the \ac{PEER}, \ac{POP}, \ac{FRQ} and \ac{PIVO} function.
Second, the comprehensive score function is implmenented which includes the components and the embedding function $f_G$.
Finally, Uncertainty Sampling will be implemented such that the first milestone of a first working version of \ac{UCGAN} is reached.
Subsequently, an evaluation to compare the approach with others, further improvements and refactoring take place.
For theoretical and practical part the final milestone of submitting the thesis as well as the code is end of March.
