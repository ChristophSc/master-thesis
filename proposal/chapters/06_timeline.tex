\chapter{Timeline}
\label{ch:timeline}

% general description of the schedule
The schedule shown in \autoref{fig:timeline} is established for the master thesis.
\begin{figure}
  \centering
    \includegraphics[width=\textwidth]{figures/gantt_chart.png}
  \caption{Gantt-Chart with tasks and milestones of master thesis time plan.}
  \label{fig:timeline}
\end{figure}
The work is basically divided into a theoretical part of writing and a practical part of implementation.
Both of them should take place at the same time.
Following the agile software development method SCRUM, the total time of five months is split into individual sprints.
Each of them has a duration of four weeks and at the end of each sprint a certain theoretical and practical part should be completed.
In addition, important completions and deadlines for the master thesis were included as milestones in the gantt-chart.
The theoretical part includes the milestone for the submission of a first, finished version and the final submission of the master thesis.

% Theoretical part
In the theoretical part of writing, the chapters of the master thesis are completed successively over the sprints until the first milestone: the submission of the first complete version of the thesis is reached.
After further corrections and rewriting, the final submission of the master thesis takes place.

%Practical Part
The practical part of the work consists of the initial creation of a simple approach.
This is further developed and improved in the course of the sprints, so that the final structure of the approach is fixed after the third sprint.
After it has been evaluated and compared with existing state-of-the-art approaches, the final implementation is submitted at the end of the overall time.

