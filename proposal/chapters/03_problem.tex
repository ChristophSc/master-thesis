\chapter{Problem}
\label{ch:problem}

% general introduction
Since \acp{KG} contain large amount of information, they provide many application possibilities.
However, to achieve good results for these applications we need to learn good embeddings which represent as much information as possible in a graph representation of a low-dimensional vector space.
Most of graph representation learning methods can be unified within a \ac{SampledNCE} framework comprising an encoder that generates node embeddings by learning to distinguish pairs of a positive and a negative triple \cite{MCNS}.
Therefore, we need negative triples for embedding learning which are not available in most \acp{KG}.
For this reason, we are left with two options to collect them \cite{safavi2021negater}: 
On one hand they can be obtained by human annotation which is very cost-prohibitive at scale, but leads to very informative and useful examples. 
On the other, they can be generated ad-hoc, which in turn does not represent a great effort, but also leads to uninformative and useless negative examples.
For this reason, we need to find a way between these two extreme approaches.

% degree-based sampling
In recent years, various machine learning approaches have proven to find more useful and informative negative examples.
One straight forward approach is to generate negative examples degree-based.
For example, negative examples can be gathered by randomly replacing either head or tail entity in a given positive triple.
Examples generated in this way are very likely to represent negative examples, but usually are uninformative and useless.
E.g by replacing the head entity in given positive triple (Steve Jobs, founded, Apple Inc.) by randomly selected entity 'Paderborn' leads to the new negative triplet (Paderborn, founded, Apple Inc.), which is a negative example but also uninformative and useless for embedding learning.
The problem with "too easy" negative examples is less severe to models using log-softmax loss function, because they usually sample a high amount of negatives for one positive triple \cite{cai2017kbgan}.
However, the performance of marginal loss functions can be seriously damaged by the low quality of uniformly sampled negatives since negative-to-positive ratio is always 1:1 \cite{cai2017kbgan}.

% hard-samples negative sampling
To find informative and more useful negative samples, they can be compared and ranked by different metrics which indicate the closeness to the positive sample.
For this purpose, various options have already been presented that use different metrics.
E.g. \cite{DNS} presented with \ac{DNS} a pair-wise \ac{CF} algorithms based on \ac{BPR}.
Other approaches like \ac{PinSAGE} \cite{PinSAGE} which is based on \ac{GraphSAGE} \cite{GraphSAGE} rely on random-walk-based \ac{GCN}. 
A producer samples the network neighborhoods of a node, while the consumer runs \ac{SGD} on computation graphs.    
One of the latest approaches for negative sampling is \ac{NegatER}
which ranks a selection of statements using fine-tuned contextual \ac{LM} \cite{safavi2021negater}.
With this approach, negative sampling process takes place in two steps:
Firstly, a language model for positive knowledge is fine-tuned by using contrastive learning.
Subsequently, statements are ranked by k-nearest-neighbors retrieval approach which are most plausible or most surprising.

With these approaches of negative sampling from a fixed distribution two different problems arise:
Since they ignore changes in the distribution of negative triples, they suffer from the vanishing gradient and biased estimation problem \cite{zhang2021efficient}.
The scoring functions tend to give observed (positive) triplets large values and most of the non-observed (probably negative) triplets will have smaller values during the training.
Therefore, if negative triplets are uniformly sampled, it is very likely to pick up one with zero gradients.

This leads to the following main challenges for negative sampling \cite{zhang2021efficient}: 
(i) The negative triple's dynamic distribution has to be captured and (ii) triples have to be effectively sampled from this distribution.

% GAN-based approaches
Two pioneering works \ac{KBGAN} \cite{cai2017kbgan} and \ac{IGAN} \cite{IGAN} attempt to address these challenges by a \ac{GAN} \cite{zhang2021efficient}.
\acp{GAN} were originally proposed for generating samples in a continuous space such as images and adapted for the generation of hard negative examples \cite{zhang2021efficient}.
The process of \ac{KBGAN} is depicted in the Figure \ref{fig:overview} and described as follows:
\begin{figure*}[t]
  \centering
    \includegraphics[width=0.95\textwidth]{figures/kbgan_original.png}
  \caption{An overview of the \textsc{kbgan} framework (based on: \cite{cai2017kbgan})}
  \label{fig:overview}
\end{figure*}

\begin{enumerate}
    \item 
    At first, a set of sampled corrupted triples $Neg(h,r,t)\subset\{(h',r,t)|h'\in\mathcal{E}\}\cup\{(h,r,t')|t'\in\mathcal{E}\}$ is created by uniformly sampling of $N_s$ entities $\in$ \entities to replace $h$ or $t$.
    For reasons of efficiency and to reduce the likelihood of creating false negatives, $N_s$ is a relatively small number compared to the number of all possible negative examples \cite{cai2017kbgan}.
    
    \item 
    Negative triples from $Neg$ are given to the generator (G) as input.
    
    \item 
    G is a \ac{KGE} model with a softmax function to provide a probability for each given negative triple, which gives us a probability distribution $p_G(h',r,t'|h,r,t)$.

    \item 
    Afterwards, one of these triples is sampled as output of G according to the $p_G$, while high quality triples should have a high probability to be sampled.
    
    \item 
    Discriminator (D) receives as input both the sampled negative triple and the ground truth triple $(h,r,t)$.
    The training objective of D is a marginal loss function, because these benefit most from high quality negative examples \cite{cai2017kbgan}.

    \item 
    Subsequently, D calculates the scores of both triples $(h,r,t)$ and $(h',r,t')$ with score function $f_D$ of D, score function models distance between points or vectors and a smaller distance indicates a higher likelihood of truth \cite{cai2017kbgan}.
    
    \item 
    Accordingly, the objective of D is to minimize the marginal loss function $L_D$ which is defined as
    \begin{equation}
        L_D=\sum_{(h,r,t)\in\mathcal{T}}[f_D(h,r,t)-f_D(h',r,t')+\gamma]_+ \in \mathbb{R}^+
    \end{equation}
    
    \item 
    To give G feedback for the sampled negative triple $(h',r,t')$ its calculated score $f_D(h',r,t')$ is send back to G as a reward, which is defined as
    \begin{equation}
        r = -f_D(h',r,t') \in \mathbb{R}^-
    \end{equation}
    since $f_D(h,r,t) \in \mathbb{R}^+$.
    Therefore, the objective of G can be formulated as maximizing the expectation of negative distances:
    \begin{equation}
        R_G=\sum_{(h,r,t)\in\mathcal{T}}\mathbb{E}[-f_D(h',r,t')]
    \end{equation}
    
    \item
    This process continues until convergence
\end{enumerate}

However, this approach still has many problems since it increases the number of training parameters because of the generator and the model suffers from instability and degeneracy.
This results from the fact, that the  \textsc{REINFORCE} gradient has high variance for training the generator and it is still not guaranteed that large-gradient negative triples are created \cite{zhang2021efficient}.
From the set $Neg$ of negative´examples only a few negative triples lead to large gradients, and therefore, \ac{KBGAN} takes a lot of effort to model the distributions.
The consequence is an instable training performance and pre-training becomes a must for generator and discriminator model.
    
In summary, there are already several approaches for Negative Sampling, but they still have problems to capture the dynamic distribution of the negative triples or effectively sample them \cite{zhang2021efficient}.
Therefore, we present a new negative sampling technique which aims to \textbf{improve the efficiency of sampling by using more informative and useful negative examples for the embedding model}.



